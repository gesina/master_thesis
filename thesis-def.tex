%%%%%%%%%%%%%%%%%%%%%%%%%%%%%%%%%%
%  Master Thesis in Mathematics
% "Immersions and Stiefel-Whitney classes of Manifolds"
% -- Definitions --
% 
% Author: Gesina Schwalbe
% Supervisor: Georgios Raptis
% University of Regensburg 2018
%%%%%%%%%%%%%%%%%%%%%%%%%%%%%%%%%%


% --------------------------------------------
% NOMENCLATURE
% ------------
% References:
% eq:  equation,   reference by \eqref
% tag: tagged equation/item, reference by \ref
% item: untagged item,    ""
% def: definition, reference by \autoref
% thm: theorem,    ""
% lem: lemma,      ""
% cor: corollary,  ""
% rem: remark,     ""
% ---------------------------------------------

\KOMAoptions{parskip=half}

% LANGUAGE
\usepackage[ngerman, main=english]{babel}
%\hyphenation{}
\usepackage[autostyle=try]{csquotes}

% INDICES
\usepackage[style=alphabetic, backend=biber]{biblatex}
\bibliography{thesis.bib}

% \usepackage{makeidx}
%\makeindex

% Nomenclature
% Add entry via
%   \nomenclature[<prefix for sorting>]{<symbol>}{<description>}
% !! no newlines within the command; all around need to be commented out !!
% Output via:
%   \printnomenclature
% Creates .nlo file with command:
%   makeindex thesis.nlo -s nomencl.ist -o thesis.nls
\usepackage[intoc, refpage]{nomencl}
\setlength{\nomitemsep}{0pt}
\makenomenclature


% MATHS PACKAGES
\usepackage{mathtools,amsthm,amssymb,dsfont,mathrsfs}
% \usepackage{amsfonts}
\usepackage{tikz-cd}
\usetikzlibrary{babel}

% MATHS-FONT
% \usepackage{mathastext} % uses text font; bad arrows -> NO
% \usepackage{euler} % for egyptienne
% \usepackage{sfmath}\usepackage{sansmathaccent} % sans serif
% \usepackage[math]{anttor} % letters look slightly crippled in comparison
% \usepackage[math]{kurier}
% \usepackage{mathpazo}
%
% \usepackage[charter]{mathdesign} % ugly longmapsto
% Fix the \logmapsto for this font
%\renewcommand*{\longmapsto}{%
%  \kern-.35pt\mapstochar\kern0.35pt\longrightarrow}

% FONT
\usepackage{fontspec}
\setmainfont{Latin Modern Roman}
\setsansfont{Latin Modern Sans}
%
% \setmainfont{Charis SIL}
%\setsansfont{Source Sans Pro}
%
%\setsansfont{Fira Sans}
%\setsansfont{Bitstream Vera Sans} % too wide

% OTHER PACKAGES
\usepackage{enumitem}
\setlist[description]{font=\normalfont\itshape, leftmargin=1.3em}
\usepackage{booktabs}
\usepackage{graphicx}
\usepackage{hyperref}
\usepackage{tabularx}
\usepackage{ifthen}

% PROTOTYPING
\newcommand*{\optcite}[2][]{{\color{red}\cite[][#1]{#2}}}

% LANGUAGE
% Shorthands with correct spacing
\newcommand*{\idest}{i.e.\ } % i.e.
\newcommand*{\Idest}{I.e.\ } % I.e.
\newcommand*{\forexample}{e.g.\ } % e.g.
\newcommand*{\Forexample}{E.g.\ } % E.g.
\newcommand*{\itemref}[2]{\autoref{#1}.\ref{#2}} % ref to item in thm

% THEOREMS
\newtheorem{Thm}{Theorem}
\newtheorem{Prop}[Thm]{Proposition}
\newtheorem{Lem}[Thm]{Lemma}
\newtheorem{Cor}[Thm]{Corollary}
\theoremstyle{definition}
\newtheorem{Def}[Thm]{Definition}
\newtheorem{LemDef}[Thm]{Lemma/Definition}
\theoremstyle{remark}
\newtheorem{Ex}[Thm]{Example}
\newtheorem{Rem}[Thm]{Remark}
\newtheorem*{claim}{claim}
\newtheorem{Not}[Thm]{Notation}


\setlist[enumerate,1]{label=\roman*)}
\setlist[enumerate,2]{label=\alph*)}
\setlist[enumerate,3]{label=\roman*.}

%
\newlist{steps}{enumerate}{1}
\setlist*[steps]{
  font={\normalfont\bfseries},
  label={Step~\arabic*:},
  ref={Step~\arabic*},
  align=left, labelindent=1em, labelwidth=*, itemindent=0pt
}


% Named axioms with \axiom instead of \item
\newlist{axioms}{enumerate}{1}
\setlist*[axioms]{
  font={\normalfont\bfseries},
  label={Axiom~\arabic*.},
  ref={Axiom~\arabic*},
  align=left, labelindent=1em, labelwidth=*, itemindent=0pt
}
\newcommand*{\axiom}[1][]{
  \ifthenelse{\equal{#1}{}}%
  {\item}
  {% see https://tex.stackexchange.com/questions/167454/how-to-change-the-label-of-one-item-in-an-enumitem-list
  \item[\refstepcounter{axiomsi}%
    Axiom~\arabic{axiomsi}~{\normalfont\emph{(#1)}}.]%
  }
}


% MATHS DEFINITIONS
% Numbers
\newcommand*{\Nat}{\mathds{N}} % natural numbers
\newcommand*{\Z}{\mathds{Z}} % integers
\newcommand*{\Zmod}[1]{\Z_{#1}} % residual class rings
\newcommand*{\Q}{\mathds{Q}} % rational numbers
\newcommand*{\R}{\mathds{R}} % real numbers
\newcommand*{\C}{\mathds{C}} % complex numbers

% Sequences, symmetric polynomials
\newcommand*{\Part}{\mathscr{P}} % partition of an integer
\newcommand*{\Emptypart}{()} % empty partition, i.e. partition of 0
\newcommand*{\PartitionsOf}[1]{\Pi({#1})} % set of partitions of the integer #1
\newcommand*{\concat}{\oplus} % concatenation of sequences/partitions

\newcommand*{\Symm}[1]{\mathcal{S}^{#1}} % ring of symmetric polynomials in #1 variables
\newcommand*{\Permutations}[1]{\Sigma_{#1}} % group of permutations of #1 elements
\newcommand*{\symm}[1]{\text{Sym}_{#1}} % symmetrisation in #1 variables of a polynomial
\newcommand*{\el}[2]{\sigma_{#1}^{#2}} % #1th elementary symmetric polynomial in #2 variables
\newcommand*{\s}[1]{s_{#1}} % partition polynomial of the partition #1
\newcommand*{\snum}[2]{s_{#1}\fundcl{#2}} % partition polynomial of the partition #1 applied to the fundamental class of #2

% Symbol shortenings
\renewcommand*{\epsilon}{\varepsilon}
\newcommand*{\longto}{\longrightarrow}
\newcommand*{\longfrom}{\longleftarrow}
\newcommand*{\isosymb}{\sim}
\newcommand*{\isoto}{\overset{\isosymb}{\to}}
\newcommand*{\longisoto}{\overset{\isosymb}{\longto}}
\DeclareMathOperator{\im}{im} % image
\newcommand*{\oversetleft}[2]{\overset{\mathllap{#1}}{#2}} % overset with left lap
\newcommand*{\oversetcenter}[2]{\overset{\mathclap{#1}}{#2}} % overset with center lap
\newcommand*{\equalsby}[1]{\oversetleft{\text{#1}}=}
\newcommand*{\cequalsby}[1]{\oversetcenter{\text{#1}}=}
\newcommand*{\leqby}[1]{\oversetleft{\text{#1}}\leq}
\newcommand*{\cleqby}[1]{\oversetcenter{\text{#1}}\leq}

% maps
\renewcommand*{\Im}{\mathrm{im}} % image
\newcommand*{\Id}[1][]{\mathrm{id}_{#1}} % identity map
\newcommand*{\incl}{\mathrm{incl}} % inclusion
\newcommand*{\proj}{\mathrm{proj}} % projection
\newcommand*{\pb}[1]{{#1}^*} % pullback by a map
\newcommand*{\pf}[1]{{#1}_*} % pushforward by a map
\newcommand*{\susp}{\Sigma} % suspension map
\newcommand*{\immto}{\hookrightarrow} % immersion arrow
\newcommand*{\immlongto}{\lhook\joinrel\longrightarrow} % long immersion arrow

% spaces
\newcommand*{\Sph}{S} % sphere symbol
\newcommand*{\Sphere}[1]{\Sph^{#1}} % {#1}-sphere
\newcommand*{\CP}[1][\infty]{\C{}\mathrm{P}^{#1}} % complex projective space
\newcommand*{\RP}[1]{\R{}\mathrm{P}^{#1}} % #1th real projektive space
\newcommand*{\RPinf}{\RP{\infty}} % infinite real projektive space
\DeclareMathOperator{\rk}{rk} % Rang (einer Matrix)
\newcommand*{\E}[1]{\mathrm{E}#1} % total space of a bundle
\newcommand*{\B}[1]{\mathrm{B}#1} % base space of a bundle
\newcommand*{\BG}{\mathrm{BG}} % classifying space of principal G-bundles
\newcommand*{\EG}{\E{\mathrm{G}}} % total space of the universal bundle
\newcommand*{\GL}[1]{\mathrm{GL}_{#1}(\R)} % general linear group over \R
\newcommand*{\Orth}{O} % orthogonal matrices
\newcommand*{\BO}{\mathrm{BO}} % classifying space of O-bundles
\newcommand*{\pt}{*} % single point space
\newcommand*{\zeroset}{\{0\}} % set containing only 0
\newcommand*{\Boundary}[1]{\partial{#1}} % boundary of a space

% category stuff
\newcommand*{\Top}{\mathrm{Top}} % category of topological spaces
\newcommand*{\Vect}{\mathrm{Vect}} % category of vector bundles
\newcommand*{\Hom}[1]{\mathrm{Hom}_{#1}} % #1-module homomorphisms

% (co)homology classes/maps
\renewcommand*{\H}{H} % co-/homology (overrides 'double accute')
\newcommand*{\relH}{\widetilde\H} % relative co-/homology
\newcommand*{\capped}[2]{\left\langle #1,\, #2 \right\rangle} % cap prod
\newcommand*{\thomiso}{t} % Thom isomorphism
\newcommand*{\fundcl}[1]{[#1]} % fundamental class of a manifold

\newcommand*{\cl}{\mathrm{cl}} % general name of a characteristic class
\newcommand*{\Cl}{\mathrm{Cl}} % general name of a characteristic class
\newcommand*{\charcl}[2]{{#1}(#2)} % Notation for char. cls of a bdl
\newcommand*{\dual}[1]{\overline{#1}} % dual characteristic class

\newcommand*{\U}{u} % Thom class without argument
\renewcommand*{\u}[1]{\charcl{\U}{#1}} % Thom class (overwrites 'cup over letter')

\newcommand*{\Ws}{w} % Stiefel-Whitney class symbol
\newcommand*{\dualWs}{\dual{\Ws}} % dual Stiefel-Whitney class symbol
\newcommand*{\ws}[1]{\Ws_{#1}} % {#1}th Stiefel-Whitney class (in H(BO))
\newcommand*{\dualws}[1]{\dualWs_{#1}} % {#1}th dual Stiefel-Whitney class (in H(BO))
\newcommand*{\w}[2]{\charcl{\ws{#1}}{#2}} % {#1}th Stiefel-Whitney class of bdl
\newcommand*{\dualw}[2]{\charcl{\dualws{#1}}{#2}} % {#1}th dual Stiefel-Whitney cls of bdl
\newcommand*{\W}[1]{\charcl{\Ws}{#1}} % total Stiefel-Whitney class of bdl
\newcommand*{\dualW}[1]{\charcl{\dualWs}{#1}} % total dual Stiefel-Whitney class of bdl

\newcommand*{\Vu}{v} % Wu class symbol
\newcommand*{\dualVu}{\dual{\Vu}} % dual Wu class symbol
\newcommand*{\vu}[1]{\Vu_{#1}} % {#1}th Wu class symbol
\newcommand*{\dualvu}[1]{\dualVu_{#1}} % {#1}th dual Wu class symbol
\renewcommand*{\v}[2]{\charcl{\vu{#1}}{#2}} % {#1}th Wu class of mfd/bdl (overwrites check over letter)
\newcommand*{\dualv}[2]{\charcl{\dualvu{#1}}{#2}} % {#1}th dual Wu class of mfd/bdl
\newcommand*{\V}[1]{\charcl{\Vu}{#1}} % total Wu class of a mfd/bdl
\newcommand*{\dualV}[1]{\charcl{\dualVu}{#1}} % dual total Wu class of mfd/bdl

% Steenrod algebra stuff
\newcommand*{\SQ}{\mathrm{Sq}} % total Steenrod Square
\newcommand*{\Sq}[1]{\SQ^{#1}}  % Steenrod Squares
\newcommand*{\A}{\mathcal{A}_2} % Steenrod algebra
\newcommand*{\Sqcup}{\circ} % multiplication of the Steenrod algebra
\newcommand*{\antipode}{\chi} % antipode of the Steenrod algebra
\renewcommand*{\l}{l} % length of a sequence (overrides l with stroke)
\renewcommand*{\d}{d} % degree of a sequence (overrides dot under letter)
\newcommand*{\e}{e} % excess of a sequence

% bundles
\newcommand*{\trivbdl}{\epsilon} % trivial rank ? vector bundle
\newcommand*{\T}[1]{\mathrm{T}#1} % tangent bundle
\newcommand*{\N}[1]{\nu_{#1}} % (stable) normal bundle
\newcommand*{\Discbdl}[1]{\mathrm{D}#1} % disc bundle of a vector bundle
\newcommand*{\Spherebdl}[1]{\mathrm{S}#1} % disc bundle of a vector bundle
\newcommand*{\Thomspace}[1]{\mathrm{T}{#1}} % Thom space of a vector bundle
\newcommand*{\collapse}{c} % collapse map from sphere to thom space of normal bundle
\newcommand*{\zerosec}[1]{0_{#1}} % zero section of a vector bundle
\newcommand*{\minuszerosec}[1]{{#1}^0} % total space minus zero section
\newcommand*{\minuszero}[1]{{#1}\setminus\{0\}} % space without 0
% pair (E, E^0) for total space E (homology is reduced thom space hom.)
\newcommand*{\thomspacepair}[1]{#1, \minuszerosec{#1}}
% pair (E_b, E_b\{0}) for fiber E_b (homology is mostly that of a sphere)
\newcommand*{\spherepair}[1]{#1, \minuszero{#1}}
\newcommand*{\rkk}{k} % arbitrary rank of a vector bundle (overwrites ogonek)
\newcommand*{\rkl}{l} % arbitrary other rank of a vector bundle (overwrites stroke bar)

% cobordism stuff
\renewcommand*{\c}{\eta} % cobordism ring symbol (overwrites cedilla)
\newcommand*{\disjointsum}{\amalg} % disjoint sum
\newcommand*{\leftgroupaction}{\curvearrowright} % group action symbol
\newcommand*{\Twistedprod}[2]{D_{#1}({#2})} % twisted product of #2 by the #1-sphere
\newcommand*{\Twistedprodcovspace}[2]{(#2\times #2)\times \Sphere{#1}} % twisted product of #2 by the #1-sphere
\newcommand*{\G}[1]{G^{#1}} % #1th constructed generator of the cobordism ring

% TODO: necessary checks
% fulfil instead of fulfill


%%% Local Variables:
%%% mode: latex
%%% TeX-master: "thesis"
%%% End:
