%%%%%%%%%%%%%%%%%%%%%%%%%%%%%%%%%% 
% Master Thesis in Mathematics
% "Immersions and Stiefel-Whitney classes of Manifolds"
% -- Chapter 4: Brown's Theorem --
% 
% Author: Gesina Schwalbe
% Supervisor: Georgios Raptis
% University of Regensburg 2018
%%%%%%%%%%%%%%%%%%%%%%%%%%%%%%%%%% 

\chapter{Brown's Theorem}

\section{The Cobordism Ring Structure}

\begin{LemDef}[Cobordism Ring] % TODO
\end{LemDef}

\begin{Thm} % TODO: cobordism ring thm
  \optcite[Thm.~1.23]{immersionconj}
  \optcite[Theorem~IV.9]{thom}
  \begin{gather*}
    \c_* \cong \Zmod2[\sigma_i| i\neq 2^r-1]
  \end{gather*}
\end{Thm}


\section{Indecomposable Elements of the Cobordism Ring}

\subsection{Special Properties of Symmetric Polynomials}
In this section a special kind of polynomials will be examined that
obey a very practical multiplication rule similar to \ref{tag:cartan}
when evaluated on Stiefel-Whitney classes. This will enable to express
certain Stiefel-Whitney numbers of product manifolds with respect to
ones of their factors.

From this the subsequent subsection will deduce a simple rule on how to
detect product manifolds and indecomposable elements of the cobordism
ring.

Beforehands, mind the following notation needed for denoting
symmetrisation of polynomials.
\begin{Def}
  Let $k,l\in\Nat$ be integers.
  \begin{itemize}
  \item
    A partition $\Part=(i_1,\dotsc,i_l)$ of $k$ is an unordered sequence
    of integers such that $k=\sum_{r=0}^{l}i_r$.
    Two partitions only differing by zeros are considered equal.
    In other words, a partition is an equivalence class of sequences in
    $\bigoplus_\infty\Nat$ under the relation $\Part\sim\sigma(\Part)$
    for any permutation $\sigma$.
  \item
    The notation $I^l$ for a sequence of integers will mean a sequence
    of length $l$. Write $I^l\in\Part$ for a sequence of length
    $l$ in the equivalence class of the partition $\Part$. 
  \item
    Denote by $\PartitionsOf{k}$ the set of partitions of $k$.
  \item
    Write short $\Emptypart$ for the partition of $0$.
  \item The concatenation of sequences and analoguesly partitions will
    be denoted by
    \begin{align*}
      \Nat^{l_1} \times \Nat^{l_2}
      &\xrightarrow{-\concat-}
        \Nat^{l_1+l_2}
      \\
      \PartitionsOf{k_1} \times \PartitionsOf{k_2}
      &\xrightarrow{-\concat-}
        \PartitionsOf{k_1+k_2}
      \\
      (i_1,\dotsc,i_r), (j_1,\dotsc,j_s)
      &\longmapsto
        (i_1,\dotsc,i_r,j_1,\dotsc,j_s)
    \end{align*}
  \end{itemize}
\end{Def}

Also as preparation, recall some properties of symmetric polynomials.
\begin{LemDef}
  Let $n\in\Nat$ and $\Zmod2[t_1,\dotsc,t_n]$ be the polynomial ring
  in $n$ variables over the fields $\Zmod2$, each $t_i$ of degree 1.
  \begin{enumerate}
  \item Let
    $\Symm{n}_*
    \coloneqq \Zmod2[t_1,\dotsc,t_n]^{\Permutations{n}}
    \subset \Zmod2[t_1,\dotsc,t_n]$
    be the graded subring of symmetric polynomials in $n$ variables.
  \item It has a basis indexed by partitions $\Part$ consisting of the
    elements
    \begin{gather*}
      \symm{n} t^\Part \coloneqq \sum_{I^n\in\Part} t^I \in \Symm{n}_k
    \end{gather*}
    where $t^{(i_1,\dotsc,i_n)}\coloneqq t_1^{i_1}\dotsm t_n^{i_n}$.
    \begin{proof}
      \optcite[footnote~2, p.~154]{thom} % TODO: formulate proof?
    \end{proof}
  \item $\Symm{n}_*$ is generated by the
    elementary symmetric polynomials in $n$ variables
    \begin{gather*}
      \el i n\coloneqq \symm{n} t^{\Part_i}
      \in \Symm{n}_i
      \qquad \text{for }
      \Part_i = (1,\dotsc,1)
      \in \PartitionsOf i
    \end{gather*}
    for $1\leq i\leq n$.
    \Forexample
    $\el 1 n=\sum_{r=1}^{n}t_r$,
    $\el 2 n=\sum_{1\leq r<s\leq n} t_r t_s$.
  \item The elementary symmetric polynomials in $n$ variables are
    algebraically independent and fulfill
    \begin{gather}\label{eq:sumelemsymmpoly}
      1 + \sum_{i=1}^{n} \el i n
      = \prod_{r=1}^{n}(1+t_n)
    \end{gather}
  \end{enumerate}
\end{LemDef}

Now the desired polynomials can be defined.
\begin{Def}
  \optcite[p.~90]{milnorlectures}
  Let $k\in\Nat$ and $\Part\in\PartitionsOf k$.
  Define the homogenous polynomial
  $\s{\Part}\in\Zmod2[x_1,\dotsc,x_k]$ of degree $k$ by
  \begin{gather*}
    \s{\Part}(\el 1 n,\dotsc, \el k n) = \symm{n} t^{\Part} \in \Symm{n}_k
  \end{gather*}
  for some $n\geq k$. Mind, that this is well-defined as the
  elementary symmetric polynomials are algebraically independent, and
  the definition does not depend on $n$ as long as $k\leq n$.

  For a any graded ring $A_*=\bigoplus_{i\geq 0} A_i$ write elements as
  $a=\sum_i a_i \coloneqq (a_0,a_1,\dotsc)$.
  For such an element define the evaluation of $\s{\Part}$ as
  \begin{gather*}
    \s{\Part}(a) \coloneqq \s{\Part}(a_1,\dotsc,a_k)
    \;,
  \end{gather*}
  \idest skip $a_0$ and all higher $a_i$.
\end{Def}


These polynomials that translate the basis $\symm{n}t^{\Part}$ of
$\Symm{n}_k$ into expressions of the generators $\el i n$
fulfill as promised the following interesting property concerning
multiplication of elements.
\begin{Lem}
  \optcite[Theorem~33, p.~91f]{milnorlectures}
  Let $k\in\Nat$, and let $A_*=\bigoplus_{i\geq 0} A_i$ be a graded ring.
  For $a,b\in A_*$ with $a_0=1=b_0$, and any partition
  $\Part\in\PartitionsOf k$ holds
  \begin{gather*}
    \s{\Part}(a\cdot b)
    = \sum_{\Part_1\concat\Part_2=\Part}
    \s{\Part_1}(a) \cdot \s{\Part_2}(b)
    \;.
  \end{gather*}
  \begin{proof}
    As for the definition of $\s{\Part}$ it suffices to check equality
    on algebraically independent elements. So, choose:
    \begin{itemize}
    \item The result will be independent of $n$, as long as
      $n$ is chosen large enough, thus choose $n=2k$.
      Also note: This is the least choice for $n$ for which a pair
      of sequences with $J_1^k\concat J_2^k\in\Part$ may contain a
      sequence of all zeros.
    \item For simplicity of notation write
      $x_i=t_i$ and $y_i=t_{2i}$ for $1\leq i\leq k$ to split the
      variables $(t_1,\dotsc,t_{2k})$ into two equal parts.
    \item In order to directly work with the definition of $\s{\Part}$ it is
      convenient to choose as algebraically independent elements
      \begin{align*}
        a &\coloneqq
            \prod_{\mathclap{r=1}}^k(1+t_r)
            \cequalsby{\eqref{eq:sumelemsymmpoly}}
            1 + \sum_{i=1}^k \el i k(x_1,\dotsc,x_k)
            \;\text{, and}\\
        b &\coloneqq
            \prod_{\mathclap{r=k+1}}^n(1+t_r)
            \cequalsby{\eqref{eq:sumelemsymmpoly}}
            1 + \sum_{i=1}^k \el i k(y_1,\dotsc,y_k)
            \;\text{, then}\\
        a\cdot b
          &=
            \prod_{r=1}^n (1+t_r)
            \cequalsby{\eqref{eq:sumelemsymmpoly}}
            1 + \sum_{i=1}^n \el i n (t_1,\dotsc,t_n)
            \;.
      \end{align*}
      Consequently, the sets 
      $(a_1,\dotsc,a_k)$, $(b_1,\dotsc,b_k)$, and
      $(a\cdot b)_{1\leq i\leq n}$ are algebraically independent, as
      was needed.
    \end{itemize}
    Now calculate
    \begin{align*}
      \s{\Part}(a\cdot b)
      &\coloneqq
        \symm{n} t^{\Part}
      \\ &\equalsby{Def.}
           \sum_{I^n\in\Part}
           t_1^{i_1}\dotsm t_n^{i_n}
      \\ &=
           \sum_{I^n\in\Part}
           x^{(i_1,\dotsc,i_k)}\cdot y^{(i_{k+1},\dotsc,i_n)}
      \\ &=
           \sum_{J_1^k\concat J_2^k\in\Part}
           x^{J_1} \cdot y^{J_2}
      \\ &\equalsby{Group by equiv.}
           \sum_{\Part_1\concat \Part_2=\Part}
           \sum_{\substack{J_1^k\in\Part_1\\ J_2^k\in\Part_2}}
      x^{J_1} \cdot y^{J_2}
      \\ &=
           \sum_{\Part_1\concat \Part_2=\Part}
           \left(\sum_{J_1^k\in\Part_1} x^{J_1}\right)
           \cdot
           \left(\sum_{J_2^k\in\Part_2} y^{J_2}\right)
      \\ &\equalsby{Def.}
           \sum_{\Part_1\concat \Part_2=\Part}
           \left(\symm{k} x^{\Part_1}\right)
           \cdot
           \left(\symm{k} y^{\Part_2}\right)
      \\ &\equalsby{Def.}
           \sum_{\Part_1\concat \Part_2=\Part}
           \s{\Part_1}\left( \el 1 k(x_1,\dotsc,x_k),\dotsc \right)
           \cdot
           \s{\Part_2}\left( \el 1 k(y_1,\dotsc,y_k),\dotsc \right)
      \\ &\equalsby{Def.}
           \sum_{\Part_1\concat \Part_2=\Part}
           \s{\Part_1}(a) \cdot \s{\Part_2}(b)
           \qedhere
    \end{align*}
  \end{proof}
\end{Lem}



\subsection{Stiefel-Whitney Numbers of Indecomposable Manifolds}
\begin{Thm}
  \optcite[Thm.~1.26]{immersionconj};
  \optcite[p.~195ff]{thom};
  for non-dyadic decomposition: \optcite[p.~156]{thom};
  for summation notation: \optcite[p.154, (4)f]{thom};
  \\
  A closed $n$-manifold $M$ represents an indecomposable element of
  $\c_*$, if and only if
  \begin{gather*}
    \langle s_m(M), \fundcl M \rangle
  \end{gather*}
\end{Thm}


%%% Local Variables:
%%% mode: latex
%%% TeX-master: "thesis"
%%% End:
