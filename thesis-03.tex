%%%%%%%%%%%%%%%%%%%%%%%%%%%%%%%%%% 
% Master Thesis in Mathematics
% "Immersions and Stiefel-Whitney classes of Manifolds"
% -- Chapter 3: The Immersion Conjecture up to Cobordism --
% 
% Author: Gesina Schwalbe
% Supervisor: Georgios Raptis
% University of Regensburg 2018
%%%%%%%%%%%%%%%%%%%%%%%%%%%%%%%%%% 

\chapter{The Immersion Conjecture up to Cobordism}\label{chap:brown}
% Review the necessary results about the structure of the unoriented
% cobordism ring and explain R. L. Brown’s proof of the immersion
% conjecture up to cobordism. [immersionconj], [brown]
The overall goal of this chapter is to prove the following theorem of
R.~L.~Brown following his paper \cite{brown},
which essentially states that the immersion conjecture is true up to
the cobordism relation.
\begin{Thm}[Brown]\label{thm:brown}
  Every closed $n$-manifold is cobordant to an $n$-manifold that immerses
  into $\R^{2n-\alpha(n)}$.
\end{Thm}

As one easily sees that this property is stable under
the ring operations (Lemma~\ref{lem:brownstableunderringops}),
the main idea for the proof is to find manifolds
fulfilling the conjecture whose cobordism classes form a generating set
of the cobordism ring.
As the latter has the form of a polynomial algebra
$\Zmod2[\sigma_i|i\neq 2^r-1]$, a set of elements
$([\G i]|i\neq2^r-1)$ will already be a generating set if all of the
$[\G i]$ are indecomposable.
Thus, the candidate generating elements that will be constructed
in \autoref{sec:proofbrown} need to be tested for:
\begin{enumerate}
\item indecomposability, which is the more lengthy part as it
  requires the preliminary results from
  \autoref{sec:indecomposabilitycriterion} respectively
  \autoref{sec:twistedprod:indecompcriterion}, and
\item the property to fulfill the immersion conjecture.
\end{enumerate}
The odd generators are going to be constructed using twisted
products, which are introduced in \autoref{sec:twistedproddef}.

This chapter is structured into some preliminary work on finding a
criterion to easily detect indecomposable elements of the cobordism
ring in \autoref{sec:indecomposableelements},
the twisted product construction and its properties---especially
concerning indecomposability---in
\autoref{sec:twistedprod}, and the final proof with the construction
of the generating set in \autoref{sec:proofbrown}, where all
preliminary work is merged.

For clarity of presentation, both a couple of results from Thom's
paper \cite{thom} within the review in
\autoref{sec:cobordismringstructure} will merely be referenced without
proof.
The reader is assumed to be familiar with symmetric polynomials.

Call a manifold \emph{indecomposable} if it
represents an indecomposable element of the cobordism ring.

\section{Detecting Indecomposable Elements of the Cobordism Ring}
\label{sec:indecomposableelements}
In order to find representatives for a set of algebraically
independent generators of the polynomial cobordism ring, one needs a
way to detect indecomposable elements. Indecomposable in this 
context means to not be expressible as a sum of products of lower
degree elements.

This section will follow an approach of Thom 
\cite[Chapters~IV.5 and~IV.6]{thom}, in which a certain characteristic
class serves as indicator. This characteristic class is constructed
out of Stiefel-Whitney classes using certain functions on symmetric
polynomials (see \autoref{sec:functions}), and becomes particularly
useful on tangent bundles of manifolds (see
\autoref{sec:swnumsofproductmfds}). The final indication lemma is then
stated and proved in \autoref{sec:indecomposabilitycriterion}.

\subsection{Special Properties of Symmetric Polynomials}\label{sec:functions}
This subsection examines a special kind of polynomials that
obey a product rule similar to \ref{tag:cartan} whenever evaluated on
elements of the form of a total Stiefel-Whitney class.

As a consequence, one can express certain combinations of
Stiefel-Whitney numbers of product manifolds in terms of ones of their
factors, which will be investigated in detail in
\autoref{sec:swnumsofproductmfds}.
From this, \autoref{sec:indecomposabilitycriterion} will deduce a
simple criterion for a manifold to be cobordant to a product of
manifolds.

Beforehand, mind the following notation of partitions needed for
symmetrising polynomials.
\begin{Def}\label{def:partition}
  Let $k,l\in\Nat$ be integers.
  \begin{itemize}
  \item
    A \emph{partition} $\Part=(i_1,\dotsc,i_l)$ of $k$ is an unordered sequence
    of integers such that $k=\sum_{r=0}^{l}i_r$.
    Two partitions only differing by zeros are considered equal.
    In other words, a partition is an equivalence class of sequences in
    $\bigoplus_\infty\Nat$ under the relation $\Part\sim\sigma(\Part)$
    for any permutation $\sigma$.
  \item
    The notation $I^l$ for a sequence of integers will mean a sequence
    of length $l$. Write $I^l\in\Part$ for a sequence of length
    $l$ in the equivalence class of the partition $\Part$. 
  \item
    Denote by $\PartitionsOf{k}$ the set of partitions of $k$.
  \item
    Write $\Emptypart$ for the unique partition of $0$.
  \item
    Call a sequence or partition \emph{non-dyadic}, if
    none of its entries is of the form $2^m-1$.
  \item The concatenation of sequences, and analogously partitions, will
    be denoted by
    \begin{align*}
      \Nat^{l_1} \times \Nat^{l_2}
      &\xrightarrow{-\concat-}
        \Nat^{l_1+l_2}
      \\
      \PartitionsOf{k_1} \times \PartitionsOf{k_2}
      &\xrightarrow{-\concat-}
        \PartitionsOf{k_1+k_2}
      \\
      (i_1,\dotsc,i_r), (j_1,\dotsc,j_s)
      &\longmapsto
        (i_1,\dotsc,i_r,j_1,\dotsc,j_s)
    \end{align*}
  \end{itemize}
\end{Def}

Also as preparation, recall some properties of symmetric polynomials
and agree on a shorthand for symmetrised monomials.
\begin{LemDef}
  Let $n\in\Nat$ and $\Zmod2[t_1,\dotsc,t_n]$ be the graded polynomial
  algebra in $n$ variables over the fields $\Zmod2$, each $t_i$ of
  degree one.
  \begin{enumerate}
  \item Let
    $\Symm{n}_*
    \coloneqq \Zmod2[t_1,\dotsc,t_n]^{\Permutations{n}}
    \subset \Zmod2[t_1,\dotsc,t_n]$
    be the graded subalgebra of symmetric polynomials in $n$ variables.
  \item $\Symm{n}_*$ has a basis indexed by partitions $\Part$
    consisting of symmetrised monomials, \idest elements of the form
    \begin{gather*}
      \symm{n} t^\Part \coloneqq \sum_{I^n\in\Part} t^I \in \Symm{n}_k
    \end{gather*}
    where $t^{(i_1,\dotsc,i_n)}\coloneqq t_1^{i_1}\dotsm t_n^{i_n}$.
    \begin{proof}
      See also \cite[footnote~2, p.~154]{thom}.
      Linear independence is clear as monomials $t^I$ and $t^{I'}$ are
      linearly independent if $I\neq I'$, and sequences belonging to
      different partitions must be unequal.
      For the generating property note that every symmetric polynomial
      $p$ is of the form $\sum_{I^n\in A}t^I$. This
      however can be written as a sum of symmetrised monomials by
      descending induction:
      For a symmetric polynomial
      \begin{gather*}
        p
        = \sum_{\Part\in B}\symm{n} t^{\Part}  + \sum_{I^n\in A}t^I
        \;,
      \end{gather*}
      and $I'\in A$, one must have $s(I')\in A$ for
      $s\in\Permutations{n}$, since
      $p-\sum_{\Part\in B}\symm{n} t^{\Part} = \sum_{I^n\in A}t^I$
      is still symmetric.
      Hence one can write
      \begin{align*}
        p
        &= \sum_{\mathclap{\Part\in B}}\symm{n} t^{\Part}
          + \sum_{s\in\Permutations{n}}t^{s(I')}
          + \sum_{\mathrlap{
          I^n\in A\setminus\{s(I')|s\in\Permutations{n}\}
          }} t^I
        \\
        &= \sum_{\mathclap{\Part\in B\cup\{[I']\}}}\symm{n} t^{\Part}
          + \sum_{\mathrlap{
            I^n\in A\setminus\{s(I')|s\in\Permutations{n}\}
          }} t^I
        \;,
      \end{align*}
      and thus inductively decrease $\# A$ to zero, expressing $p$
      as a sum of symmetrised polynomials.
    \end{proof}
  \item $\Symm{n}_*$ is generated as algebra by the algebraically
    independent, elementary symmetric polynomials in $n$ variables
    \begin{gather*}
      \el i n\coloneqq \symm{n} t^{\Part_i}
      \in \Symm{n}_i
      \qquad \text{for }
      \Part_i = (1,\dotsc,1)
      \in \PartitionsOf i
    \end{gather*}
    for $1\leq i\leq n$.
    \Forexample
    $\el 1 n=\sum_{r=1}^{n}t_r$,
    $\el 2 n=\sum_{1\leq r<s\leq n} t_r t_s$.
    \begin{proof}
      This is the well-known fundamental theorem on symmetric
      polynomials, see \forexample
      \cite[Chap.~4.4, Satz~1]{bosch2013algebra}. 
    \end{proof}
  \item As a simple calculation shows that the elementary symmetric
    polynomials in $n$ variables fulfill
    \begin{gather}\label{eq:sumelemsymmpoly}
      1 + \sum_{i=1}^{n} \el i n
      = \prod_{r=1}^{n}(1+t_n)
    \end{gather}
  \end{enumerate}
\end{LemDef}

Now the desired polynomials can be defined.
The following definition is according to \cite[p.~90]{milnorlectures}.
\begin{Def}
  Let $k\in\Nat$, $\Part\in\PartitionsOf k$, and let
  $\Zmod2[\alpha_1,\dotsc,\alpha_k]$ be the polynomial ring in $k$
  variables where $\alpha_i$ has degree $i$.
  \begin{enumerate}
  \item
    Define the homogeneous polynomial
    $\s{\Part}\in\Zmod2[\alpha_1,\dotsc,\alpha_k]$ of degree $k$ by
    \begin{gather*}
      \s{\Part}(\el 1 n,\dotsc, \el k n) = \symm{n} t^{\Part} \in \Symm{n}_k
    \end{gather*}
    for some $n\geq k$. This means, $\s{\Part}$ gives the representation
    of $\symm{n}t^\Part$ in terms of the generating set $(\el i n)_i$.
    Mind, that this
    \begin{enumerate}
    \item is well-defined, as the elementary symmetric
      polynomials are algebraically independent generators, and
    \item does not depend on $n$ as long as $k\leq n$.
    \end{enumerate}
  \item 
    Further, for a graded, commutative ring $A_*=\bigoplus_{i\geq 0}
    A_i$ write elements as
    \begin{gather*}
      a=\sum_i a_i \coloneqq (a_0,a_1,a_2,\dotsc)
      \;,
    \end{gather*}
    and
  \item 
    define the evaluation of $\s{\Part}$ on such an element $a$ as
    \begin{gather*}
      \s{\Part}(a) \coloneqq \s{\Part}(a_1,\dotsc,a_k)
      \in A_k
      \;,
    \end{gather*}
    \idest skip $a_0$ and all higher $a_i$.
  \end{enumerate}
\end{Def}

\begin{Ex}
  The first such polynomials over $\Zmod2$ are
  (see \cite[p.~90]{milnorlectures}):
  \begin{align*}
    k&=0:
    &\s{\Emptypart} &= 1
    \\ k&=1:
    &\s{(1)} &= \alpha_1
    \\ k&=2:
    &\s{(2)} &= \alpha_1^2
        &\s{(1,1)} &= \alpha_2
    \\ k&=3:
    &\s{(3)} &= \alpha_1^3 + \alpha_1\alpha_2 + \alpha_3
        &\s{(1,2)} &= \alpha_1\alpha_2 + \alpha_3
             &\s{(1,1,1)} &= \alpha_3
  \end{align*}
\end{Ex}

\begin{Rem}\label{rem:spolyevaluation}
  If $k\leq r\leq n\in\Nat$ and one is given an element $a=1+\sum_{i\geq 1} a_i$
  in a graded commutative algebra $A_*$, which can be written as
  \begin{gather*}
    a = 1 + \sum_{i=1}^{r}\el{i}{n}(f_1,\dotsc,f_n)
  \end{gather*}
  for some degree one elements $f_i\in A_*$,
  then for any partition $\Part\in\PartitionsOf{k}$ one gets
  \begin{align*}
    \s{\Part}(a)
    &= \s{\Part}\left(
      \el{1}{n}(f_1,\dotsc,f_n),\dotsc, \el{k}{n}(f_1,\dotsc,f_n)
      \right) \\
    &= \left(
      \s{\Part}
      \left( \el{1}{n}, \dotsc, \el{k}{n} \right)
      \right)
      (f_1,\dotsc,f_n) \\
    &= \symm{n} f^\Part \in A_k
      \;.
  \end{align*}
  This trick will be very useful when dealing with Stiefel-Whitney
  classes.
\end{Rem}

As promised, these translation-polynomials between the basis
$\symm{n}t^{\Part}$ of $\Symm{n}_k$ and the generators $\el i n$
fulfill the following interesting property concerning
multiplication of elements that have the form of a total
Stiefel-Whitney class.
\begin{Lem}\label{lem:productrule:general}
  Let $k\in\Nat$, and let $A_*=\bigoplus_{i\geq 0} A_i$ be a graded
  commutative ring.
  For $a,b\in A_*$ with $a_0=1=b_0$, and any partition
  $\Part\in\PartitionsOf k$ holds
  \begin{gather*}
    \s{\Part}(a\cdot b)
    = \sum_{\Part_1\concat\Part_2=\Part}
    \s{\Part_1}(a) \cdot \s{\Part_2}(b)
    \;.
  \end{gather*}
  \begin{proof}
    A proof can also be found in
    \cite[Theorem~33, p.~91f]{milnorlectures}.
    We will prove the lemma for a special which generalizes to
    the lemma.
    Beforehand, recall that for $r\geq 1$ and any
    element $f=\sum_{i\geq 0}f_i$ in a commutative graded ring $A_*$
    holds
    \begin{gather}\label{eq:translationpolycut}
      \s{\Part}\left( \sum_{i\geq 0}f_i \right)
      = \s{\Part}\left( \sum_{i=0}^{k+r}f_i \right)
      \;.
    \end{gather}
    Now let $A_*$ be the subring of $\Z[t_1,\dotsc,t_{2k}]$ that is
    generated by the algebraically independent elements
    \begin{align*}
      a_i&\coloneqq \el i k(x_1,\dotsc,x_k)
      &\text{and}&
      &b_i&\coloneqq \el i k(y_1,\dotsc,y_k)
    \end{align*}
    for $1\leq i\leq k$ and with $x_i=t_i$ and $y_i=t_{2i}$.
    Then, for any other graded commutative
    ring $\bar A_*$, and elements
    $\bar a=1+\sum_{i\geq 1}\bar a_i$,
    $\bar b=1+\sum_{i\geq 1}\bar b_i\in\bar A_*$, the ring
    homomorphism defined by
    \begin{align*}
      \phi\colon A_*&\longto \bar A_*,
      &a_i&\mapsto \bar a_i
      &b_i&\mapsto \bar b_i
    \end{align*}
    is both well-defined, since $a_i,b_i$ are algebraically independent
    generators, and surjective onto the subring of $\bar A_*$
    generated by $\bar a_i,\bar b_i$ with $1\leq i\leq k$.
    Hence, if we assume the statement is proven for
    $a=1+\sum_{i=1}^k a_i$ and $b=1+\sum_{i=1}^k b_i$ in $A_*$, one
    has
    \begin{align*}
      \s{\Part}(\bar a\cdot \bar b)
      &\cequalsby{\eqref{eq:translationpolycut}}
      \s{\Part}\left(
        \left( 1+\sum_{i=1}^{k}\bar a_i \right) \cdot
        \left( 1+\sum_{i=1}^{k}\bar b_i \right)
      \right)
      = \s{\Part}(\phi(a)\cdot \phi(b)) \\
      &= \phi(\s{\Part}(a\cdot b)) \\
      &\equalsby{Assumption} \phi\left(
      \sum_{\mathrlap{\Part_1\concat\Part_2=\Part}}
      \s{\Part_1}(a) \cdot \s{\Part_2}(b)
      \right) \\
      &= \sum_{\mathrlap{\Part_1\concat\Part_2=\Part}}
        \s{\Part_1}(\phi(a)) \cdot \s{\Part_2}(\phi(b)) 
      =
        \sum_{\mathclap{\Part_1\concat\Part_2=\Part}}
        \s{\Part_1}\left( \sum_{i=1}^{k}\bar a_i \right)
        \cdot \s{\Part_2}\left( \sum_{i=1}^{k}\bar b_i \right) \\
      &\equalsby{\eqref{eq:translationpolycut}}
        \sum_{\mathrlap{\Part_1\concat\Part_2=\Part}}
        \s{\Part_1}(\bar a) \cdot \s{\Part_2}(\bar b)
      \;,
    \end{align*}
    which is the statement of the lemma for $\bar a, \bar b$.
    Therefore, it suffices to show the lemma for $a$ and $b$ as
    above.
    
    In order to do so, observe that
    \begin{align*}
      a &\coloneqq
          \prod_{\mathclap{r=1}}^k(1+t_r)
          \cequalsby{\eqref{eq:sumelemsymmpoly}}
          1 + \sum_{i=1}^k \el i k(x_1,\dotsc,x_k)
          \;\text{, and}\\
      b &\coloneqq
          \prod_{\mathclap{r=k+1}}^n(1+t_r)
          \cequalsby{\eqref{eq:sumelemsymmpoly}}
          1 + \sum_{i=1}^k \el i k(y_1,\dotsc,y_k)
          \;\text{, then}\\
      a\cdot b
        &=
          \prod_{r=1}^n (1+t_r)
          \cequalsby{\eqref{eq:sumelemsymmpoly}}
          1 + \sum_{i=1}^{2k} \el i n (t_1,\dotsc,t_{2k})
          \;,
    \end{align*}
    which nicely fits into the defining relation for $\s{\Part}$.
    Now calculate
    \begin{align*}
      \s{\Part}(a\cdot b)
      &\coloneqq
        \symm{n} t^{\Part}
      \\ &\equalsby{Def.}
           \sum_{I^n\in\Part}
           t_1^{i_1}\dotsm t_n^{i_n}
      \\ &=
           \sum_{I^n\in\Part}
           x^{(i_1,\dotsc,i_k)}\cdot y^{(i_{k+1},\dotsc,i_n)}
      \\ &=
           \sum_{J_1^k\concat J_2^k\in\Part}
           x^{J_1} \cdot y^{J_2}
      \\ &\equalsby{Group by equiv.}
           \sum_{\Part_1\concat \Part_2=\Part}
           \sum_{\substack{J_1^k\in\Part_1\\ J_2^k\in\Part_2}}
      x^{J_1} \cdot y^{J_2}
      \\ &=
           \sum_{\Part_1\concat \Part_2=\Part}
           \left(\sum_{J_1^k\in\Part_1} x^{J_1}\right)
           \cdot
           \left(\sum_{J_2^k\in\Part_2} y^{J_2}\right)
      \\ &\equalsby{Def.}
           \sum_{\Part_1\concat \Part_2=\Part}
           \left(\symm{k} x^{\Part_1}\right)
           \cdot
           \left(\symm{k} y^{\Part_2}\right)
      \\ &\equalsby{Def.}
           \sum_{\Part_1\concat \Part_2=\Part}
           \s{\Part_1}\left( \el 1 k(x_1,\dotsc,x_k),\dotsc \right)
           \cdot
           \s{\Part_2}\left( \el 1 k(y_1,\dotsc,y_k),\dotsc \right)
      \\ &\equalsby{Def.}
           \sum_{\Part_1\concat \Part_2=\Part}
           \s{\Part_1}(a) \cdot \s{\Part_2}(b)
           \qedhere
    \end{align*}
  \end{proof}
\end{Lem}

\begin{Ex}
  The most important partition of an integer $k$ will be the trivial
  one $(k)\in\PartitionsOf k$. In this case
  Lemma~\ref{lem:productrule:general} says
  \begin{gather*}
    \s{(k)}(a\cdot b) = \s{(k)}(a) + \s{(k)}(b)
    \;.
  \end{gather*}
\end{Ex}


\subsection{Stiefel-Whitney Numbers of Product Manifolds}
\label{sec:swnumsofproductmfds}
In order to apply the special polynomials out of the preceding
section, as well as their product property from
Lemma~\ref{lem:productrule:general}, to the Stiefel-Whitney
numbers of (product) manifolds, first start with Stiefel-Whitney
classes.

So, let $M^n=M_1^{n_1}\times M_2^{n_2}$ all be closed manifolds of the
noted dimension throughout this section.

Recall that
\begin{enumerate}
\item the cohomology ring $\H^*(M)$ is a graded ring,
\item the total Stiefel-Whitney number of a manifold is of the form
  $\W{M_i} = 1 + \w 1 {M_i} + \dotsb + \w n {M_i}$, and that
\item by the Künneth isomorphism, we have
  \begin{align*}
    \H^*(M_1)\otimes \H^*(M_2)
    &\overset{\cong}\longto \H^*(M)
    \\
    c_1 \otimes c_2
    &\longmapsto c_1\times c_2
      \coloneqq \pb\proj_1 c_1 \cup \pb\proj_2 c_2
  \end{align*}
  and $\W{M} = \W{M_1} \times \W{M_2}$ by
  \ref{tag:swclassesmultiplicativity} of the Stiefel-Whitney classes.
\end{enumerate}
Thus, one can apply the multiplication
rule~\ref{lem:productrule:general} to $\W{M}$, which immediately
yields:
\begin{Cor}\label{cor:productrule:swcl}
  For $M=M_1\times M_2$ manifolds as above one gets for any partition
  $\Part\in\PartitionsOf n$:
  \begin{align*}
    \s{\Part}(\W{M})
    =
    \s{\Part}\left(\W{M_1}\times \W{M_2}\right)
    &\cequalsby{\ref{lem:productrule:general}}
      \sum_{\mathrlap{\Part_1\concat\Part_2=\Part}}
      \s{\Part_1}(\pb\proj_1\W{M_1}) \cdot \s{\Part_2}(\pb\proj_2\W{M_2})
    \\
    &=
      \sum_{\mathrlap{\Part_1\concat\Part_2=\Part}}
      \s{\Part_1}(\W{M_1}) \times \s{\Part_2}(\W{M_2})
    \\ &=
         \sum_{\mathrlap{\substack{
         \Part_1\concat\Part_2=\Part\\
    \Part_1\in\PartitionsOf{n_1}\\
    \Part_2\in\PartitionsOf{n_2}\\
    }}}
    \s{\Part_1}(\W{M_1}) \times \s{\Part_2}(\W{M_2})
  \end{align*}
  \begin{proof}
    For the last equality note that by definition of $\s{\Part}$
    for any partition $\Part$ of some $k\in\Nat$, 
    the element $\s{\Part}(\W{W})$ lies in $\H^k(M)$,
    hence must be zero if $k>\dim W$.
    Therefore, for any combination of partitions
    $\Part_1\in\PartitionsOf{k_1}$,
    $\Part_2\in\PartitionsOf{k_2}$
    where $k_1+k_2=n=n_1+n_2$ with $k_i\neq n_i$, the product
    $\s{\Part_1}(\W{M_1})\cdot\s{\Part_2}(\W{M_2})$ will have a zero
    factor and can be skipped.
  \end{proof}
\end{Cor}

In order to pass to Stiefel-Whitney numbers instead of classes, use
the following notation.
\begin{Def}
  Let $W$ be a closed manifold and $\Part\in\PartitionsOf{\dim W}$.
  Then write
  \begin{align*}
    \s{\Part}(W) &\coloneqq \s{\Part}(\W{W})\\
    \snum{\Part}{W} &\coloneqq \capped{\s{\Part}(W)}{\fundcl{W}}
                      \;.
  \end{align*}
\end{Def}

Now the product rule from Lemma~\ref{lem:productrule:general} translates to
\begin{Cor}\label{cor:swnumdecompositionmfds}
  For closed manifolds $M_1$, and $M_2$ with dimensions $n_1$ and
  $n_2$ one has
  \begin{align*}
    \snum{\Part}{M_1\times M_2}
    &= \sum_{\mathclap{\substack{
      \Part_1\concat\Part_2=\Part\\
    \Part_1\in\PartitionsOf{n_1}\\
    \Part_2\in\PartitionsOf{n_2}\\
    }}}
    \snum{\Part_1}{M_1} \cdot \snum{\Part_2}{M_2}
    \quad \in \quad \Zmod2
  \end{align*}
  and as a special case
  \begin{align}
    % \notag
    % \snum{(n_1,n_2)}{M_1\times M_2}
    % &= \snum{(n_1)}{M_1} \cdot \snum{(n_2)}{M_2}
    % \\
    \label{eq:productrule:swnum}
    \snum{(n_1+n_2)}{M_1\times M_2} &= 0
                                      \;.
  \end{align}
  In particular, if $\snum{(\dim W)}{W}\neq 0$ for a closed manifold
  $W$, then $W$ is no product manifold.
  \begin{proof}
    The statement immediately follows from the previous
    Corollary~\ref{cor:productrule:swcl} with the following
    two facts:
    \begin{enumerate}
    \item The generator $\fundcl{M_1\times
        M_2}\in\H^{n_1+n_2}(M_1\times M_2)$ corresponds to the generator
      $\fundcl{M_1}\otimes\fundcl{M_2}\in
      \H^{n_1}(M_1)\otimes\H^{n_2}(M_2)\cong \H^{n_1+n_2}(M_1\times M_2)$
      under the Künneth isomorphism.
    \item For cohomology classes $c_1\in\H^{n_1}(M_1)$, $c_2\in\H^{n_2}(M_2)$ holds
      \begin{gather*}
        \capped{c_1\otimes c_2}{\fundcl{M_1\times M_2}}
        = \capped{c_1\otimes c_2}{\fundcl{M_1}\otimes\fundcl{M_2}}
        = \capped{c_1}{\fundcl{M_1}} \cdot \capped{c_2}{\fundcl{M_2}}
        \in \Zmod2
      \end{gather*}
      by the universal property of the tensor product.
      \qedhere
    \end{enumerate}    
  \end{proof}
\end{Cor}

\autoref{eq:productrule:swnum} is the desired obstruction for
a manifold to be a product or---as will be explained in the next
subsections---to be cobordant to a product.
Finally, this statement will be an invaluable tool for detecting
manifolds that are not only not cobordant to a product manifold but
whose cobordism class is indecomposable.

\subsection{Review: The Cobordism Ring Structure}
\label{sec:cobordismringstructure}
Recall that two closed manifolds of the same dimension $n$ are
(unoriented) cobordant if their disjoint union is the boundary of an
$(n+1)$-dimensional manifold.
This is an equivalence relation amongst $n$-manifolds, and the
set of equivalence classes forms an Abelian group $\c_n$ of order two
with the disjoint sum as addition and the $n$-sphere as zero element.
The Cartesian product turns the graded $\Zmod2$-module
$\c_*\coloneqq \bigoplus_{n\geq 0}\c_n$ into an $\Zmod2$-algebra
called the (unoriented) cobordism ring.
Denote the cobordism equivalence class of a manifold $M$ by $[M]$.

Most remarkably, the cobordism relation is homotopy invariant, which
will become clear from the property described in
Theorem~\ref{thm:cobordantiffswnumscoincide}.
Further, the structure of this algebra is well-known to be as follows.
\begin{Thm}[Thom]\label{thm:cobordismringstructure}
  \begin{gather*}
    \c_*
    % \cong \pi_*(\MO)
    \cong \Zmod2[\sigma_i| i\neq 2^r-1]
  \end{gather*}
  as graded $\Zmod2$-algebras.
  \begin{proof}
    Compare \cite[Thm.~1.23]{immersionconj}, and
    \cite[Theorem~IV.9]{thom}.
    See \cite[Theorem~IV.12]{thom}, or
    \autoref{sec:indecomposabilitycriterion},
    \ref{item:generatorscobordimsring}, for a proof using
    Theorem~\ref{thm:basiscobordismring} below.
    Alternatively see \cite[Chap.~VI]{stong}.
  \end{proof}
\end{Thm}

During the proof of the above theorem, Thom constructs special
manifolds that form a basis of the cobordism ring,
and are uniquely characterized by the below properties.
For the formulation recall that a sequence or partition is called
\emph{non-dyadic}, if none of its entries is of the form $2^r-1$.
\begin{Thm}\label{thm:basiscobordismring}
  There exists a basis of the cobordism ring represented by manifolds
  $V_\Part$ that in each degree $k$ is indexed by non-dyadic
  partitions $\Part$ of $k$. Further, the $V_\Part$ are uniquely
  characterized by
  \begin{gather*}
    \s{\Part'}(\W{V_\Part}) = \delta_{\Part,\Part'}
    \in \H^k(V_{\Part})\cong \Zmod2
  \end{gather*}
  for any non-dyadic partitions $\Part,\Part'\in\PartitionsOf{k}$,
  where $\delta$ is the usual Kronecker delta.
  \begin{proof}
    See \cite[Section~IV.5, proof of Theorem~IV.9]{thom}.
    % [Milnor] better source?
  \end{proof}
\end{Thm}

In order to relate the results on the Stiefel-Whitney numbers of
manifolds---respectively certain linear combinations of them---from
before with cobordism classes, one needs the following more general
connection. Thom rather directly deduces this from the existence of
the above basis.
\begin{Thm}[Thom]\label{thm:cobordantiffswnumscoincide}
  Two closed manifolds are cobordant if and only if all of their
  Stiefel-Whitney numbers coincide.
  \begin{proof}[proof (sketch)]
    The proof that manifolds with coinciding Stiefel-Whitney numbers
    are cobordant was conducted by Thom \cite[Theorem IV.10]{thom}.
    To see that cobordant manifolds have the same Stiefel-Whitney
    numbers, let $M^n$ be a null-bordant closed manifold, \idest
    assume $M^n=\Boundary{W}$ for a closed manifold $W$. Now, consider
    any Stiefel-Whitney number 
    $\capped{\w{1}{M}^{i_1}\dotsm\w{l}{M}^{i_l}}{\fundcl{M}}$,
    of $M$, where $I=(i_1,\dotsc,i_l)$ with $\d(I)=n$.
    Then this is zero, as one calculates using the long
    exact sequence of cohomology respectively homology of the pair
    $i\colon M\immto W$ (abbreviated les) with boundary map
    $\partial$:
    \begin{align*}
      \W{M}
      &= \W{\T M}
        = \W{\T M\oplus\trivbdl}
        = \W{\T W|_{M}}
        \cequalsby{Def.} \W{\pb i \T W} = \pb i \W{W} \\
      \fundcl M
      &\cequalsby{les} \partial \fundcl{W} \\
      \intertext{which yields with the fact
      $\pf i\circ\partial\overset{\text{les}}= 0$}
      \capped{\W{M}^I}{\fundcl M}
      &= \capped{\pb i\W{W}^I}{\partial\fundcl{W}}
        = \pf i\capped{\W{W}^I}{\pf i\partial \fundcl{W}}
        = 0
        \;.
        \qedhere
    \end{align*}
  \end{proof}
\end{Thm}

\subsection{A Criterion for Indecomposability}
\label{sec:indecomposabilitycriterion}
Now, we focus on the ultimate goal of the current section, which
is to deduce the following indecomposability criterion. This
originates as a corollary from Thom's proof of the multiplicative
structure of the cobordism ring (see \cite[Section~IV.5]{thom}).
\begin{Thm}\label{thm:indecomposabilitycriterion}
  A closed $n$-manifold $M$ represents an indecomposable element of
  the cobordism ring $\c_*$, if and only if
  \begin{gather*}
    \snum{(n)}{M} \neq 0 \in \Zmod2
    \;.
  \end{gather*}
  % for non-dyadic decomposition: [p.~156]{thom};
  % for summation notation: [p.154, (4)f]{thom};
\end{Thm}

\begin{proof}[proof of Theorem~\ref{thm:indecomposabilitycriterion}]
  Using the main theorems \ref{thm:basiscobordismring} and
  \ref{thm:cobordantiffswnumscoincide} from
  \autoref{sec:cobordismringstructure}, as well as the main
  corollaries \ref{cor:productrule:swcl} and
  \ref{cor:swnumdecompositionmfds} from
  \autoref{sec:swnumsofproductmfds}, enables to obtain the
  desired result in the following steps.
  \begin{steps}
  \item\label{item:manifoldbasisrepr}
    As the classes $[V_\Part]$ form a basis of the cohomology ring by
    Theorem~\ref{thm:basiscobordismring}, any manifold $M^n$ is
    cobordant to a unique linear combination, \idest disjoint sum,
    \begin{gather*}
      [M] = \coprod_{\mathclap{\substack{
            \Part\in\PartitionsOf n \\\text{ non-dyadic}
          }}} \alpha_\Part [V_\Part]
      \;,\quad
      \alpha_\Part \in \Zmod2,
    \end{gather*}
    of the classes $[V_\Part]$.
    Now the Stiefel-Whitney numbers are determined by the cobordism
    class according to Theorem~\ref{thm:cobordantiffswnumscoincide},
    and are additive with respect to disjoint sums.
    So, one gets for any Stiefel-Whitney number $\wsnum{I}{M}$ of $M$
    \begin{align*}
      \wsnum{I}{M}
      &= \sum_{\mathclap{\substack{
        \Part\in\PartitionsOf n \\\text{ non-dyadic}
      }}} \alpha_\Part \wsnum{I}{V_\Part}
      \;,
      &\text{especially}
      &&\snum{\Part'}{M}
      &= \sum_{\mathclap{\substack{
        \Part\in\PartitionsOf n \\\text{ non-dyadic}
      }}} \alpha_\Part \snum{\Part'}{V_\Part}
      \cequalsby{Def.} \alpha_{\Part'}
      \;.
    \end{align*}
    Thus, $[V_\Part]$ is a summand of $[M]$ if and only if
    $\snum{\Part}{M}$ is non-zero. In other words, $M$ is cobordant to
    \begin{gather*}
      \coprod_{\mathclap{\substack{
            \Part\in\PartitionsOf n \text{ non-dyadic}\\
            \s{\Part}(W)\neq 0
          }}} V_\Part
      \;.
    \end{gather*}
  \item\label{item:productpartitions}
    For partitions $\Part_1'$ of $n_1$, and $\Part_2'$ of $n_2$ and
    any partition $\Part'$ holds
    \begin{align*}
      \s{\Part'}(V_{\Part_1'}\times V_{\Part_2'})
      &\cequalsby{\ref{cor:productrule:swcl}}
        \sum_{\mathclap{\substack{
        \Part_1\concat\Part_2=\Part'\\
      \Part_1\in\PartitionsOf{n_1}\\
      \Part_2\in\PartitionsOf{n_2}\\
      }}}
      \s{\Part_1}(V_{\Part_1'}) \cdot \s{\Part_2}(V_{\Part_2'})
      \overset{\text{Def.}}=
      \sum_{\mathclap{\substack{
      \Part_1\concat\Part_2=\Part'\\
      \Part_1\in\PartitionsOf{n_1}\\
      \Part_2\in\PartitionsOf{n_2}\\
      }}}
      \delta_{\Part_1,\Part_1'} \cdot \delta_{\Part_2, \Part_2'}
      = \delta_{\Part', \Part_1'\concat\Part_2'}
      \\
      &\equalsby{Def.} \s{\Part'}(V_{\Part_1'\concat\Part_2'})
        \;.
    \end{align*}
    Thus, by \ref{item:manifoldbasisrepr}
    the basis representations of
    $[V_{\Part_1'}]\times[V_{\Part_2'}]=[V_{\Part_1'}\times V_{\Part_2'}]$
    and 
    $[V_{\Part_1'\concat\Part_2'}]$
    coincide, wherefore they must be equal. 
  \item\label{item:generatorscobordimsring}
    By \ref{item:productpartitions}, all basis elements
    $[V_\Part]$ can be written as a product of lower degree basis
    elements, except for those corresponding to a trivial partition
    $(k)$, $k\in\Nat$.
    Furthermore, such a basis element $[V_{(k)}]$ cannot be
    decomposable, as otherwise $\snum{(k)}{V_{(k)}} = 0$
    by \autoref{eq:productrule:swnum} in
    Corollary~\ref{cor:swnumdecompositionmfds},
    which contradicts the definition in
    Theorem~\ref{thm:basiscobordismring}.
    
    Altogether, the basis elements represented by a $k$-dimensional
    manifold $V_{(k)}\in\c_k$ for $k\neq 2^m-1$ are
    indecomposable---hence algebraically independent---generators of
    the cobordism ring.
    This is a proof of Theorem~\ref{thm:cobordismringstructure} using
    Theorem~\ref{thm:basiscobordismring}.
  \item By \ref{item:generatorscobordimsring}, the cobordism
    class of a manifold $W$ of dimension $k$ is an indecomposable
    element of $\c_*$ if and only if its unique representation by
    basis elements $[V_\Part]$ contains as a summand the unique
    $k$-dimensional indecomposable basis element $[V_{(k)}]$,
    \idest if and only if $\snum{(k)}{W}\neq 0$ by
    \ref{item:manifoldbasisrepr}.
    \qedhere
  \end{steps}
\end{proof}

This directly yields the following example which will be a key point
in constructing a candidate generating set of the cobordism ring.
\begin{Ex}\label{ex:rpnindecomposable}
  For $k\in\Nat$ even, the projective space $\RP k$ represents an
  indecomposable element of the cobordism ring.
  \begin{proof}
    If one applies Remark~\ref{rem:spolyevaluation} to
    \begin{gather*}
      \W{\RP k}
      = (1+x)^{k+1}
      = \prod_{i=1}^{k+1}(1+x)
      = 1+\sum_{j=1}^{k+1} \el{j}{k+1}(x,\dotsc,x)
    \end{gather*}
    where $x$ is the
    generator in degree one of $\H^*(\RP k)\cong\Zmod2[x]/(x^{k+1})$,
    one gets
    \begin{align*}
      \s{(k)}(\RP k)
      &\cequalsby{\ref{rem:spolyevaluation}}
        \sum_{i=1}^{k+1} x^k
        = (k+1)x^k \\
      \snum{(k)}{\RP k}
      &= \capped{(k+1)x^k}{\fundcl{\RP k}}
        = k+1
        \equiv 1 \mod2
        \;.
        \qedhere
    \end{align*}
  \end{proof}
\end{Ex}

\section{Twisted Products}
\label{sec:twistedprod}
The candidates for a generating set needed for the proof of Brown's
Theorem~\ref{thm:brown} will be inductively constructed
using the so-called twisted product construction explained below.
The main advantage of this tool is---besides quite a couple of handy
preservation properties---the fact that a twisted product is
indecomposable if and only if its factor is and the dimension was
chosen correctly
(Theorem~\ref{thm:twistedprod:indecompcriterion}). The latter will
be the main result of this section, and is discussed in
\autoref{sec:twistedprod:indecompcriterion}.

\subsection{Definition}\label{sec:twistedproddef}
The following definition is according to
\cite[p.~83]{immersionconj} respectively 
\cite[compare~§4, Def./ of $P(m;X)$]{brown}.
\begin{Def}
  Let $X$ be a space and $k\in\Nat$ an integer.
  Define the \emph{twisted product of $X$ by $\Sphere k$}, denoted
  $\Twistedprod{k}{X}$, to be the orbit space of the properly
  discontinuous $\Zmod2$-action on $\Twistedprodcovspace{k}{X}$ given
  by
  \begin{align*}
    \Zmod2 &\leftgroupaction \Twistedprodcovspace{k}{X}
             \;,
    &[1] \actson (s, (p_1,p_2)) &\coloneqq (-s, (p_2, p_1))
                                  \;,
  \end{align*}
  which combines the antipodal action $[1]\actson s\coloneqq -s$ on
  $\Sphere k$ and twisting on $X\times X$.
  For a map $f\colon X\to Y$ of spaces, define
  \begin{gather*}
    \Twistedprod{k}{f}\coloneqq (\Id\times f\times f/\sim)\colon
    \Twistedprod{k}{X}\longto\Twistedprod{k}{Y}
    \;,\quad
    [s,(p_1,p_2)] \longmapsto [s,(f(p_1),f(p_2))]
    \;.
  \end{gather*}
\end{Def}
\begin{Ex}
  Major examples needed later are
  \begin{itemize}
  \item $\Twistedprod k {\pt} = \RP k$, and
  \item $\Twistedprod 0 {M} = M\times M$.
  \end{itemize}
\end{Ex}

First, gather some rather immediate, convenient properties. It is
especially noteworthy how well the twisted product behaves concerning
manifolds and fiber bundles.
\begin{Rem}\label{rem:twistedprodproperties}
  Let $X$ be a space and $k\in\Nat$.
  \begin{enumerate}
  \item $\Twistedprod{k}{-}$ is a functor on the category of
    topological spaces preserving injectivity.
  \item\label{item:twistedprodfiberbdl}
    $\Twistedprod{k}{-}$ preserves fiber bundles,
    \idest for a fiber bundle $\xi\colon\E\xi\to X$ with fiber $F$
    the twisted product $\Twistedprod{k}{\xi}\colon
    \Twistedprod{k}{\E\xi}\to \Twistedprod{k}{X}$
    is again a fiber bundle with fiber $F\times F$.
    This comes from the fiber bundle
    \begin{gather*}
      F\times F
      \longto \Twistedprodcovspace{k}{\E\xi}
      \longto \Twistedprodcovspace{k}{X}
    \end{gather*}
    where all maps are maps of $\Zmod2$-spaces.
    As a special case, $\Twistedprod{k}{X}$ admits a fiber bundle
    \begin{gather}\label{eq:twistedprodrpnfiberbdl}
      X\times X
      \longto \Twistedprod{k}{X}
      \longto \RP k = \Sphere k/\sim
    \end{gather}
    with fiber $X\times X$ which comes from the trivial fiber bundle
    $X\to\pt$.
    Further, let $\eta\colon\E\eta\to X$ be
    another fiber bundle, and $f\colon X'\to X$ a map.
    One has:
    \begin{enumerate}
    \item\label{item:twistedprod:preservespb}
      $\Twistedprod{k}{-}$ respects pullbacks, \idest
      \begin{gather*}
        \Twistedprod{k}{\pb f \xi}
        = \pb{\left(\Twistedprod{k}{f}\right)}
        \left(\Twistedprod{k}{\xi}\right)
      \end{gather*}
    \item $\Twistedprod{k}{-}$ respects Whitney sums of vector
      bundles, \idest 
      \begin{gather*}
        \Twistedprod{k}{\xi\oplus\eta\colon \E{(\xi\oplus\eta)}\to X}
        = \Twistedprod{k}{\xi}\oplus\Twistedprod{k}{\eta}
        \;.
      \end{gather*}
      To see this, observe that the following is a well-defined
      commutative pullback diagram of vector bundles:
      \begin{center}
        \begin{tikzcd}
          \Twistedprod{k}{\E{(\xi\oplus\eta)}}
          \ar[r]
          \ar[d, "\Twistedprod{k}{\xi\oplus\eta}"]
          &\Twistedprod{k}{\E{(\xi\times\eta)}}
          \ar[r]
          \ar[d, "\Twistedprod{k}{\xi\times\eta}"]
          &\Twistedprod{k}{\E{\xi}}\times\Twistedprod{k}{\E{\eta}}
          \ar[d, "\Twistedprod{k}{\xi}\times \Twistedprod{k}{\eta}"]
          \\
          \Twistedprod{k}{X}
          \ar[r, "\Twistedprod{k}{\Delta}"]
          &\Twistedprod{k}{X\times X}
          \ar[r, "\widetilde\Delta"]
          &\Twistedprod{k}{X}\times\Twistedprod{k}{X}
        \end{tikzcd}
      \end{center}
      where
      $\widetilde\Delta\colon
      [s,(x_1,y_1),(x_2,y_2)]\mapsto\left([s,x_1,x_2],[s,y_1,y_2]\right)$.
    \end{enumerate}
  \item\label{item:twistedprodmanifold}
    $\Twistedprod{k}{-}$ preserves closed smooth manifolds, \idest
    for a closed smooth manifold $M^n$, $\Twistedprod{k}{M}$ is a
    again a $(2m+k)$-dimensional closed smooth manifold.
    This is because the proper discontinuity comes from the antipodal
    $\Zmod2$-action, and makes the projection
    \begin{gather*}
      \Twistedprodcovspace{k}{X}
      \xrightarrow{\pi}
      \Twistedprod{k}{X}
      \coloneqq
      \left( \Twistedprodcovspace{k}{X} \right)/\sim
    \end{gather*}
    a two-leaved covering space.
    Further:
    \begin{enumerate}
    \item\label{item:twistedprodpreservesimmersions}
      $\Twistedprod{k}{-}$ preserves immersions.
    \item\label{item:twistedprod:tangentspace}
      $\T\Twistedprod{k}{M}
      \cong \pb\proj\T{\RP k} \oplus \Twistedprod{k}{\T M}$,
      \idest the tangent space of $\Twistedprod{k}{M}$ can be obtained
      from $\Twistedprod{k}{\T M}$ by adding the missing tangent space
      part of the sphere:
      \begin{alignat*}{4}
        \T{\Twistedprod{k}{M}}
        &\cong& \T{\left(\Twistedprodcovspace{k}{M}\right)}/\sim \\
        &\cong& \T \Sphere k\times \T M\times\T M/\sim
        &\overset{\cong}{\longto}
        \pb\proj \T{\RP k} \oplus \Twistedprod{k}{\T M}
        \\
        &&[(s,v), (m_1,v_1), (m_2,v_2)]
        &\longmapsto
        \left( ([s], v), [s, (m_1,v_1), (m_2, v_2)] \right)
      \end{alignat*}
      where $\proj\colon\Twistedprod{k}{M}\to\RP k$ is the projection.
      The first isomorphism is due to the covering space property, and
      the last is easily seen to be a well-defined isomorphism of
      vector bundles.
      Further note that for a map of manifolds $f\colon M\to N$,
      the differential map $\Diff\Twistedprod{k}{f}$ on tangent spaces
      will be the identity on the first summand.
    \end{enumerate}
  \end{enumerate}
\end{Rem}

As a last inside into the definition have a look at the twisted
product of real spaces.
\begin{Lem}\label{lem:twistedprodrealspace}
  Let $k, n\in\Nat$,
  % and denote by $\N{}$ is the normal line bundle of $\Sphere k$,
  and by $\gamma_k$ the tautological line bundle over
  $\RP k$.
  The fiber bundle $\Twistedprod{k}{\R^n}\to \RP k$ is isomorphic to
  the vector bundle
  \begin{gather*}
    (n\cdot\gamma_k)\oplus\trivbdl^n
    = (\gamma_k\oplus\dotsb\oplus\gamma_k)\oplus\trivbdl^n
  \end{gather*}
  \begin{proof}
    Compare also \cite[Prop.~4.3,p~1107]{brown}.
    A well-defined vector bundle isomorphism is for example
    \begin{align*}
      \Twistedprod{k}{\R^n}
      &\overset{\sim}\longto
        (\Twistedprodcovspace{k}{\R^n}/\approx)
        \cong (\Sphere{k}\times\R^n/\approx)\times\R^n
        \cong \E{((n\cdot\gamma_k)\oplus\trivbdl^n)}
      \\
      [s, v_1, v_2]
      &\longmapsto
        [s, v_1+v_2, v_1-v_2]
    \end{align*}
    where $\approx$ is the equivalence relation identifying
    $(s,v_1,v_2)$ and $(-s,-v_1,v_2)$ respectively $(s,v)$ and
    $(-s,-v)$.
    Here it was used that $\gamma_k$ is by
    construction
    \begin{align*}
      \E{\gamma_k} = (\Sphere k\times\R/\approx) &\to \RP k
      &[s,v] &\mapsto [s]
               \;,
    \end{align*}
    and hence
    $\E{(n\gamma_k)}\cong(\Sphere k \times\R^n/\approx)$.
  \end{proof}
\end{Lem}

\subsection{The Cohomology Ring of Twisted Products}
Besides the above direct properties, there is a fairly easy
description of the cohomology ring of a twisted product relating it to
the cohomology ring of its factor.
This can be revealed inductively using a diagram of long exact
sequences which relates the cohomology of two degrees of twisted
products and known sequences of spheres.
Use the following notation.

\begin{Def}
  Let $X$ be a space and $k\in\Nat$.
  Note that by the Künneth isomorphism
  $\H^*(X^2)\cong\H^*(X)\otimes\H^*(X)$, and recall that all exact
  sequences of $\Zmod2$-vector spaces split.
  Define by
  \begin{itemize}
  \item
    $\pi_k\colon \Sphere k\times X^2\to\Twistedprod k X$
    to be the quotient map from the definition,
  \item
    $\proj\colon\Twistedprod k X\to \RP k$
    to be the fiber bundle map,
  \item
    $T\colon \Sphere{k}\times X^2\to\Sphere{k}\times X^2$,
    $(s,p,q)\mapsto(-s,q,p)$,
  \item
    $N\coloneqq
    \ker(x\otimes y\mapsto x\otimes y + y\otimes x)
    = \left(
        a\otimes b + b\otimes a
        \middle|
        a,b\in \H^*(X\times X)
      \right)_{\Zmod2}$,
  \item
    $d\colon \H^*(X)\to\H^*(X\times X)$,
    $d(a)\coloneqq a\otimes a$, and
    $D\coloneqq \Im d
    = \left\{ a\otimes a \in \H^*(X\times X) \right\}$,
  \item
    $s_k\in\H^k(\Sphere k)\cong\Zmod2$ to be the generator, and
  \item
    $c_k\coloneqq \pb\proj x\in\H^1(\Twistedprod{k}{X})$, where
    $x\in\H^*(\RP k)\cong\Zmod2[x]/(x^{k+1})$ is the unique generator.
  \end{itemize}
  The indices $k$ are omitted if they are obvious from the context.
\end{Def}
\begin{Rem}
  Note that $N+D$ is closed under multiplication and addition,
  and---as a first hint on the cohomology structure---$\proj$ admits a
  section  $[s]\mapsto[s,p,p]$ for any point $p\in X$, thus making
  $\H^*(\RP k)$ a direct summand of $\H^*(\Twistedprod k X)$ of the
  form $\Zmod2[c]/(c^{k+1})$.
  From the commutative diagram factorization
  \begin{center}
    \begin{tikzcd}
      S^k\times X^2
      \ar[d, "\pi"]
      \ar[r, "\proj_{\Sphere k}"]
      & \Sphere k \ar[d, "\pi"]
      \\
      \Twistedprod k X
      \ar[r, "\Twistedprod{k}{X\to\pt}"]
      &\RP k      
    \end{tikzcd}
  \end{center}
  and with $\pb\pi\colon\RP k\to\Sphere k$ being zero, one gets $\pb\pi(c)=0$.
\end{Rem}
\begin{Thm}\label{thm:twistedprod:cohomstructure}
  Let $X$ be a space and $k\in\Nat$.
  Then the cohomology ring of $\Twistedprod{k}{X}$ has the form
  \begin{align*}
    \H^*(\Twistedprod{k}{X})
    &\cong
      \left(
      \Zmod2[c,s]/(c^{k+1},s^2,cs)
      \right)
      \otimes (N+D)
  \end{align*}
  with $c$ of degree 1, $s$ of degree $k$, and the additional
  properties
  \begin{enumerate}
  \item $c\otimes N=0=s\otimes D$, 
  \item $\pb\proj x = c\otimes d(1)$,
    hence
    \begin{gather*}
      \pb\proj\colon\H^*(\RP k)\cong
      \Zmod2[c]/(c^{k+1})\subset\H^*(\Twistedprod k X)
      \;,
    \end{gather*}
    and
  \item\label{item:twistedprodcohom:pi}
    $\pb\pi (s\otimes n) = s\otimes n$,
    $\pb\pi (1\otimes (n+d(a))) = 1\otimes (n+d(a))$
    for $n\in N$ and $a\in\H^*(X)$,
    hence
    \begin{gather*}
      \pb\pi\colon (1\otimes(N+D)) + (s\otimes N)
      \cong (1\otimes(N+D))\oplus (s_k\otimes N)
      \;.
    \end{gather*}
  \end{enumerate}
  For readability skip $1\otimes-$ and $-\otimes d(1)$ in element
  notation wherever it the meaning is clear from the context.
  Further note that $D$ is multiplicatively, but not additively closed,
  whereas $(N+D)\subset\H^*(X^2)$ is a subring via
  $d(a)+d(b) = d(a+b)+(a\otimes b+ b\otimes a)$.
\end{Thm}
A proof of Theorem~\ref{thm:twistedprod:cohomstructure} can be found
at the end of the section.

\subsubsection[Stiefel-Whitney Classes]
{Stiefel-Whitney Classes of Twisted Products of Vector Bundles}
The results of this section will yield a splitting principle
applicable to Stiefel-Whitney classes of twisted products of vector
bundles.

Some immediate consequences of the above structure theorem are:
\begin{Cor}
  Let $X$ again be a space and $k\in\Nat$.
  \begin{enumerate}
  \item
    For any section $s_p\colon\RP k\to\Twistedprod{k}{X}$, $q\mapsto[q,p,p]$,
    of the fiber bundle described in
    \eqref{eq:twistedprodrpnfiberbdl}, and $n_1,n_2\in N$, $0\neq a\in\H^*(X)$ holds
    \begin{align}\label{eq:twistedprodcohom:section}
      \pb s_p(c) &= x
      &\text{and}&
      &\pb s_p(c\otimes d(a) + 1\otimes n_1 + s\otimes n_2) &= 0
                                                              \;.
    \end{align}
  \item\label{item:twistedprod:preservescohominj}
    $\Twistedprod{k}{-}$ preserves injectivity on cohomology.
  \end{enumerate}
  \begin{proof}
    The section property is clear from the behavior of $\pb\proj$.
    Consider a map $f\colon X\to Y$ which is injective on cohomology
    and induces the map
    $F\colon\Sphere{k}\times X^2\to\Sphere{k}\times Y^2$.
    Since every element in $\H^*(\RP k)\otimes D$ can uniquely be
    written as $\sum_{i=0}^k c^i\cdot d(a_i)$,
    one only has to check injectivity of $\pb{\Twistedprod k f}$ on
    the two parts
    \begin{gather*}
      \left( \Zmod2[c]/(c^{k+1})\otimes 1 \right)
      \qquad\text{and}\qquad
      \left(
        (1\otimes (N+D)) + (s\otimes N)
      \right)
      \;.
    \end{gather*}
    \begin{description}
    \item[First part:]
      Since $\Twistedprod k f$ is a morphism of vector bundles over
      $\RP k$,
      \begin{gather*}
        \pb{\Twistedprod k f}(c)
        = \pb{\Twistedprod k f}(\pb\proj(x))
        = \pb\proj(x)
        = c
        \in\H^*(\Twistedprod k X)
        \;,
      \end{gather*}
      so $\pb{\Twistedprod k f}$ is injective on
      $\H^*(\RP k)\otimes 1\subset\H^*(\Twistedprod k Y)$.
    \item[Second part:]
      Obviously
      $\pb F\colon\H^*(\Sphere k\times Y^2)\to\H^*(\Sphere k\times X^2)$ 
      will be injective. The isomorphism property of $\pb\pi$ hence
      implies injectivity of $\pb{\Twistedprod k f}$ on
      $(1\otimes D)+(1\otimes N)+(s_k\otimes N)\subset\H^*(\Twistedprod k Y)$.
      \qedhere
    \end{description}
  \end{proof}
\end{Cor}

\begin{Rem}
  Let $\xi$ be a vector bundle over a space $X$.
  Recall that by the splitting
  principle~\cite[Theorem~(19.3.9)]{tomdieck}, $\W{\xi}$ is
  pulled back to the product $\prod_i\W{\xi_i}$ of total
  Stiefel-Whitney classes of some line bundles $\xi_i$,
  along a map $f\colon Y\to X$ which is injective on cohomology.
  Since $\Twistedprod{k}{-}$ preserves Whitney sums and injectivity on
  cohomology, $\W{\Twistedprod{k}{\xi}}$ will injectively pull
  back along $\Twistedprod{k}{f}$ to the product
  $\prod_i\W{\Twistedprod{k}{\xi_i}}$.
\end{Rem}

Thus, to get from the Stiefel-Whitney classes of a vector bundle to
the ones of its $k$th twisted product, a good approach is to
investigate how $\Twistedprod{k}{-}$ acts on the Stiefel-Whitney
classes of line bundles.
\begin{Cor}\label{cor:twistedprod:swlinebdl}
  Let $\xi\colon E\to X$ be a line bundle with total Stiefel-Whitney
  class $\W{\xi}=1+\alpha$, and let $k\in\Nat$.
  Define $e\colon\H^*(X)\to N\subset\H^*(X\times X)$,
  $e(a)\coloneqq 1\otimes a+a\otimes 1$.
  Then, along the isomorphism from
  Theorem~\ref{thm:twistedprod:cohomstructure},
  \begin{gather*}
    \W{\Twistedprod{k}{\xi}} = 1+ (c_k\otimes d(1)+1\otimes
    e(\alpha)) + 1\otimes d(\alpha)
    = 1+c_k+e(\alpha)+d(\alpha)
    \;,
  \end{gather*}
  respectively $\w1\xi = c_k+e(\alpha)$, $\w2\xi=d(\alpha)$.
  \begin{proof}
    See also \cite[Prop.~7.4, p.~1113]{brown}.
    For $k=0$ this is simply the product rule for the total
    Stiefel-Whitney class because $\Twistedprod{0}{\xi}=\xi\times\xi$
    and $c=0$. Thus, assume $k\geq0$.
    With $\deg\w{i}{\Twistedprod{k}{\xi}}\leq\rk\Twistedprod{k}{\xi}=2$,
    the total Stiefel-Whitney class of the two-dimensional vector bundle
    $\Twistedprod{k}{\xi}$ must by
    Theorem~\ref{thm:twistedprod:cohomstructure} be of the general
    form
    \begin{align*}
      \W{\Twistedprod{k}{\xi}}
      &= 1
        + \sum_{i=1}^k c^i\otimes d(a_i)
        + s\otimes n'
        + 1\otimes(n+d(a))
         \\
      &\equalsby{}
        1
        + \underbrace{
        c\otimes d(a') + c^2\otimes d(a'')
        }_{\text{check section}}
        + \underbrace{
        s\otimes n'
        + 1\otimes(n+d(a))
        }_{\text{check $\pi$}}
        \;,
        % 1 &\quad&&&{(=\ws0)} \\
      % &+ \delta_1\cdot c\otimes d(1) + \delta_2\cdot c^2\otimes d(1)
      %     &&\delta_1,\delta_2\in\{0,1\}
      %           &&{(\text{check section})}\\
      % &+ 1\otimes d(a)
      %     &&a \neq 1
      %           &&{(\text{check $\pi$})} \\
      % &+ {\textstyle \sum_{r\in I} c^{i_r}\otimes d(a_r)}
      %     &&a_r\neq 1,\; i_r>1
      %           &&{(=0\text{ by dim.})} \\
      % &+ 1\otimes n_1 + s\otimes n_2
      %     &&n_1, n_2\in N
      %           &&{(\text{check $\pi$})}
    \end{align*}
    for some $n,n'\in N$ and $a,a',a''\in\H^*(X)$,
    with $\deg a\leq 0\geq\deg a''$, $\deg n'\leq1\geq\deg a$, and
    $\deg n\leq 2$.
    In order to determine the unknown elements,
    note that by Theorem~\ref{thm:twistedprod:cohomstructure}
    \begin{itemize}
    \item $\pb\pi$ is an isomorphism on $s\otimes N+1\otimes(D+N)$, and
    \item for any point $p\in X$ and section $s_p\colon\RP
      k\to\Twistedprod{k}{X}$, $\pb s_p$ is an isomorphism on
      $\{c^i\otimes d(b)|b\in\H^0(X), 1\leq i\leq k\}$.
    \end{itemize}
    So the following remains to check:
    \begin{description}
    \item[$\pb\pi\W{\Twistedprod k \xi}$:]
      $\Twistedprod{k}{\xi}$ is the quotient of the bundle
      $\trivbdl\times\xi\times\xi\colon
      \Twistedprodcovspace{k}{E}\to\Twistedprodcovspace{k}{X}$, where
      \begin{gather*}
        \W{\Id\times\xi\times\xi}
        = 1\cdot \W{\xi}\cdot\W{\xi}
        = 1 + 1\otimes e(\alpha) + 1\otimes d(\alpha)
        \;.
      \end{gather*}
      As $\pi$ is a covering map,
      $\pb\pi\Twistedprod{k}{\xi}=\trivbdl\times\xi\times\xi$,
      and thus
      $\pb\pi\W{\Twistedprod{k}{\xi}}=\W{\Id\times\xi\times\xi}$.
      By
      \itemref{thm:twistedprod:cohomstructure}{item:twistedprodcohom:pi},
      $\pb\pi$ is the identity on
      elements of this form, so this yields
      $n'=0$, $n=e(\alpha)$, and $a=\alpha$.
    \item[$\pb s_p \W{\Twistedprod{k}{\xi}}$:]
      Consider a section $s_p\colon [s]\mapsto [s,p,p]$, $p\in X$, of
      the bundle $\Twistedprod k X\to\RP k$. Since
      $s_p=\Twistedprod k{(\ast\mapsto p\in X)}$,
      the pullback of $\Twistedprod{k}{\xi}$ along $s_p$ yields
      \begin{align*}
        \pb s_p\Twistedprod{k}{\xi}
        &= \pb{\Twistedprod{k}{\ast\mapsto p}}
          \left(\Twistedprod{k}{\xi}\right) \\
        &= \Twistedprod{k}{\pb{(\ast\mapsto p)}\xi} \\
        &= \Twistedprod{k}{\trivbdl^1}
          \cequalsby{\ref{lem:twistedprodrealspace}}
          \gamma_k\oplus\trivbdl^1
          \colon
          \Twistedprod{k}{\R} \to \RP k\;,\quad
          [s, v_1,v_2] \mapsto [s]
          \;.
      \end{align*}
      Then
      \begin{gather*}
        \pb{s_p}\W{\Twistedprod{k}{\xi}}
        = \W{\pb{s_p}\Twistedprod{k}{\xi}}
        = \W{\gamma\oplus\trivbdl}
        = 1+x\in\H^*(\RP k)
        \;,
      \end{gather*}
      .and thus $a'=1$, $a''=0$.
      \qedhere
    \end{description}
  \end{proof}
\end{Cor}


\subsubsection{A Proof of the Structure Theorem}
The rest of this section is dedicated to the proof of
Theorem~\ref{thm:twistedprod:cohomstructure} on the cohomology
structure of twisted products.

The essential step is to relate a twisted product to
\begin{enumerate}[1.]
\item its lower dimensional counterpart (analogous to the embedding of
  a projective space into a higher dimensional one), and
\item well-known sequences of spheres and disks.
\end{enumerate}
This can be done using an alternative pushout construction, which is
explained below. The proofs are omitted since all facts are easy to
check.

Having constructed a diagram of pairs of spaces with a corresponding
one of cohomology groups, the more tedious part then is to finalize
the proof with an inductive diagram chase, which is given for
completeness but may be skipped by the reader.

\begin{Fact}
  Let $X$ again be a space and $k\in\Nat_{\geq1}$.
  \begin{enumerate}
  \item
    The twisted product $\Twistedprod{k}{X}$ is the pushout
    $(\Disk k\times X^2)\cup_{T}(\Sphere{k-1}\times X^2)$ of
    \begin{center}
      \begin{tikzcd}
        \Sphere{k}\times X^2
        &\ar[from=l,leftarrow, "T"]
        \Sphere{k}\times X^2
        \ar[r, rightarrowtail, "\incl"]
        &\Disk{k}\times X^2
      \end{tikzcd}
    \end{center}
    \idest it is the product $\Disk k\times X^2$ of the closed
    $k$-disk with $X^2$ with the identification
    $(s,p,q)=T((s,p,q))\coloneqq(-s,q,p)$ on all boundary points in
    $\Boundary{\Disk k\times X^2}=\Sphere{k-1}\times X^2$.
  \item
    The pushouts
    \begin{align*}
      \Twistedprod{k}{X}
      &=(\Disk k\times X^2)\cup_{T}(\Sphere{k-1}\times X^2)
        \qquad\text{and}\\
      \Twistedprod{k-1}{X}
      &=(\Sphere{k-1}\times X^2)\cup_{T}(\Sphere{k-1}\times X^2)
    \end{align*}
    merge to the commutative pushout diagram
    \begin{center}
      \begin{tikzcd}
        \Sphere{k-1}\times X^2
        \ar[r, equals]
        \ar[d, "T"]
        &\Sphere{k-1}\times X^2
        \ar[r, rightarrowtail, "\incl"]
        \ar[d, "\pi"]
        &\Disk{k}\times X^2
        \ar[d]
        \\
        \Sphere{k-1}\times X^2
        \ar[r, "\pi"]
        &\Twistedprod{k-1}{X}
        \ar[r, rightarrowtail, "\incl"]
        &\Twistedprod{k}{X}
      \end{tikzcd}
    \end{center}
    making $(\Twistedprod{k}{X}, \Twistedprod{k-1}{X})$ a neighborhood
    deformation retract pair using the stability of cofibrations under pushout.
    Furthermore, for a smaller disk $\Disk{k}_+\subsetneq\Disk k$
    there is an induced excision
    \begin{gather*}
      (\Disk k\times X^2, \Sphere{k-1}\times X^2)
      \rightarrowtail
      (\Twistedprod{k}{X}, \overline{\Twistedprod{k-1}{X}})
    \end{gather*}
    where
    $\overline{\Twistedprod{k-1}{X}}\coloneqq
    ((\Disk k\setminus\Disk{k}_+)\times X^2)
    \cup_{T}
    (\Sphere{k-1}\times X^2)$
    is the embedded
    $\Twistedprod{k-1}{X}$ with a collar, \idest a neighborhood
    deformation retract of $\Twistedprod{k-1}{X}$ in
    $\Twistedprod{k}{X}$.
  \item
    The excision
    \begin{gather*}
      (\Disk{k}_+,\Sphere{k-1}_+) \amalg (\Disk{k}_-,\Sphere{k-1}_-)
      \immlongto
      (\Sphere k, \Sphere{k-1}\times I)
    \end{gather*}
    of an upper and a lower polar cap into the sphere relative to its
    bloated equator, is compatible with the excision from above
    in the sense that the following diagram commutes:
    \begin{center}
      \begin{tikzcd}
        (\Disk{k}_+,\Sphere{k-1}_+) \amalg (\Disk{k}_-,\Sphere{k-1}_-)
        \ar[r, hookrightarrow, "\text{excis.}"]
        \ar[d, "\pi"]
        &(\Sphere k, \Sphere{k-1}\times I)
        \ar[r, dash, "\simeq"]
        \ar[d, "\pi"]
        &(\Sphere k, \Sphere{k-1})
        \ar[d, "\pi"]
        \\
        (\Disk{k}, \Sphere{k-1})\times X^2
        \ar[r, hookrightarrow, "\text{excis.}"]
        &(\Twistedprod{k}{X},\overline{\Twistedprod{k-1}{X}})
        \ar[r, dash, "\simeq"]
        &(\Twistedprod{k}{X}, \Twistedprod{k-1}{X})
      \end{tikzcd}
    \end{center}
  \end{enumerate}
\end{Fact}

This now nicely fits into a larger diagram of pairs of spaces, which
induces a diagram of long exact sequences of cohomology.
\begin{Fact}
  For a space $X$ and $k\in\Nat_{\geq1}$ the following diagram commutes
  \begin{center}
    \begin{tikzcd}[column sep=small]
      (\Disk{k}_+,\Sphere{k-1}_+)\times X^2
      \ar[r,hookrightarrow,"\tau"]
      \ar[rr, equals, bend left=30]
      &\displaystyle\coprod_{+,-} (\Disk{k},\Sphere{k-1})\times X^2
      \ar[r,"\pi"]
      &(\Disk{k}_+,\Sphere{k-1}_+)\times X^2
      \\
      &
      \ar[from=u, hookrightarrow, "\text{excision}"]
      (\Sphere{k},\Sphere{k-1}\times I)\times X^2
      \ar[r, "\pi"]
      &
      \ar[from=u, hookrightarrow, "\text{excision}"]
      \left(
        \Twistedprod{k}{X},
        \overline{\Twistedprod{k-1}{X}}
        % \Twistedprod{k-1}{X}\cup_{\Sphere{k-1}_+} (\Sphere{k-1}\times I)
      \right)
      \\
      &
      \ar[u, dash, "\simeq"]
      (\Sphere{k},\Sphere{k-1})\times X^2
      \ar[r, "\pi_k"]
      &
      \ar[u, dash, "\simeq"]
      (\Twistedprod{k}{X},\Twistedprod{k-1}{X})
      \ar[r, "\proj"]
      &(\RP k, \RP{k-1})
      \\
      \ar[uuu, hookrightarrow, "\tilde j"]
      \Disk{k}_+\times X^2
      \ar[r, hookrightarrow, "\iota"]
      &
      \ar[u, hookrightarrow, "\hat j"]
      (\Sphere k\times X^2)
      \ar[r, "\pi_k"]
      &
      \ar[u, hookrightarrow, "j"]
      \Twistedprod{k}{X}
      \ar[r, "\proj"]
      &
      \ar[u, hookrightarrow]
      \RP k
      \\
      \ar[u, hookrightarrow, "\tilde i"]
      \Sphere{k-1}\times X^2
      \ar[r, equals]
      &
      \ar[u, hookrightarrow, "\hat i"]
      \Sphere{k-1}\times X^2
      \ar[r, "\pi_{k-1}"]
      &
      \ar[u, hookrightarrow, "i"]
      \Twistedprod{k-1}{X}
      \ar[r, "\proj"]
      &
      \ar[u, hookrightarrow]
      \RP{k-1}
    \end{tikzcd}
  \end{center}
  resulting in the commutative diagram of exact cohomology sequences
  \begin{center}
    \begin{tikzcd}[column sep=small]
      \vdots\ar[d]&\vdots\ar[d]&\vdots\ar[d]\\
      \H^l((\Disk{k},\Sphere{k-1})\times X^2)
      \ar[from=r, "\pb\tau"]
      &\H^l((\Sphere{k},\Sphere{k-1})\times X^2)
      \ar[from=r, "\pb{\pi_k}"]
      &\H^l((\Disk{k},\Sphere{k-1})\times X^2)
      \\
      \ar[from=u, "\pb{\tilde j}"]
      \H^l(\Disk{k}\times X^2)
      \ar[from=r, "\pb{\iota}"]
      &
      \ar[from=u, "\pb{\hat j}"]
      \H^l(\Sphere k\times X^2)
      \ar[from=r, "\pb{\pi_k}"]
      &
      \ar[from=u, "\pb j"]
      \H^l(\Twistedprod{k}{X})
      \\
      \ar[from=u, "\pb{\tilde i}"]
      \H^l(\Sphere{k-1}\times X^2)
      \ar[r, equals]
      &
      \ar[from=u, "\pb{\hat i}"]
      \H^l(\Sphere{k-1}\times X^2)
      \ar[from=r, "\pb{\pi_{k-1}}"]
      &
      \ar[from=u, "\pb i"]
      \H^l(\Twistedprod{k-1}{X})
      \\
      \ar[from=u, "\tilde{\delta}"]
      \H^{l-1}((\Disk{k},\Sphere{k-1})\times X^2)
      \ar[from=r, "\pb\tau"]
      &
      \ar[from=u, "\hat{\delta}"]
      \H^{l-1}((\Sphere{k},\Sphere{k-1})\times X^2)
      \ar[from=r, "\pb{\pi_k}"]
      &
      \ar[from=u, "\delta"]
      \H^{l-1}((\Disk{k},\Sphere{k-1})\times X^2)
      \\
      \vdots\ar[from=u]&\vdots\ar[from=u]&\vdots\ar[from=u]
    \end{tikzcd}
  \end{center}
  where $\pb\tau\pb\pi$ is the identity.
\end{Fact}

The useful thing about the cohomology diagrams arising from the
above diagram of pairs of spaces is that most of the columns and
sideways maps are well-known. Some facts are listed below.
\begin{Fact}
  Let $X$ again be a space and $k\in\Nat$.
  \begin{enumerate}
  \item
    The cohomology sequences of the pairs
    $(\Disk{k},\Sphere{k-1})\times X^2$ and
    $(\Sphere{k},\Sphere{k-1})\times X^2$ are well-known from the
    sequences of the pairs $(\Disk{k},\Sphere{k-1})$ and $(\Sphere{k},\Sphere{k-1})$.
    For $k>1$ they are given for $l\in\Nat$:
    \begin{center}
      \begin{tikzcd}[row sep=tiny, column sep=small]
        &\argument{(1\otimes f,s_{k-1}\otimes g)}
        \ar[r, mapsto]
        &\argument{(s_k\otimes g, s_k\otimes g)}
        \\
        \argument{(1\otimes f, s_k\otimes g)}
        \ar[r, mapsto]
        &\argument{(1\otimes f, 0)}
        &\argument{(s_k\otimes f, s_k\otimes g)}
        \ar[r, mapsto]
        &\argument{(0, s_k\otimes (f+g))}
        \\
        \H^{l-1}(\Sphere k\times X^2)
        \ar[r, "\pb{\hat i}"]
        &\H^{l-1}(\Sphere{k-1}\times X^2)
        \ar[r, "\hat{\delta}"]
        &\H^l((\Sphere{k},\Sphere{k-1})\times X^2)
        \ar[r, "\pb{\hat j}"]
        &\H^l(\Sphere k\times X)
        \\
        \Splitlineoplus{1\otimes\H^{l-1}(X^2)}{s_k\otimes\H^{l-k-1}(X^2)}
        \ar[u, equals]
        &
        \Splitlineoplus{1\otimes\H^{l-1}(X^2)}{s_{k-1}\otimes\H^{l-k}(X^2)}
        \ar[u, equals]
        \ar[ddddd, equals]
        &
        \Splitlineoplus{s_k\otimes\H^{l-k}(X^2)}{s_k\otimes\H^{l-k}(X^2)}
        \ar[u, equals]
        \\
        \argument{(1\otimes f, s_k\otimes g)}
        \ar[dd, mapsto, "\pb{\iota}"]
        &&\argument{(s_k\otimes f, s_k\otimes g)}
        \ar[dd, mapsto, "\pb{\tau}"]
        \\~\\
        \argument{f}
        && \argument{s_k\otimes f}
        \\
        \H^{l-1}(X^2)
        \ar[d, equals]
        &
        &
        s_k\otimes\H^{l-k}(X^2)
        \ar[d, equals]
        \\
        \H^{l-1}(\Disk k\times X^2)
        \ar[r, "\pb{\tilde i}"]
        &\H^{l-1}(\Sphere{k-1}\times X^2)
        \ar[r, "\tilde{\delta}"]
        &\H^l((\Disk{k},\Sphere{k-1})\times X^2)
        \ar[r, "\pb{\tilde j}", "=0"{below}]
        &\H^l(\Disk k\times X)
        \\
        \argument{f}
        \ar[r, mapsto]
        &\argument{(1\otimes f, 0)}
        \\
        &\argument{(1\otimes f,s_{k-1}\otimes g)}
        \ar[r, mapsto]
        &\argument{s_k\otimes g}
      \end{tikzcd}
    \end{center}
    For $k=1$ one has the modifications
    \begin{center}
      \begin{tikzcd}[row sep=tiny]
        &(1\otimes f, s_0\otimes g)
        \ar[r, mapsto, "\hat{\delta}"]
        &(s_1\otimes(f+g), s_1\otimes(f+g))
        \\
        (1\otimes f, s_1\otimes g)
        \ar[r, mapsto, "\pb{\hat i}"]
        &(1\otimes f, s_0\otimes f)
        \\
        f
        \ar[r, mapsto, "\pb{\tilde i}"]
        &(1\otimes f, s_0\otimes f)
        \\
        &(1\otimes f, s_0\otimes g)
        \ar[r, mapsto, "\tilde{\delta}"]
        &s_1\otimes(f+g)
      \end{tikzcd}
    \end{center}
  \item
    Furthermore, it is easily seen that
    \begin{align*}
      \pb\pi\colon
      \H^l((\Disk k, \Sphere{k-1})\times X^2)
      &\to \H^l((\Sphere k, \Sphere{k-1})\times X^2)
      \\
      s_k\otimes x\otimes y
      &\mapsto
        (s_k\otimes x\otimes y, s_k\otimes y\otimes x)
    \end{align*}
  \item 
    It is known respectively easily seen that
    \begin{center}
      \begin{tikzcd}[row sep=small, column sep=small]
        &x^k \ar[r, mapsto]
        \ar[ddl, mapsto, bend right=30]
        &x^k,\quad
        x \ar[r, mapsto]
        &x
        \\
        &\H^*(\RP k, \RP k-1)
        \ar[r]
        \ar[d, "\pb\proj"{near start}]
        &\H^*(\RP k)
        \ar[r]
        &\H^*(\RP{k-1})
        \\
        s_k\otimes 1\otimes 1
        &\H^*((\Disk k, \Sphere{k-1})\times X^2)
      \end{tikzcd}
    \end{center}
  \end{enumerate}
\end{Fact}

Some further immediate results are:
\begin{Cor}\label{cor:twistedprodcohom:prelim}
  Let $X$ be a space, $k\in\Nat$,
  $a,b\in\H^*(X)$, $f\in\H^*(X^2)$, and
  $u\in\H^*(\Twistedprod k X)$.
  \begin{enumerate}
  \item\label{item:twistedprodcohom:prelim:pij} 
    $\pb\pi\pb j
    = \pb{\hat j}\pb\pi\colon
    s_k\otimes(a\otimes b)\mapsto s_k\otimes(a\otimes b+b\otimes a)$
  \item $\delta = (\pb\tau\pb\pi)^{-1}\tilde\delta\pb\pi = \tilde\delta\pb\pi$
  \item $\pb j(s_k\otimes 1\otimes 1) = c^k \coloneqq \pb\proj(x^k)$
  \item $\pb i(c_k) = \pb i(c_{k-1})$
  \item\label{item:twistedprodcohom:prelim:j}
    $\pb j(s_k\otimes f)\cdot u
    = \pb j\left(s_k\otimes(f\cdot \pb\iota\pb\pi(u))\right)$
  \end{enumerate}
  \begin{proof}
    For \ref{item:twistedprodcohom:prelim:j} first observe that the
    ring
    \begin{gather*}
      \H^*(\Twistedprod k X, \Twistedprod{k-1} X)
      = \H^*((\Disk k,\Sphere{k-1})\times X^2)
      = s_k\otimes\H^*(X^2)
    \end{gather*}
    is both a module over $\H^*(\Twistedprod k X)$ and over
    $\H^*(\Disk k\times X^2)=\H^*(X^2)$. It is known that the
    $\H^*(X^2)$-module structure looks like
    \begin{gather*}
      (s_k\otimes a_1\otimes b_1)\cdot (a_2\otimes b_2)
      = s_k\otimes\big((a_1\otimes b_1)\cdot(a_2\otimes b_2)\big)
      = s_k\otimes(a_1a_2)\otimes(b_1b_2)
      \;.
    \end{gather*}
    Now the base change between the two module structures is the one
    along $\pb\iota\pb\pi$, and this means that
    for any $u\in\H^*(\Twistedprod k X)$
    and
    $s_k\otimes f \in s_k\otimes\H^*(X^2)$
    \begin{gather*}
      (s_k\otimes f)\cdot u
      = (s_k\otimes f)\cdot\pb\iota\pb\pi(u)
      = s_k\otimes(f\cdot \pb\iota\pb\pi(u))
      \;.
    \end{gather*}
    As a last step note that
    \begin{gather*}
      \pb j\colon
      \H^*(\Twistedprod k X, \Twistedprod{k-1} X)
      \to \H^*(\Twistedprod k X)
    \end{gather*}
    is a morphism of $\H^*(\Twistedprod k X)$-modules, \idest
    \begin{gather*}
      \pb j(f\cdot g) = \pb j(f)\cdot g
      \qquad\text{for
        $f\in\H^*(\Twistedprod k X, \Twistedprod{k-1} X)$ and
        $g\in\H^*(\Twistedprod k X)$.}
      \qedhere
    \end{gather*}
  \end{proof}
\end{Cor}

We are going to inductively prove the following reformulation of
Theorem~\ref{thm:twistedprod:cohomstructure}.
\begin{Thm}\label{thm:twistedprod:cohomstructure:alt}
  For a space $X$ and $k\in\Nat$, $\H^*(\Twistedprod k X)$ is the group
  \begin{align*}
    \H^*(\Twistedprod{k}{X})
    &\cong
      \bc k\otimes D
      \oplus \left(\bigoplus_{i=1}^{k-1} \bc i\otimes D\right)
      \oplus (1\otimes(N+D))
      \oplus (s\otimes N)
    \\
    &\cong
      \bc k\otimes D
      \oplus \left(\bigoplus_{i=1}^{k-1} \bc i\otimes D\right)
      \oplus (1\otimes D)
      \oplus (1\otimes N)
      \oplus (s\otimes N)
      % \\
      % &\cong
      % \left(\H^*(\RP k) \otimes D \right)
      % \oplus \left(\H^*(\Sphere k) \otimes N\right)
  \end{align*}
  with the additive relation
  $\bc i\otimes d(a) + \bc i\otimes d(b)=\bc i\otimes d(a+b)$ for
  $a,b\in\H^*(X)$ and $1\leq i\leq k$,
  which is equipped with a multiplication determined by:
  \begin{enumerate}
  \item\label{twistedprodcohom:proof:0}
    The following restriction of $\pb\pi$ is a ring isomorphism onto
    its image
    \begin{align*}
      \pb\pi\colon
      (1\otimes D)\oplus (1\otimes N)\oplus(s_k \otimes N)
      &\longto
        (1\otimes(D+N))\oplus(s_k\otimes N)
        \subset\H^*(\Sphere k\times X^2)
      \\
      1\otimes d + 1\otimes n_1 + s\otimes n_2
      &\longmapsto
        1\otimes(d+n_1) + s_k\otimes n_2
        \;.
    \end{align*}
  \item\label{twistedprodcohom:proof:1}
    $(\bc i\otimes d(a))\cdot (\bc j\otimes d(b))
    = \bc{i+j}\otimes d(ab)$ for $1\leq i,j,i+j\leq k-1$,
    so $\bc i\otimes d(1) = (\bc 1\otimes d(1))^i$.
  \item\label{twistedprodcohom:proof:2}
    $(\bc i\otimes d(a))\cdot(1\otimes d(b))
    = \bc i\otimes d(ab)$ for $1\leq i\leq k-1$,
    so $\bc i\otimes D = (\bc 1\otimes 1)^i\cdot (1\otimes D)$.
  \item\label{twistedprodcohom:proof:3}
    $(\bc k\otimes d(a))\cdot(1\otimes d(b))
    = \bc k\otimes d(ab)$,
    so $\bc k\otimes D = (\bc 1\otimes 1)^i\cdot (1\otimes D)$.
  \item\label{twistedprodcohom:proof:4}
    $\pb\proj(x^k) = \bc k\otimes d(1)$ and
    $\pb\proj(x) = \bc 1\otimes d(1)$,
    so $(\bc 1\otimes d(1))^k = \bc k\otimes d(1)$ and
    \begin{gather*}
      \ker\pb\pi
      = (\bc k\otimes D)
      \oplus \left( \bigoplus_{i=1}^{k-1}\bc i\otimes D \right)
      = \sum_{i=1}^k c^i\cdot (1\otimes D)
      \;.
    \end{gather*}
  \item\label{twistedprodcohom:proof:5}
    $c\cdot (1\otimes N + s\otimes N) = 0$,
    so $(\bc i \otimes D)\cdot (1\otimes N + s\otimes N) = 0$
    for $1\leq i\leq k$.
  \end{enumerate}
\end{Thm}
\begin{Rem}
  Note that the demanded multiplication properties of the
  $\Zmod2$-vector space in the theorem do already fully qualify a
  multiplication, as all cases of combinations of components are
  covered ($1\leq i,j\leq k$):
  \begin{description}
  \item[$(1\otimes(D+N)\oplus s\otimes N)\cdot (1\otimes(D+N)\oplus s\otimes N)$:]
    \ref{twistedprodcohom:proof:0}
  \item[$(\bc i\otimes D)\cdot (\bc j\otimes D)$:]
    By \ref{twistedprodcohom:proof:1} and \ref{twistedprodcohom:proof:2}
    $(\bc i\otimes D)=(\bc 1\otimes 1)\cdot (1\otimes D)$,
    which can be simplified with
    $(\bc 1\otimes 1)=c_k\coloneqq \pb\proj x_k$
    from \ref{twistedprodcohom:proof:4}
    to $(\bc i\otimes D)=c^i\cdot(1\otimes D)$.
    Analogously with \ref{twistedprodcohom:proof:3}
    and \ref{twistedprodcohom:proof:4}
    $(\bc k\otimes D) = c^k\cdot(1\otimes D)$.
  \item[$(\bc i\otimes D)\cdot (1\otimes D)$:]
    \ref{twistedprodcohom:proof:2} resp. \ref{twistedprodcohom:proof:3}
    for the case $i=k$
  \item[$(\bc i\otimes D)\cdot (1\otimes N) = 0$:]
    \ref{twistedprodcohom:proof:5}
  \item[$(\bc i\otimes D)\cdot (s\otimes N) = 0$:]
    \ref{twistedprodcohom:proof:5}
  \end{description}
\end{Rem}
\begin{proof}[proof of
  Theorem~\ref{thm:twistedprod:cohomstructure:alt}
  respectively Theorem~\ref{thm:twistedprod:cohomstructure}]
  This is roughly geared to the proof in \cite[Theorem~7.1]{brown}.
  With the above reformulation, the proof is a straight forward
  induction on $k$.
  \Idest split up the $\Zmod2$-vector space $\H^*(\Twistedprod k X)$
  into direct summands of the form in the theorem, which are then
  either known from the case $k-1$, or calculable, and then check the
  needed multiplication and isomorphism properties.
  The split looks as follows:
  \begin{align*}
    \Im\pb j
    &\cong \ker(\pb\pi|_{\Im\pb j})
      \oplus \left(\Im\pb j/\ker(\pb\pi|_{\Im\pb j})\right) \\
    &\cong \pb j(\ker(\pb\pi\pb j))
      \oplus \Im(\pb\pi|_{\Im\pb j}) \\
    &\cong \pb j(\ker(\pb\pi\pb j))
      \oplus \Im(\pb\pi\pb j) \\
    \Im\pb i
    &= \ker\delta = \ker(\tilde\delta\pb\pi_{k-1})\\
    &= \ker(\tilde\delta|_{\Im\pb\pi_{k-1}})
      \oplus \ker\pb\pi_{k-1} \\
    \H^*(\Twistedprod k X)
    &\cong \Im\pb j
      \oplus \left( \H^*(\Twistedprod k X)/\Im\pb j \right)\\
    &= \Im\pb j
      \oplus \left( \H^*(\Twistedprod k X)/\ker\pb i \right)\\
    &\cong \Im\pb j \oplus \Im\pb i\\
    &\cong \pb j(\ker(\pb\pi\pb j))
      \oplus \Im(\pb\pi\pb j)
      \oplus \ker(\tilde\delta|_{\Im\pb\pi_{k-1}})
      \oplus \ker\pb\pi_{k-1}
  \end{align*}
  In the end this is supposed to look like
  \begin{align*}
    \pb j(\ker(\pb\pi\pb j))
    &= c_k^k\cdot(\pb\pi)^{-1}(1\otimes D)\cong c^k\otimes D
    \\
    \ker\pb\pi_{k-1}
    &= {\textstyle\bigoplus_{i=1}^{k-1}} c_{k-1}^i\otimes D
    \\
    \ker(\tilde\delta|_{\Im\pb\pi_{k-1}})
    &= 1\otimes(N+D)
    \\
    \Im(\pb\pi\pb j) &= s_k\otimes N
  \end{align*}
  so the correspondents to the symbols from the theorem's notation are
  $\bc i = c_{k-1}^i$ for $i\leq k-1$, $\bc k = c_k^k$, $s=s_k$.

  \minisec{Induction step}
  We will start with the induction step, so assume for a space $X$ and
  $k>1$ that Theorem~\ref{thm:twistedprod:cohomstructure:alt} holds in
  the case $k-1$.
  Begin with identifying the direct summands of the vector space,
  carefully tracking where the induction assumption
  \begin{description}
  \item[$\Im(\pb\pi\pb j)=s_k\otimes N$:]
    Recall from
    Corollary~\itemref{cor:twistedprodcohom:prelim}{item:twistedprodcohom:prelim:pij}
    that
    \begin{align*}
      \pb\pi\pb j\colon
      s_k\otimes\H^*(X^2)
      &\to 1\otimes\H^*(X^2)\oplus s_k\otimes\H^*(X^2)\\
      s_k\otimes(a\otimes b)
      &\mapsto s_k\otimes(a\otimes b+b\otimes a)
    \end{align*}
    and hence $\Im(\pb\pi\pb j)=s_k\otimes N$.
    Obviously $\pb\pi$ maps this part of $\H^*(\Twistedprod k X)$
    isomorphically to $s_k\otimes N\subset\H^*(\Sphere k\times X^2)$,
    already inducing multiplication.
  \item[$\ker(\tilde\delta|_{\Im\pb\pi_{k-1}})\cong 1\otimes (D+N)$:]
    By the assumption on $k-1$
    \begin{gather*}
      \Im(\pb\pi_{k-1})
      =(1\otimes(D+N))\oplus (s_k\otimes N)
      \in\H^*(\Sphere{k-1}\times X^2)
      \;.
    \end{gather*}
    With $\ker(\tilde\delta) = 1\otimes\H^*(X^2)$, one gets
    $\ker(\tilde\delta|_{\Im\pb\pi_{k-1}}) = 1\otimes (D+N)$.
    Since $\pb\pi_{k-1}$ is the identity on
    $1\otimes(D+N)\in\H^*(\Twistedprod{k-1} X)$ and $\pb{\tilde i}$ is
    the identity on $1\otimes(D+N)\in\H^*(\Sphere k\times X^2)$,
    $\pb\pi_{k}$ and $\pb i$ are both the identity on
    $\ker(\tilde\delta|_{\Im\pb\pi_{k-1}})$, inducing ring structure
    on this subring.
  \item[$\ker(\pb\pi_{k-1})\cong \bigoplus_{i=1}^{k-1} c_{k-1}^i\otimes D$:]
    This holds by the induction assumption.
    Note that by construction $\pb i$ is injective on this direct
    summand which is a direct summand of $\Im\pb i$,
    hence the ring structure is inherited.
    In order to see that this summand lies in the kernel of
    $\pb\pi_k$, one has to first see that
    $c_k=c_{k-1}\otimes d(1)\in\H^*(\Twistedprod k X)$, respectively
    in the direct sum notation
    \begin{gather*}
      c_k = f + 1\otimes d(a) + 1\otimes n + s\otimes n'
      + \sum_{i=1}^{k-1} c_{k-1}^i\otimes d(a_i)
    \end{gather*}
    all summands are zero except for $c_{k-1}\otimes d(1)$.
    So check all possible components:
    \begin{itemize}
    \item $f\in\pb j(\ker\pb\pi\pb j)$ has degree greater $k$ as will
      be seen below, so with $k>1$ from the induction assumption has
      to be zero.
    \item $\pb i(c_k) = c_{k-1}$, so $1\otimes d(a)+1\otimes n=0$.
    \item $\pb\pi(c_k)= 0$, so $s\otimes n'=0$.
    \end{itemize}
    Thus
    $\ker(\pb\pi_{k-1})
    \cong \sum_{k=1}^{k-1}c_k^i\cdot (1\otimes D)
    \subset\ker\pb\pi$.
  \item[$\pb j(\ker(\pb\pi\pb j))\cong c_k^k\otimes D$:]
    Again recall
    Corollary~\itemref{cor:twistedprodcohom:prelim}{item:twistedprodcohom:prelim:pij}
    to directly obtain
    $\ker(\pb\pi\pb j) = s_k\otimes(N+D)$.
    So one has in general
    \begin{align*}
      \pb j(\ker(\pb\pi\pb j))
      &= \pb j(s_k\otimes(N+D)) \\
      &\equalsby{Corollary~\itemref{cor:twistedprodcohom:prelim}{item:twistedprodcohom:prelim:j}}
        \pb j(s_k\otimes 1\otimes 1) \cdot (\pb\iota\pb\pi)^{-1}(N+D) \\
      &= c^k \cdot \left( (\pb\pi)^{-1}(1\otimes (D+N))\right)
        \subset \ker\pb\pi
        \;.
    \end{align*}
    Note that this lives, as was used above, in degree greater $k-1$.
    Since by
    Corollary~\itemref{cor:twistedprodcohom:prelim}{item:twistedprodcohom:prelim:pij}
    $c^k\cdot \ker\pb\pi=\pb
    j(s_k\otimes\pb\iota\pb\pi(\ker\pb\pi))=\pb j(s_k\otimes 0)=0$
    and the effect of $\pb\pi$ on all direct summands is known:
    \begin{align*}
      \pb j(\ker(\pb\pi\pb j))
      &= c^k \cdot \left( (\pb\pi)^{-1}(1\otimes (D+N))\right) \\
      &= c^k \cdot \left( \ker\pb\pi + (1\otimes (D+N))\right) \\
      &= c^k \cdot (1\otimes (D+N))
    \end{align*}
    The next step is to use that by induction assumption and the form
    of $\tilde\delta$ holds
    \begin{gather*}
      \ker\pb j=\Im\delta=\Im\tilde\delta\pb\pi = s_k\otimes N
    \end{gather*}
    hence $c^k\cdot (\pb\pi)^{-1}(1\otimes N) = \pb j(s_k\otimes N)=0$,
    $c^k\cdot -$ is injective on the set $(1\otimes D)$, and
    \begin{align*}
      c^k\cdot(1\otimes d(a)) + c^k \cdot(1\otimes d(b))
      &= c^k\cdot \left( 1\otimes (d(a)+d(b)) \right) \\
      &= c^k\cdot \left( 1\otimes
        (d(a+b) + a\otimes b + b\otimes a)
        \right) \\
      &= c^k\cdot (1\otimes d(a+b)) + c^k\cdot(a\otimes b + b\otimes a)\\
      &= c^k\cdot (1\otimes d(a+b))
        \;.
    \end{align*}
    Altogether, this yields the desired isomorphism
    \begin{gather*}
      \pb j(\ker(\pb\pi\pb j))
      = c_k^k\cdot(\pb\pi)^{-1}(1\otimes D)\cong c^k\otimes D
      \;.
    \end{gather*}
  \end{description}
  Now it remains to check the multiplication properties:
  \begin{description}
  \item[\ref{twistedprodcohom:proof:0}:]
    Checked during construction.
  \item[\ref{twistedprodcohom:proof:1}, \ref{twistedprodcohom:proof:2}:]
    These are the multiplication properties induced by $\pb i$ being
    injective on the corresponding summands, thus inheriting the
    multiplication properties from the induction assumption.
  \item[\ref{twistedprodcohom:proof:3}:]
    It was shown that $c^k\otimes D\cong c^k\cdot (1\otimes D)$.
  \item[\ref{twistedprodcohom:proof:4}:]
    This was shown during the identification of $\ker(\pb\pi_{k-1})$
    as part of $\ker\pb\pi$.
  \item[\ref{twistedprodcohom:proof:5}:]
    $c_k\cdot(1\otimes N)$ is inherited via $\pb i$ from the assumed
    structure of $\H^*(\Twistedprod{k-1} X)$ using
    $c_k = c_{k-1}\otimes d(1)$.
    The fact $c_k\cdot(s_k\otimes N)=0$ follows from
    \begin{gather*}
      \ker\pb\pi\pb j\cdot\Im\pb j
      \subset \pb j(s_k\otimes(\H^*(X)\cdot \pb\iota\pb\pi(\ker\pb\pi)))
      = \pb j(s_k\otimes 0) = 0
    \end{gather*}
    using $c_k\in\ker\pb\pi$, and $(s_k\otimes N)\subset\Im\pb j$.
  \end{description}
  
  \minisec{Induction Start}
  The case $k=0$ is a bit simpler as
  $\Twistedprod 0 X=X^2$. One still has
  $\Im\pb\pi\pb j=s_1\otimes N$, and, using the known formula
  \begin{gather*}
    \pb\pi(a\otimes b)=(1\otimes a\otimes b, s_0\otimes b\otimes a)
    \in \H^*(S^0\times X^2)=\H^*(X^2)^2
    \;,
  \end{gather*}
  one gets
  $\delta(a\otimes b)
  = \tilde\delta\pb\pi(a\otimes b)
  = s_1\otimes(a\otimes b + b\otimes a)$, giving
  $\ker\delta=N+D$, and $\Im\delta = \ker\pb j = s_1\otimes N$.
  Hence, analogously to above, one obtains
  $\pb j (\ker(\pb\pi\pb j)) \cong c\otimes D$ and
  $c\cdot (1\otimes N + s_1\otimes N) = 0$.
\end{proof}


\subsection{Indecomposability of Twisted Product Cobordism Classes}
\label{sec:twistedprod:indecompcriterion}
Now that the cohomology ring and Stiefel-Whitney classes of line
bundles of a twisted product are well-known, one can investigate
general Stiefel-Whitney numbers of twisted product manifolds.
As promised, the following result will be the cornerstone when inductively
defining the desired basis for the cobordism ring in the subsequent section.
\begin{Thm}\label{thm:twistedprod:indecompcriterion}
  Let $M^n$ be a manifold and $k\in\Nat_{\geq1}$.
  Then $\Twistedprod{k}{M}$ represents an indecomposable class of the
  cobordism ring if and only if $M$ does and $\binom{k+n-1}{n}$ is
  non-zero modulo two.
\end{Thm}
Recall Theorem~\ref{thm:indecomposabilitycriterion}, saying
a manifold $M^n$ represents an indecomposable element if and only if
$\snum{(n)}{M}$ is non-zero,
and recall that $\Twistedprod{k}{-}$ preserves injectivity on
cohomology by
Theorem~\itemref{thm:twistedprod:cohomstructure}{item:twistedprod:preservescohominj}.
Then Theorem~\ref{thm:twistedprod:indecompcriterion} is a
direct consequence of the following Lemma.
\begin{Lem}\label{lem:twistedprod:indecompcriterion}
  Let $M$, $n$, $k$ be as in
  Theorem~\ref{thm:twistedprod:indecompcriterion} above.
  Then there is a map $f\colon X\to M$ of spaces which is injective on
  cohomology and fulfills
  \begin{gather*}
    \pb{\Twistedprod{k}{f}} \s{(2n+k)}(\Twistedprod{k}{M})
    = \binom{k+n-1}{n} \cdot c^k
    \cdot d\left( \pb f \s{(n)}(M) \right)
    \in \H^{2n+k}(\Twistedprod{k}{X})
    \;.
  \end{gather*}
\end{Lem}
\begin{proof}[proof of Lemma~\ref{lem:twistedprod:indecompcriterion}]
  Take $f$ to be a reduction of $\T M$ to line bundles
  using the splitting principle~\cite[Theorem~(19.3.9)]{tomdieck}.
  \Idest choose a space $X$ and a map $f\colon X\to M$ which is
  injective on cohomology and fulfills
  $\pb f \T M = \xi_1\oplus\dotsb\oplus\xi_n$ for line bundles
  $\xi_i$ over $X$, each with total Stiefel-Whitney class
  $\W{\xi_i}=1+\alpha_i$.
  With the fiber bundle properties from
  Remark~\itemref{rem:twistedprodproperties}{item:twistedprodfiberbdl}%
  \ref{item:twistedprod:preservespb}
  and the tangent space structure from
  Remark~\itemref{rem:twistedprodproperties}{item:twistedprodmanifold}%
  \ref{item:twistedprod:tangentspace}
  this yields on vector bundles:
  \begin{align*}
    \pb{\Twistedprod{k}{f}} \Twistedprod{k}{\T M}
    &\cequalsby{\ref{rem:twistedprodproperties}}
      \Twistedprod{k}{\pb f \T M}
      = \Twistedprod{k}{\bigoplus_{i\leq n}\xi_i}
      = \bigoplus_{i\leq n}\Twistedprod{k}{\xi_i}
    \\
    \pb{\Twistedprod{k}{f}} \T{\Twistedprod{k}{M}}
    &\cequalsby{\ref{rem:twistedprodproperties}}
      \pb{\Twistedprod{k}{f}} \left(
      \T{\RP k} \oplus \Twistedprod{k}{\T M}
      \right)\\
    &= \pb{\Twistedprod{k}{f}} \T{\RP k}
      \oplus
      \pb{\Twistedprod{k}{f}} \Twistedprod{k}{\T M}
      = \T{\RP k} \oplus \bigoplus_{i\leq n}\Twistedprod{k}{\xi_i}\\
    \intertext{And on Stiefel-Whitney classes:}
    \pb{\Twistedprod{k}{f}} \W{\Twistedprod{k}{\T M}}
    &= \prod_{i\leq n} \W{\Twistedprod{k}{\xi_i}}
      \cequalsby{\ref{cor:twistedprod:swlinebdl}}
      \prod_{i\leq n} \left(1 + c + e(\alpha_i) + d(\alpha_i)\right)
    \\
    \pb{\Twistedprod{k}{f}} \W{\T\Twistedprod{k}{M}}
    &= \pb{\Twistedprod{k}{f}}\W{\T{\RP k}}
      \cdot \prod_{i\leq n} \W{\Twistedprod{k}{\xi_i}} \\
    &= (c+1)^{k+1}
      \cdot \prod_{i\leq n} \left(1+c+e(\alpha_i)+d(\alpha_i)\right)
  \end{align*}
  In order to work with these Stiefel-Whitney class expressions as
  symmetric polynomials, introduce variables $u_i, v_i$ of degree
  one such that
  \begin{align*}
    \w1{\xi_i} &= c+e(\alpha_i) = \el 1 2(u_i, v_i) = u_i+v_i \\
    \w2{\xi_i} &= d(\alpha_i)   = \el 2 2(u_i, v_i) = u_iv_i \\
    \W{\xi_i}  &= 1+c+e(\alpha_i)+d(\alpha_i)
                 = 1+\sum_{i\leq2}\el i 2(u_i,v_i) = (1+u_i)(1+v_i)
                 \;.
  \end{align*}
  The key point now qualifying $f$ for the proof, is that both $f$ and
  thus by
  Theorem~\itemref{thm:twistedprod:cohomstructure}{item:twistedprod:preservescohominj}
  also $\Twistedprod{k}{f}$ are injective on cohomology, and fulfill
  \begin{enumerate}
  \item with $\W{\pb f\T M} = \prod_{i=1}^n (1+\alpha_i)$ that
    \begin{gather}\label{eq:lem:twistedprod:indecompcriterion:sM}
      \pb f \s{(n)}(M)
      = \s{(n)}(\W{\pb f\T M})
      = \sum_{i=1}^n \alpha_i^n
      \;
    \end{gather}
  \item and with
    $\W{\pb f\T{\Twistedprod{k}{M}}}
    = \prod_{i=1}^{k+1}(1+c) \cdot \prod_{i=1}^n
    (1+u_i)(1+v_i)$
    that
    \begin{align}\notag
      \pb{\Twistedprod{k}{f}} \s{(2n+k)}(\Twistedprod{k}{M})
      &= \s{(2n+k)}(\W{\pb f\T{\Twistedprod{k}{M}}}) \\
      \label{eq:lem:twistedprod:indecompcriterion:sDkM}
      &= (k+1)c^{2n+k} + \sum_{i=1}^n (u_i^{2n+k} + v_i^{2n+k})
        \;.
    \end{align}
  \end{enumerate}
  In order to formulate
  $\pb{\Twistedprod{k}{f}}\s{(2n+k)}(\Twistedprod{k}{M})$ 
  in terms of $c$ and $d(\alpha_i)$,
  use one of the Newton-Girard formulas saying:
  \begin{Lem}[Newton-Girard]
    For integers $l, m\in\Nat$ holds
    \begin{align*}
      \symm{l} t^m
      &= \sum_{\mathclap{r_1+2r_2+\dotsb+mr_=m}}
        (-1)^m \frac{m\cdot(r_1+\dotsb+r_m-1)!}{(r_1)!\dotsm(r_m)!}
        \cdot \prod_{i=1}^m (-\el i l)^{r_i}
        \;.
    \end{align*}
    As a special case for $l=2$ this becomes modulo 2
    \begin{align*}
      t_1^m + t_2^m
      &= \sum_{\mathclap{r_1+2r_2=m}}
        \frac{m\cdot(r_1+r_2-1)!}{(r_1)!(r_2)!}
        \cdot (\el 1 2)^{r_1} \cdot (\el 2 2)^{r_2}
        = \sum_{\mathclap{r_1+2r_2=m}}
        \altbinom{r_1-1}{r_2} (t_1+t_2)^{r_1}(t_1t_2)^{r_2}
    \end{align*}
    where $\frac{(r_1+2r_2)(r_1+r_2-1)!}{(r_1)!(r_2)!}
    = \binom{r_1+r_2-1}{r_2} + 2\binom{r_1+r_2-1}{r_1}$
    and the notation $\altbinom{p}{q}\coloneqq \binom{p+q}{q}$
    was used.
    \begin{proof}
      A proof of the main Newton-Girard formula can be found in
      \cite[Theorem~10.12.2]{raymond}. This special case can
      immediately be obtained by iterated application of the formula.
    \end{proof}
  \end{Lem}
  Before applying this to $t_1=u_i$, $t_2=v_i$, $m=2n+k$, and $l=2$,
  recall the following special properties from
  Theorem~\ref{thm:twistedprod:cohomstructure} of the mentioned symbols in
  $\H^*(\Twistedprod{k}{X})$:
  \begin{itemize}
  \item $c^{k+1}=0$,
  \item $c\cdot e(\alpha_i)=0$, and
  \item $e(\alpha_i)^{r_1}d(\alpha_i)^{r_2} = 0$ for $r_1+2r_2>2n$,
    as $\H^{*>2n}(M\times M)=0$.
  \end{itemize}
  Then simplify in terms of $c$, $e(\alpha_i)$, $d(\alpha_i)$:
  \begin{align}\notag
    u_i^{2n+k} + v_i^{2n+k}
    &=
      \sum_{{r_1+2r_2=2n+k}}
      \altbinom{r_1-1}{r_2} (u_i+v_i)^{r_1}(u_iv_i)^{r_2} \\\notag
    &\equalsby{Def.}
      \sum_{{r_1+2r_2=2n+k}}
      \altbinom{r_1-1}{r_2}
      (c+e(\alpha_i))^{r_1}d(\alpha_i)^{r_2} \\\notag
    &\equalsby{$c\cdot e(\alpha_i)=0$}
      \sum_{{r_1+2r_2=2n+k}} \altbinom{r_1-1}{r_2}
      \left(c^{r_1}d(\alpha_i)^{r_2}+e(\alpha_i)^{r_1}d(\alpha_i)^{r_2}\right) \\\notag
    &\equalsby{$c^{k+1}=0$, $d(\alpha_i)^{n+s}=0$}
      \altbinom{k-1}{n} c^k d(\alpha_i)^n
      + \sum_{\mathclap{r_1+2r_2=2n+k}} \altbinom{r_1-1}{r_2}
      e(\alpha_i)^{r_1}d(\alpha_i)^{r_2} \\
    \label{eq:lem:twistedprod:indecompcriterion:1}
    &= \altbinom{k-1}{n} c^k d(\alpha_i)^n      
  \end{align}
  Altogether it turns out that
  \begin{align*}
    \pb{\Twistedprod{k}{f}}\s{(2n+k)}(\Twistedprod{k}{M})
    &\cequalsby{\eqref{eq:lem:twistedprod:indecompcriterion:sDkM}}
      (k+1)c^{2n+k} + \sum_{i=1}^n (u_i^{2n+k} + v_i^{2n+k}) \\
    &\equalsby{$c^{k+1}=0$}
      \sum_{i=1}^n (u_i^{2n+k} + v_i^{2n+k}) \\
    &\equalsby{\eqref{eq:lem:twistedprod:indecompcriterion:1}}
      \sum_{i=1}^n \altbinom{k-1}{n} c^k d(\alpha_i)^n \\
    &= \altbinom{k-1}{n} c^k d(\sum_{i=1}^n \alpha_i^n) \\
    &\equalsby{\eqref{eq:lem:twistedprod:indecompcriterion:sM}}
      \altbinom{k-1}{n} c^k d(\pb f \s{(n)}(M))
  \end{align*}
  as was stated.
\end{proof}

\begin{Rem}\label{rem:splitdim}
  By Theorem~\ref{thm:twistedprod:indecompcriterion}, one has to
  both choose
  \begin{itemize}
  \item the dimension combination, as well as
  \item the factor (see later)
  \end{itemize}
  of a twisted product correctly, in order to obtain the representative
  of an indecomposable cobordism class.
  For the correct dimension choice note that any integer $m$ with
  binary expansion
  \begin{gather*}
    m=2^{i_0}+\dotsb+2^{i_l}
    \;,\qquad
    0 \leq i_1<\dotsb<i_r
  \end{gather*}
  can be split at any splitting point $0\leq l_k\leq l$ to
  \begin{gather*}
    m = \underbrace{\left(
        \sum_{r=0}^{l_k} 2^{i_r}
      \right)}_{\eqqcolon k}
    + 2\cdot\underbrace{\left(
        \sum_{r=l_k+1}^l 2^{i_r-1}
      \right)}_{\eqqcolon n}
    \eqqcolon k+2n
    \;,
  \end{gather*}
  where:
  \begin{itemize}
  \item $\alpha(m)=\alpha(k)+\alpha(n)$.
  \item For $m$ not of the form $2^s-1$, $n$ will also not be of the
    form $2^s-1$.
  \end{itemize}
\end{Rem}
Such splits as described above are desirable because the binomial
coefficient $\binom{k+n-1}{n}$ will never be zero modulo two by the
following Lemma, 
as is required by the indecomposability criterion for twisted products
in Theorem~\ref{thm:twistedprod:indecompcriterion}.
\begin{Lem}\label{lem:lucas}
  For $a,b\in\Nat$, $\binom{a+b}{b}$ will be non-zero modulo 2 if and only
  if $\alpha(a+b)=\alpha(a)+\alpha(b)$.
  \begin{proof}
    This is a direct consequence of Lucas' well-known theorem
    which states
    \begin{gather*}
      \binom{a+b}{b} \equiv \prod_{i=0}^s \binom{a_i+b_i}{b_i} \mod 2
    \end{gather*}
    where $a=\sum_{i=1}^s a_i2^{i}$, $b=\sum_{i=1}^s b_i2^{i}$ are the
    binary expansions of $a$ and $b$, and $\binom{0}{1}\coloneqq 0$.
    This expression will be non-zero, if and only if $b_i$ is one for
    any $i$ where $a_i+b_i$ is one. In other words, if and only if
    $\alpha(a+b)=\sum_{i=1}^s(a_i+b_i)$.    
  \end{proof}
\end{Lem}

Therefore:
\begin{Cor}\label{cor:twistedprod:indecompcriterion}
  For $m=k+2n\in\Nat$ as above and $M^n$ an indecomposable manifold,
  the $m$-dimensional twisted product $\Twistedprod{k}{M}$ will be
  indecomposable.
  \begin{proof}
    The previous Lemma~\ref{lem:lucas} can be applied to the
    combination $k-1$, $n$ from above, as one simply calculates
    \begin{align*}
      \alpha(k-1+n)
      &=\textstyle
        \alpha\left(
        \left(\sum_{r=0}^{l_k} 2^{i_r}\right)
        - 1
        + \left(\sum_{r=l_k+1}^{l} 2^{i_r-1}\right)
        \right) \\
      &=\textstyle
        \alpha\left(
        (2^{i_0}-1)
        + \left(\sum_{r=1}^{l_k} 2^{i_r}\right)
        + \left(\sum_{r=l_k+1}^{l} 2^{i_r-1}\right)
        \right) \\
      &=\textstyle
        \alpha\left(
        \left(\sum_{r=0}^{i_0-1} 2^r \right)
        + \left(\sum_{r=1}^{l_k} 2^{i_r}\right)
        + \left(\sum_{r=l_k+1}^{l} 2^{i_r-1}\right)
        \right) \\
      &\textstyle
        \equalsby{$i_0<i_1$}
        \alpha\left(
        \left(\sum_{r=0}^{i_0-1} 2^r \right)
        + \left(\sum_{r=1}^{l_k} 2^{i_r}\right)
        \right)
        + \alpha\left(\sum_{r=l_k+1}^{l} 2^{i_r-1}\right)
      \\
      &= \alpha(k-1) + \alpha(n)
    \end{align*}
  \end{proof}
\end{Cor}


\section
{Brown's Theorem: Finding a Convenient Generating Set}
\label{sec:proofbrown}
Recall that the goal of this chapter is to prove R.~L.~Brown's
theorem \ref{thm:brown} which states that the immersion conjecture is
true up to cobordism.
Also recall, that if one can show that the conjecture is stable
under the ring operations of the cobordism ring
$\c_*\cong\Zmod2[\sigma_i|i\neq2^r-1]$, it suffices to find a
generating set $([\G i]|i\neq2^r-1)$ of $\c_*$ which fulfills the conjecture.
Thus, for the proof one needs to construct a set of manifolds
$(\G i|i\neq2^r-1)$ that both fulfill the conjecture and represent
indecomposable cobordism classes.

The major work required to check indecomposability of manifolds in the
cobordism ring has been done in the previous chapters.
For checking the immersion property of the selected candidates,
several well-known results on immersions of projective spaces
will be used.

Such generators in even dimension can easily be constructed as
codimension-one submanifolds of odd-dimensional products of
real projective spaces. The immersion property can in this case be
checked using results of Sanderson~\cite{sanderson} which say that
odd dimensional real projective spaces fulfill---and partly
overfulfill---the immersion conjecture.

For odd dimensional generators the trick will be to take a twisted
product of an even dimensional one, and use an immersion theorem by
Mahowald and Milgram~\cite{milgram} on the twisted product of an
Euclidean space to conclude the immersion property.

\subsection{Stability Properties of the Conjecture}
The stability properties needed are the following.
\begin{Lem}\label{lem:brownstableunderringops}
  Let $M_i^{n_i}$ be a closed manifold immersing into
  $\R^{2n_i-\alpha(n_i)}$ for $i=1,2$.
  Then both manifolds
  \begin{enumerate}
  \item $M_1\disjointsum M_2$ for $n_1=n_2=n$, and
  \item $M_1\times M_2$ for $n_1$, $n_2$ arbitrary
  \end{enumerate}
  immerse into $\R^{2(n_1+n_2)-\alpha(n_1+n_2)}$.
  \begin{proof}
    $M_1\times M_2$ immerses into the real space of dimension
    \begin{align*}
      \left( 2n_1-\alpha(n_1) \right)
      + \left( 2n_2-\alpha(n_2) \right)
      &= 2(n_1+n_2) - \left(\alpha(n_1)+\alpha(n_2)\right)\\
      &\leq 2(n_1+n_2) - \left(\alpha(n_1 + n_2)\right)
    \end{align*}
    where the inequality is due to the number theoretic fact
    $\alpha(n_1+n_2) \leq \alpha(n_1)+\alpha(n_2)$.

    For $n_1=n=n_2$ the images of the immersions
    \begin{gather*}
      \iota_1\colon M_1\to\R^{2n-\alpha(n)}
      \qquad \text{and} \qquad
      \iota_2\colon M_2\to\R^{2n-\alpha(n)}
    \end{gather*}
    are compact. So by
    concatenation with translation they can be assumed to be disjoint,
    wherefore the disjoint union
    $\iota_1\disjointsum\iota_2\colon M_1\disjointsum M_2\to\R^{2n-\alpha(n)}$
    is again an immersion.
  \end{proof}
\end{Lem}

\subsection{Generating Set}
\subsubsection{Even Dimensional Generators}
The idea in finding even dimensional generators is to use the
following immersion properties of real projective spaces, which have
been investigated in detail throughout the last decades:
\begin{Lem}\label{lem:immersionrealprojspace}
  \begin{enumerate}
  \item
    $\RP{2^i}$ immerses into Euclidean space of dimension
    $2\cdot2^i-1$.
  \item
    $\RP{k}$ immerses into Euclidean space of dimension
    $2k-3$ for $k\geq 5$ odd, \idest overfulfills the immersion
    property in case $\alpha(k)=2$.
  \end{enumerate}
  \begin{proof}
    The first statement is Whitney's immersion
    theorem~\cite{whitneyimmersiontheorem}.
    The second one is a collection of results by Sanderson
    presented in \cite{sanderson}, more precisely it is
    \cite[Theorem~(5.3)]{sanderson} for the case $k>8$, and
    \cite[Theorem~(4.1)]{sanderson} yields immersions
    $\RP 5\immto\R^7$, and $\RP 7\immto\R^8$.
  \end{proof}
\end{Lem}
The trick now will be to utilize the above nice immersion property of
odd dimensional projective spaces. More precisely, the following odd
dimensional product of projective spaces overfulfills the immersion
property by at least three, wherefore all (even-dimensional)
submanifolds of codimension one fulfill the immersion property.
\begin{Def}
  Let $m\in\Nat$ be even and $\alpha(m)>1$ (\idest $m$ is not of the
  form $2^{i}$) with minimal binary expansion 
  $m=2^{i_1}+\dotsb+2^{i_l}$, $k_r\coloneqq 2^{i_r}$.
  Define
  \begin{gather*}
    K^{m+1}\coloneqq \left(\prod_{r=1}^{l-1} \RP{k_r}\right) \times \RP{k_l+1}
  \end{gather*}
  with cohomology ring
  \begin{align*}
    \H^*(K^{m+1})
    &= \left(\bigotimes_{r=1}^{k-1}\H^*(\RP{k_r})\right)
      \otimes \H^*(\RP{k_l+1}) \\
    &= \Zmod2[x_{k_1},\dotsc,x_{k_{l-1}}, x_{k_l+1}]/
      \left( x_{k_1}^{k_1+1},\dotsc,x_{k_{l-1}}^{k_{l-1}+1},
      x_{k_l+1}^{k_l+2} \right)
  \end{align*}
  given by the Künneth isomorphism.
\end{Def}
\begin{Lem}
  For $m$ as above $K^{m+1}$ immerses into $\R^{2m-\alpha(m)}$
  \begin{proof}
    Observe that
    $m=2^{i_1}+\dotsb +2^{i_{\alpha(m)}}$ with $0<i_r< i_{r+1}$ and
    $k_l=2^{i_{\alpha(m)}}\geq 5$ by assumption, so $\RP{k_l+1}$
    immerses into Euclidean space of dimension $2(k_l+1)-3=2k_l-1$ by
    the Lemma.
    Altogether this yields an immersion of $K^{m+1}$ into Euclidean
    space of dimension
  \begin{gather*}
    \sum_{r=1}^{k-1}(2k_r-1) + 2k_l-1
    = 2\left(\sum_{r=1}^{k}\right) - l
    = 2m - \alpha(m)
    \;.
    \qedhere
  \end{gather*}
  \end{proof}
\end{Lem}


In order to define submanifolds of the above $K^{m+1}$ which are
additionally indecomposable in the cobordism ring, recall that the
Steenrod problem on realizing homology classes by submanifolds is
solved for homology classes in degree of codimension one, or more
precisely:
\begin{Lem}
  Let $M^n$ be a manifold. Then for every homology class
  $\alpha\in\H_{n-1}(M)$ there is a submanifold $N$ and an embedding
  $\iota\colon N\immto M$ such that $\pf\iota(\fundcl N) = \alpha$.
  \begin{proof}
    See \cite[Theorem~II.26]{thom}.
  \end{proof}
\end{Lem}

Thus the following candidates for even-dimensional generators are
well-defined.
\begin{Def}
  Let $m$ be even. Define for
  \begin{description}
  \item[$m=\alpha(m)=0$:] $\G m = \pt$.
  \item[$\alpha(m)=1$:] $\G m = \RP m$.
  \item[$\alpha(m)>1$:]
    $\emb\colon\G m\subset K^{m+1}$ is the submanifold of
    codimension one realizing the Poincaré dual of
    $y=\sum_{r=1}^{l-1}x_{i_r} + x_{i_{l+1}}\in\H^1(K^{m+1})$,
    \idest $\pf\emb\fundcl{\G m}=y\cap\fundcl{K^{m+1}}$.
  \end{description}
\end{Def}


\begin{Lem}\label{lem:evengen}
  Let $m=\sum_{r=1}^l k_r$ be even as in the definition above.
  \begin{enumerate}
  \item\label{lem:evengen:immersionprop}
    $\G m$ immerses into $\R^{2m-\alpha(m)}$, and
  \item\label{lem:evengen:indecomposable}
    $\G m$ represents an indecomposable element of the
    cobordism ring.
  \end{enumerate}
\end{Lem}
\begin{proof}[proof of
  Lemma~\ref{lem:evengen}]
  For $m=0$ both claims are trivial.
  
  $\G m$ clearly immerses into $\R^{2m-\alpha(m)}$, either factoring
  over $K^{n+1}$ in case $\alpha(m)>1$ or by Whitney's immersion
  theorem in case $\alpha(m)=1$.

  For the indecomposability we want to use the criterion from
  Theorem~\ref{thm:indecomposabilitycriterion}, \idest one has to show
  that $\snum{m}{\G m}\neq 0$.
  In the case $\alpha(m)=1$ this is Example~\ref{ex:rpnindecomposable}.
  In the other case $\alpha(m)>1$ we first need to identify the
  Stiefel-Whitney numbers of $N$ in degrees greater 0 and express them
  as elementary symmetric polynomials in some variables.
  As one will see the factor $\RP{k_l+1}$ has no special role here,
  and the following holds for any product $\prod_{r=1}^{l}\RP{n_r}$ of
  real projective spaces with
  \begin{enumerate}
  \item $m+1=\sum_{r=1}^l n_r$,
  \item $n_r<m-1$, and
  \item $\alpha(n_i+n_j)=\alpha(n_i)+\alpha(n_j)$
    for $1\leq i<j\leq l$,
    hence $\alpha(n_1+\dotsb+n_r)=\sum_{i=1}^{r}\alpha(n_i)$
    for $2\leq r\leq l$.
  \end{enumerate}
  In our case take $n_l\coloneqq k_l+1$ and 
  $n_r\coloneqq k_r$ for $1\leq k<l$.
  Now, to determine the Stiefel-Whitney numbers 
  denote the normal bundle of the embedding
  $\emb\colon\G m\subset K^{n+1}$ by $\N{}$. 
  As $\T{\G m}\oplus\N{\emb}=\pb\emb\T{K^{m+1}}$ there is the relation
  of Stiefel-Whitney classes
  \begin{align*}
    \W{\T{\G m}} \cdot \W{\N\emb}
    &= \pb\emb\W{\T{K^{m+1}}} \\
    &= \pb\emb\left(
      \W{\RP{k_l+1}}
      \cdot \prod_{r=1}^{l-1}\W{\RP{k_r}}
      \right) \\
    &= \pb\emb\left(
      (1+x_{k_l+1})^{k_l+1}
      \cdot \prod_{r=1}^{l-1}(1+x_{k_r})^{k_r}
      \right)
      \;.
  \end{align*}
  Note that $\N{}$ is a line bundle, so one only has to find out
  $\w1{\N{}}$. For this one needs the following
  generalization of Lemma~\ref{lem:thomisofundcl} which gives the
  highest Stiefel-Whitney class of normal bundles of such embeddings.
  \begin{Lem}
    Let $M^n, W^{n+\rkk}$ be compact manifolds, $\rkk>0$,
    $\emb\colon M\immto W$ be an embedding with corresponding normal
    bundle $\N{\emb}$, $\infty\in W\setminus\emb(N)$, and
    Thom-Pontryagin collapse map on pairs of spaces corresponding to
    some tubular neighborhood embedding of $\N{\emb}$
    \begin{gather*}
      \collapse\colon
      W
      \to W/\left( W\setminus \E{\N{\emb}} \right)
      \cong \Discbdl{\N{\emb}} / \Spherebdl{\N{\emb}}
      \cong \Thomspace{\N{\emb}}
      \xrightarrow{\incl} (\Thomspace{\N\emb},\infty)
      \;.
    \end{gather*}
    Then
    \begin{enumerate}
    \item\label{item:fundclsubmfd}
      $\pf\emb\fundcl M = \pb\collapse\u{\N{\emb}}\cap\fundcl W$,
      \idest one has a description of the Poincaré dual of
      $\pf\emb\fundcl M$, and
    \item\label{item:highestswclasssubmfd}
      $\w{\rk\N{\emb}}{\N{\emb}}
      = \pb\emb\pb\collapse\u{\N{\emb}}$, 
      so the pushforward of the Poincaré dual of $\pf\emb\fundcl N$
      along the embedding is the $\rk\N{}$-Stiefel-Whitney class.
    \end{enumerate}
    \begin{proof}
      See also \cite[p.~371]{bredon}.
      To directly obtain \ref{item:highestswclasssubmfd} directly from
      \ref{item:fundclsubmfd}, one can use the fact that
      the Stiefel-Whitney class $\w{\rk\N{\emb}}{\N{\emb}}$ in degree
      $\rk{\N{\emb}}$ is the Euler class of $\N{\emb}$, which is
      defined as $\pb{(\zerosec{\N{\emb}})}(\u{\N{\emb}})$
      (see \forexample \cite[Prop.~17.2]{bredon}), because then
      one has with $\zerosec{\N{\emb}} = \collapse\emb$ that
      \begin{gather*}
        \w{\rk\N{\emb}}{\N{\emb}}
        = \pb{(\zerosec{\N{\emb}})}(\u{\N{\emb}})
        = \pb\emb\pb\collapse\u{\N{\emb}}
        \;.
      \end{gather*}
      
      For \ref{item:fundclsubmfd} recall from
      Lemma~\ref{lem:thomisofundcl}, that 
      $\fundcl M = \pf p(\u{\N{\emb}} \cap \pb\collapse\fundcl W)$,
      so
      \begin{gather}\label{eq:fundclsubmfd}
        \pf{\zerosec{\N{\emb}}}\fundcl M
        = \pf{\zerosec{\N{\emb}}}\pf p
        (\u{\N{\emb}} \cap \pb\collapse\fundcl W)
        = \u{\N{\emb}} \cap \pb\collapse\fundcl W
        \;.
      \end{gather}
      Now observe that for the inclusion maps
      \begin{center}
        \begin{tikzcd}
          W
          \ar[r, "j"]
          &(W, W\setminus M)
          &\ar[l,"\text{exc.}"{below}, "i"{above}]
          (\Discbdl{\N{\emb}},\Discbdl{\N{\emb}}\setminus M)
        \end{tikzcd}
      \end{center}
      $i$ is an excision, and 
      $\pf c = (\pf i)^{-1} \pf j$, $\pb c= \pb j(\pb i)^{-1}$.
      Denote $u\coloneqq \u{\gamma_1}$ and the preimage of
      $\u{\gamma_1}$ under the excision by
      $u'\coloneqq (\pb i)^{-1}\u{\gamma_1}\in\H^*(M, M\setminus N)$.
      Then calculate using the cap-product formula from
      \autoref{eq:capprod2},
      and $\emb=i\circ\zerosec{\N{\emb}}$:
      \begin{align*}
        \pb\collapse u \cap \fundcl W
        &=\pb j u' \cap \fundcl W
        \cequalsby{$(*)$}
          \pf j(\pb j u' \cap \fundcl W) \\
        &\equalsby{\eqref{eq:capprod2}}
          u' \cap \pf j\fundcl W 
          = u' \cap \pf i\pf c\fundcl W \\
        &\equalsby{\eqref{eq:capprod2}}
          \pf i(\pb i u' \cap \pf c\fundcl W)
        \cequalsby{\eqref{eq:fundclsubmfd}}
          \pf i\pf{\zerosec{\N{\emb}}}\fundcl M
          = \pf\emb \fundcl M
          \;,
      \end{align*}
      where equality $(*)$ uses, that the map of triple of spaces
      $j\colon (W, \emptyset, \emptyset)\to(W, W\setminus N, \emptyset)$
      is the identity when restricted to $(W,\emptyset)\to(W, \emptyset)$.
    \end{proof}
  \end{Lem}
  By the above Lemma
  \begin{align*}
    \w1{\N{}}
    &= \pb\emb y \\
    \W{\N{}}
    &= 1 + \w1{\N{}}
      = \pb\emb(1+y)
      \cequalsby{(*)} \pb\emb\left((1+y)^{-(2^s-1)}\right)
      = \left(\pb\emb(1+y)\right)^{-(2^s-1)}
  \end{align*}
  for some $2^s>n+1$, where equality (*) then comes from $y^{2^s}=0$ and
  \begin{gather*}
    (1+y)^{2^s-1} \cdot (1+y)
    = (1+y)^{2^s}
    = (1+y^{2^s})
    = 1
    \;.
  \end{gather*}
  Hence, $\W{\T{\G m}}$ can be expressed as
  \begin{align*}
    \W{\T{\G m}}
    &= \pb\emb(1+y)^{2^s-1}
      \cdot \pb\emb\left(
      \prod_{r=1}^{l}(1+x_{n_r})^{n_r+1}
      \right) \\
    &= \pb\emb(1 + \el{1}{}(
      \underbrace{y,\dotsc,y}_{2^s-1},
      \underbrace{x_{n_1},\dotsc,x_{n_1}}_{n_1+1},
      \dotsc,
      \underbrace{x_{n_{l}},\dotsc,x_{n_{l}}}_{n_{l}+1}
      )) \\
    &= 1 + \el{1}{}(
      \pb\emb y,\dotsc,\pb\emb y,
      \pb\emb x_{n_1},\dotsc, \pb\emb x_{n_1},
      \dotsc,
      \pb\emb x_{n_l},\dotsc, \pb\emb x_{n_l}
      ))
  \end{align*}
  which is a representation as a pullback of a sum of elementary
  symmetric polynomials in the given one dimensional variables.
  This yields
  \begin{align*}
    \s{m}(\G m)
    &= (2^s-1) \pb\emb y^m
      + (n_1+1) \pb\emb x_{n_1}^m
      + \dotsb
      + (n_l+1)\pb\emb x_{n_l}^m
    \\
    &= \pb\emb\left(
      (2^s-1)y^m
      + (n_1+1)x_{n_1}^m
      + \dotsb
      + (n_l+1)x_{n_l}^m
      \right) \\
    &\equalsby{$x_{n_r}^{n_r+1}=0$, $n_r+1<m$}
      \pb\emb((2^s-1)y^m) \\
    &\equalsby{$2^s=0$}
      \pb\emb y^m
  \end{align*}
  So
  \begin{align*}
    \pf\emb\snum{m}{\G m}
    &= \pf\emb\capped{\s{m}(\G m)}{\fundcl{\G m}} \\
    &= \pf\emb\capped{\pb\emb y^m}{\fundcl{\G m}} \\
    &= \capped{y^m}{\pf\emb\fundcl{\G m}} \\
    &\equalsby{Def.} \capped{y^m}{y\cap\fundcl{K^{m+1}}}
      = \capped{y^{m+1}}{\fundcl{K^{m+1}}}
      \;.
  \end{align*}
  As capping with $\fundcl{K^{m+1}}$ is the Poincaré isomorphism, it
  suffices for the proof to show that $y^{m+1}$, hence the image of
  $\snum{m}{\G m}$ under $\pf\emb$, is non-zero. But, by the relations
  of the $x_r$, combinatorics give that the only non-zero product
  $\prod_{r=1}^{l}x_{n_r}^{a_r}$ of degree $m+1$ is the one with
  $a_r=n_r$.
  With the multinomial theorem $y^{m+1}$ reformulates to
  \begin{gather*}
    y^{m+1}
    = (x_{n_1}+\dotsb+x_{n_l})^{m+1}
    = \frac{(n_1 + \dotsb + n_l)!} {(n_1)! \dotsm (n_l)!}
    \cdot \prod_{r=1}^{l}x_{n_r}^{n_r}
  \end{gather*}
  which is non-zero if and only if the binomial expression is.
  The latter however simplifies with descending induction using the
  easy formula
  \begin{gather*}
    \frac{(a_1+\dotsb+a_j)!}{(a_1)!\dotsm(a_j)!}
    = \binom{a_1+\dotsb+a_j}{a_j}
    \cdot \frac{(a_1+\dotsb+a_{j-1})!}{(a_1)!\dotsm(a_{j-1})!}
  \end{gather*}
  for $a_r\in\Nat$ to
  \begin{gather*}
    \frac{(n_1+\dotsb+n_l)!}{(n_1)! \dotsm (n_l)!}
    = \prod_{r=2}^{l}\binom{n_1+\dotsb+n_r}{n_r}
  \end{gather*}
  all factors of which are one by Lemma~\ref{lem:lucas} due to the
  condition on the $n_r$.
  So altogether $\G m$ represents an indecomposable element of
  the cobordism ring according to the indecomposability criterion
  in Theorem~\ref{thm:indecomposabilitycriterion}.
  \qedhere
\end{proof}

\subsubsection{Odd Dimensional Generators}
For the odd dimensional generators the twisted product construction
comes into play.
As already discussed, the twisted product of a manifold which is
indecomposable in the cobordism ring will be so, too, if the product
dimension was chosen correctly. So the selected choice of candidates
will be twisted products of some $\G n$ for $n$ even, whose
indecomposability was proven during the previous section.
The trick here will be to select for a candidate in odd degree $m$ a
split of $m$ of the form $m=k+2n$ with $n$ even(!).
\begin{Rem}\label{rem:splitodddimension}
  The following is a convenient split in the above sense:
  Let $m$ be odd and write the binary expansion as 
  \begin{gather*}
    m
    = \underbrace{2^{i_0} + \dotsb + 2^{i_{l_k}}}_{\eqqcolon k}
    + \underbrace{2^{i_{l_k+1}}
      + \dotsb + 2^{i_{l_k+l_n}}}_{\eqqcolon 2n}
    = k+2n
  \end{gather*}
  such that as usual $i_r<i_{r+1}$, and additionally $i_r=r$ for
  $0\leq r\leq l_k$ and $i_{l_k+1} > l_k+1$.
  In other words split the binary expansion of $m$ at the first point
  where it skips a power of two and $k$ to the first part and $2n$ to
  the second part. Since $m$ is odd, this trick guarantees that $n$ is
  even. Also, by Remark~\ref{rem:splitdim},
  if $m$ is not of the form $2^s-1$, $n$ will not be either.
\end{Rem}

For the immersion property we will again utilize nice immersion
properties of projective spaces, more precisely the following one by
Mahowald and Milgram for the total space of line bundles over such:
\begin{Lem}[{\cite[Theorem~4.1, p.~418]{mahowald}}]
  \label{lem:immersionuniversalbdl}
  % Wrong attempt [p.~85]{immersionconj} (?):
  % """
  % The total space of a vector bundle $E\to M^n$ of rank $k$ over a
  % closed manifold of dimension $n$ can be immersed into $\R^{2n+k}$.
  % """
  For $p,q\in\Nat$ odd and the total space of $(p+1)\gamma_q$ immerses
  in Euclidean space of dimension
  \begin{gather*}
    2q+p+1-\alpha(p+q+1)+\alpha(p+1)
    \;.
  \end{gather*}
\end{Lem}
The trick for odd $m$ now will be to find an immersion of the
$m$-candidate into some $\Twistedprod{k}{\R^{s}}$, which is
the total space of $(s\gamma_k)\oplus\trivbdl^s$
by Corollary~\ref{lem:twistedprodrealspace}, and then apply the above
lemma to a super-bundle $((s+s')\gamma_k)\oplus\trivbdl^s$ of
the latter.
At that point it will be essential that $k$ from the above split of
$m$ is odd in order to make the lemma applicable in the first
place.

With these preliminary considerations one can make a choice of
generator candidates:
\begin{Def}
  Let $m=k+2n$ be odd and not of the form $2^s-1\in\Nat$ with the
  split from Remark~\ref{rem:splitodddimension} above. Define
  \begin{gather*}
    \G m \coloneqq \Twistedprod{k}{\G n}
  \end{gather*}
\end{Def}
\begin{Lem}\label{lem:oddgen}
  For $m$ odd as in the above definition, $\G m$ fulfills:
  \begin{enumerate}
  \item\label{item:brownimmersionproperty}
    $\G m$ immerses into $\R^{2m-\alpha(m)}$.
  \item\label{item:indecomposabilityproperty}
    The cobordism classes of the $\G m$, $m\neq 2^s-1$, are
    indecomposable.
  \end{enumerate}
\end{Lem}
\begin{proof}[proof of
  Lemma~\itemref{lem:oddgen}{item:indecomposabilityproperty}]
  The used split of $m$ from Remark~\ref{rem:splitodddimension} is a
  special case of splits described in Remark~\ref{rem:splitdim}, for
  all of which the twisted product preserves indecomposability by
  Corollary~\ref{cor:twistedprod:indecompcriterion}.
  Thus, as $n$ from the split is even and $\G n$ is indecomposable by
  Lemma~\ref{lem:evengen}, $\G m=\Twistedprod{k}{\G n}$ is also an
  indecomposable element in the cobordism ring.
\end{proof}
\begin{proof}[proof of
  Lemma~\itemref{lem:oddgen}{item:brownimmersionproperty}]
  Here one needs that for $m=k+2n$ as above $\G n$ admits by
  Lemma~\ref{lem:evengen} an immersion
  \begin{gather*}
    \iota\colon\G n \immlongto \R^{2n-\alpha(n)}
    \;.
  \end{gather*}
  As described in
  Remark~\itemref{rem:twistedprodproperties}{item:twistedprodpreservesimmersions},
  this induces an immersion of twisted products
  \begin{center}
    \begin{tikzcd}[row sep=small, column sep=large]
      (\Sphere{2^k} \times \G n\times \G n/\sim)
      \ar[d, equals]
      \ar[r, hookrightarrow, "1\times\iota\times\iota/\sim"]
      &
      (\Sphere{2^k} \times \R^{2n-\alpha(n)} \times \R^{2n-\alpha(n)}/\sim)
      \;.
      \ar[d, equals]
      \\
      \G m
      &\Twistedprod{k}{\R^{2n-\alpha(n)}}
    \end{tikzcd}
  \end{center}
  The latter twisted product however is a vector bundle over
  $\RP{k}$ as in
  Remark~\itemref{rem:twistedprodproperties}{item:twistedprodfiberbdl}.
  More precisely, by Lemma~\ref{lem:twistedprodrealspace} it is the
  well-known vector bundle 
  $((2n-\alpha(n))\gamma_{k})\oplus\trivbdl^{2n-\alpha(n)}$
  where $\gamma_{k}$
  is the tautological line bundle over $\RP k$.
  As $p=2n-1$ and $q=k$ are both odd, 
  Lemma~\ref{lem:immersionuniversalbdl} is applicable to $p$ and $q$,
  and yields an immersion of $(p+1)\gamma_q = 2n\gamma_k$ into
  Euclidean space of dimension
  \begin{align*}
    2q+p+1-\alpha(q+p+1) + \alpha(p+1)
    &= 2k + 2n - 1 + 1 - \alpha(k+2n) + \alpha(2n) \\
    &= k + m - \alpha(m) + \alpha(n)
    \;.
  \end{align*}
  Hence, the total space of
  $(2n\gamma_k)\oplus\trivbdl^{2n-\alpha(n)}$
  immerses into Euclidean space of codimension
  \begin{gather*}
    \left(k + m - \alpha(m) + \alpha(n)\right)
    + 2n-\alpha(n)
    = (k+2n) + m - \alpha(m)
    = 2m - \alpha(m)
    \;,
  \end{gather*}
  altogether yielding an immersion
  \begin{align*}
    \Twistedprod{k}{\G m}
    &\immto
    \Twistedprod{k}{\R^{2n-\alpha(n)}}
    = \E\left(((2n-\alpha(n))\gamma_k)\oplus\trivbdl^{2n-\alpha(n)}\right)
    \\
    &\immto 
    \E\left((2n\gamma_k)\oplus\trivbdl^{2n-\alpha(n)}\right)
    \\
    &\immto
    \R^{2m-\alpha(m)}
  \end{align*}
  as was needed.
\end{proof}


This finally gives Brown's theorem:
\begin{Thm}
  Lemmata~\ref{lem:evengen} and \ref{lem:oddgen} together yield a
  complete generating set of the cobordism ring, \idest
  \begin{gather*}
    \c_*\cong \left( [\G m] \,\big|\, m\neq 2^s-1 \right)_{\Zmod2}
    = \Zmod2\left[[\G m] \,\big|\, m\neq 2^s-1\right]
    \;,
  \end{gather*}
  each element of which fulfills the immersion property as was
  desired.
\end{Thm}


%%% Local Variables:
%%% mode: latex
%%% TeX-master: "thesis"
%%% ispell-local-dictionary: "en_US"
%%% End:
