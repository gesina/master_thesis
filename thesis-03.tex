%%%%%%%%%%%%%%%%%%%%%%%%%%%%%%%%%% 
% Master Thesis in Mathematics
% "Immersions and Stiefel-Whitney classes of Manifolds"
% -- Chapter 4: Brown's Theorem --
% 
% Author: Gesina Schwalbe
% Supervisor: Georgios Raptis
% University of Regensburg 2018
%%%%%%%%%%%%%%%%%%%%%%%%%%%%%%%%%%

\chapter{Brown's Theorem}
% Review the necessary results about the structure of the unoriented cobor-
% dism ring and explain R. L. Brown’s proof of the immersion conjecture up
% to cobordism. [immersionconj], [brown]

\section{The Cobordism Ring Structure}

\begin{LemDef}[Cobordism Ring] % TODO
\end{LemDef}

\begin{Thm} % TODO: cobordism ring thm
  \optcite[Thm.~1.23]{immersionconj}
  \optcite[Theorem~IV.9]{thom}
  \begin{gather*}
    \c_* \cong \Zmod2[\sigma_i| i\neq 2^r-1]
  \end{gather*}
\end{Thm}


\section{Indecomposable Elements of the Cobordism Ring}
In order to find representatives for a set of algebraically
independent generators of the polynomial ring $\c_*$, one needs to
find a way to detect indecomposable elements. Indecomposable in this
context means not expressable as a sum of products of lower degree
elements.
In order to do so the subsequent sections will follow an approach of
Thom 
\cite[Chapters~IV.5 and~IV.6]{thom}.


\subsection{Special Properties of Symmetric Polynomials}
This section examines a special kind of polynomials that
obey a product rule similar to \ref{tag:cartan} whenever evaluated on
elements which have the form of a total Stiefel-Whitney class.
This will make it possible to express certain combinations of
Stiefel-Whitney numbers of product manifolds in terms of ones of their
factors.

From this the subsequent subsection will deduce a simple criterion for
a manifold to be cobordant to a product of manifolds. 

Beforehands, mind the following notation of partitions needed for
symmetrisation of polynomials.
\begin{Def}
  Let $k,l\in\Nat$ be integers.
  \begin{itemize}
  \item
    A partition $\Part=(i_1,\dotsc,i_l)$ of $k$ is an unordered sequence
    of integers such that $k=\sum_{r=0}^{l}i_r$.
    Two partitions only differing by zeros are considered equal.
    In other words, a partition is an equivalence class of sequences in
    $\bigoplus_\infty\Nat$ under the relation $\Part\sim\sigma(\Part)$
    for any permutation $\sigma$.
  \item
    The notation $I^l$ for a sequence of integers will mean a sequence
    of length $l$. Write $I^l\in\Part$ for a sequence of length
    $l$ in the equivalence class of the partition $\Part$. 
  \item
    Denote by $\PartitionsOf{k}$ the set of partitions of $k$.
  \item
    Write $\Emptypart$ for the unique partition of $0$.
  \item The concatenation of sequences and analoguesly partitions will
    be denoted by
    \begin{align*}
      \Nat^{l_1} \times \Nat^{l_2}
      &\xrightarrow{-\concat-}
        \Nat^{l_1+l_2}
      \\
      \PartitionsOf{k_1} \times \PartitionsOf{k_2}
      &\xrightarrow{-\concat-}
        \PartitionsOf{k_1+k_2}
      \\
      (i_1,\dotsc,i_r), (j_1,\dotsc,j_s)
      &\longmapsto
        (i_1,\dotsc,i_r,j_1,\dotsc,j_s)
    \end{align*}
  \end{itemize}
\end{Def}

Also as preparation, recall some properties of symmetric polynomials.
\begin{LemDef}
  Let $n\in\Nat$ and $\Zmod2[t_1,\dotsc,t_n]$ be the polynomial ring
  in $n$ variables over the fields $\Zmod2$, each $t_i$ of degree 1.
  \begin{enumerate}
  \item Let
    $\Symm{n}_*
    \coloneqq \Zmod2[t_1,\dotsc,t_n]^{\Permutations{n}}
    \subset \Zmod2[t_1,\dotsc,t_n]$
    be the graded subring of symmetric polynomials in $n$ variables.
  \item It has a basis indexed by partitions $\Part$ consisting of
    symmetrized monomials, \idest elements of the form
    \begin{gather*}
      \symm{n} t^\Part \coloneqq \sum_{I^n\in\Part} t^I \in \Symm{n}_k
      % TODO: Maybe leave the special notation for symmetrized stuff out?
    \end{gather*}
    where $t^{(i_1,\dotsc,i_n)}\coloneqq t_1^{i_1}\dotsm t_n^{i_n}$.
    \begin{proof}
      Each symmetric polynomial is the sum of homogenous symmetric
      polynomials $\sum_{I^n\in A}t^I$ which can be written as a sum
      of symmetrized monomials by descending induction on the number
      of monomial summands $\#A$.
      Linear independence is clear as monomials $t^I$ and $t^{I'}$ are
      linearly independent if $I\neq I'$, and any two partitions have
      empty intersection.
      \optcite[footnote~2, p.~154]{thom}
    \end{proof}
  \item $\Symm{n}_*$ is generated by the algebraically independent
    elementary symmetric polynomials in $n$ variables
    \begin{gather*}
      \el i n\coloneqq \symm{n} t^{\Part_i}
      \in \Symm{n}_i
      \qquad \text{for }
      \Part_i = (1,\dotsc,1)
      \in \PartitionsOf i
    \end{gather*}
    for $1\leq i\leq n$.
    \Forexample
    $\el 1 n=\sum_{r=1}^{n}t_r$,
    $\el 2 n=\sum_{1\leq r<s\leq n} t_r t_s$.
    \begin{proof}
      % TODO: ref
      This is the fundamental theorem on symmetric polynomials
      \cite[Chapter~4.4, Satz~1]{bosch2013algebra}.
    \end{proof}
  \item As a simple calculation shows, the elementary symmetric
    polynomials in $n$ variables fulfill
    \begin{gather}\label{eq:sumelemsymmpoly}
      1 + \sum_{i=1}^{n} \el i n
      = \prod_{r=1}^{n}(1+t_n)
    \end{gather}
  \end{enumerate}
\end{LemDef}

Now the desired polynomials can be defined.
\begin{Def}
  \optcite[p.~90]{milnorlectures}
  Let $k\in\Nat$, $\Part\in\PartitionsOf k$, and let
  $\Zmod2[\alpha_1,\dotsc,\alpha_k]$ be the polynomial ring in $k$
  variables where $\alpha_i$ has degree $i$.
  Define the homogenous polynomial
  $\s{\Part}\in\Zmod2[\alpha_1,\dotsc,\alpha_k]$ of degree $k$ by
  \begin{gather*}
    \s{\Part}(\el 1 n,\dotsc, \el k n) = \symm{n} t^{\Part} \in \Symm{n}_k
  \end{gather*}
  for some $n\geq k$. Mind, that this is well-defined as the
  elementary symmetric polynomials are algebraically independent, and
  the definition does not depend on $n$ as long as $k\leq n$.
  
  For a any graded ring $A_*=\bigoplus_{i\geq 0} A_i$ write elements as
  $a=\sum_i a_i \coloneqq (a_0,a_1,\dotsc)$.
  For such an element define the evaluation of $\s{\Part}$ as
  \begin{gather*}
    \s{\Part}(a) \coloneqq \s{\Part}(a_1,\dotsc,a_k)
    \;,
  \end{gather*}
  \idest skip $a_0$ and all higher $a_i$.
\end{Def}
\begin{Ex}
  The first such polynomials over $\Zmod2$ are
  \begin{align*}
    k&=0:
    &\s{\Emptypart} &= 1
    \\ k&=1:
    &\s{(1)} &= \alpha_1
    \\ k&=2:
    &\s{(2)} &= \alpha_1^2
        &\s{(1,1)} &= \alpha_2
    \\ k&=3:
    &\s{(3)} &= \alpha_1^3 + \alpha_1\alpha_2 + \alpha_3
        &\s{(1,2)} &= \alpha_1\alpha_2 + \alpha_3
             &\s{(1,1,1)} &= \alpha_3
  \end{align*}
  \cite[p.~90]{milnorlectures}
\end{Ex}

These polynomials that translate the basis $\symm{n}t^{\Part}$ of
$\Symm{n}_k$ into expressions of the generators $\el i n$
fulfill as promised the following interesting property concerning
multiplication of elements.
\begin{Lem}\label{lem:productrule:general}
  \optcite[Theorem~33, p.~91f]{milnorlectures}
  Let $k\in\Nat$, and let $A_*=\bigoplus_{i\geq 0} A_i$ be a graded ring.
  For $a,b\in A_*$ with $a_0=1=b_0$, and any partition
  $\Part\in\PartitionsOf k$ holds
  \begin{gather*}
    \s{\Part}(a\cdot b)
    = \sum_{\Part_1\concat\Part_2=\Part}
    \s{\Part_1}(a) \cdot \s{\Part_2}(b)
    \;.
  \end{gather*}
  \begin{proof}
    As for the definition of $\s{\Part}$ it suffices to check equality
    on algebraically independent elements. So, make the following choices:
    \begin{itemize}
    \item The result will be independent of $n$, as long as
      $n$ is large enough, thus choose $n=2k$.
      Note that this is the smallest choice for $n$ for which there
      may be a pair of sequences $J_1^k\concat J_2^k\in\Part$
      where one is all zeros.
    \item For simplicity of notation write
      $x_i=t_i$ and $y_i=t_{2i}$ for $1\leq i\leq k$ to split the
      variables $(t_1,\dotsc,t_{2k})$ into two equal parts.
    \item In order to directly work with the definition of $\s{\Part}$ it is
      convenient to choose as algebraically independent elements
      \begin{align*}
        a &\coloneqq
            \prod_{\mathclap{r=1}}^k(1+t_r)
            \cequalsby{\eqref{eq:sumelemsymmpoly}}
            1 + \sum_{i=1}^k \el i k(x_1,\dotsc,x_k)
            \;\text{, and}\\
        b &\coloneqq
            \prod_{\mathclap{r=k+1}}^n(1+t_r)
            \cequalsby{\eqref{eq:sumelemsymmpoly}}
            1 + \sum_{i=1}^k \el i k(y_1,\dotsc,y_k)
            \;\text{, then}\\
        a\cdot b
          &=
            \prod_{r=1}^n (1+t_r)
            \cequalsby{\eqref{eq:sumelemsymmpoly}}
            1 + \sum_{i=1}^n \el i n (t_1,\dotsc,t_n)
            \;.
      \end{align*}
      Consequently, the sets 
      $(a_1,\dotsc,a_k)$, $(b_1,\dotsc,b_k)$, and
      $(a\cdot b)_{1\leq i\leq n}$ are algebraically independent, as
      was needed.
    \end{itemize}
    Now calculate
    \begin{align*}
      \s{\Part}(a\cdot b)
      &\coloneqq
        \symm{n} t^{\Part}
      \\ &\equalsby{Def.}
           \sum_{I^n\in\Part}
           t_1^{i_1}\dotsm t_n^{i_n}
      \\ &=
           \sum_{I^n\in\Part}
           x^{(i_1,\dotsc,i_k)}\cdot y^{(i_{k+1},\dotsc,i_n)}
      \\ &=
           \sum_{J_1^k\concat J_2^k\in\Part}
           x^{J_1} \cdot y^{J_2}
      \\ &\equalsby{Group by equiv.}
           \sum_{\Part_1\concat \Part_2=\Part}
           \sum_{\substack{J_1^k\in\Part_1\\ J_2^k\in\Part_2}}
      x^{J_1} \cdot y^{J_2}
      \\ &=
           \sum_{\Part_1\concat \Part_2=\Part}
           \left(\sum_{J_1^k\in\Part_1} x^{J_1}\right)
           \cdot
           \left(\sum_{J_2^k\in\Part_2} y^{J_2}\right)
      \\ &\equalsby{Def.}
           \sum_{\Part_1\concat \Part_2=\Part}
           \left(\symm{k} x^{\Part_1}\right)
           \cdot
           \left(\symm{k} y^{\Part_2}\right)
      \\ &\equalsby{Def.}
           \sum_{\Part_1\concat \Part_2=\Part}
           \s{\Part_1}\left( \el 1 k(x_1,\dotsc,x_k),\dotsc \right)
           \cdot
           \s{\Part_2}\left( \el 1 k(y_1,\dotsc,y_k),\dotsc \right)
      \\ &\equalsby{Def.}
           \sum_{\Part_1\concat \Part_2=\Part}
           \s{\Part_1}(a) \cdot \s{\Part_2}(b)
           \qedhere
    \end{align*}
  \end{proof}
\end{Lem}

\begin{Ex}
  The most important partition of an integer $k$ will be the trivial
  one $(k)\in\PartitionsOf k$. In this case
  Lemma~\autoref{lem:productrule:general} says
  \begin{gather*}
    \s{(k)}(a\cdot b) = \s{(k)}(a) + \s{(k)}(b)
    \;.
  \end{gather*}
\end{Ex}


\subsection{Stiefel-Whitney Numbers of Product Manifolds}
In order to apply the special polynomials from the preceeding section
and their product property \autoref{lem:productrule:general} to
the Stiefel-Whitney numbers of (product) manifolds, first start with application
to Stiefel-Whitney classes.

So, let $M^n=M_1^{n_1}\times M_2^{n_2}$ all be closed manifolds of the
noted dimension throughout the section.

Recall that
\begin{enumerate}
\item the cohomology ring $\H^*(M)$ of a space respectively
  manifold is a graded ring,
\item the total Stiefel-Whitney number of a manifold is of the form
  $\W{M_i} = 1 + \w 1 {M_i} + \dotsb + \w n {M_i}$, and that
\item by Künneth holds
  \begin{align*}
    \H^*(M_1)\otimes \H^*(M_2)
    &\overset{\cong}\longto \H^*(M)
    \\
    c_1 \otimes c_2
    &\longmapsto c_1\cdot c_2
      \coloneqq \pb\proj_1 c_1 \cdot \pb\proj_2 c_2
  \end{align*}
  and $\W{M} = \W{M_1} \cdot \W{M_2}$ by
  \ref{tag:swclassesmultiplicativity} of the Stiefel-Whitney classes.
\end{enumerate}
Thus, one can apply the multiplication
Lemma~\autoref{lem:productrule:general} to $\W{M}$, which immediately
yields:
\begin{Cor}\label{cor:productrule:swcl}
  For $M=M_1\times M_2$ manifolds as above one gets for any partition
  $\Part\in\PartitionsOf n$:
  \begin{align*}
    \s{\Part}(\W{M})
    =
    \s{\Part}\left(\W{M_1}\cdot \W{M_2}\right)
    &\cequalsby{\autoref{lem:productrule:general}}
      \sum_{\mathclap{\Part_1\concat\Part_2=\Part}}
      \s{\Part_1}(\W{M_1}) \cdot \s{\Part_2}(\W{M_2})
    \\ &=
         \sum_{\mathclap{\substack{
         \Part_1\concat\Part_2=\Part\\
    \Part_1\in\PartitionsOf{n_1}\\
    \Part_2\in\PartitionsOf{n_2}\\
    }}}
    \s{\Part_1}(\W{M_1}) \cdot \s{\Part_2}(\W{M_2})
  \end{align*}
  \begin{proof}
    The last equality is due to the following dimension reasons.
    For any any manifold $W$ and a partition $\Part$ of $k$,
    $\s{\Part}(\W{W})$ is of degree $k$ if $k\leq\dim W$ and zero else.
    Therefore, for any combination of partitions
    $\Part_1\in\PartitionsOf{k_1}$,
    $\Part_2\in\PartitionsOf{k_2}$
    where $k_1+k_2=n=n_1+n_2$ with $k_i\neq n_i$, the product
    $\s{\Part_1}(\W{M_1})\cdot\s{\Part_2}(\W{M_2})$ will have a zero
    factor and can be skipped.
  \end{proof}
\end{Cor}

In order to pass to Stiefel-Whitney numbers instead of classes, use
the following notation.
\begin{Def}
  Let $W$ be a closed manifold and $\Part\in\PartitionsOf{\dim W}$.
  Then write
  \begin{gather*}
    \snum{\Part}{W} \coloneqq \capped{\s{\Part}(\W{W})}{\fundcl{W}}
    \;.
  \end{gather*}
\end{Def}

Now the product rule \autoref{lem:productrule:general} translates to
\begin{Cor}
  For closed manifolds $M_1$, and $M_2$ with dimensions $n_1$ and
  $n_2$ one has
  \begin{align*}
    \snum{\Part}{M_1\times M_2}
    &= \sum_{\mathclap{\substack{
      \Part_1\concat\Part_2=\Part\\
    \Part_1\in\PartitionsOf{n_1}\\
    \Part_2\in\PartitionsOf{n_2}\\
    }}}
    \snum{\Part_1}{M_1} \cdot \snum{\Part_2}{M_2}
    \quad \in \quad \Zmod2
  \end{align*}
  and as a special case
  \begin{align}
    % \notag
    % \snum{(n_1,n_2)}{M_1\times M_2}
    % &= \snum{(n_1)}{M_1} \cdot \snum{(n_2)}{M_2}
    % \\
    \label{eq:productrule:swnum}
    \snum{(n_1+n_2)}{M_1\times M_2} &= 0
                                      \;.
  \end{align}
  In particular, if $\snum{(\dim W)}{W}\neq 0$ for a closed manifold
  $W$, then $W$ is no product manifold.
  \begin{proof}
    The statement immediately follows from the previous
    Corollary~\autoref{cor:productrule:swcl} with the following
    two facts:
    \begin{enumerate}
    \item The generator $\fundcl{M_1\times
        M_2}\in\H^{n_1+n_2}(M_1\times M_2)$ corresponds to the generator
      $\fundcl{M_1}\otimes\fundcl{M_2}\in
      \H^{n_1}(M_1)\otimes\H^{n_2}(M_2)\cong \H^{n_1+n_2}(M_1\times M_2)$
      under the Künneth isomorphism.
    \item For cohomology classes $c_1\in\H^{n_1}(M_1)$, $c_2\in\H^{n_2}(M_2)$ holds
      \begin{gather*}
        \capped{c_1\otimes c_2}{\fundcl{M_1\times M_2}}
        = \capped{c_1\otimes c_2}{\fundcl{M_1}\otimes\fundcl{M_2}}
        = \capped{c_1}{\fundcl{M_1}} \cdot \capped{c_2}{\fundcl{M_2}}
        \in \Zmod2
      \end{gather*}
      by the universal property of the tensor product.
      \qedhere
    \end{enumerate}    
  \end{proof}
\end{Cor}

\autoref{eq:productrule:swnum} is the desired obstruction for
a manifold to be a product or---as will be explained in the next
subsection---to be cobordant to a product.
Finally, this statement will be an invaluable tool for detecting
manifolds that are not only not cobordant to a product manifold but
whose cobordism class is indecomposable.


\subsection{A Criterion for Indecomposability}
The ultimate goal of this subsection is to deduce the following
theorem which is a consequence of Thom's proof of the multiplicative
structure of the cobordism ring (see \cite[Section~IV.5]{thom}).
\begin{Thm}\label{thm:indecomposabilitycriterion}
  A closed $n$-manifold $M$ represents an indecomposable element of
  the cobordism ring $\c_*$, if and only if
  \begin{gather*}
    \snum{(n)}{M} \neq 0 \in \Zmod2
    \;.
  \end{gather*}
  % for non-dyadic decomposition: \optcite[p.~156]{thom};
  % for summation notation: \optcite[p.154, (4)f]{thom};
\end{Thm}

For the mentioned proof, which includes the above statement, Thom
constructs special manifolds that form a basis of the cobordism ring,
and are uniquely characterized by the below properties.
For the formulation call a sequence or partition \emph{non-dyadic}, if
none of its entries is of the form $2^m-1$. 
\begin{Thm}\label{thm:basiscobordismring} % TODO: Proof
  There exists a basis of the cobordism ring represented by manifolds
  $V_\Part$ that is in each degree $k$ indexed by non-dyadic
  partitions $\Part$ of $k$. Further, the $V_\Part$ are uniquely
  characterized by
  \begin{gather*}
    \s{\Part'}(\W{V_\Part}) = \delta_{\Part,\Part'}
    \in \H^k(V_{\Part})\cong \Zmod2
  \end{gather*}
  for any non-dyadic partitions $\Part,\Part'\in\PartitionsOf{k}$,
  where $\delta$ is the usual Kronecker delta.
  \begin{proof}
    % TODO: proof special basis cobordism ring [Thom]; [Milnor] better source?
  \end{proof}
\end{Thm}

Before starting with the proof of the desired result one needs the
following connection between Stiefel-Whitney classes of a manifold and
its cobordism class. Thom rather directly deduces this from the
existence of the above basis.
\begin{Thm}[Thom]\label{thm:cobordantiffswnumscoincide}
  Two closed manifolds are cobordant if and only if all of their
  Stiefel-Whitney numbers coincide.
% TODO: Proof/ref: SW Nums = 0 <=> null-bordant
\end{Thm}

\begin{proof}[proof of \autoref{thm:indecomposabilitycriterion}]
  Using Theorems~\autoref{thm:basiscobordismring} and
  \autoref{thm:cobordantiffswnumscoincide} enables to proof the
  desired result using the following intermediate conclusions.
  \begin{enumerate}
  \item Any manifold $W$ of dimension $k$ is
    cobordant to
    \begin{gather*}
      \coprod_{\mathclap{\substack{
            \Part\in\PartitionsOf k \text{ non-dyadic}\\
            \s{\Part}(\W{W})\neq 0
          }}} V_\Part
      \;,
    \end{gather*}
    as the Stiefel-Whitney numbers are additive with respect to
    disjoint union. % TODO: check
  \item\label{item:manifoldbasisrepr}
    $V_{\Part_1}\times V_{\Part_2}$ is cobordant to
    $V_{\Part_1\concat\Part_2}$ since
    $\s{\Part'}(\W{V_{\Part_1}\times V_{\Part_2}})
    = \delta_{\Part', \Part_1\concat\Part_2}$
    by \ref{cor:productrule:swcl} and thus all their Stiefel-Whitney
    numbers coincide by definition.
  \item\label{item:generatorscobordimsring}
    By conclusion \ref{item:manifoldbasisrepr},
    the basis elements represented by a $k$-dimensional manifold
    $V_{(k)}\in\c_k$ for $k\neq 2^m-1$ are algebraically independent
    generators of the cobordism ring.
  \item By conclusion \ref{item:generatorscobordimsring}, the bordism
    class of a manifold $W$ of dimension $k$ is an indecomposable
    element of $\c_*$ if and only if its representation by basis
    elements $[V_\Part]$ contains the unique $k$-dimensional
    indecomposable basis element $[V_{(k)}]$ as summand,
    \idest if and only if $\snum{(k)}{W}\neq 0$.
    \qedhere
  \end{enumerate}
\end{proof}



%%% Local Variables:
%%% mode: latex
%%% TeX-master: "thesis"
%%% End:
