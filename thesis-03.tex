%%%%%%%%%%%%%%%%%%%%%%%%%%%%%%%%%%
% Master Thesis in Mathematics
% "Immersions and Stiefel-Whitney classes of Manifolds"
% -- Chapter 3: The Immersion Conjecture up to Cobordism --
% 
% Author: Gesina Schwalbe
% Supervisor: Georgios Raptis
% University of Regensburg 2018
%%%%%%%%%%%%%%%%%%%%%%%%%%%%%%%%%% 

\chapter{The Immersion Conjecture up to Cobordism}\label{chap:brown}
% Review the necessary results about the structure of the unoriented
% cobordism ring and explain R. L. Brown’s proof of the immersion
% conjecture up to cobordism. [immersionconj], [brown]
The overall goal of this chapter is to proof the following theorem of
R.~L.~Brown following his paper \cite{brown},
which essentially states that the immersion conjecture is true up to
the cobordism relation.
\begin{Thm}[Brown]\label{thm:brown}
  Every closed $n$-manifold is cobordant to an $n$-manifold that immerses
  into $\R^{2n-\alpha(n)}$.
\end{Thm}

As one easily sees that this property is stable under
the ring operations (Lemma~\ref{lem:brownstableunderringops}),
the main idea for the proof is to find manifolds
fulfilling the conjecture whose cobordism classes form a generating set
of the cobordism ring.
As the latter has the form of a polynomial algebra
$\Zmod2[\sigma_i|i\neq 2^r-1]$, a set of elements
$([\G i]|i\neq2^r-1)$ will already be a generating set if all of the
$\G i$ are indecomposable.
Thus, the candidates generating elements that will be constructed
in \autoref{sec:proofbrown} need to be tested for (see
Theorem~\ref{thm:brownproof}):
\begin{enumerate}
\item indecomposability, which is the more lengthy part as it
  requires the preliminary results from
  \autoref{sec:indecomposabilitycriterion} respectively
  \autoref{sec:twistedprod:indecompcriterion}, and
\item the property to fulfil the immersion conjecture.
\end{enumerate}
The generating set is going to be constructed using so-called twisted
products.

This chapter is structured into some preliminary work on finding a
criterion to easily detect indecomposable elements of the cobordism
ring in \autoref{sec:indecomposableelements},
the twisted product construction and its properties---especially
concerning indecomposability---in
\autoref{sec:twistedprod}, and the final proof with the construction
of the generating set in \autoref{sec:proofbrown}, where all
preliminary work is merged.

For clarity of presentation, both a couple of results from Thom's
paper \cite{thom} within the review in
\autoref{sec:cobordismringstructure},
as well as the concrete calculation of the cohomology ring structure
of twisted products, will merely be referenced without proof.
The reader is assumed to be familiar with symmetric polynomials.

\section{Detecting Indecomposable Elements of the Cobordism Ring}
\label{sec:indecomposableelements}
In order to find representatives for a set of algebraically
independent generators of the polynomial ring $\c_*$, one needs a way
to detect indecomposable elements. Indecomposable in this 
context means to not be expressible as a sum of products of lower
degree elements.

This section will follow an approach of Thom 
\cite[Chapters~IV.5 and~IV.6]{thom}, in which a certain characteristic
class serves as indicator. This characteristic class is constructed
out of Stiefel-Whitney classes using certain functions on symmetric
polynomials (see \autoref{sec:functions}), and becomes particularly
handy on tangent bundles of manifolds (see
\autoref{sec:swnumsofproductmfds}). The final indication lemma is then
stated and proved in \autoref{sec:indecomposabilitycriterion}.

\subsection{Special Properties of Symmetric Polynomials}\label{sec:functions}
This subsection examines a special kind of polynomials that
obey a product rule similar to \ref{tag:cartan} whenever evaluated on
elements of the form of a total Stiefel-Whitney class
(see Lemma~\ref{lem:productrule:general}).

As a consequence, one can express certain combinations of
Stiefel-Whitney numbers of product manifolds in terms of ones of their
factors, which will be investigated in detail in
\autoref{sec:swnumsofproductmfds}.
From this \autoref{sec:indecomposabilitycriterion} will deduce a
simple criterion for a manifold to be cobordant to a product of
manifolds.

Beforehand, mind the following notation of partitions needed for
symmetrising polynomials.
\begin{Def}\label{def:partition}
  Let $k,l\in\Nat$ be integers.
  \begin{itemize}
  \item
    A \emph{partition} $\Part=(i_1,\dotsc,i_l)$ of $k$ is an unordered sequence
    of integers such that $k=\sum_{r=0}^{l}i_r$.
    Two partitions only differing by zeros are considered equal.
    In other words, a partition is an equivalence class of sequences in
    $\bigoplus_\infty\Nat$ under the relation $\Part\sim\sigma(\Part)$
    for any permutation $\sigma$.
  \item
    The notation $I^l$ for a sequence of integers will mean a sequence
    of length $l$. Write $I^l\in\Part$ for a sequence of length
    $l$ in the equivalence class of the partition $\Part$. 
  \item
    Denote by $\PartitionsOf{k}$ the set of partitions of $k$.
  \item
    Write $\Emptypart$ for the unique partition of $0$.
  \item
    Call a sequence or partition \emph{non-dyadic}, if
    none of its entries is of the form $2^m-1$.
  \item The concatenation of sequences, and analogously partitions, will
    be denoted by
    \begin{align*}
      \Nat^{l_1} \times \Nat^{l_2}
      &\xrightarrow{-\concat-}
        \Nat^{l_1+l_2}
      \\
      \PartitionsOf{k_1} \times \PartitionsOf{k_2}
      &\xrightarrow{-\concat-}
        \PartitionsOf{k_1+k_2}
      \\
      (i_1,\dotsc,i_r), (j_1,\dotsc,j_s)
      &\longmapsto
        (i_1,\dotsc,i_r,j_1,\dotsc,j_s)
    \end{align*}
  \end{itemize}
\end{Def}

Also as preparation, recall some properties of symmetric polynomials
and agree on a shorthand for symmetrised monomials.
\begin{LemDef}
  Let $n\in\Nat$ and $\Zmod2[t_1,\dotsc,t_n]$ be the graded polynomial
  algebra in $n$ variables over the fields $\Zmod2$, each $t_i$ of
  degree one.
  \begin{enumerate}
  \item Let
    $\Symm{n}_*
    \coloneqq \Zmod2[t_1,\dotsc,t_n]^{\Permutations{n}}
    \subset \Zmod2[t_1,\dotsc,t_n]$
    be the graded subalgebra of symmetric polynomials in $n$ variables.
  \item $\Symm{n}_*$ has a basis indexed by partitions $\Part$
    consisting of symmetrised monomials, \idest elements of the form
    \begin{gather*}
      \symm{n} t^\Part \coloneqq \sum_{I^n\in\Part} t^I \in \Symm{n}_k
    \end{gather*}
    where $t^{(i_1,\dotsc,i_n)}\coloneqq t_1^{i_1}\dotsm t_n^{i_n}$.
    \begin{proof}
      \optcite[footnote~2, p.~154]{thom}
      Every symmetric polynomial is the sum of homogeneous symmetric
      polynomials, \idest is of the form $\sum_{I^n\in A}t^I$. This
      however can be written as a sum of symmetrised monomials by
      descending induction on the number of monomial summands $\#A$,
      which makes the symmetrised monomials a generating set of $\Symm{n}_*$.
      Linear independence is clear as monomials $t^I$ and $t^{I'}$ are
      linearly independent if $I\neq I'$, and sequences belonging to
      different partitions must be unequal.
    \end{proof}
  \item $\Symm{n}_*$ is generated by the algebraically independent,
    elementary symmetric polynomials in $n$ variables
    \begin{gather*}
      \el i n\coloneqq \symm{n} t^{\Part_i}
      \in \Symm{n}_i
      \qquad \text{for }
      \Part_i = (1,\dotsc,1)
      \in \PartitionsOf i
    \end{gather*}
    for $1\leq i\leq n$.
    \Forexample
    $\el 1 n=\sum_{r=1}^{n}t_r$,
    $\el 2 n=\sum_{1\leq r<s\leq n} t_r t_s$.
    \begin{proof}
      This is the well-known fundamental theorem on symmetric
      polynomials, see \forexample
      \cite[Chap.~4.4, Satz~1]{bosch2013algebra}. 
    \end{proof}
  \item As a simple calculation shows that the elementary symmetric
    polynomials in $n$ variables fulfil
    \begin{gather}\label{eq:sumelemsymmpoly}
      1 + \sum_{i=1}^{n} \el i n
      = \prod_{r=1}^{n}(1+t_n)
    \end{gather}
  \end{enumerate}
\end{LemDef}

Now the desired polynomials can be defined.
\begin{Def}
  \optcite[p.~90]{milnorlectures}
  Let $k\in\Nat$, $\Part\in\PartitionsOf k$, and let
  $\Zmod2[\alpha_1,\dotsc,\alpha_k]$ be the polynomial ring in $k$
  variables where $\alpha_i$ has degree $i$.
  Define the homogeneous polynomial
  $\s{\Part}\in\Zmod2[\alpha_1,\dotsc,\alpha_k]$ of degree $k$ by
  \begin{gather*}
    \s{\Part}(\el 1 n,\dotsc, \el k n) = \symm{n} t^{\Part} \in \Symm{n}_k
  \end{gather*}
  for some $n\geq k$. This means, $\s{\Part}$ gives the representation
  of $\symm{n}t^\Part$ in terms of the generating set $(\el i n)_i$.
  Mind, that this
  \begin{enumerate}
  \item is well-defined, as the elementary symmetric
    polynomials are algebraically independent, and
  \item the definition does not depend on $n$ as long as $k\leq n$.
  \end{enumerate}
  Further, for a graded ring $A_*=\bigoplus_{i\geq 0} A_i$ write
  elements as
  \begin{gather*}
    a=\sum_i a_i \coloneqq (a_0,a_1,a_2,\dotsc)
    \;,
  \end{gather*}
  and define the evaluation of $\s{\Part}$ on such an element $a$ as
  \begin{gather*}
    \s{\Part}(a) \coloneqq \s{\Part}(a_1,\dotsc,a_k)
    \;,
  \end{gather*}
  \idest skip $a_0$ and all higher $a_i$.
\end{Def}

\begin{Ex}
  The first such polynomials over $\Zmod2$ are
  \begin{align*}
    k&=0:
    &\s{\Emptypart} &= 1
    \\ k&=1:
    &\s{(1)} &= \alpha_1
    \\ k&=2:
    &\s{(2)} &= \alpha_1^2
        &\s{(1,1)} &= \alpha_2
    \\ k&=3:
    &\s{(3)} &= \alpha_1^3 + \alpha_1\alpha_2 + \alpha_3
        &\s{(1,2)} &= \alpha_1\alpha_2 + \alpha_3
             &\s{(1,1,1)} &= \alpha_3
  \end{align*}
  \cite[p.~90]{milnorlectures}
\end{Ex}

As promised, these translation-polynomials between the basis
$\symm{n}t^{\Part}$ of $\Symm{n}_k$ and the generators $\el i n$
fulfil the following interesting property concerning
multiplication of elements that have the form of a total
Stiefel-Whitney class.
\begin{Lem}\label{lem:productrule:general}
  \optcite[Theorem~33, p.~91f]{milnorlectures}
  Let $k\in\Nat$, and let $A_*=\bigoplus_{i\geq 0} A_i$ be a graded ring.
  For $a,b\in A_*$ with $a_0=1=b_0$, and any partition
  $\Part\in\PartitionsOf k$ holds
  \begin{gather*}
    \s{\Part}(a\cdot b)
    = \sum_{\Part_1\concat\Part_2=\Part}
    \s{\Part_1}(a) \cdot \s{\Part_2}(b)
    \;.
  \end{gather*}
  \begin{proof}
    Just as for the definition of $\s{\Part}$, it suffices to check
    the equality on algebraically independent elements. More
    precisely, it even suffices to take $a=\sum_{i=1}^k a_i$ and
    $b=\sum_{i=1}^k b_i$ that do not exceed degree $k$, since all higher
    elements $a_i$ respectively $b_i$ are ignored during evaluation by
    definition.
    So, make the following choices:
    \begin{itemize}
    \item The result will be independent of $n$, as long as
      $n$ is large enough, thus choose $n=2k$.
      Note that this is the smallest choice for $n$ for which there
      may be a pair of sequences $J_1^k\concat J_2^k\in\Part$
      where one is all zeros.
    \item For simplicity of notation write
      $x_i=t_i$ and $y_i=t_{2i}$ for $1\leq i\leq k$ to split the
      variables $(t_1,\dotsc,t_{2k})=(x_1,\dotsc,x_k,y_1,\dotsc,y_k)$
      into two equal parts.
    \item In order to directly work with the definition of $\s{\Part}$ it is
      convenient to choose as algebraically independent elements
      \begin{align*}
        a &\coloneqq
            \prod_{\mathclap{r=1}}^k(1+t_r)
            \cequalsby{\eqref{eq:sumelemsymmpoly}}
            1 + \sum_{i=1}^k \el i k(x_1,\dotsc,x_k)
            \;\text{, and}\\
        b &\coloneqq
            \prod_{\mathclap{r=k+1}}^n(1+t_r)
            \cequalsby{\eqref{eq:sumelemsymmpoly}}
            1 + \sum_{i=1}^k \el i k(y_1,\dotsc,y_k)
            \;\text{, then}\\
        a\cdot b
          &=
            \prod_{r=1}^n (1+t_r)
            \cequalsby{\eqref{eq:sumelemsymmpoly}}
            1 + \sum_{i=1}^n \el i n (t_1,\dotsc,t_n)
            \;.
      \end{align*}
      Consequently, the sets 
      $(a_1,\dotsc,a_k)$, $(b_1,\dotsc,b_k)$, and
      $(a\cdot b)_{1\leq i\leq n}$ are algebraically independent as
      needed.
    \end{itemize}
    Now calculate
    \begin{align*}
      \s{\Part}(a\cdot b)
      &\coloneqq
        \symm{n} t^{\Part}
      \\ &\equalsby{Def.}
           \sum_{I^n\in\Part}
           t_1^{i_1}\dotsm t_n^{i_n}
      \\ &=
           \sum_{I^n\in\Part}
           x^{(i_1,\dotsc,i_k)}\cdot y^{(i_{k+1},\dotsc,i_n)}
      \\ &=
           \sum_{J_1^k\concat J_2^k\in\Part}
           x^{J_1} \cdot y^{J_2}
      \\ &\equalsby{Group by equiv.}
           \sum_{\Part_1\concat \Part_2=\Part}
           \sum_{\substack{J_1^k\in\Part_1\\ J_2^k\in\Part_2}}
      x^{J_1} \cdot y^{J_2}
      \\ &=
           \sum_{\Part_1\concat \Part_2=\Part}
           \left(\sum_{J_1^k\in\Part_1} x^{J_1}\right)
           \cdot
           \left(\sum_{J_2^k\in\Part_2} y^{J_2}\right)
      \\ &\equalsby{Def.}
           \sum_{\Part_1\concat \Part_2=\Part}
           \left(\symm{k} x^{\Part_1}\right)
           \cdot
           \left(\symm{k} y^{\Part_2}\right)
      \\ &\equalsby{Def.}
           \sum_{\Part_1\concat \Part_2=\Part}
           \s{\Part_1}\left( \el 1 k(x_1,\dotsc,x_k),\dotsc \right)
           \cdot
           \s{\Part_2}\left( \el 1 k(y_1,\dotsc,y_k),\dotsc \right)
      \\ &\equalsby{Def.}
           \sum_{\Part_1\concat \Part_2=\Part}
           \s{\Part_1}(a) \cdot \s{\Part_2}(b)
           \qedhere
    \end{align*}
  \end{proof}
\end{Lem}

\begin{Ex}
  The most important partition of an integer $k$ will be the trivial
  one $(k)\in\PartitionsOf k$. In this case
  Lemma~\ref{lem:productrule:general} says
  \begin{gather*}
    \s{(k)}(a\cdot b) = \s{(k)}(a) + \s{(k)}(b)
    \;.
  \end{gather*}
\end{Ex}


\subsection{Stiefel-Whitney Numbers of Product Manifolds}
\label{sec:swnumsofproductmfds}
In order to apply the special polynomials out of the preceding
section, as well as their product property from
Lemma~\ref{lem:productrule:general}, to the Stiefel-Whitney
numbers of (product) manifolds, first start with Stiefel-Whitney
classes.

So, let $M^n=M_1^{n_1}\times M_2^{n_2}$ all be closed manifolds of the
noted dimension throughout this section.

Recall that
\begin{enumerate}
\item the cohomology ring $\H^*(M)$ of a space respectively
  manifold is a graded ring,
\item the total Stiefel-Whitney number of a manifold is of the form
  $\W{M_i} = 1 + \w 1 {M_i} + \dotsb + \w n {M_i}$, and that
\item by Künneth holds
  \begin{align*}
    \H^*(M_1)\otimes \H^*(M_2)
    &\overset{\cong}\longto \H^*(M)
    \\
    c_1 \otimes c_2
    &\longmapsto c_1\cdot c_2
      \coloneqq \pb\proj_1 c_1 \cdot \pb\proj_2 c_2
  \end{align*}
  and $\W{M} = \W{M_1} \cdot \W{M_2}$ by
  \ref{tag:swclassesmultiplicativity} of the Stiefel-Whitney classes.
\end{enumerate}
Thus, one can apply the multiplication
rule~\ref{lem:productrule:general} to $\W{M}$, which immediately
yields:
\begin{Cor}\label{cor:productrule:swcl}
  For $M=M_1\times M_2$ manifolds as above one gets for any partition
  $\Part\in\PartitionsOf n$:
  \begin{align*}
    \s{\Part}(\W{M})
    =
    \s{\Part}\left(\W{M_1}\cdot \W{M_2}\right)
    &\cequalsby{\ref{lem:productrule:general}}
      \sum_{\mathclap{\Part_1\concat\Part_2=\Part}}
      \s{\Part_1}(\W{M_1}) \cdot \s{\Part_2}(\W{M_2})
    \\ &=
         \sum_{\mathclap{\substack{
         \Part_1\concat\Part_2=\Part\\
    \Part_1\in\PartitionsOf{n_1}\\
    \Part_2\in\PartitionsOf{n_2}\\
    }}}
    \s{\Part_1}(\W{M_1}) \cdot \s{\Part_2}(\W{M_2})
  \end{align*}
  \begin{proof}
    The last equality is due to the following dimension reasons.
    For any manifold $W$ and partition $\Part$ of $k$,
    $\s{\Part}(\W{W})$ is of degree $k$ if $k\leq\dim W$ and zero else.
    Therefore, for any combination of partitions
    $\Part_1\in\PartitionsOf{k_1}$,
    $\Part_2\in\PartitionsOf{k_2}$
    where $k_1+k_2=n=n_1+n_2$ with $k_i\neq n_i$, the product
    $\s{\Part_1}(\W{M_1})\cdot\s{\Part_2}(\W{M_2})$ will have a zero
    factor and can be skipped.
  \end{proof}
\end{Cor}

In order to pass to Stiefel-Whitney numbers instead of classes, use
the following notation.
\begin{Def}
  Let $W$ be a closed manifold and $\Part\in\PartitionsOf{\dim W}$.
  Then write
  \begin{align*}
    \s{\Part}(W) &\coloneqq \s{\Part}(\W{W})\\
    \snum{\Part}{W} &\coloneqq \capped{\s{\Part}(W)}{\fundcl{W}}
    \;.
  \end{align*}
\end{Def}

Now the product rule from Lemma~\ref{lem:productrule:general} translates to
\begin{Cor}\label{cor:swnumdecompositionmfds}
  For closed manifolds $M_1$, and $M_2$ with dimensions $n_1$ and
  $n_2$ one has
  \begin{align*}
    \snum{\Part}{M_1\times M_2}
    &= \sum_{\mathclap{\substack{
      \Part_1\concat\Part_2=\Part\\
    \Part_1\in\PartitionsOf{n_1}\\
    \Part_2\in\PartitionsOf{n_2}\\
    }}}
    \snum{\Part_1}{M_1} \cdot \snum{\Part_2}{M_2}
    \quad \in \quad \Zmod2
  \end{align*}
  and as a special case
  \begin{align}
    % \notag
    % \snum{(n_1,n_2)}{M_1\times M_2}
    % &= \snum{(n_1)}{M_1} \cdot \snum{(n_2)}{M_2}
    % \\
    \label{eq:productrule:swnum}
    \snum{(n_1+n_2)}{M_1\times M_2} &= 0
                                      \;.
  \end{align}
  In particular, if $\snum{(\dim W)}{W}\neq 0$ for a closed manifold
  $W$, then $W$ is no product manifold.
  \begin{proof}
    The statement immediately follows from the previous
    Corollary~\ref{cor:productrule:swcl} with the following
    two facts:
    \begin{enumerate}
    \item The generator $\fundcl{M_1\times
        M_2}\in\H^{n_1+n_2}(M_1\times M_2)$ corresponds to the generator
      $\fundcl{M_1}\otimes\fundcl{M_2}\in
      \H^{n_1}(M_1)\otimes\H^{n_2}(M_2)\cong \H^{n_1+n_2}(M_1\times M_2)$
      under the Künneth isomorphism.
    \item For cohomology classes $c_1\in\H^{n_1}(M_1)$, $c_2\in\H^{n_2}(M_2)$ holds
      \begin{gather*}
        \capped{c_1\otimes c_2}{\fundcl{M_1\times M_2}}
        = \capped{c_1\otimes c_2}{\fundcl{M_1}\otimes\fundcl{M_2}}
        = \capped{c_1}{\fundcl{M_1}} \cdot \capped{c_2}{\fundcl{M_2}}
        \in \Zmod2
      \end{gather*}
      by the universal property of the tensor product.
      \qedhere
    \end{enumerate}    
  \end{proof}
\end{Cor}

\autoref{eq:productrule:swnum} is the desired obstruction for
a manifold to be a product or---as will be explained in the next
subsections---to be cobordant to a product.
Finally, this statement will be an invaluable tool for detecting
manifolds that are not only not cobordant to a product manifold but
whose cobordism class is indecomposable.

\subsection{Review: The Cobordism Ring Structure}
\label{sec:cobordismringstructure}
Recall that two closed manifolds of the same dimension $n$ are
(unoriented) cobordant if their disjoint union is the border of an
$(n+1)$-dimensional manifold.
This is an equivalence relation amongst $n$-manifolds, and the
set of equivalence classes forms an Abelian group $\c_n$ of order two
with the disjoint sum as addition and the $n$-sphere as zero element.
The Cartesian product turns the graded $\Zmod2$-module
$\c_*\coloneqq \bigoplus_{n\geq 0}\c_n$ into an $\Zmod2$-algebra
called the (unoriented) cobordism ring.
Denote the cobordism equivalence class of a manifold $M$ by $[M]$.

Most remarkably, the cobordism relation is homotopy invariant, \idest
homotopic manifolds of the same dimension will be cobordant.
Further, the structure of this algebra is well-known.
\begin{Thm}[Thom]\label{thm:cobordismringstructure}
  \optcite[Thm.~1.23]{immersionconj}
  \optcite[Theorem~IV.9]{thom}
  \begin{gather*}
    \c_*
    % \cong \pi_*(\MO)
    \cong \Zmod2[\sigma_i| i\neq 2^r-1]
  \end{gather*}
  as graded $\Zmod2$-algebras.
  \begin{proof}
    See \cite[Theorem~IV.12]{thom} or
    \autoref{sec:indecomposabilitycriterion},
    \ref{item:generatorscobordimsring}, for a proof using
    Theorem~\ref{thm:basiscobordismring} below.
  \end{proof}
\end{Thm}

During the proof of the above theorem, Thom constructs special
manifolds that form a basis of the cobordism ring,
and are uniquely characterised by the below properties.
For the formulation recall that a sequence or partition is called
\emph{non-dyadic}, if none of its entries is of the form $2^r-1$.
\begin{Thm}\label{thm:basiscobordismring}
  There exists a basis of the cobordism ring represented by manifolds
  $V_\Part$ that in each degree $k$ is indexed by non-dyadic
  partitions $\Part$ of $k$. Further, the $V_\Part$ are uniquely
  characterised by
  \begin{gather*}
    \s{\Part'}(\W{V_\Part}) = \delta_{\Part,\Part'}
    \in \H^k(V_{\Part})\cong \Zmod2
  \end{gather*}
  for any non-dyadic partitions $\Part,\Part'\in\PartitionsOf{k}$,
  where $\delta$ is the usual Kronecker delta.
  \begin{proof}
    See \cite[Section~IV.5, proof of Theorem~IV.9]{thom}.
    \TODO{Do proof / construction?} % [Milnor] better source?
  \end{proof}
\end{Thm}

In order to relate the results on the Stiefel-Whitney numbers of
manifolds---respectively certain linear combinations of them---from
before with cobordism classes, one needs the following more general
connection. Thom rather directly deduces this from the existence of
the above basis.
\begin{Thm}[Thom]\label{thm:cobordantiffswnumscoincide}
  Two closed manifolds are cobordant if and only if all of their
  Stiefel-Whitney numbers coincide.
  \begin{proof}[proof (sketch)]
    The proof that manifolds with coinciding Stiefel-Whitney numbers
    are cobordant was conducted by Thom \cite[Theorem IV.10]{thom}.
    To see that cobordant manifolds have the same Stiefel-Whitney
    numbers, let $M^n$ be a null-bordant closed manifold, \idest
    assume $M^n=\Boundary{W}$ for a closed manifold $W$. Now, consider
    any Stiefel-Whitney number 
    $\capped{\w{1}{M}^{i_1}\dotsm\w{l}{M}^{i_l}}{\fundcl{M}}$,
    of $M$, where $I=(i_1,\dotsc,i_l)$ with $\d(I)=n$.
    Then this is zero, as one calculates using the long
    exact sequence of cohomology respectively homology of the pair
    $i\colon M\immto W$ (abbreviated les):
    \begin{align*}
      \W{M}
      &= \W{\T M}
        = \W{\T M\oplus\trivbdl}
        = \W{\T W|_{M}}
        \cequalsby{Def.} \W{\pb i \T W} = \pb i \W{W} \\
      \fundcl M
      &\cequalsby{les} \partial \fundcl{W} \\
      \intertext{which yields with the fact
      $\pf i\circ\partial\overset{\text{les}}= 0$}
      \capped{\W{M}^I}{\fundcl M}
      &= \capped{\pb i\W{W}^I}{\partial\fundcl{W}}
        = \pf i\capped{\W{W}^I}{\pf i\partial \fundcl{W}}
        = 0
        \;.
        \qedhere
    \end{align*}
  \end{proof}
\end{Thm}

\subsection{A Criterion for Indecomposability}
\label{sec:indecomposabilitycriterion}
Now, one can focus on the ultimate goal of the current section, which
is to deduce the following indecomposability criterion. This
originates as a corollary from Thom's proof of the multiplicative
structure of the cobordism ring (see \cite[Section~IV.5]{thom}).
\begin{Thm}\label{thm:indecomposabilitycriterion}
  A closed $n$-manifold $M$ represents an indecomposable element of
  the cobordism ring $\c_*$, if and only if
  \begin{gather*}
    \snum{(n)}{M} \neq 0 \in \Zmod2
    \;.
  \end{gather*}
  % for non-dyadic decomposition: \optcite[p.~156]{thom};
  % for summation notation: \optcite[p.154, (4)f]{thom};
\end{Thm}

\begin{proof}[proof of Theorem~\ref{thm:indecomposabilitycriterion}]
  Using the main theorems \ref{thm:basiscobordismring} and
  \ref{thm:cobordantiffswnumscoincide} from
  \autoref{sec:cobordismringstructure}, as well as the main
  corollaries \ref{cor:productrule:swcl} and
  \ref{cor:swnumdecompositionmfds} from
  \autoref{sec:swnumsofproductmfds}, enables to obtain the
  desired result in the following steps.
  \begin{steps}
  \item\label{item:manifoldbasisrepr}
    As the classes $[V_\Part]$ form a basis of the cohomology ring by
    Theorem~\ref{thm:basiscobordismring}, any manifold $M^n$ is
    cobordant to a unique linear combination, \idest disjoint sum,
    \begin{gather*}
      [M] = \coprod_{\mathclap{\substack{
            \Part\in\PartitionsOf n \\\text{ non-dyadic}
        }}} \alpha_\Part [V_\Part]
        \;,\quad
        \alpha_\Part \in \Zmod2,
    \end{gather*}
    of the classes $[V_\Part]$.
    Now the Stiefel-Whitney numbers are determined by the cobordism
    class according to Theorem~\ref{thm:cobordantiffswnumscoincide}additive with
    and are additive with respect to disjoint sums.
    So, one gets for any Stiefel-Whitney number $\wsnum{I}{M}$ of $M$
    \begin{align*}
      \wsnum{I}{M}
      &= \sum_{\mathclap{\substack{
        \Part\in\PartitionsOf n \\\text{ non-dyadic}
      }}} \alpha_\Part \wsnum{I}{V_\Part}
      \;,
      &\text{especially}
      &&\snum{\Part'}{M}
      &= \sum_{\mathclap{\substack{
        \Part\in\PartitionsOf n \\\text{ non-dyadic}
      }}} \alpha_\Part \snum{\Part'}{V_\Part}
      \cequalsby{Def.} \alpha_{\Part'}
      \;.
    \end{align*}
    Thus, $[V_\Part]$ is a summand of $[M]$ if and only if
    $\snum{\Part}{M}$ is non-zero. In other words, $M$ is cobordant to
    \begin{gather*}
      \coprod_{\mathclap{\substack{
            \Part\in\PartitionsOf k \text{ non-dyadic}\\
            \s{\Part}(W)\neq 0
          }}} V_\Part
      \;.
    \end{gather*}
  \item\label{item:productpartitions}
    For partitions $\Part_1'$ of $n_1$, and $\Part_2'$ of $n_2$ and
    any partition $\Part'$ holds
    \begin{align*}
      \s{\Part'}(V_{\Part_1'}\times V_{\Part_2'})
      &\cequalsby{\ref{cor:productrule:swcl}}
        \sum_{\mathclap{\substack{
        \Part_1\concat\Part_2=\Part'\\
      \Part_1\in\PartitionsOf{n_1}\\
      \Part_2\in\PartitionsOf{n_2}\\
      }}}
      \s{\Part_1}(V_{\Part_1'}) \cdot \s{\Part_2}(V_{\Part_2'})
      \overset{\text{Def.}}=
        \sum_{\mathclap{\substack{
        \Part_1\concat\Part_2=\Part'\\
      \Part_1\in\PartitionsOf{n_1}\\
      \Part_2\in\PartitionsOf{n_2}\\
      }}}
      \delta_{\Part_1,\Part_1'} \cdot \delta_{\Part_2, \Part_2'}
      = \delta_{\Part', \Part_1'\concat\Part_2'}
      \\
      &\equalsby{Def.} \s{\Part'}(V_{\Part_1'\concat\Part_2'})
        \;.
    \end{align*}
    Thus, by \ref{item:manifoldbasisrepr}
    the basis representations of
    $[V_{\Part_1'}]\times[V_{\Part_2'}]=[V_{\Part_1'}\times V_{\Part_2'}]$
    and 
    $[V_{\Part_1'\concat\Part_2'}]$
    coincide, wherefore they must be equal. 
  \item\label{item:generatorscobordimsring}
    By \ref{item:productpartitions}, all basis elements
    $[V_\Part]$ can be written as a product of lower degree basis
    elements, except for those corresponding to a trivial partition
    $(k)$, $k\in\Nat$.
    Furthermore, such a basis element $[V_{(k)}]$ cannot be
    decomposable, as otherwise $\snum{(k)}{V_{(k)}} = 0$
    by \autoref{eq:productrule:swnum} in
    Corollary~\ref{cor:swnumdecompositionmfds},
    which contradicts the definition in
    Theorem~\ref{thm:basiscobordismring}.
    
    Altogether, the basis elements represented by a $k$-dimensional
    manifold $V_{(k)}\in\c_k$ for $k\neq 2^m-1$ are
    indecomposable---hence algebraically independent---generators of
    the cobordism ring.
    This is a proof of Theorem~\ref{thm:cobordismringstructure} using
    Theorem~\ref{thm:basiscobordismring}.
  \item By \ref{item:generatorscobordimsring}, the cobordism
    class of a manifold $W$ of dimension $k$ is an indecomposable
    element of $\c_*$ if and only if its unique representation by
    basis elements $[V_\Part]$ contains as a summand the unique
    $k$-dimensional indecomposable basis element $[V_{(k)}]$,
    \idest if and only if $\snum{(k)}{W}\neq 0$ by
    \ref{item:manifoldbasisrepr}.
    \qedhere
  \end{steps}
\end{proof}

This directly yields the following example which will be a key point
in constructing a candidate generating set of the cobordism ring.
\begin{Ex}\label{ex:rpnindecomposable}
  For $k\in\Nat$ even, the projective space $\RP k$ represents an
  indecomposable element of the cobordism ring.
  \begin{proof}
    $\W{\RP k} = (1+x)^{k+1}=\prod_{i=1}^{k+1}$ where $x$ is the
    generator in degree one of $\H^*(\RP k)\cong\Zmod2[x]/(x^{k+1})$,
    and thus
    \begin{align*}
      \s{(k)}(\RP k)
      &\cequalsby{Def.} \sum_{i=1}^{k+1} x^k
      = (k+1)x^k \\
      \snum{(k)}{\RP k}
      &= \capped{(k+1)x^k}{\fundcl{\RP k}}
      = k+1
        \equiv 1 \mod2
        \;.
        \qedhere
    \end{align*}
  \end{proof}
\end{Ex}

\section{Twisted Products}
\label{sec:twistedprod}
The candidates for a generating set needed for the proof of Brown's
Theorem~\ref{thm:brown} will be inductively constructed
using the so-called twisted product construction explained below.
The main advantage of this tool is---besides quite a couple of handy
preservation properties---the fact that a twisted product is
indecomposable if and only if its factor is and the dimension was
chosen correctly
(Theorem~\ref{thm:twistedprod:indecompcriterion}). The latter will
be the main result of this section, and is discussed in
\autoref{sec:twistedprod:indecompcriterion}.

\subsection{Definition}
\begin{Def}
  \optcite[p.~83]{immersionconj}
  \optcite[compare §4, Def.~of~$P(m;X)$]{brown}
  Let $X$ be a space and $k\in\Nat$ an integer.
  Define the \emph{twisted product of $X$ by $\Sphere k$}, denoted
  $\Twistedprod{k}{X}$, to be the orbit space of the properly
  discontinuous $\Zmod2$-action on $\Twistedprodcovspace{k}{X}$ given
  by
  \begin{align*}
    \Zmod2 &\leftgroupaction \Twistedprodcovspace{k}{X}
             \;,
    &[1] \actson (s, (p_1,p_2)) &\coloneqq (-s, (p_2, p_1))
                                \;,
  \end{align*}
  which combines the antipodal action $[1]\actson s\coloneqq -s$ on
  $\Sphere k$ and twisting on $X\times X$.
  For a map $f\colon X\to Y$ of spaces, define
  \begin{gather*}
    \Twistedprod{k}{f}\coloneqq (\Id\times f\times f/\sim)\colon
    \Twistedprod{k}{X}\longto\Twistedprod{k}{Y}
    \;,\quad
    [s,(p_1,p_2)] \longmapsto [s,(f(p_1),f(p_2))]
    \;.
  \end{gather*}
\end{Def}
\begin{Ex}
  Major examples needed later are
  \begin{itemize}
  \item $\Twistedprod k {\pt} = \RP k$, and
  \item $\Twistedprod 0 {M} = M\times M$.
  \end{itemize}
\end{Ex}

First, gather some rather immediate, convenient properties. It is
especially noteworthy how well the twisted product behaves concerning
manifolds and fibre bundles.
\begin{Rem}\label{rem:twistedprodproperties}
  Let $X$ be a space and $k\in\Nat$.
  \begin{enumerate}
  \item $\Twistedprod{k}{-}$ is a functor on the category of
    topological spaces preserving injectivity.
  \item\label{item:twistedprodfibrebdl}
    $\Twistedprod{k}{-}$ preserves fibre bundles,
    \idest for a fibre bundle $\xi\colon\E\xi\to X$ with fibre $F$
    the twisted product $\Twistedprod{k}{\xi}\colon
    \Twistedprod{k}{\E\xi}\to \Twistedprod{k}{X}$
    is again a fibre bundle with fibre $F\times F$.
    This comes from the fibre bundle
    \begin{gather*}
      F\times F
      \longto \Twistedprodcovspace{k}{\E\xi}
      \longto \Twistedprodcovspace{k}{X}
    \end{gather*}
    where all maps are maps of $\Zmod2$-sets.
    As a special case, $\Twistedprod{k}{X}$ admits a fibre bundle
    \begin{gather}\label{eq:twistedprodrpnfibrebdl}
      X\times X
      \longto \Twistedprod{k}{X}
      \longto \RP k = \Sphere k/\sim
    \end{gather}
    with fibre $X\times X$ which comes from the trivial fibre bundle
    $X\to\pt$.
    Further, let $\eta\colon\E\eta\to X$ be
    another fibre bundle, and $f\colon X'\to X$ a map.
    One has:
    \begin{enumerate}
    \item $\Twistedprod{k}{-}$ respects sums of fibre bundles, \idest
      \begin{gather*}
        \Twistedprod{k}{\xi\oplus\eta\colon \E\xi\oplus\E\eta\to X}
        = \Twistedprod{k}{\xi}\oplus\Twistedprod{k}{\eta}
      \end{gather*}
    \item\label{item:twistedprod:preservespb}
      $\Twistedprod{k}{-}$ respects pullbacks, \idest
      \begin{gather*}
        \Twistedprod{k}{\pb f \xi}
        = \pb{\left(\Twistedprod{k}{f}\right)}
        \left(\Twistedprod{k}{\xi}\right)
      \end{gather*}
    \end{enumerate}
  \item\label{item:twistedprodmanifold}
    $\Twistedprod{k}{-}$ preserves closed smooth manifolds, \idest
    for a closed smooth manifold $M^n$, $\Twistedprod{k}{M}$ is a
    again a $(2m+k)$-dimensional closed smooth manifold.
    This is because the proper discontinuity comes from the antipodal
    $\Zmod2$-action, and makes the projection
    \begin{gather*}
      \Twistedprodcovspace{k}{X}
      \xrightarrow{\pi}
      \Twistedprod{k}{X}
      \coloneqq
      \left( \Twistedprodcovspace{k}{X} \right)/\sim
    \end{gather*}
    a two-leaved covering space.
    Further:
    \begin{enumerate}
    \item\label{item:twistedprodpreservesimmersions}
      $\Twistedprod{k}{-}$ preserves immersions.
    \item\label{item:twistedprod:tangentspace}
      $\T\Twistedprod{k}{M}
      \cong \pb\proj\T{\RP k} \oplus \Twistedprod{k}{\T M}$,
      \idest the tangent space of $\Twistedprod{k}{M}$ can be obtained
      from $\Twistedprod{k}{\T M}$ by adding the missing tangent space
      part of the sphere:
      \begin{alignat*}{4}
        \T{\Twistedprod{k}{M}}
        &\cong& \T{\left(\Twistedprodcovspace{k}{M}\right)}/\sim \\
        &\cong& \T \Sphere k\times \T M\times\T M/\sim
        &\overset{\cong}{\longto}
        \pb\proj \T{\RP k} \oplus \Twistedprod{k}{\T M}
        \\
        &&[(s,v), (m_1,v_1), (m_2,v_2)]
        &\longmapsto
          \left( ([s], v), [s, (m_1,v_1), (m_2, v_2)] \right)
      \end{alignat*}
      where $\proj\colon\Twistedprod{k}{M}\to\RP k$ is the projection.
      The first isomorphism is due to the covering space property, and
      the last is easily seen to be a well-defined isomorphism of
      vector bundles.
      Further note that for a map of manifolds $f\colon M\to N$,
      the differential map $\Diff\Twistedprod{k}{f}$ on tangent spaces
      will be the identity on the first summand.
    \end{enumerate}
  \end{enumerate}
\end{Rem}

\subsection{The Cohomology Ring of Twisted Products}
Besides the above direct properties, there is a fairly easy
description of the cohomology ring of a twisted product relating it to
the cohomology ring of its factor.
This can be revealed inductively using a diagram of long exact
sequences which relates the cohomology of two degrees of twisted
products and known sequences of spheres.
Use the following notation.

\begin{Def}
  Let $X$ be a space and $k\in\Nat$.
  Note that by the Künneth isomorphism
  $\H^*(X^2)\cong\H^*(X)\otimes\H^*(X)$, and recall that all exact
  sequences of $\Zmod2$-vector spaces split.
  Define
  \begin{itemize}
  \item
    $\pi_k\colon \Sphere k\times X^2\to\Twistedprod k X$
    to be the quotient map from the definition (index omitted if
    obvious),
  \item
    $\proj\colon\Twistedprod k X\to \RP k$
    to be the fibre bundle map,
  \item
    $T\colon \Sphere{k}\times X^2\to\Sphere{k}\times X^2$,
    $(s,p,q)\mapsto(-s,q,p)$,
  \item
    $N\coloneqq
    \ker(x\otimes y\mapsto x\otimes y + y\otimes x)
    = \left(\left\{
        a\otimes b + b\otimes a
        \in \H^*(X\times X)
      \right\}\right)_{\Zmod2}$,
  \item
    $d\colon \H^*(X)\to\H^*(X\times X)$,
    $d(a)\coloneqq a\times a$, and
    $D\coloneqq
    \left\{ d(a)=a\otimes a \in \H^*(X\times X) \right\}$,
  \item
    $s_k\in\H^k(\Sphere k)\cong\Zmod2$ to be the generator, and
  \item
    $c_k\coloneqq \pb\proj x\in\H^1(\Twistedprod{k}{X})$ to be the
    pullback of the generator $x\in\H^*(\RP k)\cong\Zmod2[x]/(x^{k+1})$.
  \end{itemize}
  Note that $N+D$ is closed under multiplication and addition,
  and---as a first hint on the cohomology structure---$\proj$ admits a
  section  $[s]\mapsto[s,p,p]$ for any point $p\in X$, thus making
  $\H^*(\RP k)$ a direct summand of $\H^*(\Twistedprod k X)$ of the
  form $\Zmod2[c]/(c^{k+1})$.
\end{Def}

\begin{Thm}\label{thm:twistedprod:cohomstructure}
  Let $X$ be a space and $k\in\Nat$.
  Then the cohomology ring of $\Twistedprod{k}{X}$ has the form
  \begin{align*}
    \H^*(\Twistedprod{k}{X})
    &\cong
      \left(
      \Zmod2[c,s]/(c^{k+1},s^2,cs)
      \right)
      \otimes (N+D)
  \end{align*}
  with $c$ of degree 1, $s$ of degree $k$, and the additional
  properties
  \begin{enumerate}
  \item $c\otimes N=0=s\otimes D$, 
  \item $\pb\proj x = c\otimes d(1)$,
    hence
    \begin{gather*}
      \pb\proj\colon\H^*(\RP k)\cong
      \Zmod2[c]/(c^{k+1})\subset\H^*(\Twistedprod k X)
      \;,
    \end{gather*}
  \item $\pb\pi (c\otimes d(1)) = 0$, and
  \item\label{item:twistedprodcohom:pi}
    $\pb\pi (s\otimes n) = s\otimes n$,
    $\pb\pi (1\otimes (n+d(a))) = 1\otimes (n+d(a))$
    for $n\in N$ and $a\in\H^*(X)$,
    hence
    \begin{gather*}
      \pb\pi\colon (1\otimes(N+D)) + (s\otimes N)
      \cong (1\otimes(N+D))\oplus (s_k\otimes N)
      \;.
    \end{gather*}
  \end{enumerate}
  For readability skip $1\otimes-$ and $-\otimes d(1)$ in element
  notation wherever it is clear to which part the summand
  belongs.
\end{Thm}

\begin{Cor}
  Some immediate consequences from
  Theorem~\ref{thm:twistedprod:cohomstructure} are:
  \begin{enumerate}
  \item
    For any section $s_p\colon\RP k\to\Twistedprod{k}{X}$, $q\mapsto[q,p,p]$,
    of the fibre bundle described in
    \eqref{eq:twistedprodrpnfibrebdl}, and $n_1,n_2\in N$, $0\neq a\in\H^*(X)$ holds
    \begin{align}\label{eq:twistedprodcohom:section}
      \pb s_p(c) &= x
      &\text{and}&
      &\pb s_p(c\otimes d(a) + 1\otimes n_1 + s\otimes n_2) &= 0
      \;.
    \end{align}
  \item\label{item:twistedprod:preservescohominj}
    $\Twistedprod{k}{-}$ preserves injectivity on cohomology.
  \end{enumerate}
  \begin{proof}
    The section property is clear from the behaviour of $\pb\proj$.
    Consider a map $f\colon X\to Y$ which is injective on cohomology
    and induces the map
    $F\colon\Sphere{k}\times X^2\to\Sphere{k}\times Y^2$.
    Since every element in $\H^*(\RP k)\otimes D$ can uniquely be
    written as $\sum_{i=0}^k c^i\cdot d(a_i)$,
    one only has to check injectivity of $\pb{\Twistedprod k f}$ on
    the two parts
    \begin{gather*}
      \left( \Zmod2[c]/(c^{k+1})\otimes 1 \right)
      \qquad\text{and}\qquad
      \left(
        (1\otimes (N+D)) + (s\otimes N)
      \right)
      \;.
    \end{gather*}
    \begin{description}
    \item[First part:]
      Since $\Twistedprod k f$ is a morphism of fibre bundles over
      $\RP k$,
      \begin{gather*}
        \pb{\Twistedprod k f}(c)
        = \pb{\Twistedprod k f}(\pb\proj(x))
        = \pb\proj(x)
        = c
        \in\H^*(\Twistedprod k X)
        \;,
      \end{gather*}
      so $\pb{\Twistedprod k f}$ is injective on
      $\H^*(\RP k)\otimes 1\subset\H^*(\Twistedprod k Y)$.
    \item[Second part:]
      Obviously
      $\pb F\colon\H^*(\Sphere k\times Y^2)\to\H^*(\Sphere k\times X^2)$ 
      will be injective. The isomorphism property of $\pb\pi$ hence
      implies injectivity of $\pb{\Twistedprod k f}$ on
      $(1\otimes D)+(1\otimes N)+(s_k\otimes N)\subset\H^*(\Twistedprod k Y)$.
      \qedhere
    \end{description}
  \end{proof}
\end{Cor}

This then immediately gives a result on the Stiefel-Whitney classes of
a twisted product of line bundles that will be used below to simplify
calculations involving higher dimensional bundles.
\begin{Cor}\label{cor:twistedprod:swlinebdl}
  \optcite[Prop.~7.4, p.~1113]{brown}
  Let $\xi\colon E\to X$ be a line bundle with total Stiefel-Whitney
  class $\W{\xi}=1+\alpha$.
  Define $e\colon\H^*(X)\to N\subset\H^*(X\times X)$,
  $e(a)\coloneqq 1\otimes a+a\otimes 1$.
  Then
  \begin{gather*}
    \W{\Twistedprod{k}{\xi}} = 1+ (c\otimes d(1)+1\otimes
    e(\alpha)) + 1\otimes d(\alpha)
    = 1+c+e(\alpha)+d(\alpha)
    \;,
  \end{gather*}
  respectively $\w1\xi = c+e(\alpha)$, $\w2\xi=d(\alpha)$.
  \begin{proof}
    For $k=0$ this is simply the product rule for the total
    Stiefel-Whitney class because $\Twistedprod{0}{\xi}=\xi\times\xi$
    and $c=0$. Thus, assume $k\geq0$.
    Then, by Theorem~\ref{thm:twistedprod:cohomstructure}, the total
    Stiefel-Whitney class of the two-dimensional vector bundle
    $\Twistedprod{k}{\xi}$ must be of the general form
    \begin{align*}
      \W{\Twistedprod{k}{\xi}}
      &=
        1 &\quad&&&{(=\ws0)} \\
      &+ \delta_1\cdot c\otimes d(1) + \delta_2\cdot c^2\otimes d(1)
          &&\delta_1,\delta_2\in\{0,1\}
            &&{(\text{check section})}\\
      &+ 1\otimes d(a)
          &&a \neq 1
                &&{(\text{check $\pi$})} \\
      &+ {\textstyle \sum_{r\in I} c^{i_r}\otimes d(a_r)}
          &&a_r\neq 1,\; i_r>1
            &&{(=0\text{ by dim.})} \\
      &+ 1\otimes n_1 + s\otimes n_2
          &&n_1, n_2\in N
                &&{(\text{check $\pi$})}
    \end{align*}
    The index set $I$ must be empty as any $c^{i_r}\otimes d(a_r)$
    would have dimension greater two.
    So, it remains to check the pullbacks of $\W{\Twistedprod{k}{\xi}}$
    along a section $s_p\colon\RP k\to\Twistedprod{k}{X}$ and along
    the projection
    $\pi\colon\Twistedprodcovspace{k}{X}\to\Twistedprod{k}{X}$
    in order to identify the other summands.
    \begin{description}
    \item[$\pb\pi\W{\Twistedprod k \xi}$:]
      $\Twistedprod{k}{\xi}$ is the quotient of the bundle
      $\trivbdl\times\xi\times\xi\colon
      \Twistedprodcovspace{k}{E}\to\Twistedprodcovspace{k}{X}$, where
      \begin{gather*}
        \W{\Id\times\xi\times\xi}
        = 1\cdot \W{\xi}\cdot\W{\xi}
        = 1 + 1\otimes e(\alpha) + 1\otimes d(\alpha)
        \;.
      \end{gather*}
      As $\pi$ is a covering map,
      $\pb\pi\Twistedprod{k}{\xi}=\trivbdl\times\xi\times\xi$,
      and thus
      $\pb\pi\W{\Twistedprod{k}{\xi}}=\W{\Id\times\xi\times\xi}$.
      By \itemref{thm:twistedprod:cohomstructure}{item:twistedprodcohom:pi} $\pb\pi$ is the identity on
      elements of this form, so this yields the third
      summand of $\W{\Twistedprod{k}{\xi}}$.
    \item[$\pb s_p \W{\Twistedprod{k}{\xi}}$:]
        The pullback of $\Twistedprod{k}{\xi}$ along this section is
        \begin{gather*}
          \Twistedprod{k}{E_p}=\Twistedprod{k}{\R}\to \RP k
          \;,\quad
          [s, v_1,v_2] \mapsto [s]
          \;.
        \end{gather*}
        In order to simplify this, use the well-defined vector bundle
        isomorphism
        \optcite[Prop.~4.3,p~1107]{brown}
        \begin{align*}
          \Twistedprod{k}{\R}
          &\overset{\sim}\longto
            (\Twistedprodcovspace{k}{\R}/\approx)
            \cong \E{(\gamma\oplus\trivbdl)} \\
          [s, v_1, v_2]
          &\longmapsto
            [s, v_1+v_2, v_1-v_2]
        \end{align*}
        where $\approx$ is the equivalence relation identifying
        $(s,v_1,v_2)$ and $(-s,-v_1,v_2)$, and
        $\gamma$ is the tautological line bundle for $\RP k$.
        Then $\pb{s_p}\Twistedprod{k}{\xi} \cong
        \gamma\oplus\trivbdl$ and $\pb{s_p}\W{\xi} =
        \W{\gamma\oplus\trivbdl} = 1+x$, completing the proof.
        \qedhere
      \end{description}
    \end{proof}
\end{Cor}

The rest of this subsection is dedicated to the proof of
Theorem~\ref{thm:twistedprod:cohomstructure} on the cohomology
structure of twisted products.

The essential step is to find the relation of a twisted product to
\begin{enumerate}[1.]
\item its lower dimensional counterpart analogous to the embedding of
  a projective space into a higher dimensional one, and
\item well-known sequences of spheres and disks
\end{enumerate}
in a useful way.
This can be done using an alternative pushout construction, which is
explained below. The proofs are omitted since all facts are easy to check.
\begin{Fact}
  Let $X$ again be a space and $k\in\Nat_{\geq1}$.
  \begin{enumerate}
  \item
    The twisted product $\Twistedprod{k}{X}$ is the pushout
    $(\Disk k\times X^2)\cup_{T}(\Sphere{k-1}\times X^2)$ of
    \begin{center}
      \begin{tikzcd}
        \Sphere{k}\times X^2
        &\ar[from=l,leftarrow, "T"]
        \Sphere{k}\times X^2
        \ar[r, rightarrowtail, "\incl"]
        &\Disk{k}\times X^2
      \end{tikzcd}
    \end{center}
    \idest it is the product $\Disk k\times X^2$ of the closed
    $k$-disk with $X^2$ with the identification
    $(s,p,q)=T((s,p,q))\coloneqq(-s,q,p)$ on all boundary points in
    $\Boundary{\Disk k\times X^2}=\Sphere{k-1}\times X^2$.
  \item
    The pushouts
    \begin{align*}
      \Twistedprod{k}{X}
      &=(\Disk k\times X^2)\cup_{T}(\Sphere{k-1}\times X^2)
      \qquad\text{and}\\
      \Twistedprod{k-1}{X}
      &=(\Sphere{k-1}\times X^2)\cup_{T}(\Sphere{k-1}\times X^2)
    \end{align*}
    merge to the commutative pushout diagram
    \begin{center}
      \begin{tikzcd}
        \Sphere{k-1}\times X^2
        \ar[r, equals]
        \ar[d, "T"]
        &\Sphere{k-1}\times X^2
        \ar[r, rightarrowtail, "\incl"]
        \ar[d, "\pi"]
        &\Disk{k}\times X^2
        \ar[d]
        \\
        \Sphere{k-1}\times X^2
        \ar[r, "\pi"]
        &\Twistedprod{k-1}{X}
        \ar[r, rightarrowtail, "\incl"]
        &\Twistedprod{k}{X}
      \end{tikzcd}
    \end{center}
    making $(\Twistedprod{k}{X}, \Twistedprod{k-1}{X})$ a neighbourhood
    deformation retract pair using the stability of cofibrations under pushout.
    Furthermore, for a smaller disk $\Disk{k}_+\subsetneq\Disk k$
    there is an obvious excision
    \begin{gather*}
      (\Disk k\times X^2, \Sphere{k-1}\times X^2)
      \rightarrowtail
      (\Twistedprod{k}{X}, \overline{\Twistedprod{k-1}{X}})
    \end{gather*}
    where
    $\overline{\Twistedprod{k-1}{X}}\coloneqq
    ((\Disk k\setminus\Disk{k}_+)\times X^2)
    \cup_{T}
    (\Sphere{k-1}\times X^2)$
    is the embedded
    $\Twistedprod{k-1}{X}$ with a collar, \idest a neighbourhood
    deformation retract of $\Twistedprod{k-1}{X}$ in
    $\Twistedprod{k}{X}$.
  \item
    The excision
    \begin{gather*}
      (\Disk{k}_+,\Sphere{k-1}_+) \amalg (\Disk{k}_-,\Sphere{k-1}_-)
      \immlongto
      (\Sphere k, \Sphere{k-1}\times I)
    \end{gather*}
    of an upper and a lower polar cap into the sphere relative to its
    bloated equator, is compatible with the excision from above
    in the sense that the following diagram commutes:
    \begin{center}
      \begin{tikzcd}
        (\Disk{k}_+,\Sphere{k-1}_+) \amalg (\Disk{k}_-,\Sphere{k-1}_-)
        \ar[r, hookrightarrow, "\text{excis.}"]
        \ar[d, "\pi"]
        &(\Sphere k, \Sphere{k-1}\times I)
        \ar[r, dash, "\simeq"]
        \ar[d, "\pi"]
        &(\Sphere k, \Sphere{k-1})
        \ar[d, "\pi"]
        \\
        (\Disk{k}, \Sphere{k-1})\times X^2
        \ar[r, hookrightarrow, "\text{excis.}"]
        &(\Twistedprod{k}{X},\overline{\Twistedprod{k-1}{X}})
        \ar[r, dash, "\simeq"]
        &(\Twistedprod{k}{X}, \Twistedprod{k-1}{X})
      \end{tikzcd}
    \end{center}
  \end{enumerate}
\end{Fact}

This now nicely fits into a larger diagram of pairs of spaces, which
induces a diagram of long exact sequences of cohomology.
\begin{Fact}
  For a space $X$ and $k\in\Nat_{\geq1}$ the following diagram commutes
  \begin{center}
    \begin{tikzcd}[column sep=small]
      (\Disk{k}_+,\Sphere{k-1}_+)\times X^2
      \ar[r,hookrightarrow,"\tau"]
      \ar[rr, equals, bend left=30]
      &\displaystyle\coprod_{+,-} (\Disk{k},\Sphere{k-1})\times X^2
      \ar[r,"\pi"]
      &(\Disk{k}_+,\Sphere{k-1}_+)\times X^2
      \\
      &
      \ar[from=u, hookrightarrow, "\text{excision}"]
      (\Sphere{k},\Sphere{k-1}\times I)\times X^2
      \ar[r, "\pi"]
      &
      \ar[from=u, hookrightarrow, "\text{excision}"]
      \left(
        \Twistedprod{k}{X},
        \overline{\Twistedprod{k-1}{X}}
        % \Twistedprod{k-1}{X}\cup_{\Sphere{k-1}_+} (\Sphere{k-1}\times I)
      \right)
      \\
      &
      \ar[u, dash, "\simeq"]
      (\Sphere{k},\Sphere{k-1})\times X^2
      \ar[r, "\pi_k"]
      &
      \ar[u, dash, "\simeq"]
      (\Twistedprod{k}{X},\Twistedprod{k-1}{X})
      \ar[r, "\proj"]
      &(\RP k, \RP{k-1})
      \\
      \ar[uuu, hookrightarrow, "\tilde j"]
      \Disk{k}_+\times X^2
      \ar[r, hookrightarrow, "\iota"]
      &
      \ar[u, hookrightarrow, "\hat j"]
      (\Sphere k\times X^2)
      \ar[r, "\pi_k"]
      &
      \ar[u, hookrightarrow, "j"]
      \Twistedprod{k}{X}
      \ar[r, "\proj"]
      &
      \ar[u, hookrightarrow]
      \RP k
      \\
      \ar[u, hookrightarrow, "\tilde i"]
      \Sphere{k-1}\times X^2
      \ar[r, equals]
      &
      \ar[u, hookrightarrow, "\hat i"]
      \Sphere{k-1}\times X^2
      \ar[r, "\pi_{k-1}"]
      &
      \ar[u, hookrightarrow, "i"]
      \Twistedprod{k-1}{X}
      \ar[r, "\proj"]
      &
      \ar[u, hookrightarrow]
      \RP{k-1}
    \end{tikzcd}
  \end{center}
  resulting in the commutative diagram of exact cohomology sequences
\begin{center}
  \begin{tikzcd}[column sep=small]
    \vdots\ar[d]&\vdots\ar[d]&\vdots\ar[d]\\
    \H^l((\Disk{k},\Sphere{k-1})\times X^2)
    \ar[from=r, "\pb\tau"]
    &\H^l((\Sphere{k},\Sphere{k-1})\times X^2)
    \ar[from=r, "\pb{\pi_k}"]
    &\H^l((\Disk{k},\Sphere{k-1})\times X^2)
    \\
    \ar[from=u, "\pb{\tilde j}"]
    \H^l(\Disk{k}\times X^2)
    \ar[from=r, "\pb{\iota}"]
    &
    \ar[from=u, "\pb{\hat j}"]
    \H^l(\Sphere k\times X^2)
    \ar[from=r, "\pb{\pi_k}"]
    &
    \ar[from=u, "\pb j"]
    \H^l(\Twistedprod{k}{X})
    \\
    \ar[from=u, "\pb{\tilde i}"]
    \H^l(\Sphere{k-1}\times X^2)
    \ar[r, equals]
    &
    \ar[from=u, "\pb{\hat i}"]
    \H^l(\Sphere{k-1}\times X^2)
    \ar[from=r, "\pb{\pi_{k-1}}"]
    &
    \ar[from=u, "\pb i"]
    \H^l(\Twistedprod{k-1}{X})
    \\
    \ar[from=u, "\tilde{\delta}"]
    \H^{l-1}((\Disk{k},\Sphere{k-1})\times X^2)
    \ar[from=r, "\pb\tau"]
    &
    \ar[from=u, "\hat{\delta}"]
    \H^{l-1}((\Sphere{k},\Sphere{k-1})\times X^2)
    \ar[from=r, "\pb{\pi_k}"]
    &
    \ar[from=u, "\delta"]
    \H^{l-1}((\Disk{k},\Sphere{k-1})\times X^2)
    \\
    \vdots\ar[from=u]&\vdots\ar[from=u]&\vdots\ar[from=u]
    \end{tikzcd}
  \end{center}
  where $\pb\tau\pb\pi$ is the identity.
\end{Fact}

% TODO: Maybe in Appendix?
The useful thing about the cohomology diagrams arising from the
above diagram of pairs of spaces is that most of the columns and
sideways maps are well-known. Some facts are listed below.
\begin{Fact}
  Let $X$ again be a space and $k\in\Nat$.
  \begin{enumerate}
  \item
    The cohomology sequences of the pairs
    $(\Disk{k},\Sphere{k-1})\times X^2$ and
    $(\Sphere{k},\Sphere{k-1})\times X^2$ are well-known from the
    sequences of the pairs $(\Disk{k},\Sphere{k-1})$ and $(\Sphere{k},\Sphere{k-1})$.
    For $k>1$ they are given for $l\in\Nat$:
    \begin{center}
      \newcommand*{\argument}[1]{{\scriptstyle #1}}
      \begin{tikzcd}[row sep=tiny, column sep=small]
        &\argument{(1\otimes a,s_{k-1}\otimes b)}
        \ar[r, mapsto]
        &\argument{(s_k\otimes b, s_k\otimes b)}
        \\
        \argument{(1\otimes a, s_k\otimes b)}
        \ar[r, mapsto]
        &\argument{(1\otimes a, 0)}
        &\argument{(s_k\otimes a, s_k\otimes b)}
        \ar[r, mapsto]
        &\argument{(0, s_k\otimes (a+b))}
        \\
        \H^{l-1}(\Sphere k\times X^2)
        \ar[r, "\pb{\hat i}"]
        &\H^{l-1}(\Sphere{k-1}\times X^2)
        \ar[r, "\hat{\delta}"]
        &\H^l((\Sphere{k},\Sphere{k-1})\times X^2)
        \ar[r, "\pb{\hat j}"]
        &\H^l(\Sphere k\times X)
        \\
        \Splitlineoplus{1\otimes\H^{l-1}(X^2)}{s_k\otimes\H^{l-k-1}(X^2)}
        \ar[u, equals]
        &
        \Splitlineoplus{1\otimes\H^{l-1}(X^2)}{s_{k-1}\otimes\H^{l-k}(X^2)}
        \ar[u, equals]
        \ar[ddddd, equals]
        &
        \Splitlineoplus{s_k\otimes\H^{l-k}(X^2)}{s_k\otimes\H^{l-k}(X^2)}
        \ar[u, equals]
        \\
        \argument{(1\otimes a, s_k\otimes b)}
        \ar[dd, mapsto, "\pb{\iota}"]
        &&\argument{(s_k\otimes a, s_k\otimes b)}
        \ar[dd, mapsto, "\pb{\tau}"]
        \\~\\
        \argument{a}
        && \argument{s_k\otimes a}
        \\
        \H^{l-1}(X^2)
        \ar[d, equals]
        &
        &
        s_k\otimes\H^{l-k}(X^2)
        \ar[d, equals]
        \\
        \H^{l-1}(\Disk k\times X^2)
        \ar[r, "\pb{\tilde i}"]
        &\H^{l-1}(\Sphere{k-1}\times X^2)
        \ar[r, "\tilde{\delta}"]
        &\H^l((\Disk{k},\Sphere{k-1})\times X^2)
        \ar[r, "\pb{\tilde j}", "=0"{below}]
        &\H^l(\Disk k\times X)
        \\
        \argument{a}
        \ar[r, mapsto]
        &\argument{(1\otimes a, 0)}
        \\
        &\argument{(1\otimes a,s_{k-1}\otimes b)}
        \ar[r, mapsto]
        &\argument{s_k\otimes b}
      \end{tikzcd}
    \end{center}
    For $k=1$ one has the modifications
    \begin{center}
      \begin{tikzcd}[row sep=tiny]
        &(1\otimes a, s_0\otimes b)
        \ar[r, mapsto, "\hat{\delta}"]
        &(s_1\otimes(a+b), s_1\otimes(a+b))
        \\
        (1\otimes a, s_1\otimes b)
        \ar[r, mapsto, "\pb{\hat i}"]
        &(1\otimes a, s_0\otimes a)
        \\
        a
        \ar[r, mapsto, "\pb{\tilde i}"]
        &(1\otimes a, s_0\otimes a)
        \\
        &(1\otimes a, s_0\otimes b)
        \ar[r, mapsto, "\tilde{\delta}"]
        &s_1\otimes(a+b)
      \end{tikzcd}
    \end{center}
  \item
    Furthermore, it is easily seen that
    \begin{align*}
      \pb\pi\colon
      \H^l((\Disk k, \Sphere{k-1})\times X^2)
      &\to \H^l((\Sphere k, \Sphere{k-1})\times X^2)
      \\
      s_k\otimes x\otimes y
      &\mapsto
        (s_k\otimes x\otimes y, s_k\otimes y\otimes x)
    \end{align*}
  \item 
    It is known respectively easily seen that
    \begin{center}
      \begin{tikzcd}[row sep=small, column sep=small]
        &x^k \ar[r, mapsto]
        \ar[ddl, mapsto, bend right=30]
        &x^k,\quad
        x \ar[r, mapsto]
        &x
        \\
        &\H^*(\RP k, \RP k-1)
        \ar[r]
        \ar[d, "\pb\proj"{near start}]
        &\H^*(\RP k)
        \ar[r]
        &\H^*(\RP{k-1})
        \\
        s_k\otimes 1\otimes 1
        &\H^*((\Disk k, \Sphere{k-1})\times X^2)
      \end{tikzcd}
    \end{center}
  \end{enumerate}
\end{Fact}

Some further immediate results are:
\begin{Cor}
  \begin{enumerate}
  \item 
    $\pb\pi\pb j
    = \pb{\hat j}\pb\pi\colon
    s_k\otimes(x\otimes y)\mapsto s_k\otimes(x\otimes y+y\otimes x)$
  \item $\delta = (\pb\tau\pb\pi)^{-1}\tilde\delta\pb\pi = \tilde\delta\pb\pi$
  \item $\pb j(s_k\otimes 1\otimes 1) = c^k \coloneqq \pb\proj(x^k)$
  \item $\pb i(c_k) = \pb i(c_{k-1})$
    % TODO: other immediate results
  \end{enumerate}
\end{Cor}
  
We are going to inductively proof the following reformulation of
Theorem~\ref{thm:twistedprod:cohomstructure}.
\begin{Thm}\label{thm:twistedprod:cohomstructure:alt}
  For a space $X$ and $k\in\Nat$, $\H^*(\Twistedprod k X)$ is the group
  \begin{align*}
    \H^*(\Twistedprod{k}{X})
    &\cong
      \left(\bigoplus_{i=1}^k c^i\otimes D\right)
      \oplus (1\otimes(N+D))
      \oplus (s\otimes N)
    \\
    &\cong
       \left(\bigoplus_{i=1}^k c^i\otimes D\right)
      \oplus (1\otimes D)
      \oplus (1\otimes N)
      \oplus (s\otimes N)
    \\
    &\cong
      \left(\H^*(\RP k) \otimes D \right)
      \oplus \left(\H^*(\Sphere k) \otimes N\right)
  \end{align*}
  with the additive relation $c\otimes d(a)+c\otimes d(b)=c\otimes d(a+b)$ for
  $a,b\in\H^*(X)$, and which is equipped with a multiplication determined by:
  \begin{enumerate}
  \item\label{twistedprodcohom:proof:proj}
    $\pb\proj$ is a ring isomorphism onto its image and looks like
    \begin{align*}
      \pb\proj\colon \H^*(\RP k) &\to \H^*(\RP k)\otimes d(1)
      \subset \H^*(\Twistedprod k X),
      &x^i &\mapsto c^i\otimes d(1)
    \end{align*}
    \idest
    $(c^i\otimes d(1))\cdot(c^j\otimes d(1)) = c^{i+j}\otimes d(1)$
    for $0\leq i+j\leq k$, and $0$ else.
  \item
    $(c^i\otimes d(b)) \cdot (1\otimes d(a)) = c^i\otimes d(ab)$,
    so
    $c^i\otimes d(a)
    \cequalsby{\ref{twistedprodcohom:proof:proj}}
    (c\otimes d(1))^i\cdot (1\otimes d(a))$.
  \item 
    $(c\otimes d(1)) \cdot (1\otimes N) = 0$.
  \item 
    $(c\otimes d(1)) \cdot (s_k\otimes N) = 0$.
  \item The following restriction of $\pb\pi$ is a ring isomorphism onto
    its image
    \begin{align*}
      \pb\pi\colon
      (1\otimes D)\oplus (1\otimes N)\oplus(s_k \otimes N)
      &\longto
        (1\otimes(D+N))\oplus(s_k\otimes N)
        \subset\H^*(\Sphere k\times X^2)
      \\
      1\otimes d + 1\otimes n_1 + s_k\otimes n_2
      &\longmapsto
        1\otimes(d+n_1) + s_k\otimes n_2
        \;.
    \end{align*}
    % \idest for $f,g\in N+D$, $n,n'\in N$:
    % \begin{enumerate}
    % \item $(1\otimes f) \cdot (1\otimes g)= 1\otimes fg$,
    % \item $(1\otimes f) \cdot (s\otimes g)= s\otimes fg$, and
    % \item $(s\otimes n)\cdot (s\otimes n')=0$.
    % \end{enumerate}
  \item And finally, even though not necessary for determining the
    multiplicative structure,
    $\pb\pi(c\otimes d(1)) = 0$,
    respectively with the above
    \begin{gather*}
      \ker(\pb\pi)
      = (c\otimes d(1))\cdot \H^*(\Twistedprod k X)
      = \left(\bigoplus_{i=1}^k c^i\otimes D\right)
      \;.
    \end{gather*}
  \end{enumerate}
\end{Thm}
\begin{proof}[proof of
  Theorem~\ref{thm:twistedprod:cohomstructure:alt}
  respectively Theorem~\ref{thm:twistedprod:cohomstructure}]
  \TODO{Do proof of \ref{thm:twistedprod:cohomstructure}?} % or explicitly say that it's missing!
  % TODO: Idea/outline of proof
  Compare also \cite[Theorem~7.1]{brown}.    
\end{proof}


\subsection{Indecomposability of Twisted Product Cobordism Classes}
\label{sec:twistedprod:indecompcriterion}
Now that the cohomology ring and Stiefel-Whitney classes of line
bundles of a twisted product are well-known, one can investigate
general Stiefel-Whitney numbers of twisted product manifolds.
As promised, the following result will be the cornerstone when inductively
defining the desired basis for the cobordism ring in the subsequent section.
\begin{Thm}\label{thm:twistedprod:indecompcriterion}
  Let $M^n$ be a manifold and $k\in\Nat_{\geq1}$.
  Then $\Twistedprod{k}{M}$ represents an indecomposable class of the
  cobordism ring if and only if $M$ does and $\binom{k+n-1}{n}$ is
  non-zero modulo two.
\end{Thm}
Recall Theorem~\ref{thm:indecomposabilitycriterion}, saying
a manifold $M^n$ represents an indecomposable element if and only if
$\snum{(n)}{M}$ is non-zero,
and recall that $\Twistedprod{k}{-}$ preserves injectivity on
cohomology by
Theorem~\itemref{thm:twistedprod:cohomstructure}{item:twistedprod:preservescohominj}.
Then Theorem~\ref{thm:twistedprod:indecompcriterion} is a
direct consequence of the following Lemma.
\begin{Lem}\label{lem:twistedprod:indecompcriterion}
  Let $M$, $n$, $k$ be as in
  Theorem~\ref{thm:twistedprod:indecompcriterion} above.
  Then there is a map $f\colon X\to M$ of spaces which is injective on
  cohomology and fulfils
  \begin{gather*}
    \pb{\Twistedprod{k}{f}} \s{(2n+k)}(\Twistedprod{k}{M})
    = \binom{k+n-1}{n} \cdot c^k
    \cdot d\left( \pb f \s{(n)}(M) \right)
    \in \H^{2n+k}(\Twistedprod{k}{M})
    \;.
  \end{gather*}
\end{Lem}
\begin{proof}[proof of Lemma~\ref{lem:twistedprod:indecompcriterion}]
  As the Stiefel-Whitney classes of twisted products of line bundles
  are well-known, take $f$ to be a reduction of $\T M$ to line bundles
  using the splitting principle.
  \Idest choose a space $X$ and a map $f\colon X\to M$ which is
  injective on cohomology and fulfils
  $\pb f \T M = \xi_1\oplus\dotsb\oplus\xi_n$ for line bundles
  $\xi_i$ over $X$, each with total Stiefel-Whitney class
  $\W{\xi_i}=1+\alpha_i$.
  With the fibre bundle properties from
  Remark~\itemref{rem:twistedprodproperties}{item:twistedprodfibrebdl}%
  \ref{item:twistedprod:preservespb}
  and the tangent space structure from
  Remark~\itemref{rem:twistedprodproperties}{item:twistedprodmanifold}%
  \ref{item:twistedprod:tangentspace}
  this yields on vector bundles:
  \begin{align*}
    \pb{\Twistedprod{k}{f}} \Twistedprod{k}{\T M}
    &\cequalsby{\ref{rem:twistedprodproperties}}
      \Twistedprod{k}{\pb f \T M}
      = \Twistedprod{k}{\bigoplus_{i\leq n}\xi_i}
      = \bigoplus_{i\leq n}\Twistedprod{k}{\xi_i}
    \\
    \pb{\Twistedprod{k}{f}} \T{\Twistedprod{k}{M}}
    &\cequalsby{\ref{rem:twistedprodproperties}}
      \pb{\Twistedprod{k}{f}} \left(
      \T{\RP k} \oplus \Twistedprod{k}{\T M}
      \right)\\
    &= \pb{\Twistedprod{k}{f}} \T{\RP k}
      \oplus
      \pb{\Twistedprod{k}{f}} \Twistedprod{k}{\T M}
      = \T{\RP k} \oplus \bigoplus_{i\leq n}\Twistedprod{k}{\xi_i}\\
    \intertext{And on Stiefel-Whitney classes:}
    \pb{\Twistedprod{k}{f}} \W{\Twistedprod{k}{\T M}}
    &= \prod_{i\leq n} \W{\Twistedprod{k}{\xi_i}}
      \cequalsby{\ref{cor:twistedprod:swlinebdl}}
      \prod_{i\leq n} \left(1 + c + e(\alpha_i) + d(\alpha_i)\right)
    \\
    \pb{\Twistedprod{k}{f}} \W{\T\Twistedprod{k}{M}}
    &= \pb{\Twistedprod{k}{f}}\W{\T{\RP k}}
      \cdot \prod_{i\leq n} \W{\Twistedprod{k}{\xi_i}} \\
    &= (c+1)^{k+1}
      \cdot \prod_{i\leq n} \left(1+c+e(\alpha_i)+d(\alpha_i)\right)
  \end{align*}
  In order to work with these Stiefel-Whitney class expressions as
  symmetric polynomials, introduce variables $u_i, v_i$ of degree
  one such that
  \begin{align*}
    \w1{\xi_i} &= c+e(\alpha_i) = \el 1 2(u_i, v_i) = u_i+v_i \\
    \w2{\xi_i} &= d(\alpha_i)   = \el 2 2(u_i, v_i) = u_iv_i \\
    \W{\xi_i}  &= 1+c+e(\alpha_i)+d(\alpha_i)
                 = 1+\sum_{i\leq2}\el i 2(u_i,v_i) = (1+u_i)(1+v_i)
                 \;.
  \end{align*}
  The key point now qualifying $f$ for the proof, is that both $f$ and
  thus by
  Theorem~\itemref{thm:twistedprod:cohomstructure}{item:twistedprod:preservescohominj}
  also $\Twistedprod{k}{f}$ are injective on cohomology, and fulfil
  \begin{enumerate}
  \item with $\W{\pb f\T M} = \prod_{i=1}^n (1+\alpha_i)$ that
    \begin{gather}\label{eq:lem:twistedprod:indecompcriterion:sM}
      \pb f \s{(n)}(M)
      = \s{(n)}(\W{\pb f\T M})
      = \sum_{i=1}^n \alpha_i^n
      \;
    \end{gather}
  \item and with
    $\W{\pb f\T{\Twistedprod{k}{M}}}
    = \prod_{i=1}^{k+1}(1+c) \cdot \prod_{i=1}^n
    (1+u_i)(1+v_i)$
    that
    \begin{align}\notag
      \pb{\Twistedprod{k}{f}} \s{(2n+k)}(\Twistedprod{k}{M})
      &= \s{(2n+k)}(\W{\pb f\T{\Twistedprod{k}{M}}}) \\
      \label{eq:lem:twistedprod:indecompcriterion:sDkM}
      &= (k+1)c^{2n+k} + \sum_{i=1}^n (u_i^{2n+k} + v_i^{2n+k})
        \;.
    \end{align}
  \end{enumerate}
  In order to formulate
  $\pb{\Twistedprod{k}{f}}\s{(2n+k)}(\Twistedprod{k}{M})$ 
  in terms of $c$ and $d(\alpha_i)$,
  use one of the Newton-Girard formulas saying:
  \begin{Lem}[Newton-Girard]
    For integers $l, m\in\Nat$ holds
    \begin{align*}
      \symm{l} t^m
      &= \sum_{\mathclap{r_1+2r_2+\dotsb+mr_=m}}
        (-1)^m \frac{m\cdot(r_1+\dotsb+r_m-1)!}{(r_1)!\dotsm(r_m)!}
        \cdot \prod_{i=1}^m (-\el i l)^{r_i}
        \;.
    \end{align*}
    As a special case for $l=2$ this becomes modulo 2
    \begin{align*}
      t_1^m + t_2^m
      &= \sum_{\mathclap{r_1+2r_2=m}}
        \frac{m\cdot(r_1+r_2-1)!}{(r_1)!(r_2)!}
        \cdot (\el 1 2)^{r_1} \cdot (\el 2 2)^{r_2}
      = \sum_{\mathclap{r_1+2r_2=m}}
        \altbinom{r_1-1}{r_2} (t_1+t_2)^{r_1}(t_1t_2)^{r_2}
    \end{align*}
    where $\frac{(r_1+2r_2)(r_1+r_2-1)!}{(r_1)!(r_2)!}
    = \binom{r_1+r_2-1}{r_2} + 2\binom{r_1+r_2-1}{r_1}$
    and the notation $\altbinom{p}{q}\coloneqq \binom{p+q}{q}$
    was used.
    \begin{proof}
      A proof of the main Newton-Girard formula can be found in
      \cite[Theorem~10.12.2]{raymond}. This special case can
      immediately be obtained by iterated application of the formula.
    \end{proof}
  \end{Lem}
  Before applying this to $t_1=u_i$, $t_2=v_i$, $m=2n+k$, and $l=2$,
  recall the following special properties from
  Theorem~\ref{thm:twistedprod:cohomstructure} of the mentioned symbols in
  $\H^*(\Twistedprod{k}{X})$:
  \begin{itemize}
  \item $c^{k+1}=0$,
  \item $c\cdot e(\alpha_i)=0$, and
  \item $e(\alpha_i)^{r_1}d(\alpha_i)^{r_2} = 0$ for $r_1+2r_2>2n$,
    as $\H^{*>2n}(M\times M)=0$.
  \end{itemize}
  Then simplify in terms of $c$, $e(\alpha_i)$, $d(\alpha_i)$:
  \begin{align}\notag
    u_i^{2n+k} + v_i^{2n+k}
    &=
      \sum_{{r_1+2r_2=2n+k}}
      \altbinom{r_1-1}{r_2} (u_i+v_i)^{r_1}(u_iv_i)^{r_2} \\\notag
    &\equalsby{Def.}
      \sum_{{r_1+2r_2=2n+k}}
      \altbinom{r_1-1}{r_2}
      (c+e(\alpha_i))^{r_1}d(\alpha_i)^{r_2} \\\notag
    &\equalsby{$c\cdot e(\alpha_i)=0$}
      \sum_{{r_1+2r_2=2n+k}} \altbinom{r_1-1}{r_2}
      \left(c^{r_1}d(\alpha_i)^{r_2}+e(\alpha_i)^{r_1}d(\alpha_i)^{r_2}\right) \\\notag
    &\equalsby{$c^{k+1}=0$, $d(\alpha_i)^{n+s}=0$}
      \altbinom{k-1}{n} c^k d(\alpha_i)^n
      + \sum_{\mathclap{r_1+2r_2=2n+k}} \altbinom{r_1-1}{r_2}
      e(\alpha_i)^{r_1}d(\alpha_i)^{r_2} \\
    \label{eq:lem:twistedprod:indecompcriterion:1}
    &= \altbinom{k-1}{n} c^k d(\alpha_i)^n      
  \end{align}
  Altogether it turns out that
  \begin{align*}
    \pb{\Twistedprod{k}{f}}\s{(2n+k)}(\Twistedprod{k}{M})
    &\cequalsby{\eqref{eq:lem:twistedprod:indecompcriterion:sDkM}}
      (k+1)c^{2n+k} + \sum_{i=1}^n (u_i^{2n+k} + v_i^{2n+k}) \\
    &\equalsby{$c^{k+1}=0$}
      \sum_{i=1}^n (u_i^{2n+k} + v_i^{2n+k}) \\
    &\equalsby{\eqref{eq:lem:twistedprod:indecompcriterion:1}}
      \sum_{i=1}^n \altbinom{k-1}{n} c^k d(\alpha_i)^n \\
    &= \altbinom{k-1}{n} c^k d(\sum_{i=1}^n \alpha_i^n) \\
    &\equalsby{\eqref{eq:lem:twistedprod:indecompcriterion:sM}}
      \altbinom{k-1}{n} c^k d(\pb f \s{(n)}(M))
  \end{align*}
  as was stated.
\end{proof}

By Theorem~\ref{thm:twistedprod:indecompcriterion}, one has to
both choose
\begin{itemize}
\item the dimension combination, as well as
\item the factor (see later)
\end{itemize}
of a twisted product correctly, in order to obtain the representative
of an indecomposable cobordism class.
For the correct dimension choice note, that any integer $m$ with
binary expansion
\begin{gather*}
  m=2^{i_1}+\dotsb+2^{i_l}
  \;,\qquad
  0 \leq i_1<\dotsb<i_r
\end{gather*}
can be written as $m=2^{i_1}+2n\eqqcolon k+2n$, where:
\begin{itemize}
\item $n=0$ for $l=1$, else $n=2^{i_2-1}+\dotsb+2^{i_l-1}$
\item $\alpha(m)=\alpha(n)+1$.
\item For $m$ not of the form $2^s-1$, $n$ will also not be of the
  form $2^s-1$.
\end{itemize}
This is favourable because the binomial coefficient
$\binom{k-1}{n}$ will never be zero modulo two by the following Lemma,
as is required by the indecomposability criterion for twisted products
in Theorem~\ref{thm:twistedprod:indecompcriterion}.
\begin{Lem}
  For $a,b\in\Nat$, $\binom{a}{b}$ will be non-zero modulo 2 if and only
  if $\alpha(a+b)=\alpha(a)+\alpha(b)$.
  \begin{proof}
    This is a direct consequence of Lucas' well-known theorem
    which states
    \begin{gather*}
      \binom{a+b}{b} \equiv \prod_{i=0}^s \binom{a_i+b_i}{b_i} \mod 2
    \end{gather*}
    where $a=\sum_{i=1}^s a_i2^{i}$, $b=\sum_{i=1}^s b_i2^{i}$ are the
    binary expansions of $a$ and $b$, and $\binom{0}{1}\coloneqq 0$.
    This expression will be non-zero, if and only if $b_i$ is one for
    any $i$ where $a_i+b_i$ is one. In other words, if and only if
    $\alpha(a+b)=\sum_{i=1}^s(a_i+b_i)$.    
  \end{proof}
\end{Lem}
This lemma can be applied to the combination $k-1$, $n$ from above, as one
simply calculates
\begin{align*}
  \alpha(k-1+n)
  &= \alpha\left((2^{i_1}-1)+ (2^{i_2-1}+\dotsb+2^{i_l-1})\right) \\
  &= \alpha\left(1+2^1+\dotsb 2^{i_1-1} + 2^{i_2-1}+\dotsb+2^{i_l-1}\right) \\
  &\equalsby{$i_1<i_2$}
  \alpha\left(1+2^1+\dotsb 2^{i_1-1}) + \alpha(2^{i_2-1}+\dotsb+2^{i_l-1}\right)\\
  &= \alpha(k-1) + \alpha(n)
\end{align*}

Therefore:
\begin{Cor}\label{cor:twistedprod:indecompcriterion}
  For $m=2^{i_1}+2n\in\Nat$ as above, the twisted product
  $\Twistedprod{2^{i_1}}{M}$ of an indecomposable manifold $M^n$ will be
  indecomposable.
\end{Cor}


\section
{Brown's Theorem: Finding a Convenient Generating Set}
\label{sec:proofbrown}
Remember that the goal of this chapter was to prove R.~L.~Brown's
theorem \ref{thm:brown} which states that the immersion conjecture is
true up to cobordism.

Also recall, that if one can show the conjecture's statement is stable
under the ring operations of the cobordism ring
$\c_*\cong\Zmod2[\sigma_i|i\neq2^r-1]$, it suffices to find a
generating set $([\G i]|i\neq2^r-1)$ of $\c_*$ which fulfils the conjecture.
Thus, for the proof one needs to construct a set of manifolds
$(\G i|i\neq2^r-1)$ that both fulfil the conjecture and represent
indecomposable cobordism classes.

\subsection{Stability Properties of the Conjecture}
The stability properties needed are the following.
\begin{Lem}\label{lem:brownstableunderringops}
  Let $M_i^{n_i}$ be a closed manifold immersing into
  $\R^{2n_i-\alpha(n_i)}$ for $i=1,2$.
  Then both manifolds
  \begin{enumerate}
  \item $M_1\disjointsum M_2$ for $n_1=n_2=n$, and
  \item $M_1\times M_2$ for $n_1$, $n_2$ arbitrary
  \end{enumerate}
  immerse into $\R^{2(n_1+n_2)-\alpha(n_1+n_2)}$.
  \begin{proof}
    $M_1\times M_2$ immerses into the real space of dimension
    \begin{align*}
      \left( 2n_1-\alpha(n_1) \right)
      + \left( 2n_2-\alpha(n_2) \right)
      &= 2(n_1+n_2) - \left(\alpha(n_1)+\alpha(n_2)\right)\\
      &\leq 2(n_1+n_2) - \left(\alpha(n_1 + n_2)\right)
    \end{align*}
    where the inequality is due to the number theoretic fact
    $\alpha(n_1+n_2) \leq \alpha(n_1)+\alpha(n_2)$.

    For $n_1=n=n_2$ the images of the immersions
    \begin{gather*}
      \iota_1\colon M_1\to\R^{2n-\alpha(n)}
      \qquad \text{and} \qquad
      \iota_2\colon M_2\to\R^{2n-\alpha(n)}
    \end{gather*}
    are compact. So by
    concatenation with translation they can be assumed to be disjoint,
    wherefore the disjoint union
    $\iota_1\disjointsum\iota_2\colon M_1\disjointsum M_2\to\R^{2n-\alpha(n)}$
    is again an immersion.
  \end{proof}
\end{Lem}


\subsection{Generating Set}
Now define the representatives of the generating set's elements as follows.
\begin{Thm}\label{thm:brownproof}
  Consider the manifolds $\G m$ for $m\neq 2^s-1\in\Nat$ that are
  defined by
  \begin{align*}
    \G 0 &\coloneqq \pt \;, \\
    \G m &\coloneqq \Twistedprod{2^{i_1}}{\G n}
        &\text{
          where
          $\textstyle m=\sum_{r=1}^l 2^{i_r}=2^{i_1}+2n$
          with $0\leq i_1<\dotsb<i_l$.
          }
  \end{align*}
  Then:
  \begin{enumerate}
  \item\label{item:brownimmersionproperty}
    $\G m$ immerses into $\R^{2m-\alpha(m)}$.
  \item\label{item:indecomposabilityproperty}
    The cobordism classes of the $\G m$, $m\neq 2^s-1$, are
    indecomposable, and thus
    \begin{gather*}
      \c_*\cong \left( [\G m] \,\big|\, m\neq 2^s-1 \right)_{\Zmod2}
      = \Zmod2\left[[\G m] \,\big|\, m\neq 2^s-1\right]
    \end{gather*}
  \end{enumerate}
\end{Thm}
\begin{proof}[proof of
  Theorem~\itemref{thm:brownproof}{item:indecomposabilityproperty}]
  By definition, all $\G m$ either are some $\Twistedprod{2^i}{\pt}$
  or arise from one as iterated twisted product. More precisely, for a binary
  representation
  \begin{align*}
    m &= 2^{i_1}+\dotsb + 2^{i_l} \\
      &= 2^{i_1}+ 2\cdot\big( 2^{i_2-1} + 2\cdot(\dotsc + 2\cdot 2^{i_l-l+1})\big)
    \\\intertext{one gets}
    \G m &\coloneqq
           \Twistedprod{2^{i_1}}{\G{2^{i_2-1} + 2\cdot(2^{i_3-1}+\dotsb)}}\\
      &= \Twistedprod{2^{i_1}}{
        \Twistedprod{2^{i_2-1}}{
        \dotsc
        \Twistedprod{2^{i_l-l+1}}{\pt}
        }}
        \;.
  \end{align*}
  But $\Twistedprod{2^{i}}{\pt}=\RP{2^i}$ represents an indecomposable
  element of the Cobordism ring by Example~\ref{ex:rpnindecomposable}.
  Thus, by Corollary~\ref{cor:twistedprod:indecompcriterion} all
  $\G m$ also represent indecomposable cobordism classes.
\end{proof}
\begin{proof}[proof of
  Theorem~\itemref{thm:brownproof}{item:brownimmersionproperty}]
  \optcite[p.~84, proof Thm.~1.30]{immersionconj}
  In order to proof the immersion property of the manifolds $\G m$,
  conduct an induction over $\alpha(m)$.
  \begin{description}
  \item[$\alpha(m)=1$:]
    As an induction start note that in the case of $m=2^i$ one has
    \begin{gather*}
      \G m
      =\Twistedprod{2^i}{\G 0}
      =\Twistedprod{2^i}{\pt}
      =\RP{2^i}
    \end{gather*}
    which immerses into $\R^{2m-1}=\R^{2m-\alpha(m)}$ by
    Whitney's immersion theorem~\cite{whitneyimmersiontheorem}.
  \item[$\alpha(m)\geq2$:]
    For the induction step write $m=2^{i_1}+2n$ as above, and assume
    there is an immersion
    \begin{gather*}
      \iota\colon\G n \immlongto \R^{2n-\alpha(n)}
      \;.
    \end{gather*}
  \end{description}
  As described in
  Remark~\itemref{rem:twistedprodproperties}{item:twistedprodpreservesimmersions}
  this induces an immersion of twisted products
  \begin{center}
    \begin{tikzcd}[row sep=small, column sep=large]
      (\Sphere{2^{i_1}} \times \G n\times \G n/\sim)
      \ar[d, equals]
      \ar[r, hookrightarrow, "1\times\iota\times\iota/\sim"]
      &
      (\Sphere{2^{i_1}} \times \R^{2n-\alpha(n)} \times \R^{2n-\alpha(n)}/\sim)
      \;.
      \ar[d, equals]
      \\
      \G m
      &\Twistedprod{2^{i_1}}{\R^{2n-\alpha(n)}}
    \end{tikzcd}
  \end{center}
  The latter twisted product however is a fibre bundle over
  $\RP{2^i_1}$ as in
  Remark~\itemref{rem:twistedprodproperties}{item:twistedprodfibrebdl}.
  More precisely, it is a vector bundle of dimension $4n-2\alpha(n)$.
  The needed immersion will be given by the following Lemma.
  \begin{Lem}\label{lem:immersionvbtotalspaces}
    \TODO{Ref. for proof?!}
    \optcite[p.~85]{immersionconj}
    \optcite[compare Prop. 4.3]{brown}
    The total space of a vector bundle $E\to M^n$ of rank $k$ over a
    closed manifold of dimension $n$ can be immersed into $\R^{2n+k}$.
  \end{Lem}
  As a result of the Lemma, $\Twistedprod{2^{i_1}}{\R^{2n-\alpha(n)}}$
  immerses into real space of dimension
  \begin{align*}
    2\cdot 2^{i_1} + \left(4n-2\alpha(n)\right)
    &= 2\cdot (2^{i_1}+2n) - 2\alpha(n) \\
    &= 2m - 2\alpha(n) \\
    &= 2m - 2\alpha(m)+2 \\
    &\leqby{m\geq 2}
    2m - \alpha(m)
    \;,
  \end{align*}
  so altogether one gets an immersion
  \begin{gather*}
    \G m
    \immlongto \Twistedprod{2^{i_1}}{\R^{2n-\alpha(n)}}
    \immlongto \R^{2m - \alpha(m)}
    \;.
  \end{gather*}
  %\begin{proof}[proof of Lemma~\ref{lem:immersionvbtotalspaces}]
  %\end{proof}
\end{proof}

%%% Local Variables:
%%% mode: latex
%%% TeX-master: "thesis"
%%% End:
