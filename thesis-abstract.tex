%%%%%%%%%%%%%%%%%%%%%%%%%%%%%%%%%%
%  Master Thesis in Mathematics
% "Immersions and Stiefel-Whitney classes of Manifolds"
% -- Abstract --
%
% Author: Gesina Schwalbe
% Supervisor: Georgios Raptis
% University of Regensburg 2018
%%%%%%%%%%%%%%%%%%%%%%%%%%%%%%%%%%

\begin{abstract}
  In 1944, Whitney's immersion
  theorem~\cite{whitneyimmersiontheorem} showed that every $n$-manifold
  can be immersed into real space of dimension $2n-1$.
  Thereby the following question arose:
  \begin{quote}
    What is the minimum codimension $k$ such that for every
    $n$-dimensional compact manifold there exists an
    immersion $M\immto\R^{n+k}$ into real space?
  \end{quote}
  The immersion conjecture asserts that $k=n-\alpha(n)$ where
  $\alpha(n)$ is the number of ones in the binary notation of $n$.
  The goal of this thesis is to motivate this initial guess for $k$,
  and to prove the conjecture up to cobordism,
  while outlining the important role of characteristic classes as the
  main tooling.
  
  More precisely, details on the following aspects will be given:
  In \autoref{chap:reformulation}, a famous result by Hirsch and
  Smale~\cite{hirschimmersions} is used to turn the problem of
  finding an immersion of certain codimension into the one of finding
  a certain vector bundle monomorphism.
  This then admits an obstruction by characteristic classes,
  which is shown to vanish exactly for codimensions
  $k\geq n-\alpha(n)$ due to a theorem of Massey~\cite{massey}
  in \autoref{chap:massey}, thus motivating the value of $k$ in the
  immersion conjecture.
  Finally, a criterion given by characteristic classes for
  indecomposability in the cobordism ring due to Thom~\cite{thom}
  is the main ingredient for a theorem of R.~L.~Brown~\cite{brown},
  which states that the immersion conjecture is true up to cobordism.
  This is shown in detail in \autoref{chap:brown}.

  Both of the main theorems outlined in this thesis---the one by
  Massey and the one by R.~L.~Brown---were important steps towards the
  proof of the immersion conjecture, which was finalised as
  recently as 1985 by R.~L.~Cohen~\cite{cohen} with previous
  efforts by E.~H.~Brown and F.~P.~Peterson.
\end{abstract}

%%% Local Variables:
%%% mode: latex
%%% TeX-master: "thesis"
%%% End:
