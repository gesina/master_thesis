%%%%%%%%%%%%%%%%%%%%%%%%%%%%%%%%%%
%  Master Thesis in Mathematics
% "Immersions and Stiefel-Whitney classes of Manifolds"
% -- Main File --
%
% Author: Gesina Schwalbe
% Supervisor: Georgios Raptis
% University of Regensburg 2018
%%%%%%%%%%%%%%%%%%%%%%%%%%%%%%%%%%

\documentclass[a4paper, english, bibliography=totoc, oneside, BCOR=3mm]{scrreprt}
%%%%%%%%%%%%%%%%%%%%%%%%%%%%%%%%%%
%  Master Thesis in Mathematics
% "Immersions and Stiefel-Whitney classes of Manifolds"
% -- Definitions --
% 
% Author: Gesina Schwalbe
% Supervisor: Georgios Raptis
% University of Regensburg 2018
%%%%%%%%%%%%%%%%%%%%%%%%%%%%%%%%%%

\KOMAoptions{parskip=half}

% LANGUAGE
\usepackage[ngerman, main=english]{babel}
%\hyphenation{}
\usepackage[autostyle=try]{csquotes}

% INDICES
\usepackage[style=alphabetic, backend=biber]{biblatex}
\bibliography{thesis.bib}

% \usepackage{makeidx}
%\makeindex

% Nomenclature
% Add entry via
%   \nomenclature[<prefix for sorting>]{<symbol>}{<description>}
% !! no newlines within the command; all around need to be commented out !!
% Output via:
%   \printnomenclature
% Creates .nlo file with command:
%   makeindex thesis.nlo -s nomencl.ist -o thesis.nls
\usepackage[intoc, refpage]{nomencl}
\setlength{\nomitemsep}{0pt}
\makenomenclature


% MATHS PACKAGES
\usepackage{mathtools,amsthm,dsfont}
% \usepackage{amssymb}
% \usepackage{amsfonts}
\usepackage{tikz-cd}
\usetikzlibrary{babel}

% MATHS-FONT
% \usepackage{mathastext} % uses text font; bad arrows -> NO
% \usepackage{euler} % for egyptienne
% \usepackage{sfmath}\usepackage{sansmathaccent} % sans serif
% \usepackage[math]{anttor} % letters look slightly crippled in comparison
% \usepackage[math]{kurier}
% \usepackage{mathpazo}
%
% \usepackage[charter]{mathdesign} % ugly longmapsto
% Fix the \logmapsto for this font
%\renewcommand*{\longmapsto}{%
%  \kern-.35pt\mapstochar\kern0.35pt\longrightarrow}

% FONT
\usepackage{fontspec}
\setmainfont{Latin Modern Roman}
\setsansfont{Latin Modern Sans}
%
% \setmainfont{Charis SIL}
%\setsansfont{Source Sans Pro}
%
%\setsansfont{Fira Sans}
%\setsansfont{Bitstream Vera Sans} % too wide

% OTHER PACKAGES
\usepackage{enumitem}
\setlist[description]{font=\normalfont\itshape, leftmargin=1.3em}
\usepackage{booktabs}
\usepackage{graphicx}
\usepackage{hyperref}
\usepackage{tabularx}

% PROTOTYPING
\newcommand*{\optcite}[2][]{\cite[][#1]{#2}}

% LANGUAGE
% Shorthands with correct spacing
\newcommand*{\idest}{i.e.\ } % i.e.
\newcommand*{\Idest}{I.e.\ } % I.e.
\newcommand*{\forexample}{e.g.\ } % e.g.
\newcommand*{\Forexample}{E.g.\ } % E.g.

% THEOREMS
\newtheorem{Thm}{Theorem}
\newtheorem{Prop}[Thm]{Proposition}
\newtheorem{Lem}[Thm]{Lemma}
\newtheorem{Cor}[Thm]{Corollary}
\theoremstyle{definition}
\newtheorem{Def}[Thm]{Definition}
\newtheorem{LemDef}[Thm]{Lemma/Definition}
\theoremstyle{remark}
\newtheorem{Ex}[Thm]{Example}
\newtheorem{Rem}[Thm]{Remark}

% MATHS DEFINITIONS
% Numbers
\newcommand*{\Nat}{\mathds{N}} % natural numbers
\newcommand*{\Z}{\mathds{Z}} % integers
\newcommand*{\Zmod}[1]{\Z_{#1}} % residual class rings
\newcommand*{\Q}{\mathds{Q}} % rational numbers
\newcommand*{\R}{\mathds{R}} % real numbers
\newcommand*{\C}{\mathds{C}} % complex numbers

% Symbol shortenings
\renewcommand*{\epsilon}{\varepsilon}
\newcommand*{\longto}{\longrightarrow}
\newcommand*{\longfrom}{\longleftarrow}
\DeclareMathOperator{\im}{im} % image

% maps
\renewcommand*{\Im}{\text{im}} % image
\newcommand*{\Id}[1][]{\mathrm{id}_{#1}} % identity map
\newcommand*{\incl}{\text{incl}} % inclusion
\newcommand*{\proj}{\text{proj}} % projection
\newcommand*{\pb}[1]{{#1}^*} % pullback by a map
\newcommand*{\pf}[1]{{#1}_*} % pushforward by a map
\newcommand*{\susp}{\Sigma} % suspension map

% spaces
\newcommand*{\CP}[1][\infty]{\text{$\C{}$P}^{#1}} % complex projective space
\newcommand*{\RP}[1][\infty]{\text{$\R{}$P}^{#1}} % real projektive space
\DeclareMathOperator{\rk}{rk} % Rang (einer Matrix)
\newcommand*{\E}[1]{\text{E}#1} % total space of universal bundle
\newcommand*{\BG}{\text{BG}} % classifying space of principal G-bundles
\newcommand*{\EG}{\E{\text{G}}} % total space of the universal bundle of principal G-bundles
\newcommand*{\GL}[1]{\text{GL}_{#1}(\R)} % general linear group over \R
\newcommand*{\Orth}{O} % orthogonal matrices
\newcommand*{\BO}{\text{BO}} % classifying space of O-bundles
\newcommand*{\pt}{*} % single point space
\newcommand*{\zeroset}{\{0\}} % set containing only 0

% categories
\newcommand*{\Top}{\text{Top}} % category of topological spaces
\newcommand*{\Vect}{\text{Vect}} % category of vector bundles

% cohomology classes/maps
\newcommand*{\capped}[2]{\left\langle #1, #2 \right\rangle} % cap prod
\newcommand*{\Sq}[1]{\text{Sq}^{#1}}  % Steenrod Squares
\newcommand*{\cl}{\text{cl}} % general name of a characteristic class
\newcommand*{\Cl}{\text{Cl}} % general name of a characteristic class
\newcommand*{\w}{w} % Stiefel-Whitney class
\newcommand*{\dualw}{\overline{\w}} % dual Stiefel-Whitney class
\renewcommand*{\u}{u} % Thom class (overwrites cup over letter)
\newcommand*{\thomiso}{t} % Thom isomorphism
\renewcommand*{\v}{v} % Wu class (overwrites check over letter)


% bundles
\newcommand*{\zerosec}[1]{0_{#1}}
\newcommand*{\trivbdl}{\epsilon} % trivial rank ? vector bundle
\newcommand*{\T}[1]{\mathrm{T}#1} % tangent bundle
\newcommand*{\N}[1]{\nu_{#1}} % (stable) normal bundle

%%% Local Variables:
%%% mode: latex
%%% TeX-master: "thesis"
%%% End:


% TITEL
\hypersetup{
  pdfauthor={Gesina Schwalbe},
  pdftitle={Immersions and Stiefel-Whitney Classes of Manifolds}  % TODO: Title
}
\newlength{\urlogowidth}\setlength{\urlogowidth}{11.5cm}
\titlehead{%
  \parbox{2\urlogowidth}{%
    \includegraphics[width=.98\urlogowidth]{UniLogo_Text.jpg}\\%
    \hspace*{0.3333\urlogowidth}%
    {\sffamily\bfseries\scshape%
      \fontsize{0.041667\urlogowidth}{0.041667\urlogowidth}\selectfont%
      Fakultät für Mathematik}\\[2em] % TODO: Translate Logo?
  }%
}

\subject{Master Thesis}
\title{Immersions and Stiefel-Whitney Classes of Manifolds}
\author{Gesina Schwalbe}
\publishers{Supervisor: Dr.~Georgios Raptis}  % TODO: Supervisor
\newcommand{\handindate}{}  % TODO: Hand-In date
\date{\handindate}

% VORVERSION:
%\includeonly{thesis-02}

\begin{document}

\maketitle
\tableofcontents


% Introduction
%%%%%%%%%%%%%%%%%%%%%%%%%%%%%%%%%%
%  Master Thesis in Mathematics
% "Immersions and Stiefel-Whitney classes of Manifolds"
% -- Chapter 1: Introduction --
% 
% Author: Gesina Schwalbe
% Supervisor: Georgios Raptis
% University of Regensburg 2018
%%%%%%%%%%%%%%%%%%%%%%%%%%%%%%%%%%

\chapter*{Introduction}

\begin{itemize}
\item space means topological space
\item all maps between spaces are continuous
  % \item $\Zmod2 \coloneqq \Z/2\Z$
\item For $X$, $Y$ spaces $[X,Y]$ denotes the set of homotopy classes
  of basepoint preserving maps $X\to Y$.
\item All manifolds are topological manifolds.
\item All homology groups have $\Zmod2$ coefficients if not stated otherwise.
\item All vector bundles are real if not stated otherwise.
\item The following notation for a fiber bundle $\xi\colon E\to B$ is used:
  \begin{itemize}
  \item $\E\xi\coloneqq E$ denotes the total space.
  \item $\B\xi\coloneqq B$ denotes the base space.
  \item $p_\xi\coloneqq p$ denotes the underlying surjection.
  \item $\E_b$ denotes the fiber over a point $b\in \B$.
  \item $0_\xi$ denotes the zero section if it exists.
  \item $\minuszerosec{E}\coloneqq E\setminus\Im\zerosec{\xi}$
    is the total space without the zero section.
  \end{itemize}
\end{itemize}

% Nomenclature:
\nomenclature{$H^i(-; R)$}{Singular cohomology of degree $i$ with coefficients
  in $R$}
\nomenclature{$K(i,G)$}{$ith$ Eilenberg-MacLane space over group $G$}
\nomenclature{$\GL n$}{Group of $n$-dimensional matrices over $\R$}
\nomenclature{$\Orth(n)$}{Group of $n$-dimensional orthogonal matrices over $\R$}
\nomenclature{$\Orth$}{Group of orthogonal matrices over $\R$}
\nomenclature{$\Sigma$}{Suspension of a space $X$
  respectively suspension morphism on cohomology: %TODO}
\nomenclature{$\zerosec\xi$}{Zero section of a vector bundle $\xi$}
  
%%% Local Variables:
%%% mode: latex
%%% TeX-master: "thesis"
%%% End:


% Preliminaries, e.g. Wu's Theorem
%%%%%%%%%%%%%%%%%%%%%%%%%%%%%%%%%% 
% Master Thesis in Mathematics
% "Immersions and Stiefel-Whitney classes of Manifolds"
% -- Chapter 1: Formulation of the Immersion Conjecture --
% 
% Author: Gesina Schwalbe
% Supervisor: Georgios Raptis
% University of Regensburg 2018
%%%%%%%%%%%%%%%%%%%%%%%%%%%%%%%%%% 

\chapter{Formulation of the Immersion Conjecture}
\label{chap:reformulation}
% Explain carefully how the immersion problem can be reformulated purely
% in terms of homotopy theory. [immersionconj] (The necessary results from differential
% topology and Hirsch-Smale theory should be stated clearly but may be
% presented without proofs.)
This chapter is dedicated to reviewing the concepts and results,
which are needed to formulate the immersion conjecture and its
connection to the theory of characteristic classes.

The (re)formulation in \autoref{sec:reformulation} uses as main
ingredient a theorem by Hirsch and Smale on the relation between
immersions and vector bundle monomorphisms presented in
\autoref{sec:hirschsmale}.
This, together with other required definitions and properties of
immersions, is contained in \autoref{sec:immersions}.
Thereafter, in \autoref{sec:charclsofvb}, characteristic classes of vector
bundles are reviewed. Most importantly, the Stiefel-Whitney
characteristic classes are recalled in \autoref{sec:swclasses},
together with an outline of the way these will be particularly
useful throughout this thesis.
The last section then contains an explanation on how they contribute
to the immersion problem via obstructions.

This chapter is meant as revision and outline, therefore a
couple of preliminary results are merely referenced without proof.

\section{Immersions}\label{sec:immersions}
This section recapitulates the definition some properties
of immersions.

\subsection{Definition}
As already mentioned, immersions are technically local embeddings of
manifolds.
From a different point of view, immersions are merely a special case
of monomorphisms of vector bundles. So, recall that a morphism
$(\xi_1\colon\E_1\to X_1)\to(\xi_2\colon E_2\to X_2)$ 
of vector bundles over different spaces is a map $F\colon E_1\to E_2$
which is linear on fibres, and that covers its restriction to the zero
section, \idest it makes the following diagram commute
\begin{center}
  \begin{tikzcd}
    E_1 \ar[r,"F"]\ar[d, "\xi_1"]
    & E_2 \ar[d, "\xi_2"]
    \\
    X_1 \ar[r, "F|_{\zerosec{\xi_1}}"]
    & X_2
  \end{tikzcd}.
\end{center}
Further, remember the fact that such a morphism is a monomorphism in
the category of vector bundles if and only if its restriction to each
fibre is injective.
\begin{Def}
  A smooth map $f\colon M\to N$ of smooth manifolds is called
  an \emph{immersion}, written $M\immto N$, if its differential
  $\Diff f\colon\T M\to\T N$ is a monomorphism of vector
  bundles.
  A homotopy $H\colon M\times I\to N$ which is an immersion in each
  stage is called \emph{regular}.
\end{Def}
\begin{Rem}
  Let $M$ and $N$ be manifolds.
  \begin{enumerate}
  \item
    Immersions are local embeddings of manifolds, \idest for an
    immersion $f\colon M\immto N$, around every point in $M$
    there is an open neighbourhood on which $f$ is a diffeomorphism
    onto its image.
    More descriptive, immersions are mappings that do not allow creases,
    respectively sharp bends, or puncturing
    (see \forexample \cite{outsidein} for nicely illustrated examples)
    However, globally, immersions need not be injective since
    \forexample self-intersections of the image are allowed.
    \begin{proof}
      This is a conclusion from the implicit function theorem,
      see \forexample \cite[Chap.~1, Theorem~3.1]{hirsch}.
    \end{proof}
  \item Embeddings of manifolds are exactly those injective immersions
    that are topological embeddings.
    If $M$ is compact, any injective immersion $f\colon M\to N$ is an
    embedding.
    \begin{proof}
      % Argumentation:
      % Every manifold has Riemannian metric
      % Every Riemannian manifold is a metric space, topologies agree
      % For metric spaces, sequentially compact is equivalent to compact
      Use the fact that for manifolds compact is equivalent to
      sequentially compact, in order to get
      \begin{gather*}
        \left\{
          y=\lim_n f(x_n) \,\middle|\,
          (x_n)_{n\in\Nat}\subset M \text{ without limit}
        \right\} \cap M = \emptyset
      \end{gather*}
      directly from injectivity of $f$, and compactness of $M$ and
      $f(M)$.
      Then apply \cite[Chap.~II, Lemma~2.6]{adachi}.
    \end{proof}
  \end{enumerate}
\end{Rem}

\subsection{The Hirsch-Smale Theorem}\label{sec:hirschsmale}
It is easy to see that in general not all vector bundle monomorphisms
between tangent bundles of smooth manifolds need to be the
differential of an immersion. However, taking the differential gives a
canonical inclusion of the set of immersions into the set of vector
bundle monomorphisms.
A theorem of Hirsch and Smale states that this inclusion
actually is a homotopy equivalence,
translating questions on the existence of immersions into the context
of characteristic classes of vector bundles.

For the formulation, one has to equip the respective sets with a
topology as follows.
\begin{Def}
  Let $M$, $N$ be closed smooth manifolds of dimensions $\dim M<\dim N$.
  \begin{enumerate}
  \item
    Equip the set of all vector bundle monomorphisms from $\xi_1$ to
    $\xi_2$ with the compact-open topology (see \forexample
    \cite{hatcher}), and denote that space by $\Mono{\xi_1}{\xi_2}$.
    Note that a path between monomorphisms $F_1$ and $F_2$ in the
    space $\Mono \xi \eta$ is an homotopy from $F_1$ to $F_2$ which is
    a vector bundle monomorphism in each stage.
  \item
    % compare Lecture_Notes_on_Immersions_of_Surfaces_in_3-Space--Nowik.ps
    % Whitney $C^r$-topology; see [Hirsch, Differential Topology, Chap 2, p.35]
    The set $\Imm M N$ of all immersions from $M$ to $N$ injects
    into $\Mono{\T M}{\T N}$ by taking the differential
    $f\mapsto\Diff f$.
    Equip $\Imm M N$ in the following with the subspace topology.
    This results in the weak topology described in
    \cite[Section~2.1]{hirsch}, which equals the Whitney
    $C^1$-topology since $M$ was chosen compact.
    By the way, $\Imm M N$ is open in $C^1(M,N)$ equipped with the
    Whitney $C^1$-topology
    (see \cite[Section~2.1, Theorem~1.1]{hirsch}).
  \end{enumerate}
\end{Def}

Now one can state the major result in immersion theory by
Hirsch using preliminary work of Smale
\cite[Sections~5 and 6]{hirschimmersions}.
The following formulation is according to
\cite[Theorem~1.2]{immersionconj}.
\begin{Thm}[Hirsch-Smale]\label{thm:hirschsmale}
  Let $M$, $N$ be closed manifolds with $\dim M<\dim N$.
  Then the differential map
  $\Diff\colon \Imm M N\to \Mono{\T M}{\T N}$
  induces isomorphisms on the homotopy groups.
  Especially,
  \begin{gather*}
    \Diff_*\colon
    \pi_0(\Imm M N) \overset\sim\longto \pi_0(\Mono{\T M}{\T N})
  \end{gather*}
  describes an isomorphism of path-connected components.
  % Original formulation by Hirsch in [hirschimmersions], sec. 5:
  Therefore, every vector bundle monomorphism
  $F\colon\T M\to\T N$ is homotopic (through vector bundle
  monomorphisms) to a monomorphism which is the differential
  $\Diff f$ of a smooth map $f\colon M\to N$, \idest of an
  immersion.
  \begin{proof}
    See \cite[Theorem~8.2.1]{introductionhprinciple} or the original
    paper \cite{hirschimmersions}.
  \end{proof}
\end{Thm}
Thus, any monomorphism of vector bundles over smooth, closed manifolds
$M$ and $N$ implies the existence of an immersion from $M$ to $N$.
This conclusion will be needed to reformulate the immersion problem.


\subsection{Normal Bundles}
Another nice property of immersions is that every immersion gives
rise to a normal bundle. This will finally make it possible to
translate the existence of an immersion of certain codimension into
the existence of a vector bundle of certain rank fulfilling a homotopy
invariant lifting property.
\begin{Def}
  Let $\imm\colon M^n\immto N^{n+r}$ be an immersion of smooth
  manifolds.
  The \emph{normal bundle $\N{\emb}$ of $\imm$}
  is the well-defined, quotient bundle $\pb{\imm}\T N/\T M$ of rank $r$,
  respectively the one fulfilling
  $\N{\imm}\oplus\T M\cong\pb{\imm}\T N$.
\end{Def}
Recall that manifolds have the very handy property that they admit a
unique tangent bundle.
Similarly, a normal bundle of an immersion into Euclidean space is
unique up to a notion of stable equivalence, under which
characteristic classes will turn out to be invariant.
\begin{Def}
  Call two vector bundles $\xi_1$, $\xi_2$ over the same space 
  \emph{stably equivalent} in case there are $s_1, s_2\in\Nat$ such
  that $\xi_1\oplus\trivbdl^{s_1}\cong\xi_2\oplus\trivbdl^{s_2}$.
\end{Def}
Now the promised notion of the stable normal bundle can be clarified.
\begin{LemDef}
  Let $M^n$ be a closed, smooth manifold.
  Then all normal bundles of immersions of $M$ into
  Euclidean spaces are stably equivalent.
  The resulting equivalence class is called the
  \emph{stable normal bundle of $M$}, written $\N M$.
  When working with vector bundles in a context that is stable in the
  above sense, like \forexample characteristic classes, the stable
  normal bundle of $M$ may be identified with an arbitrary
  representative of its class.
  \begin{proof}[proof (sketch)]
    First show that every normal bundle of an immersion is stably
    equivalent to the 
    normal bundle of \emph{some} embedding (\idest some injective
    immersion). Then ensure that all normal bundles of embeddings are
    stably equivalent.
    \begin{description}
    \item[Immersions]
      Any immersion $\imm\colon M\immto\R^{n+r}$ can be raised to
      higher codimension by concatenation with the linear embedding
      $l\colon\R^{n+r}\immto\R^{n+r+s}$ into the first components.
      As the normal bundles $\N{\imm}$ and
      $\N{l\circ\imm}\cong\N{\imm}\oplus\trivbdl^s$ are stably
      equivalent, raising the codimension does not change the stable
      equivalence class.
      Furthermore, since a regular homotopy yields an
      isomorphism on the normal bundles, it suffices to show:
      \begin{claim}
        For $r>n$ every immersion $M\immto\R^{n+r}$ is regularly
        homotopic to an embedding.
      \end{claim}
      For the claim use bumping techniques
      %as needed for the Thom transversality theorem
      to show that for $r>n$ every
      immersion $\imm\colon M\immto\R^{n+r}$ is regularly homotopic to
      an injective immersion
      (see \forexample \cite[Chap.~II, Lemma~2.5]{adachi}).
      However, as $M$ is compact, injective immersions
      are embeddings.
    \item[Embeddings]
      By Whitney's embedding theorem
      (see \forexample \cite[Chap.~II.2]{adachi})
      it is known that every manifold admits an embedding into some
      real space.
      Further, by \forexample the General Position theorem
      (compare \cite[Chap.~2]{embeddingsummary})
      or Haefliger's theorem (see \forexample \cite[Chap.~II.1]{adachi}),
      it is known that for sufficiently large $k\in\Nat$ all embeddings
      $M\immto\R^{n+k}$ are isotopic, \idest homotopic through embeddings,
      and hence their normal bundles are isomorphic.
      Therefore, all normal bundles of embeddings of a manifold are
      stably equivalent.
      \qedhere
    \end{description}
  \end{proof}
\end{LemDef}

\section{Characteristic Classes of Vector Bundles}\label{sec:charclsofvb}
The theory of characteristic classes provides the key tools
for the rest of this thesis. Therefore, this section
revises basic results, and recalls in detail several properties
of Stiefel-Whitney classes. The latter are generators of all
characteristic classes of vector bundles, and will be essential in
proving the theorems of the subsequent chapters.

\subsection{General Definition and Properties}
Before starting off with the definition of characteristic classes,
we recall the definition of universal bundles and Steenrod's
classification theorem (compare \cite[Chapter~14.4]{tomdieck}).
\begin{LemDef}\label{def:charcls}
  \begin{enumerate}
  \item Any topological group $G$ admits a contractible space $\EG$ with a
    free $G$-action, and a corresponding principal $G$-bundle
    $\gamma^G\colon \EG\to\BG\coloneqq \EG/G$, called the
    \emph{universal $G$-bundle},
    where $\gamma_G$, $\EG$, and $\BG$ are all unique up to
    homotopy.
    $\BG$ is called the \emph{classifying space} for principal
    $G$-bundles.
    For construction and uniqueness see \cite[Example~1B.7~ff.]{hatcher},
    respectively note that universal coverings are unique up to homotopy.
  \item\label{item:classificationthm}
    $\gamma^G$ fulfils the following universal property:
    For any space $X$ admitting the homotopy type of a CW-complex
    there is a bijection between $[X,\BG]$, which denotes the homotopy
    classes of maps from $X$ to $\BG$, and the isomorphism classes of
    principal $G$-bundles over $X$, given by
    \begin{gather*}
      \left(f\colon X\to\BG \right) \longmapsto \pb f \gamma^G
      \;.
    \end{gather*}
    This correspondence is natural in $X$, and is a version of
    Steenrod's classification theorem
    (see 
    \cite[Theorem~14.4.1]{tomdieck},
    \cite[Theorem~1.4, p.~75]{immersionconj}).
  \end{enumerate}
\end{LemDef}
As becomes clear directly from the statement, the classifying
theorem serves in translating bundle theoretic problems into homotopy
theoretic ones, which will be a crucial step in reformulating the
immersion problem in \autoref{sec:reformulation}.
Such homotopy theoretic questions can then be tackled using known
cohomological tools, which yields the general concept of
characteristic classes.
Some important examples for vector bundles, namely the
Stiefel-Whitney classes, will be discussed in detail in
\autoref{sec:swclasses}.
\begin{Def}
  A \emph{characteristic class}
  \begin{itemize}
  \item of degree $i$
  \item with coefficients in a ring $R$
  \item for principal $G$-bundles for a group $G$
  \end{itemize}
  is a natural transformation
  \begin{gather*}
    \Cl\colon [-, \BG] \Longrightarrow \H^i(-; R)\;.
  \end{gather*}
  of contravariant functors from the category of spaces with the
  homotopy type of a CW-complex to the category of sets.
\end{Def}
\begin{Rem}
  By Brown's representation theorem
  (see \forexample \cite[Chap.~4.E]{hatcher}),
  $\H^i(-;R)$ is a representable functor represented by the
  Eilenberg-MacLane space $K(i,R)$.
  Thus, by the Yoneda lemma, a characteristic class is
  represented by a morphism
  \begin{gather*}
    \cl\colon \BG \longto K(i, R)
  \end{gather*}
  in $\Top$, \idest by a cohomology class $\cl$ of $\BG$.
  Thus, for a space $X$, which admits the homotopy type of a
  CW-complex, 
  applying $\Cl$ to a principal $G$-bundle over $X$ that is
  represented by a morphism $\eta\colon X\to\BG$
  as in Definition~\itemref{def:charcls}{item:classificationthm},
  yields
  \begin{gather*}
    \Cl(X) = \pb{\eta} \cl \in \H^i(X;R)
    \;.
  \end{gather*}
  This describes a one-to-one correspondence between
  characteristic classes as above and cohomology classes in
  $\H^i(BG;R)$, and in the following any characteristic class will be
  identified with its corresponding cohomology class.
\end{Rem}

As this thesis is mainly concerned with vector bundles, we have a
closer look at the significance of classifying spaces in that context,
especially at their stability property and its implications for normal
bundles.
\begin{Lem}\label{lem:classificationvb}
  Let $X$ be any space, let $M^n$ be a manifold, and $r,s\in\Nat$.
  \begin{enumerate}
  \item\label{item:vbcharacterisation}
    There is a natural equivalence between the category of
    $n$-dimensional vector bundles and that of principal
    $\Orth(n)$- respectively $\GL n$-bundles, where $\Orth(n)$ is the
    $n$-dimensional real orthogonal group, and $\Orth$ their limit
    correspondingly, \idest the group of all orthogonal matrices over
    $\R$.
  \item\label{item:boincl}
    The inclusion $\B\Orth(r)\to\B\Orth(r+s)$ is $r$-connected.
  \item\label{item:bomaps}
    On vector bundles, the inclusion $\B\Orth(r)\to\B\Orth(r+s)$
    represents the direct sum with the trivial bundle $\trivbdl^s$.
    \Idest for vector bundles $\xi_1$ and $\xi_2$ over $X$ with
    classifying maps $f_1$ respectively $f_2$, there is a lift up to
    homotopy of the form
    \begin{center}
      \begin{tikzcd}
        X \ar[r, "f_1"]
        \ar[dr, bend right, "f_2"{left}]
        & \B\Orth(r)
        \ar[d, "\incl"]
        \\
        & \B\Orth(r+s)
      \end{tikzcd}
    \end{center}
    if and only if $\xi_1\oplus\trivbdl^{s}\cong\xi_2$.
  \item\label{item:charclsstablenormalbundle}
    Taking the limit of all classifying maps of the normal bundles of
    embeddings of $M$ yields a homotopy class in
    $[M,\B\Orth=\varinjlim_{k}\B\Orth(k)]$
    that classifies the stable normal bundle of $M$ uniquely.
    Any lift of it in $[M,\B\Orth(r)]$ represents a vector bundle
    $\N{}$ with the property $\N{}\oplus\T M\cong\trivbdl^{n+r}$.
  \end{enumerate}  
  \begin{proof}[proof]
    The natural equivalence is given by the known construction of
    associated vector bundles, and the stability property of the
    family $(\B\Orth(n))_n$ becomes clear from
    $\B\Orth(n)\cong\lim_{k\to\infty}G_n(\R^k)$ 
    where $G_n(\R^k)$ is the Grassmann manifold of $n$-dimensional
    vector subspaces of $\R^k$.

    For the $r$-connectivity in \ref{item:boincl}, observe that
    the diagram
    \begin{center}
      \begin{tikzcd}[row sep=small, column sep=small]
        \Orth(r) \ar[r]\ar[d,"\incl"{near start}]
        &\E\Orth(r) \ar[r] \ar[d,"\incl"{near start}]
        &\BO(r) \ar[d,"\incl"{near start}] \\
        \Orth(r+1) \ar[r]\ar[d]
        &\E\Orth(r+1) \ar[r]
        &\BO(r+1) \\
        \Sphere{r}
      \end{tikzcd}
    \end{center}
    commutes, where the rows are the defining fibre bundles for the
    classifying spaces, and $\Orth(r)\to\Orth(r+1)\to\Sphere r$
    is the well-known fibre bundle of the inclusion of orthogonal groups.
    Since $\E\Orth(s)$ is contractible for any $s\in\Nat$, 
    the long exact sequences of homotopy for the horizontal fibre bundles yield
    $\pi_i(\Orth(r))\cong\pi_{i+1}(\BO(r))$ for $i\in\Nat$, analogously for $r+1$.
    The sequence for the vertical fibre bundle yields that
    $\Orth(r)\to\Orth(r+1)$ is $r$-connected, and commutativity gives
    the same for $\incl\colon\BO(r)\to\BO(r+1)$.
    
    For the lifting property of the stable normal bundle, consider a
    lift $\N{}\in[M,\BO(r)]$, $r>1$, by \ref{item:vbcharacterisation}
    classifying a vector bundle over $M$, which we will also call
    $\N{}$. By \ref{item:bomaps} and the definition
    of the stable normal bundle, there is some $s\in\Nat$
    such that $\T M\oplus\N{}\oplus\trivbdl^s\cong\trivbdl^{n+r+s}$.
    Assume $s>0$.
    In order to show that this still implies
    $\T M\oplus\N{}\cong\trivbdl^{n+r}$ (\idest that their
    classifying maps are homotopic), consider the corresponding
    homotopy commutative diagram of classifying maps
    \begin{center}
      \begin{tikzcd}
        M
        \ar[r, "\T M\oplus\N{}", shift left]
        \ar[r, "\trivbdl^{n+r}"{below}, shift right]
        \ar[dr, bend right, "\trivbdl^{n+r+s}"{left}]
        & \B\Orth(n+r)
        \ar[d, "\incl"]
        \\
        & \B\Orth(n+r+s)
      \end{tikzcd}
    \end{center}
    This says that $\incl\circ\T M\oplus\N{}$ and
    $\incl\circ\trivbdl^{n+r}$ must be homotopic via
    some homotopy
    \begin{gather*}
      H\colon M\times I\to\BO(n+r+s) \;.
    \end{gather*}
    The trick now is to use that by Morse theory $M$ has the homotopy
    type of an $n$-dimensional CW-complex, together with $\incl$ being
    $(n+r)$-connected, $r>0$, and both $\BO(n+r+s)$ and $\BO(n+r)$
    being path-connected. Because with these assumptions
    obstruction theory yields that
    $H$ lifts to a homotopy $M\times I\to\BO(n+r)$ between
    the classifying maps of $\T M\oplus\N{}$ and $\trivbdl^{n+r}$, as
    was needed
    (see \forexample \cite[Lemma~4.6]{hatcher}).
  \end{proof}
\end{Lem}
\begin{Not}
  Throughout this thesis assume $R=\Zmod2$ and $G=\Orth(n)$
  respectively $G=\Orth$ if not stated otherwise.
\end{Not}

\subsection{Stiefel-Whitney Classes}
\label{sec:swclasses}
Now that the general concept is known, this section reviews the
defining and immediate properties of the Stiefel-Whitney
classes, a generating set for the ring $\H^*(\BO)$ of characteristic
classes of vector bundles.
This makes them especially interesting for investigation,
as any property stable under cohomology ring operations only needs to
be checked on the generating set.
Furthermore, they will be invaluable for constructing new
characteristic classes used as obstructions or indicators.
More precisely, their duals, their inverses under certain Steenrod
operations called Wu classes, and a couple of special polynomials
evaluated on them will be used.

First start with the defining properties of the Stiefel-Whitney
classes. Compare \forexample \cite[compare §4, p.~37]{milnor}.
\begin{Def}\label{def:swclasses}
  The \emph{Stiefel-Whitney classes} are
  characteristic classes for principal $\Orth$-bundles
  respectively vector bundles,
  \idest cohomology classes
  $\ws{i}\in \H^i(\BO;\Zmod2)$, $i\in\Nat$,
  fulfilling the following properties for any vector bundles $\xi$ and
  $\eta$ over a space $B$, and any map $f\colon A\to B$ of spaces:
  \begin{axioms}
  \axiom[Naturality] $\pb f\w{i}{\xi} = \w{i}{\pb f \xi}$,
  \axiom $\w{0}{\xi}=1$,
  \axiom $\W{\gamma_1} = 1 + x$,
  \axiom[Multiplicativity]\label{tag:swclassesmultiplicativity}
  $\W{\xi \oplus \eta} = \W{\xi}\cup \W{\eta}$
    \\\idest in degree $n$ we have
    $\w{n}{\xi\oplus\eta} = \sum_{i+j=n}\w{i}{\xi} \cup \w{j}{\eta}$,
  \end{axioms}
  where the \emph{total Stiefel-Whitney class}
  $\Ws\coloneqq\sum_{i\geq 0}\ws{i}$ is the formal sum of all
  Stiefel-Whitney classes,
  $\gamma_1$ is the tautological line bundle over $\RP 1$,
  $\RPinf\cong\BO(1)\to\BO$,
  and $x$ is the%
  \footnote{
    This is well-defined: A ring $R$ of the form $\Zmod2[x]$
    with $\deg(x)=1$ only admits two elements in degree~1, $0$ and a
    generator. Therefore, there exists exactly one ring
    isomorphism, and this sends the unique generator in
    degree 1 to $x$.
  }
  generator of $\H^*(\RPinf;\Zmod2)\cong\Zmod2[x]$.
\end{Def}
Note that naturality is already implied by the requirements for a
characteristic class. However, given only the above axioms:
\begin{Thm}
  Stiefel-Whitney classes exist and are uniquely defined by the above
  properties. Furthermore, they generate the ring
  $\H^*(\BO;\Zmod2)$, which is isomorphic to $\Zmod2[\ws{i}|i\geq1]$.
  \begin{proof}[proof]
      A possible concrete construction utilises the Euler
      class. For a definition via Steenrod squares can be 
      found in Theorem~\ref{thm:altdefswclasses}.
      For uniqueness see \cite[Uniqueness Theorem~7.3]{milnor}.
      For the generating property see \forexample
      \cite[Theorem~7.1~ff.]{milnor}, or \cite[Chap.~7.6]{may}.
  \end{proof}
\end{Thm}

As already mentioned, the above generating property means that every
characteristic class of vector bundles of a fixed dimension can be
represented as a certain combination of Stiefel-Whitney classes.
Moreover, they---and hence all characteristic classes---behave
extremely well with respect to vector bundle operations as emphasised
below.
\begin{Rem}
  \label{rem:propswclasses}
  Let $\xi$, $\eta$ be vector bundles over a space $X$.
  \begin{enumerate} 
  \item\label{item:propswclasses:dimesioncut} $\w{i}{\eta} = 0$
    for any vector bundle $\eta$ with $\rk\eta < i$.
    Therefore, the total Stiefel-Whitney class $\W{\xi}$ is
    well-defined (\idest the sum is finite)
    for any vector bundle $\xi$ of finite rank.
    \begin{proof}
      See \cite[Sec.~19.4]{tomdieck}.
    \end{proof}
  \item\label{item:swoftrivbdl}
    $\w{i}{\trivbdl}=0$ for $i>0$, and one immediately concludes
    from multiplicativity:
    \begin{enumerate}
    \item\label{item:swclassesstable}
      The Stiefel-Whitney classes are stable, \idest
      $\w{i}{\xi\oplus\trivbdl} = \w{i}{\xi}$, which once more proves
      the stability property of characteristic classes of vector bundles.
      Thus, for a manifold $M^n$, all normal bundles $\N{\emb}$ of
      embeddings $\emb\colon M\to\R^{n+k+r}$ share the same
      Stiefel-Whitney classes, written $\W{\N M}$ accordingly.
      Note that $\W{\N M} = \pb{\N{M}}\ws{i}$, where $\N M$ denotes the
      classifying map of the stable normal bundle.
    \item\label{item:wuclassmfdinverse}
      If $\xi\oplus\eta = \trivbdl^{\rk\xi+\rk\eta}$, then
      $\W{\xi}\cup\W{\eta}=1$.
      Especially, for any choice of embedding $\emb\colon M^n\to\R^{n+k}$
      with normal bundle $\N{\emb}$ of a smooth manifold $M$ we have
      $\T M\oplus\N{\emb} = \trivbdl$, and therefore
      $1 = \W{\T M} \cup \W{\N{\emb}} = \W{\T M}\cup\W{\N M}$.
    \end{enumerate}
    \begin{proof}
      The trivial rank $n$ bundle over $X$ is defined as the pullback
      $\pb \pi \trivbdl^n$ of the rank $n$ bundle
      $\trivbdl^n\colon \R^n\to\pt$ over the point by the trivial map
      $\pi\colon X\to\pt$. The naturality of the Stiefel-Whitney
      classes gives $\W{\trivbdl^n} = \pb\pi \W{\trivbdl^n}
      \in\pb\pi \left(\H^*(\pt;\Zmod2)\right)$,
      and the result follows from $\H^i(\pt;\Zmod2) = 0$ for $i>0$.
    \end{proof}
  \end{enumerate}
\end{Rem}

In order to algebraically work with the Stiefel-Whitney classes, the
formal inverse is often handy. Especially, since it is well-known for
manifolds as explained below.
\begin{Def}
  Define the \emph{dual Stiefel-Whitney (characteristic) classes}
  $\dualws{i}$ in degree $i$ inductively by
  \begin{align*}
    1 &= \dualws{0}\cup\ws{0} = \dualws{0}    &&\text{in degree 0}\\
    0 &= \sum_{i+j=n} \dualws{i}\cup\ws{j}  &&\text{in degree n>0}
  \end{align*}
  With the notation $\dualWs\coloneqq \sum_{i\geq0} \dualws{i}$ as above
  for the formal sum this can be reformulated as
  \begin{gather*}
    1 = \Ws\cup\dualWs
  \end{gather*}
  in the completion of the polynomial ring $\H^*(\BO)\cong\Zmod2[\ws{i}|i\in\Nat]$.
\end{Def}
By
Remark~\itemref{rem:propswclasses}{item:swoftrivbdl}\ref{item:wuclassmfdinverse},
a first example of dual Stiefel-Whitney classes is given by the
canonical tangent and normal bundle of a manifold, which makes them
especially handy in the context relevant for the immersion conjecture.
\begin{Def}
  For a manifold $M$ use the following abbreviation
  \begin{align*}
    \W{M} &\coloneqq \W{\T M}
            \;,
    &\text{and thus}&
    &\dualW{M} &\coloneqq \dualW{\T M} = \W{\N M}
    \;.
  \end{align*}
\end{Def}

\section{Reformulation of the Immersion Conjecture}
\label{sec:reformulation}

The immersion problem can finally be clearly stated with the
definitions from \autoref{sec:immersions}.
The goal of this section is to reformulate the immersion conjecture to
a statement that can be analysed with means of homotopy theory of
vector bundles, and show how characteristic classes relate to this by
finding a powerful obstruction.
The latter will be followed up in the subsequent chapter.

Before reformulating, recall the actual immersion conjecture.
\begin{Def}
  For $n\in\Nat$ consider the unique minimal binary expansion
  \begin{gather*}
    n=2^{i_1}+\dotsb+2^{i_{l_n}},
    \quad\text{with}\quad
    i_1<\dotsb<i_{l_n}
    \;.
  \end{gather*}
  Define $\alpha(n)\coloneqq l_n$, \idest $\alpha(n)$ is the number of
  ones in the binary notation of $n$.
\end{Def}
\begin{Thm}\label{thm:immersionconj}
  For $n\in\Nat$, every closed, smooth, $n$-dimensional manifold
  immerses into $\R^{2n-\alpha(n)}$.
\end{Thm}
In the style of this conjecture, an $n$-manifold that immerses into
some $\R^{2n-\alpha(n)}$ will be said to have the
\emph{immersion property}.
And the question, whether a particular manifold does have the
immersion property, will be referred to as the \emph{immersion problem} for
this manifold.

Recall, that by the Theorem~\ref{thm:hirschsmale} of Hirsch and Smale,
any vector bundle monomorphism between tangent bundles of smooth
manifolds implies the existence of an immersion.
This is the main ingredient for the subsequent reformulation of the
immersion problem.
\begin{Thm}\label{thm:immersionconj:equivalences}
  Let $n,k\in\Nat$ and $M^n$ be a closed, smooth, $n$-dimensional manifold.
  The following statements are equivalent.
  \begin{enumerate}
  \item\label{item:immersionconj:1}
    $M$ immerses into $\R^{n+k}$.
  \item\label{item:immersionconj:2}
    There is a vector bundle monomorphism $F\colon\T M\to\T{\R^{n+k}}$.
  \item\label{item:immersionconj:3}
    There is a $k$-dimensional vector bundle
    $\N{}\colon\E{\N{}}\to M$ over $M$ with
    \begin{gather*}
      \N{}\oplus\T M\cong\trivbdl^{n+k}
      \;.
    \end{gather*}
  \item\label{item:immersionconj:4}
    For the map $\N M\colon M\to\B\Orth$ classifying the stable
    normal bundle over $M$ there is a lift $\N{}\colon M\to\B\Orth(k)$
    making the following diagram commute up to homotopy
    \begin{center}
      \begin{tikzcd}
        M
        \ar[r, "\N{}"]
        \ar[dr, "\N{M}"{left}, bend right]
        & \BO(k) \ar[d, "\incl", hookrightarrow] \\
        & \BO
      \end{tikzcd}
    \end{center}
  \end{enumerate}
\end{Thm}
% From a homotopy theoretical viewpoint, one is most
% interested in statement \ref{item:immersionconj:4}
% because it allows to apply the huge arsenal of characteristic classes
% of vector bundles, an example of which will be discussed in
% \autoref{chap:massey}.

\begin{proof}[proof of Theorem~\ref{thm:immersionconj:equivalences}]
  The strategy is to show
  \ref{item:immersionconj:1}$\Rightarrow$%
  \ref{item:immersionconj:4}$\Rightarrow$%
  \ref{item:immersionconj:3}$\Rightarrow$%
  \ref{item:immersionconj:2}$\Leftrightarrow$%
  \ref{item:immersionconj:1}.
  \begin{description}
  \item[\ref{item:immersionconj:1}$\Rightarrow$\ref{item:immersionconj:4}:]
    The classifying map of an immersion's normal bundle
    lifts $\N{M}$ as required,
    using Steenrod's classification theorem
    \itemref{def:charcls}{item:classificationthm} and the properties
    of the stable normal bundle from
    Lemma~\itemref{lem:classificationvb}{item:charclsstablenormalbundle}.
  \item[\ref{item:immersionconj:4}$\Rightarrow$\ref{item:immersionconj:3}:]
    Also, by
    Lemma~\itemref{lem:classificationvb}{item:charclsstablenormalbundle},
    any rank $k$ vector bundle that is represented by a lift of
    $\N{M}$ to $[M,\B\Orth(k)]$ as in \ref{item:immersionconj:4},
    has the property needed for \ref{item:immersionconj:3}.
  \item[\ref{item:immersionconj:3}$\Rightarrow$\ref{item:immersionconj:2}:]
    % In order to get from a vector bundle monomorphism
    % $F\colon\T M\to\T{\R^{n+k}}$ to a vector bundle dual to the
    % tangent bundle in the sense of \ref{item:immersionconj:3}, note that
    % $\pb f\T{\R^{n+k}}$ for $f=F|_{\zerosec{\T M}}$ is 
    % trivial, and that $F\colon \T M\to \pb f\T{\R^{n+k}}$ is not only
    %% Wrong according to George??
    % fibre-wise, but globally injective of constant rank
    % $\dim M$.
    % Thus, as in the definition of normal bundles,
    % $\N{}\coloneqq\pb f\T{\R^{n+k}}/\T M$ is a vector bundle over $M$
    % that obviously fulfils the required property of
    % \ref{item:immersionconj:3}.
    %
    When given some rank $k$ vector bundle $\N{}$
    such that $\N{}\oplus\T M\cong\trivbdl^{n+k}$, there is a vector
    bundle monomorphism $\T M\to\trivbdl^{n+k}$ over $M$. Then the
    following chain of vector bundle morphisms
    \begin{center}
      \begin{tikzcd}
        \T M \ar[d]
        \ar[r, hookrightarrow]
        & M\times\R^{n+k} \ar[d,"\trivbdl^{n+k}"]
        \ar[r]
        & \pt\times \R^{n+k} \ar[d,"\trivbdl^{n+k}"]
        \ar[r, hookrightarrow]
        & \R^{n+k}\times \R^{n+k} \ar[d,"\trivbdl^{n+k}"]
        \\
        M
        \ar[r,equals]
        & M
        \ar[r]
        & \pt
        \ar[r, hookrightarrow]
        & \R^{n+k}
      \end{tikzcd}
    \end{center}
    is fibre-wise injective in each stage, and hence a monomorphism as
    was needed.
  \item[\ref{item:immersionconj:1}$\Leftrightarrow$\ref{item:immersionconj:2}:]
    The tricky part is to relate \ref{item:immersionconj:1} and
    \ref{item:immersionconj:2}, even though it is easily seen that
    \ref{item:immersionconj:1} implies \ref{item:immersionconj:2} by
    simply taking $F$ to be the differential $\Diff f$ of the
    immersion from \ref{item:immersionconj:1}.
    
    The converse direction is an application of the Hirsch-Smale
    theorem~\ref{thm:hirschsmale}.
    However, first substitute the non-compact manifold $\R^{n+k}$ with
    the compact sphere $N=\Sphere{n+k}$, to make $M$ and $N$ comply
    with the assumptions of the theorem:
    As $\dim M<n+k$ by assumption,
    the image of $M$ under any immersion $M\to\Sphere{n+k}$ is a
    zero-set by Sard's theorem (see \forexample \cite[Chap.~3,
    Theorem~1.3]{hirsch}), and hence every such immersion misses a
    point in $\Sphere{n+k}$, thus factoring over an immersion
    $M\to\R^{n+k}$.
    This then shows that also \ref{item:immersionconj:2} implies
    \ref{item:immersionconj:1} which makes them equivalent.
    \qedhere
  \end{description}
\end{proof}

This now gives rise to involve the powerful obstruction theory of
characteristic classes of vector bundles as follows.
\begin{Cor}\label{cor:obstruction}
  Let $n,k\in\Nat$, and $M^n$ be a smooth, closed manifold.
  If $M$ immerses into $\R^{n+k}$, then $\dualw{i}{M}=0$ for all
  $i>k$.
  \begin{proof}
    By Theorem~\ref{thm:immersionconj:equivalences} $M$ immerses
    into $\R^{n+k}$ if and only if there is a rank-$k$
    normal bundle $\N{}$ of $M$.
    Since the Stiefel-Whitney classes are stable,
    $\W{\N{}}=\W{\N M}=\dualW{M}$.
    However, as explained in
    Remark~\itemref{rem:propswclasses}{item:propswclasses:dimesioncut},
    all Stiefel-Whitney classes $\w{i}{\N{}}$ of degree $i$ exceeding
    the rank $k$ of $\N{}$ are zero.
  \end{proof}
\end{Cor}
As a result, the immersion conjecture requires that all $n$-manifolds
have vanishing dual Stiefel-Whitney classes in degrees $i>n-\alpha(n)$.
That this is true, is a theorem of Massey which will be proven in
\autoref{chap:massey}. It was an inspiration to state the conjecture
with the value $k=n-\alpha(n)$ in the first place.


%%% Local Variables:
%%% mode: latex
%%% TeX-master: "thesis"
%%% End:


% Massey's Theorem
%%%%%%%%%%%%%%%%%%%%%%%%%%%%%%%%%% 
% Master Thesis in Mathematics
% "Immersions and Stiefel-Whitney classes of Manifolds"
% -- Chapter 3: Massey's Theorem --
% 
% Author: Gesina Schwalbe
% Supervisor: Georgios Raptis
% University of Regensburg 2018
%%%%%%%%%%%%%%%%%%%%%%%%%%%%%%%%%% 

\chapter{Massey's Theorem}

Massey's main theorem on the Stiefel-Whitney classes of manifolds
gives a very concrete obstruction on what degrees of the dual
Stiefel-Whitney classes may be non-zero
by connecting this condition with the existence of binary
representations of the manifold's dimension.
\begin{Thm}[Massey]\label{thm:massey}
  \optcite[Theorem~I.]{massey}
  Let $M$ be a compact, $n$-dimensional manifold.
  Given an integer $q$ with $0<q<n$ such that $\dualw{n-q}{M}\neq0$,
  there is a sequence of integers $h_1\geq\dotsb\geq h_q\geq0$ of
  length $q$ that fulfills
  \begin{gather*}
    n = \sum_{i=1}^{q} 2^{h_i}
  \end{gather*}
\end{Thm}

As an immediate consequence all dual Stiefel-Whitney classes of degree
greater than $n-\alpha(n)$ of any manifold must be zero, because there
cannot be any shorter representation of $n$ by powers of two
than its binary representation of length $\alpha(n)$.

The following sections are dedicated to the proof of Massey's Theorem.
Let $M$ be a compact, $n$-dimensional manifold throughout the proof.
The latter consists of several steps:
\begin{steps}
\item\label{tag:masseystep1}
  Show that for any $q$, admissible iterated Steenrod square $\Sq I$, and cohomology
  class $x\in\H^q(M)$ of degree $q$ such that $\Sq I(x)$ is non-trivial there exists
  some representation of the form
  \begin{gather*}
    \deg \left(\Sq I(x)\right)
    = 2^k\cdot
    \left( 2^{k_1}+\dotsb+2^{k_{q-1}} + 1 \right)
    \;.
  \end{gather*}
\item\label{tag:masseystep2}
  Find some iterated Steenrod square which is non-trivial in degree
  $\Sq I\colon\H^q(M)\to\H^n(M)$.
\end{steps}
Applying \ref{tag:masseystep1} to the Steenrod square $\Sq I$ from
\ref{tag:masseystep2} and some $x\in\H^q(M)$ with $\Sq I(x)\neq 0$
immediately yields the result as
\begin{gather*}
  n = \deg\left(\Sq I(x)\right) = \underbrace
  {2^{k_1+k}+\dotsb+2^{k_{q-1}+k} + 2^{k}}_{\text{$q$ summands}}
  \;.
\end{gather*}

\section{Step 1: Disecting Degrees of Iterated Steenrod Squares}% TODO: titel
\ref{tag:masseystep1} requires to proof the following claim.
\begin{Lem}[\ref{tag:masseystep1}]\label{lem:masseystep1}
  Let $q\geq 0$ be an integer,
  and $I\in\Nat^{\l(I)}$ an admissible sequence of integers.
  Further, let $x\in\H^q(M)$ be a cohomology class of degree $q$
  such that $\Sq I(x)$ is non-trivial.
  Then there exists $k\in\Nat$ and a sequence of integers
  $k_1\geq\dotsb\geq k_{q-1}\geq0$ of length $q-1$ such that the
  degree of $\Sq I(x)$ can be represented as
  \begin{gather*}
    \deg \Sq I(x)
    = \deg x + \d(I)
    = 2^k\cdot
    \left( 2^{k_1}+\dotsb+2^{k_{q-1}} + 1 \right)
    \;.
  \end{gather*}
  % Mind that the maximum length of such a sequence is $\deg\Sq I(x)$.
\end{Lem}

In order to split the proof into several cases, recall that
$\Sq I(x) = 0$ for $\e(I)>\deg x$ by
\itemref{rem:sq}{item:squpperboundgeneral}. This leaves the two cases
$\e(I)<\deg x$ and $\e(I)=\deg x$.
Inductively applying the following Lemma by Serre restricts the
proof of \autoref{lem:masseystep1} to the first case where $\e(I)<q$.
\begin{Lem}[Serre]
  \label{lem:serre}
  Every admissible sequence $I\in\Nat^{\l(I)}$ of excess $\e(I)>0$
  admits an admissible sequence $J$ with $\e(J)<\e(I)$,
  together with some $k\in\Nat$
  such that for any cohomology class $x\in\H^{\e(I)}$ holds
  \begin{gather*}
    \Sq I(x) = \left(\Sq J(x)\right) ^{2^k}
    \qquad\text{respectively}\qquad
    \deg\left(\Sq I(x)\right) = 2^k\cdot \deg\left(\Sq J(x)\right)
  \end{gather*}
\end{Lem}

Before proving \autoref{lem:serre} one can finish the argumentation for
the case $\e(I)<\deg x$.
\begin{proof}[proof of \autoref{lem:masseystep1}]
  Let $q\in\Nat$ and $I=(i_1,\dotsc,i_l)$ be an admissible sequence such that
  $\e(I)<q$.
  Assume there is a cohomology class $x\in\H^q(M)$ such that $\Sq I(x)\neq0$.
  Set
  \begin{alignat*}{4}
    \alpha_0 &= q-1-\e(I)\geq 0 &\qquad&\text{which is positive as $\e(I)<q$,} \\
    \alpha_r &= i_{r}-2i_{r+1}  &&\text{for $1\leq r< \l(I)$, and} \\ 
    \alpha_{\l(I)} &= i_{\l(I)}
    \;.
  \end{alignat*}
  It is an easy excercise that the excess of $I$ can be rewritten as
  $\e(I)=\sum_{r=1}^{\l(I)}\alpha_r$, so
  \begin{gather}\label{eq:alpha0}
    \sum_{r=0}^{\l(I)}\alpha_r = \alpha_0 + \e(I) \cequalsby{Def.} q-1
    \;.
  \end{gather}
  Just as easily one sees
  \begin{align}\label{eq:proofmassey:eq1}
    i_s
    &= \sum_{r=0}^{s}2^r\alpha_{s+r}\\\notag
  \end{align}
  The above definitions then directly imply the following reformulation
  of $\d(I)$ in terms of $\alpha_i$:
  \begin{align}\notag
    \d(I) \coloneqq \sum_{s=1}^{\l(I)}i_s 
    &\cequalsby{\eqref{eq:proofmassey:eq1}}
      \sum_{s=1}^{\l(I)}\sum_{r=0}^{s}2^r\alpha_{s+r}\\\notag
    &\equalsby{Reorder} \sum_{j=1}^{\l(I)}
      \left(\sum_{m=0}^{j-1}2^m\right)\alpha_j
      = \sum_{j=1}^{\l(I)}(2^j-1)\alpha_j 
      = \sum_{j=1}^{\l(I)}2^j\alpha_j
      - \sum_{j=1}^{\l(I)}\alpha_j\\
    \label{eq:proofmassey:eq2}
    &= \sum_{j=1}^{\l(I)}2^j\alpha_j
      - \e(I)
      \;.
  \end{align}
  All put together yields
  \begin{align*}
    \deg\left(\Sq I(x)\right)
    &= \deg(x) + \d(I)\\
    &= 1 + \deg(x) -1 +\d(I) \\
    &\equalsby{\eqref{eq:proofmassey:eq2}}
      1 + q - 1 - \e(I) + \sum_{j=1}^{\l(I)}2^j\alpha_j \\
    &\equalsby{Def.}
      1 + \alpha_0 + \sum_{j=1}^{\l(I)}2^j\alpha_j \\
    &= 1 + \sum_{j=0}^{\l(I)}2^j\alpha_j
    = 1 + \left(
      \underbrace{2^0 +\dotsb+ 2^0}_{\text{$\alpha_0$ times}}
      + \dotsb
      + \underbrace{2^{\l(I)}+\dotsb+2^{\l(I)}}_{\text{$\alpha_{\l(I)}$ times}}
      \right)
  \end{align*}
  which is one plus a sum of exactly
  $\sum_{j=0}^{\l(I)}\alpha_j\cequalsby{\eqref{eq:alpha0}} q-1$
  powers of two as was to be shown.
  The $k_j$ are in this case
  \begin{gather*}
    k_j = \begin{cases}
      0 & 0< j\leq \alpha_0\\
      1 & \alpha_0< j\leq \alpha_1\\
      \vdots\\
      \l(I) & \alpha_{\l(I)-1}<j\leq \alpha_{\l(I)}
    \end{cases}
    \qedhere
  \end{gather*}
\end{proof}

\begin{proof}[proof of \autoref{lem:serre}]
  \optcite[Lemma~1, converse part, p.~159]{serre}
  First note that any admissible sequence $I\coloneqq(i_1,\dotsc,i_l)\in\Nat^l$
  of excess $\e(I)>0$ can be written as
  $I=(2^{k-1}i_k,\dotsc,2i_k,i_k,i_{k+1},\dotsc,i_l)$
  with $l>k\geq 1$ chosen maximal, \idest $i_k>2i_{k+1}$. If $I$ is admissible, the subsequence
  $J\coloneqq(i_{k+1},\dotsc,i_l)$ will be admissible as well.
  In order to see that such $J$ and $k$ fulfil the requirements from
  \autoref{lem:serre} one only needs to show
  \begin{claim}
    For $I$ and $J$ as above, and $x\in\H^{\e(I)}(X)$ a cohomology
    class of a space $X$ holds
    \begin{enumerate}
    \item $\Sq I(x) = \left(\Sq J(x)\right)^{2^k}$, and
    \item $\e(I)<\e(J)$.
    \end{enumerate}
  \end{claim}
  For the first part one simply has to check that
  $\deg(\Sq J(x))=i_k$, as the statement then inductively follows from
  property \eqref{eq:sqsquared} that $\Sq i(y)=y^2$ for any cohomology
  class with $i=\deg(y)$.
  So calculate
  \begin{align*}
    \deg(\Sq J(x))
    &= \d(J) + \deg(x)\\
    &= \d(J) + \e(I)\\
    &= \d(J) 
      + \left(\sum_{r=1}^{\l(I)-1}(i_r-2i_{r+1})\right)
      + i_{\l(I)}\\
    &= \d(J)
      + \left(\sum_{r=1}^{k-1}\underbrace{(i_r-2i_{r+1})}_{=0}\right)
      + (i_k-2i_{k+1})
      +\left(\sum_{r=k+1}^{\l(I)-1}(i_r-2i_{r+1})\right) + i_{\l(I)}\\
    &= \d(J)
      + (i_k-2i_{k+1})
      + \e(J) \\
    &= \d(J) + i_k - 2i_{k+1} + 2i_{k+1} - \d(J) \\
    &= i_k
  \end{align*}
  Comparing the excesses yields the second part:
  \begin{gather*}
    \e(I) - \e(J)
    = \left(\sum_{r=1}^{k-1}\underbrace{(i_r-2i_{r+1})}_{=0}\right)
    + \underbrace{(i_k - 2i_{k+1})}_{\text{$>0$ by def. of k}}
    > 0
    \qedhere
  \end{gather*}
\end{proof}


\section{Step 2: Find Proper Candidate for Iterated Steenrod Square}




%%% Local Variables:
%%% mode: latex
%%% TeX-master: "thesis"
%%% End:




% Brown's Theorem
%%%%%%%%%%%%%%%%%%%%%%%%%%%%%%%%%%
% Master Thesis in Mathematics
% "Immersions and Stiefel-Whitney classes of Manifolds"
% -- Chapter 4: Brown's Theorem --
% 
% Author: Gesina Schwalbe
% Supervisor: Georgios Raptis
% University of Regensburg 2018
%%%%%%%%%%%%%%%%%%%%%%%%%%%%%%%%%% 

\chapter{Brown's Theorem}
% Review the necessary results about the structure of the unoriented cobor-
% dism ring and explain R. L. Brown’s proof of the immersion conjecture up
% to cobordism. [immersionconj], [brown]
The overall goal of this chapter is to proof the following theorem of
R.~L.~Brown first given in his paper \cite{brown},
which essentially states that the immersion conjecture is true up to
the cobordism relation.
\begin{Thm}[Brown]\label{thm:brown}
  Every closed $n$-manifold is cobordant to an $n$-manifold that immerses
  into $\R^{2n-\alpha(n)}$.
\end{Thm}

As one easily sees that this property is stable under
the ring operations (Lemma~\autoref{lem:brownstableunderringops}),
the main idea and hard part for the proof is to find manifolds
fullfilling the conjecture whose cobordism classes form an
algebraically independent generating set of the cobordism ring. This
requires
\begin{enumerate}[1)]
\item to know the structure of the cobordism ring
  (\autoref{sec:ringstructure}), in order to see how many
  indecomposable elements are needed in each degree to make up a
  generating set, and
\item a simple criterion when an element of the cobordism ring is
  indecomposable (\autoref{sec:indecomposabilitycriterion}).
\end{enumerate}
When these prerequisites are available, the generating set is
constructed in \autoref{sec:proofbrown}.

For clarity of presentation a couple of results from Thom's paper
\cite{thom} will merely be referenced without proof.

\section{Review: The Cobordism Ring Structure}
\label{sec:ringstructure}
Recall that two closed manifolds of the same dimension $n$ are
(unoriented) cobordant if their disjoint union is the border of an
$(n+1)$-dimensional manifold.
This is an equivalence relation amongst $n$-manifolds, and the
set of equivalence classes forms an abelian group $\c_n$ of order two
with the disjoint sum as addition and the $n$-sphere as zero element.
The cartesian product turns the graded $\Zmod2$-module
$\c_*\coloneqq \bigoplus_{n\geq 0}\c_n$ into an $\Zmod2$-algebra
called the (unoriented) cobordism ring.

Most remarkably, the cobordism relation is homotopy invariant, \idest
homotopic manifolds of the same dimension will be cobordant.
Further, the structure of this algebra is well-known since Thom's
research on this topic \cite{thom}:

\begin{Thm} % TODO: cobordism ring thm
  \optcite[Thm.~1.23]{immersionconj}
  \optcite[Theorem~IV.9]{thom}
  \begin{gather*}
    \c_* \cong \pi_*(\MO)\cong \Zmod2[\sigma_i| i\neq 2^r-1]
  \end{gather*}
  \cite[Theorem~IV.12]{thom}
\end{Thm}


\section{Detecting Indecomposable Elements of the Cobordism Ring}
\label{sec:indecomposabilitycriterion}
In order to find representatives for a set of algebraically
independent generators of the polynomial ring $\c_*$, one needs a way
to detect indecomposable elements. Indecomposable in this 
context means not expressable as a sum of products of lower degree
elements.
This section will follow an approach of Thom 
\cite[Chapters~IV.5 and~IV.6]{thom}.


\subsection{Special Properties of Symmetric Polynomials}
This section examines a special kind of polynomials that
obey a product rule similar to \ref{tag:cartan} whenever evaluated on
elements which have the form of a total Stiefel-Whitney class.
This will make it possible to express certain combinations of
Stiefel-Whitney numbers of product manifolds in terms of ones of their
factors.

From this the subsequent subsection will deduce a simple criterion for
a manifold to be cobordant to a product of manifolds. 

Beforehands, mind the following notation of partitions needed for
symmetrisation of polynomials.
\begin{Def}
  Let $k,l\in\Nat$ be integers.
  \begin{itemize}
  \item
    A partition $\Part=(i_1,\dotsc,i_l)$ of $k$ is an unordered sequence
    of integers such that $k=\sum_{r=0}^{l}i_r$.
    Two partitions only differing by zeros are considered equal.
    In other words, a partition is an equivalence class of sequences in
    $\bigoplus_\infty\Nat$ under the relation $\Part\sim\sigma(\Part)$
    for any permutation $\sigma$.
  \item
    The notation $I^l$ for a sequence of integers will mean a sequence
    of length $l$. Write $I^l\in\Part$ for a sequence of length
    $l$ in the equivalence class of the partition $\Part$. 
  \item
    Denote by $\PartitionsOf{k}$ the set of partitions of $k$.
  \item
    Write $\Emptypart$ for the unique partition of $0$.
  \item The concatenation of sequences and analoguesly partitions will
    be denoted by
    \begin{align*}
      \Nat^{l_1} \times \Nat^{l_2}
      &\xrightarrow{-\concat-}
        \Nat^{l_1+l_2}
      \\
      \PartitionsOf{k_1} \times \PartitionsOf{k_2}
      &\xrightarrow{-\concat-}
        \PartitionsOf{k_1+k_2}
      \\
      (i_1,\dotsc,i_r), (j_1,\dotsc,j_s)
      &\longmapsto
        (i_1,\dotsc,i_r,j_1,\dotsc,j_s)
    \end{align*}
  \end{itemize}
\end{Def}

Also as preparation, recall some properties of symmetric polynomials.
\begin{LemDef}
  Let $n\in\Nat$ and $\Zmod2[t_1,\dotsc,t_n]$ be the polynomial ring
  in $n$ variables over the fields $\Zmod2$, each $t_i$ of degree 1.
  \begin{enumerate}
  \item Let
    $\Symm{n}_*
    \coloneqq \Zmod2[t_1,\dotsc,t_n]^{\Permutations{n}}
    \subset \Zmod2[t_1,\dotsc,t_n]$
    be the graded subring of symmetric polynomials in $n$ variables.
  \item It has a basis indexed by partitions $\Part$ consisting of
    symmetrized monomials, \idest elements of the form
    \begin{gather*}
      \symm{n} t^\Part \coloneqq \sum_{I^n\in\Part} t^I \in \Symm{n}_k
      % TODO: Maybe leave the special notation for symmetrized stuff out?
    \end{gather*}
    where $t^{(i_1,\dotsc,i_n)}\coloneqq t_1^{i_1}\dotsm t_n^{i_n}$.
    \begin{proof}
      Each symmetric polynomial is the sum of homogenous symmetric
      polynomials $\sum_{I^n\in A}t^I$ which can be written as a sum
      of symmetrized monomials by descending induction on the number
      of monomial summands $\#A$.
      Linear independence is clear as monomials $t^I$ and $t^{I'}$ are
      linearly independent if $I\neq I'$, and any two partitions have
      empty intersection.
      \optcite[footnote~2, p.~154]{thom}
    \end{proof}
  \item $\Symm{n}_*$ is generated by the algebraically independent
    elementary symmetric polynomials in $n$ variables
    \begin{gather*}
      \el i n\coloneqq \symm{n} t^{\Part_i}
      \in \Symm{n}_i
      \qquad \text{for }
      \Part_i = (1,\dotsc,1)
      \in \PartitionsOf i
    \end{gather*}
    for $1\leq i\leq n$.
    \Forexample
    $\el 1 n=\sum_{r=1}^{n}t_r$,
    $\el 2 n=\sum_{1\leq r<s\leq n} t_r t_s$.
    \begin{proof}
      % TODO: ref
      This is the fundamental theorem on symmetric polynomials
      \cite[Chapter~4.4, Satz~1]{bosch2013algebra}.
    \end{proof}
  \item As a simple calculation shows, the elementary symmetric
    polynomials in $n$ variables fulfill
    \begin{gather}\label{eq:sumelemsymmpoly}
      1 + \sum_{i=1}^{n} \el i n
      = \prod_{r=1}^{n}(1+t_n)
    \end{gather}
  \end{enumerate}
\end{LemDef}

Now the desired polynomials can be defined.
\begin{Def}
  \optcite[p.~90]{milnorlectures}
  Let $k\in\Nat$, $\Part\in\PartitionsOf k$, and let
  $\Zmod2[\alpha_1,\dotsc,\alpha_k]$ be the polynomial ring in $k$
  variables where $\alpha_i$ has degree $i$.
  Define the homogenous polynomial
  $\s{\Part}\in\Zmod2[\alpha_1,\dotsc,\alpha_k]$ of degree $k$ by
  \begin{gather*}
    \s{\Part}(\el 1 n,\dotsc, \el k n) = \symm{n} t^{\Part} \in \Symm{n}_k
  \end{gather*}
  for some $n\geq k$. Mind, that this is well-defined as the
  elementary symmetric polynomials are algebraically independent, and
  the definition does not depend on $n$ as long as $k\leq n$.
  
  For a any graded ring $A_*=\bigoplus_{i\geq 0} A_i$ write elements as
  $a=\sum_i a_i \coloneqq (a_0,a_1,\dotsc)$.
  For such an element define the evaluation of $\s{\Part}$ as
  \begin{gather*}
    \s{\Part}(a) \coloneqq \s{\Part}(a_1,\dotsc,a_k)
    \;,
  \end{gather*}
  \idest skip $a_0$ and all higher $a_i$.
\end{Def}
\begin{Ex}
  The first such polynomials over $\Zmod2$ are
  \begin{align*}
    k&=0:
    &\s{\Emptypart} &= 1
    \\ k&=1:
    &\s{(1)} &= \alpha_1
    \\ k&=2:
    &\s{(2)} &= \alpha_1^2
        &\s{(1,1)} &= \alpha_2
    \\ k&=3:
    &\s{(3)} &= \alpha_1^3 + \alpha_1\alpha_2 + \alpha_3
        &\s{(1,2)} &= \alpha_1\alpha_2 + \alpha_3
             &\s{(1,1,1)} &= \alpha_3
  \end{align*}
  \cite[p.~90]{milnorlectures}
\end{Ex}

These polynomials that translate the basis $\symm{n}t^{\Part}$ of
$\Symm{n}_k$ into expressions of the generators $\el i n$
fulfill as promised the following interesting property concerning
multiplication of elements.
\begin{Lem}\label{lem:productrule:general}
  \optcite[Theorem~33, p.~91f]{milnorlectures}
  Let $k\in\Nat$, and let $A_*=\bigoplus_{i\geq 0} A_i$ be a graded ring.
  For $a,b\in A_*$ with $a_0=1=b_0$, and any partition
  $\Part\in\PartitionsOf k$ holds
  \begin{gather*}
    \s{\Part}(a\cdot b)
    = \sum_{\Part_1\concat\Part_2=\Part}
    \s{\Part_1}(a) \cdot \s{\Part_2}(b)
    \;.
  \end{gather*}
  \begin{proof}
    As for the definition of $\s{\Part}$ it suffices to check equality
    on algebraically independent elements. So, make the following choices:
    \begin{itemize}
    \item The result will be independent of $n$, as long as
      $n$ is large enough, thus choose $n=2k$.
      Note that this is the smallest choice for $n$ for which there
      may be a pair of sequences $J_1^k\concat J_2^k\in\Part$
      where one is all zeros.
    \item For simplicity of notation write
      $x_i=t_i$ and $y_i=t_{2i}$ for $1\leq i\leq k$ to split the
      variables $(t_1,\dotsc,t_{2k})$ into two equal parts.
    \item In order to directly work with the definition of $\s{\Part}$ it is
      convenient to choose as algebraically independent elements
      \begin{align*}
        a &\coloneqq
            \prod_{\mathclap{r=1}}^k(1+t_r)
            \cequalsby{\eqref{eq:sumelemsymmpoly}}
            1 + \sum_{i=1}^k \el i k(x_1,\dotsc,x_k)
            \;\text{, and}\\
        b &\coloneqq
            \prod_{\mathclap{r=k+1}}^n(1+t_r)
            \cequalsby{\eqref{eq:sumelemsymmpoly}}
            1 + \sum_{i=1}^k \el i k(y_1,\dotsc,y_k)
            \;\text{, then}\\
        a\cdot b
          &=
            \prod_{r=1}^n (1+t_r)
            \cequalsby{\eqref{eq:sumelemsymmpoly}}
            1 + \sum_{i=1}^n \el i n (t_1,\dotsc,t_n)
            \;.
      \end{align*}
      Consequently, the sets 
      $(a_1,\dotsc,a_k)$, $(b_1,\dotsc,b_k)$, and
      $(a\cdot b)_{1\leq i\leq n}$ are algebraically independent, as
      was needed.
    \end{itemize}
    Now calculate
    \begin{align*}
      \s{\Part}(a\cdot b)
      &\coloneqq
        \symm{n} t^{\Part}
      \\ &\equalsby{Def.}
           \sum_{I^n\in\Part}
           t_1^{i_1}\dotsm t_n^{i_n}
      \\ &=
           \sum_{I^n\in\Part}
           x^{(i_1,\dotsc,i_k)}\cdot y^{(i_{k+1},\dotsc,i_n)}
      \\ &=
           \sum_{J_1^k\concat J_2^k\in\Part}
           x^{J_1} \cdot y^{J_2}
      \\ &\equalsby{Group by equiv.}
           \sum_{\Part_1\concat \Part_2=\Part}
           \sum_{\substack{J_1^k\in\Part_1\\ J_2^k\in\Part_2}}
      x^{J_1} \cdot y^{J_2}
      \\ &=
           \sum_{\Part_1\concat \Part_2=\Part}
           \left(\sum_{J_1^k\in\Part_1} x^{J_1}\right)
           \cdot
           \left(\sum_{J_2^k\in\Part_2} y^{J_2}\right)
      \\ &\equalsby{Def.}
           \sum_{\Part_1\concat \Part_2=\Part}
           \left(\symm{k} x^{\Part_1}\right)
           \cdot
           \left(\symm{k} y^{\Part_2}\right)
      \\ &\equalsby{Def.}
           \sum_{\Part_1\concat \Part_2=\Part}
           \s{\Part_1}\left( \el 1 k(x_1,\dotsc,x_k),\dotsc \right)
           \cdot
           \s{\Part_2}\left( \el 1 k(y_1,\dotsc,y_k),\dotsc \right)
      \\ &\equalsby{Def.}
           \sum_{\Part_1\concat \Part_2=\Part}
           \s{\Part_1}(a) \cdot \s{\Part_2}(b)
           \qedhere
    \end{align*}
  \end{proof}
\end{Lem}

\begin{Ex}
  The most important partition of an integer $k$ will be the trivial
  one $(k)\in\PartitionsOf k$. In this case
  Lemma~\autoref{lem:productrule:general} says
  \begin{gather*}
    \s{(k)}(a\cdot b) = \s{(k)}(a) + \s{(k)}(b)
    \;.
  \end{gather*}
\end{Ex}


\subsection{Stiefel-Whitney Numbers of Product Manifolds}
In order to apply the special polynomials from the preceeding section
and their product property \autoref{lem:productrule:general} to
the Stiefel-Whitney numbers of (product) manifolds, first start with application
to Stiefel-Whitney classes.

So, let $M^n=M_1^{n_1}\times M_2^{n_2}$ all be closed manifolds of the
noted dimension throughout the section.

Recall that
\begin{enumerate}
\item the cohomology ring $\H^*(M)$ of a space respectively
  manifold is a graded ring,
\item the total Stiefel-Whitney number of a manifold is of the form
  $\W{M_i} = 1 + \w 1 {M_i} + \dotsb + \w n {M_i}$, and that
\item by Künneth holds
  \begin{align*}
    \H^*(M_1)\otimes \H^*(M_2)
    &\overset{\cong}\longto \H^*(M)
    \\
    c_1 \otimes c_2
    &\longmapsto c_1\cdot c_2
      \coloneqq \pb\proj_1 c_1 \cdot \pb\proj_2 c_2
  \end{align*}
  and $\W{M} = \W{M_1} \cdot \W{M_2}$ by
  \ref{tag:swclassesmultiplicativity} of the Stiefel-Whitney classes.
\end{enumerate}
Thus, one can apply the multiplication
Lemma~\autoref{lem:productrule:general} to $\W{M}$, which immediately
yields:
\begin{Cor}\label{cor:productrule:swcl}
  For $M=M_1\times M_2$ manifolds as above one gets for any partition
  $\Part\in\PartitionsOf n$:
  \begin{align*}
    \s{\Part}(\W{M})
    =
    \s{\Part}\left(\W{M_1}\cdot \W{M_2}\right)
    &\cequalsby{\autoref{lem:productrule:general}}
      \sum_{\mathclap{\Part_1\concat\Part_2=\Part}}
      \s{\Part_1}(\W{M_1}) \cdot \s{\Part_2}(\W{M_2})
    \\ &=
         \sum_{\mathclap{\substack{
         \Part_1\concat\Part_2=\Part\\
    \Part_1\in\PartitionsOf{n_1}\\
    \Part_2\in\PartitionsOf{n_2}\\
    }}}
    \s{\Part_1}(\W{M_1}) \cdot \s{\Part_2}(\W{M_2})
  \end{align*}
  \begin{proof}
    The last equality is due to the following dimension reasons.
    For any any manifold $W$ and a partition $\Part$ of $k$,
    $\s{\Part}(\W{W})$ is of degree $k$ if $k\leq\dim W$ and zero else.
    Therefore, for any combination of partitions
    $\Part_1\in\PartitionsOf{k_1}$,
    $\Part_2\in\PartitionsOf{k_2}$
    where $k_1+k_2=n=n_1+n_2$ with $k_i\neq n_i$, the product
    $\s{\Part_1}(\W{M_1})\cdot\s{\Part_2}(\W{M_2})$ will have a zero
    factor and can be skipped.
  \end{proof}
\end{Cor}

In order to pass to Stiefel-Whitney numbers instead of classes, use
the following notation.
\begin{Def}
  Let $W$ be a closed manifold and $\Part\in\PartitionsOf{\dim W}$.
  Then write
  \begin{align*}
    \s{\Part}(W) &\coloneqq \s{\Part}(\W{W})\\
    \snum{\Part}{W} &\coloneqq \capped{\s{\Part}(W)}{\fundcl{W}}
    \;.
  \end{align*}
\end{Def}

Now the product rule \autoref{lem:productrule:general} translates to
\begin{Cor}
  For closed manifolds $M_1$, and $M_2$ with dimensions $n_1$ and
  $n_2$ one has
  \begin{align*}
    \snum{\Part}{M_1\times M_2}
    &= \sum_{\mathclap{\substack{
      \Part_1\concat\Part_2=\Part\\
    \Part_1\in\PartitionsOf{n_1}\\
    \Part_2\in\PartitionsOf{n_2}\\
    }}}
    \snum{\Part_1}{M_1} \cdot \snum{\Part_2}{M_2}
    \quad \in \quad \Zmod2
  \end{align*}
  and as a special case
  \begin{align}
    % \notag
    % \snum{(n_1,n_2)}{M_1\times M_2}
    % &= \snum{(n_1)}{M_1} \cdot \snum{(n_2)}{M_2}
    % \\
    \label{eq:productrule:swnum}
    \snum{(n_1+n_2)}{M_1\times M_2} &= 0
                                      \;.
  \end{align}
  In particular, if $\snum{(\dim W)}{W}\neq 0$ for a closed manifold
  $W$, then $W$ is no product manifold.
  \begin{proof}
    The statement immediately follows from the previous
    Corollary~\autoref{cor:productrule:swcl} with the following
    two facts:
    \begin{enumerate}
    \item The generator $\fundcl{M_1\times
        M_2}\in\H^{n_1+n_2}(M_1\times M_2)$ corresponds to the generator
      $\fundcl{M_1}\otimes\fundcl{M_2}\in
      \H^{n_1}(M_1)\otimes\H^{n_2}(M_2)\cong \H^{n_1+n_2}(M_1\times M_2)$
      under the Künneth isomorphism.
    \item For cohomology classes $c_1\in\H^{n_1}(M_1)$, $c_2\in\H^{n_2}(M_2)$ holds
      \begin{gather*}
        \capped{c_1\otimes c_2}{\fundcl{M_1\times M_2}}
        = \capped{c_1\otimes c_2}{\fundcl{M_1}\otimes\fundcl{M_2}}
        = \capped{c_1}{\fundcl{M_1}} \cdot \capped{c_2}{\fundcl{M_2}}
        \in \Zmod2
      \end{gather*}
      by the universal property of the tensor product.
      \qedhere
    \end{enumerate}    
  \end{proof}
\end{Cor}

\autoref{eq:productrule:swnum} is the desired obstruction for
a manifold to be a product or---as will be explained in the next
subsection---to be cobordant to a product.
Finally, this statement will be an invaluable tool for detecting
manifolds that are not only not cobordant to a product manifold but
whose cobordism class is indecomposable.


\subsection{A Criterion for Indecomposability}
The ultimate goal of this subsection is to deduce the following
theorem which is a consequence of Thom's proof of the multiplicative
structure of the cobordism ring (see \cite[Section~IV.5]{thom}).
\begin{Thm}\label{thm:indecomposabilitycriterion}
  A closed $n$-manifold $M$ represents an indecomposable element of
  the cobordism ring $\c_*$, if and only if
  \begin{gather*}
    \snum{(n)}{M} \neq 0 \in \Zmod2
    \;.
  \end{gather*}
  % for non-dyadic decomposition: \optcite[p.~156]{thom};
  % for summation notation: \optcite[p.154, (4)f]{thom};
\end{Thm}

For the mentioned proof, which includes the above statement, Thom
constructs special manifolds that form a basis of the cobordism ring,
and are uniquely characterized by the below properties.
For the formulation call a sequence or partition \emph{non-dyadic}, if
none of its entries is of the form $2^m-1$. 
\begin{Thm}\label{thm:basiscobordismring} % TODO: Proof
  There exists a basis of the cobordism ring represented by manifolds
  $V_\Part$ that is in each degree $k$ indexed by non-dyadic
  partitions $\Part$ of $k$. Further, the $V_\Part$ are uniquely
  characterized by
  \begin{gather*}
    \s{\Part'}(\W{V_\Part}) = \delta_{\Part,\Part'}
    \in \H^k(V_{\Part})\cong \Zmod2
  \end{gather*}
  for any non-dyadic partitions $\Part,\Part'\in\PartitionsOf{k}$,
  where $\delta$ is the usual Kronecker delta
  \cite[Section~IV.5, p.~194ff]{thom}.
  \begin{proof}
    % TODO: proof/construction special basis cobordism ring [Thom]; [Milnor] better source?
  \end{proof}
\end{Thm}

Before starting with the proof of the desired result one needs the
following connection between Stiefel-Whitney classes of a manifold and
its cobordism class. Thom rather directly deduces this from the
existence of the above basis.
\begin{Thm}[Thom]\label{thm:cobordantiffswnumscoincide}
  Two closed manifolds are cobordant if and only if all of their
  Stiefel-Whitney numbers coincide.
  \begin{proof}[proof (sketch)]
    The proof that manifolds with coinciding Stiefel-Whitney numbers
    are cobordant was conducted by Thom \cite[Theorem IV.10]{thom}.
    To see that cobordant manifolds have the same Stiefel-Whitney
    numbers, let $M^n$ be a null-bordant closed manifold, \idest
    assume $M^n=\Boundary{W}$ for a closed manifold $W$. Now, consider
    any Stiefel-Whitney number 
    $\capped{\w{1}{M}^{i_1}\dotsm\w{l}{M}^{i_l}}{\fundcl{M}}$,
    of $M$, where $I=(i_1,\dotsc,i_l)$ with $\d(I)=n$.
    Then this will be zero, as one calculates using the long
    exact sequence of cohomology respectively homology of the pair
    $i\colon M\immto W$ (abbreviated les)
    \begin{align*}
      \W{M}
      &= \W{\T M}
        = \W{\T M\oplus\trivbdl}
        = \W{\T W|_{M}}
        \cequalsby{Def.} \W{\pb i \T W} = \pb i \W{W} \\
      \fundcl M
      &\cequalsby{les} \partial \fundcl{W} \\
      \intertext{which yields with the fact
      $\pf i\circ\partial\overset{\text{les}}= 0$}
      \capped{\W{M}^I}{\fundcl M}
      &= \capped{\pb i\W{W}^I}{\partial\fundcl{W}}
        = \pf i\capped{\W{W}^I}{\pf i\partial \fundcl{W}}
        = 0
        \;.
        \qedhere
    \end{align*}
  \end{proof}
\end{Thm}

\begin{proof}[proof of \autoref{thm:indecomposabilitycriterion}]
  Using Theorems~\autoref{thm:basiscobordismring} and
  \autoref{thm:cobordantiffswnumscoincide} enables to proof the
  desired result using the following intermediate conclusions.
  \begin{enumerate}
  \item Any manifold $W$ of dimension $k$ is
    cobordant to
    \begin{gather*}
      \coprod_{\mathclap{\substack{
            \Part\in\PartitionsOf k \text{ non-dyadic}\\
            \s{\Part}(\W{W})\neq 0
          }}} V_\Part
      \;,
    \end{gather*}
    as the Stiefel-Whitney numbers are additive with respect to
    disjoint union. % TODO: check
  \item\label{item:manifoldbasisrepr}
    $V_{\Part_1}\times V_{\Part_2}$ is cobordant to
    $V_{\Part_1\concat\Part_2}$ since
    $\s{\Part'}(\W{V_{\Part_1}\times V_{\Part_2}})
    = \delta_{\Part', \Part_1\concat\Part_2}$
    by \ref{cor:productrule:swcl}, and thus all their Stiefel-Whitney
    numbers coincide by definition.
  \item\label{item:generatorscobordimsring}
    By conclusion \ref{item:manifoldbasisrepr},
    the basis elements represented by a $k$-dimensional manifold
    $V_{(k)}\in\c_k$ for $k\neq 2^m-1$ are algebraically independent
    generators of the cobordism ring.
  \item By conclusion \ref{item:generatorscobordimsring}, the bordism
    class of a manifold $W$ of dimension $k$ is an indecomposable
    element of $\c_*$ if and only if its representation by basis
    elements $[V_\Part]$ contains the unique $k$-dimensional
    indecomposable basis element $[V_{(k)}]$ as summand,
    \idest if and only if $\snum{(k)}{W}\neq 0$.
    \qedhere
  \end{enumerate}
\end{proof}

\section
{Proof of Brown's Theorem: Finding a Convenient Generating Set}
\label{sec:proofbrown}
Remember that the goal of this chapter was to prove R.~L.~Brown's
theorem \ref{thm:brown} which states that the immersion conjecture is
true up to cobordism.

If one can show that the conjecture's statement is stable
under the ring operations of the cobordism ring, it will suffice to
find a generating set of $\c_*$ which fulfills the conjecture.
The latter will heavily rely on the indecomposability criterion found
in \autoref{thm:indecomposabilitycriterion} and the structure of the
cobordism ring.

The stability properties needed are the following.
\begin{Lem}\label{lem:brownstableunderringops}
  Let $M_i^{n_i}$ be a closed manifold immersing into
  $\R^{2n_i-\alpha(n_i)}$ for $i=1,2$.
  Then both manifolds
  \begin{enumerate}
  \item $M_1\disjointsum M_2$ for $n_1=n_2=n$, and
  \item $M_1\times M_2$ for $n_1$, $n_2$ arbitrary
  \end{enumerate}
  immerse into $\R^{2(n_1+n_2)-\alpha(n_1+n_2)}$.
  \begin{proof}
    $M_1\times M_2$ immerses into the real space of dimension
    \begin{align*}
      \left( 2n_1-\alpha(n_1) \right)
      + \left( 2n_2-\alpha(n_2) \right)
      &= 2(n_1+n_2) - \left(\alpha(n_1)+\alpha(n_2)\right)\\
      &\leq 2(n_1+n_2) - \left(\alpha(n_1 + n_2)\right)
    \end{align*}
    where the inequality is due to the number theoretic fact
    $\alpha(n_1+n_2) \leq \alpha(n_1)+\alpha(n_2)$.

    For $n_1=n=n_2$ the images of the immersions
    \begin{gather*}
      \iota_1\colon M_1\to\R^{2n-\alpha(n)}
      \qquad \text{and} \qquad
      \iota_2\colon M_2\to\R^{2n-\alpha(n)}
    \end{gather*}
    are compact. So by
    concatenation with translation they can be assumed to be disjoint,
    wherefore the disjoint union
    $\iota_1\disjointsum\iota_2\colon M_1\disjointsum M_2\to\R^{2n-\alpha(n)}$
    is again an immersion.
  \end{proof}
\end{Lem}

Thus, for the proof of Brown's theorem only a generating set of $\c_*$
is needed each member of which fulfills the immersion conjecture.

\subsection{Twisted Products}
The candidates for a generating set will be inductively constructed
using the so-called twisted product construction explained below.
The main advantage of this tool is---besides quite a couple of handy
preservation properties---the fact that a twisted product is
indecomposable if and only if its factor is and the dimension was
chosen correctly. The latter will be the main result of this section.

\begin{Def}
  \optcite[p.~83]{immersionconj}
  \optcite[compare §4, Def.~of~$P(m;X)$]{brown}
  Let $X$ be a space and $k\in\Nat$ an integer.
  Define the \emph{twisted product of $X$ by $\Sphere k$}, denoted
  $\Twistedprod{k}{X}$, to be the orbit space of the properly
  discontinuous $\Zmod2$-action on $\Twistedprodcovspace{k}{X}$ given
  by
  \begin{align*}
    \Zmod2 &\leftgroupaction \Twistedprodcovspace{k}{X}
             \;,
    &[1] \actson (s, (p_1,p_2)) &\coloneqq (-s, (p_2, p_1))
                                \;,
  \end{align*}
  which combines the antipodal action $[1]\actson s\coloneqq -s$ on
  $\Sphere k$ and twisting on $X\times X$.
  For a map $f\colon X\to Y$ of spaces, define
  \begin{gather*}
    \Twistedprod{k}{f}\coloneqq \Id\times f\times f/\sim \colon
    \Twistedprod{k}{X}\longto\Twistedprod{k}{Y}
    \;,\quad
    [s,(p_1,p_2)] \longmapsto [s,(f(p_1),f(p_2))]
    \;.
  \end{gather*}
\end{Def}
\begin{Ex}
  \begin{itemize}
  \item $\Twistedprod k {\pt} = \RP k$
  \item $\Twistedprod 0 {M} = M\times M$
  \end{itemize}
\end{Ex}

First, gather some rather immediate, convenient properties. It is
especially noteworthy how well the twisted product behaves concerning
manifolds and fibre bundles.
\begin{Rem}\label{rem:twistedprodproperties}
  Let $X$ be a space and $k\in\Nat$.
  \begin{enumerate}
  \item $\Twistedprod{-}{k}$ is a functor on the category of
    topological spaces preserving injectivity.
  \item\label{item:twistedprodfibrebdl}
    $\Twistedprod{k}{-}$ preserves fibre bundles,
    \idest for a fibre bundle $\xi\colon\E\xi\to X$ with fibre $F$
    the twisted product $\Twistedprod{k}{\xi}\colon
    \Twistedprod{k}{\E\xi}\to \Twistedprod{k}{X}$
    is again a fibre bundle with fibre $F\times F$.
    This comes from the fibre bundle
    \begin{gather*}
      F\times F
      \longto \Twistedprodcovspace{k}{\E\xi}
      \longto \Twistedprodcovspace{k}{X}
    \end{gather*}
    where all maps are maps of $\Zmod2$-sets.
    As a special case, $\Twistedprod{k}{X}$ admits a fibre bundle
    \begin{gather}\label{eq:twistedprodrpnfibrebdl}
      X\times X
      \longto \Twistedprod{k}{X}
      \longto \RP k = \Sphere k/\sim
    \end{gather}
    with fibre $X\times X$ which comes from the trivial fibre bundle
    $X\to\pt$.
    Further, let $\eta\colon\E\eta\to X$ be
    another fibre bundle, and $f\colon X'\to X$ a map.
    One has:
    \begin{enumerate}
    \item $\Twistedprod{k}{-}$ respects sums of fibre bundles, \idest
      \begin{gather*}
        \Twistedprod{k}{\xi\oplus\eta\colon \E\xi\oplus\E\eta\to X}
        = \Twistedprod{k}{\xi}\oplus\Twistedprod{k}{\eta}
      \end{gather*}
    \item\label{item:twistedprod:preservespb}
      $\Twistedprod{k}{-}$ respects pullbacks, \idest
      \begin{gather*}
        \Twistedprod{k}{\pb f \xi}
        = \pb{\left(\Twistedprod{k}{f}\right)}
        \left(\Twistedprod{k}{\xi}\right)
      \end{gather*}
    \end{enumerate}
  \item\label{item:twistedprodmanifold}
    $\Twistedprod{k}{-}$ preserves closed smooth manifolds, \idest
    for a closed smooth manifold $M^n$, $\Twistedprod{k}{M}$ is a
    again a $(2m+k)$-dimensional closed smooth manifold.
    This is because the proper discontinuity comes from the antipodal
    $\Zmod2$-action, and makes the projection
    \begin{gather*}
      \Twistedprodcovspace{k}{X}
      \xrightarrow{\pi}
      \Twistedprod{k}{X}
      \coloneqq
      \left( \Twistedprodcovspace{k}{X} \right)/\sim
    \end{gather*}
    a two-leaved covering space.
    Further:
    \begin{enumerate}
    \item\label{item:twistedprodpreservesimmersions}
      $\Twistedprod{k}{-}$ preserves immersions.
    \item\label{item:twistedprod:tangentspace}
      $\T\Twistedprod{k}{M}
      \cong \pb\proj\T{\RP k} \oplus \Twistedprod{k}{\T M}$,
      \idest the tangent space of $\Twistedprod{k}{M}$ can be obtained
      from $\Twistedprod{k}{\T M}$ by adding the missing tangent space
      part of the sphere:
      \begin{alignat*}{4}
        \T{\Twistedprod{k}{M}}
        &\cong& \T{\left(\Twistedprodcovspace{k}{M}\right)}/\sim \\
        &\cong& \T \Sphere k\times \T M\times\T M/\sim
        &\overset{\cong}{\longto}
        \pb\proj \T{\RP k} \oplus \Twistedprod{k}{\T M}
        \\
        &&[(s,v), (m_1,v_1), (m_2,v_2)]
        &\longmapsto
          \left( ([s], v), [s, (m_1,v_1), (m_2, v_2)] \right)
      \end{alignat*}
      where $\proj\colon\Twistedprod{k}{M}\to\RP k$ is the projection.
      The first isomorphism is due to the covering space property, and
      the last is easily seen to be a well-defined isomorphism of
      vector bundles.
      Further note that for a map of manifolds $f\colon M\to N$,
      the differential map $\mathrm{D}\Twistedprod{k}{f}$ on tangent spaces
      will be the identity on the first summand.
    \end{enumerate}
  \end{enumerate}
\end{Rem}

Besides the above direct properties, there is a fairly easy
description of the cohomology ring of a twisted product relating it to
the cohomology ring of its factor.
\begin{Thm}\label{thm:twistedprod:cohomstructure}
  \optcite[Thm~7.1, p.~1111]{brown}
  Let $X$ be a space and $k\in\Nat$.
  Define
  \begin{itemize}
  \item
    $N\coloneqq
    \left\{
        a\otimes b + b\otimes a
        \in \H^*(X\times X)\cong \H^*(X)\otimes\H^*(X)
      \right\}$,
  \item
    $d\colon \H^*(X)\to\H^*(X\times X)$,
    $d(a)\coloneqq a\times a$, and
    $D\coloneqq
    \left\{ d(a)=a\otimes a \in \H^*(X\times X) \right\}$,
  \item
    $s\in\H^k(\Sphere k)\cong\Zmod2$ to be the generator, and
  \item
    $c\coloneqq \pb\proj x$ to be the pullback of the generator
    $x\in\H^*(\RP k)\cong\Zmod2[x]$.
  \end{itemize}
  Note that $N+D$ is closed under multiplication and addition.
  Then the cohomology ring of $\Twistedprod{k}{X}$ has the form
  \begin{gather*}
    \H^*(\Twistedprod{k}{X})
    \cong
    \left(\H^*(\RP k)\otimes D\right)
    + \left( 1\otimes N \right)
    + \left( \H^m(\Sphere m) \otimes N \right)
  \end{gather*}
  with the induced grading and multiplication which respects the
  relation $c\otimes N=0$.
  For readability skip $1\otimes$ and $\otimes d(1)$ in element
  notation wherever it is clear to which part the summand belongs.
  Further it fulfills
  \begin{enumerate}
  \item
    For $\pi\colon \Twistedprodcovspace{k}{X}\to\Twistedprod{k}{X}$
    holds
    \begin{align}\label{eq:twistedprodcohom:pi}
      \pb\pi(c)&=0
      &\text{and}&
      &\pb\pi(c^r\otimes d(a_1) + 1\otimes n_1 + s\otimes n_2)
      &= 1\otimes n_1 + s\otimes n_2
        \;.
    \end{align}
  \item
    For any section $s_p\colon\RP k\to\Twistedprod{k}{X}$, $q\mapsto[q,p,p]$,
    of the fibre bundle described in
    \eqref{eq:twistedprodrpnfibrebdl}, and $n_1,n_2\in N$, $a\in\H^*(X)$ holds
    \begin{align}\label{eq:twistedprodcohom:section}
      \pb s_p(c^r\otimes d(a)) &= \delta_{a,1} \cdot x^r
      &\text{and}&
      &\pb s_p(1\otimes n_1 + s\otimes n_2) &= 0
      \;.
    \end{align}
    where $\delta_{a,1}$ is the Kronecker delta.
  \item\label{item:twistedprod:preservescohominj}
    $\Twistedprod{k}{-}$ preserves injectivity on cohomology.
  \end{enumerate}
\end{Thm}

This then immediately gives a result on the Stiefel-Whitney classes of
a twisted product of line bundles that will be used below to simplify
calculations involving higher dimensional bundles.
\begin{Cor}\label{cor:twistedprod:swlinebdl}
  \optcite[Prop.~7.4, p.~1113]{brown}
  Let $\xi\colon E\to X$ be a line bundle with total Stiefel-Whitney
  class $\W{\xi}=1+\alpha$.
  Define $e\colon\H^*(X)\to N\subset\H^*(X\times X)$,
  $e(a)\coloneqq 1\otimes a+a\otimes 1$.
  Then
  \begin{gather*}
    \W{\Twistedprod{k}{\xi}} = 1+ (c\otimes d(1)+1\otimes
    e(\alpha)) + 1\otimes d(\alpha)
    = 1+c+e(\alpha)+d(\alpha)
    \;,
  \end{gather*}
  respectively $\w1\xi = c+e(\alpha)$, $\w2\xi=d(\alpha)$.
  \begin{proof}
    For $k=0$ this is simply the product rule for the total
    Stiefel-Whitney class because $\Twistedprod{0}{\xi}=\xi\times\xi$
    and $c=0$. Thus, assume $k\geq0$.
    Then by \autoref{thm:twistedprod:cohomstructure} the total
    Stiefel-Whitney class of the two-dimensional vector bundle
    $\Twistedprod{k}{\xi}$ must be of the general form
    \begin{align*}
      \W{\Twistedprod{k}{\xi}}
      &=
        1 &\quad&&&{(=\ws0)} \\
      &+ \delta_1\cdot c\otimes d(1) + \delta_2\cdot c^2\otimes d(1)
          &&\delta_1,\delta_2\in\{0,1\}
            &&{(\text{check section})}\\
      &+ 1\otimes d(a)
          &&a \neq 1
                &&{(\text{check $\pi$})} \\
      &+ {\textstyle \sum_{r\in I} c^{i_r}\otimes d(a_r)}
          &&a_r\neq 1,\; i_r>1
            &&{(=0\text{ by dim.})} \\
      &+ 1\otimes n_1 + s\otimes n_2
          &&n_2, 1\neq n_1\in N
                &&{(\text{check $\pi$})}
    \end{align*}
    The index set $I$ must be empty as any $c^{i_r}\otimes d(a_r)$
    would have dimension greater two.
    So, it remains to check the pullbacks of $\W{\Twistedprod{k}{\xi}}$
    along a section $s_p\colon\RP k\to\Twistedprod{k}{X}$ and along
    the projection
    $\pi\colon\Twistedprodcovspace{k}{X}\to\Twistedprod{k}{X}$
    in order to identify the other summands.
    \begin{description}
    \item[$\pb\pi\W{\Twistedprod k \xi}$:]
      $\Twistedprod{k}{\xi}$ is the quotient of the bundle
      $\trivbdl\times\xi\times\xi\colon
      \Twistedprodcovspace{k}{E}\to\Twistedprodcovspace{k}{X}$, where
      \begin{gather*}
        \W{\Id\times\xi\times\xi}
        = 1\cdot \W{\xi}\cdot\W{\xi}
        = 1 + 1\otimes e(\alpha) + 1\otimes d(\alpha)
        \;.
      \end{gather*}
      As $\pi$ is a covering map,
      % TODO: nicify reasoning?
      $\pb\pi\Twistedprod{k}{\xi}=\trivbdl\times\xi\times\xi$,
      and thus
      $\pb\pi\W{\Twistedprod{k}{\xi}}=\W{\Id\times\xi\times\xi}$.
      By \eqref{eq:twistedprodcohom:pi} $\pb\pi$ is the identity on
      elements of this form, this must be a summand of
      $\W{\Twistedprod{k}{\xi}}$.
    \item[$\pb s_p \W{\Twistedprod{k}{\xi}}$:]
        The pullback of $\Twistedprod{k}{\xi}$ along this section is
        \begin{gather*}
          \Twistedprod{k}{E_p}=\Twistedprod{k}{\R}\to \RP k
          \;,\quad
          [s, v_1,v_2] \mapsto [s]
          \;.
        \end{gather*}
        In order to simplify this, use the well-defined vector bundle
        isomorphism
        \optcite[Prop.~4.3,p~1107]{brown}
        \begin{align*}
          \Twistedprod{k}{\R}
          &\longto
            (\Twistedprodcovspace{k}{\R}/\approx)
            \cong \E{(\gamma\oplus\trivbdl)} \\
          [s, v_1, v_2]
          &\longmapsto
            [s, v_1+v_2, v_1-v_2]
        \end{align*}
        where $\approx$ is the equivalence relation identifying
        $(s,v_1,v_2)$ and $(-s,-v_1,v_2)$, and
        $\gamma$ is the tautological line bundle for $\RP k$.
        % TODO: Give a fitting description of the tautological line bdl somewhere
        Then $\pb{s_p}\Twistedprod{k}{\xi} \cong
        \gamma\oplus\trivbdl$ and $\pb{s_p}\W{\xi} =
        \W{\gamma\oplus\trivbdl} = 1+x$ completing the proof.
      \end{description}
    \end{proof}
\end{Cor}

Now that the cohomology ring of a twisted product is well-known, one
can investigate the Stiefel-Whitney numbers of twisted product manifolds.
As promised, the following result will be the cornerstone when inductively
defining the desired basis for the cobordism ring in the subsequent section.
\begin{Thm}\label{thm:twistedprod:indecompcriterion}
  Let $M^n$ be a manifold and $k\in\Nat_{\geq1}$.
  then $\Twistedprod{k}{M}$ represents an indecomposable class of the
  cobordism ring if and only if $M$ does and $\binom{k+n-1}{n}$ is
  non-zero modulo two.
\end{Thm}
Using Theorem~\autoref{thm:indecomposabilitycriterion}, saying
a manifold $M^n$ represents an indecomposable element if and only if
$\snum{(n)}{M}$ is non-zero,
Theorem~\autoref{thm:twistedprod:indecompcriterion} is a direct
consequence of the following Lemma.
\begin{Lem}\label{lem:twistedprod:indecompcriterion}
  Let $M$, $n$, $k$ be as in
  \autoref{thm:twistedprod:indecompcriterion} above.
  Then there is a map $f\colon X\to M$ of spaces which is injective on
  cohomology and fulfills
  \begin{gather*}
    \pb{\Twistedprod{k}{f}} \s{(2n+k)}(\Twistedprod{k}{M})
    = \binom{k+n-1}{n} \cdot c^k
    \cdot d\left( \pb f \s{(n)}(M) \right)
    \in \H^{2n+k}(\Twistedprod{k}{M})
    \;.
  \end{gather*}
\end{Lem}
\begin{proof}[proof of \autoref{lem:twistedprod:indecompcriterion}]
  As the Stiefel-Whitney classes of twisted products of line bundles
  are well-known, take $f$ to be a reduction of $\T M$ to line bundles
  using the splitting principle.
  \Idest choose a space $X$ and a map $f\colon X\to M$ which is
  injective on cohomology and fulfills
  $\pb f \T M = \xi_1\oplus\dotsb\oplus\xi_n$ for line bundles
  $\xi_i$ over $X$, each with total Stiefel-Whitney class
  $\W{\xi_i}=1+\alpha_i$.
  With the fibre bundle properties from
  \itemref{rem:twistedprodproperties}{item:twistedprodfibrebdl}%
  \ref{item:twistedprod:preservespb}
  and the tangent space structure from
  \itemref{rem:twistedprodproperties}{item:twistedprodmanifold}%
  \ref{item:twistedprod:tangentspace}
  this yields on vector bundles:
  \begin{align*}
    \pb{\Twistedprod{k}{f}} \Twistedprod{k}{\T M}
    &\cequalsby{\autoref{rem:twistedprodproperties}}
      \Twistedprod{k}{\pb f \T M}
      = \Twistedprod{k}{\bigoplus_{i\leq n}\xi_i}
      = \bigoplus_{i\leq n}\Twistedprod{k}{\xi_i}
    \\
    \pb{\Twistedprod{k}{f}} \T{\Twistedprod{k}{M}}
    &\cequalsby{\autoref{rem:twistedprodproperties}}
      \pb{\Twistedprod{k}{f}} \left(
      \T{\RP k} \oplus \Twistedprod{k}{\T M}
      \right)\\
    &= \pb{\Twistedprod{k}{f}} \T{\RP k}
      \oplus
      \pb{\Twistedprod{k}{f}} \Twistedprod{k}{\T M}
      = \T{\RP k} \oplus \bigoplus_{i\leq n}\Twistedprod{k}{\xi_i}\\
    \intertext{And on Stiefel-Whitney classes:}
    \pb{\Twistedprod{k}{f}} \W{\Twistedprod{k}{\T M}}
    &= \prod_{i\leq n} \W{\Twistedprod{k}{\xi_i}}
      \cequalsby{\autoref{cor:twistedprod:swlinebdl}}
      \prod_{i\leq n} \left(1 + c + e(\alpha_i) + d(\alpha_i)\right)
    \\
    \pb{\Twistedprod{k}{f}} \W{\T\Twistedprod{k}{M}}
    &= \pb{\Twistedprod{k}{f}}\W{\T{\RP k}}
      \cdot \prod_{i\leq n} \W{\Twistedprod{k}{\xi_i}} \\
    &= (c+1)^{k+1}
      \cdot \prod_{i\leq n} \left(1+c+e(\alpha_i)+d(\alpha_i)\right)
  \end{align*}
  In order to work with these Stiefel-Whitney class expressions as
  symmetric polynomials, introduce variables $u_i, v_i$ of degree
  one such that
  \begin{align*}
    \w1{\xi_i} &= c+e(\alpha_i) = \el 1 2(u_i, v_i) = u_i+v_i \\
    \w2{\xi_i} &= d(\alpha_i)   = \el 2 2(u_i, v_i) = u_iv_i \\
    \W{\xi_i}  &= 1+c+e(\alpha_i)+d(\alpha_i)
                 = 1+\sum_{i\leq2}\el i 2(u_i,v_i) = (1+u_i)(1+v_i)
                 \;.
  \end{align*}
  The key point now qualifying $f$ for the proof, is that both $f$ and
  thus by
  \itemref{thm:twistedprod:cohomstructure}{item:twistedprod:preservescohominj}
  also $\Twistedprod{k}{f}$ are injective on cohomology, and fulfill
  \begin{enumerate}
  \item with $\W{\pb f\T M} = \prod_{i=1}^n (1+\alpha_i)$
    \begin{gather}\label{eq:lem:twistedprod:indecompcriterion:sM}
      \pb f \s{(n)}(M)
      = \s{(n)}(\W{\pb f\T M})
      = \sum_{i=1}^n \alpha_i^n
      \;
    \end{gather}
  \item and with
    $\W{\pb f\T{\Twistedprod{k}{M}}}
    = \prod_{i=1}^{m+1}(1+c) \cdot \prod_{i=1}^n
    (1+u_i)(1+v_i)$
    that
    \begin{align}\notag
      \pb{\Twistedprod{k}{f}} \s{(2n+k)}(\Twistedprod{k}{M})
      &= \s{(2n+k)}(\W{\pb f\T{\Twistedprod{k}{M}}}) \\
      \label{eq:lem:twistedprod:indecompcriterion:sDkM}
      &= (k+1)c^{2n+k} + \sum_{i=1}^n (u_i^{2n+k} + v_i^{2n+k})
        \;.
    \end{align}
  \end{enumerate}
  In order to formulate
  $\pb{\Twistedprod{k}{f}}\s{(2n+k)}(\Twistedprod{k}{M})$ 
  in terms of $c$ and $d(\alpha_i)$,
  use one of the Newton-Girard formulas % TODO: ref
  saying:
  \begin{Lem}[Newton-Girard]
    For integers $l, m\in\Nat$ holds
    \begin{align*}
      \symm{l} t^m
      &= \sum_{\mathclap{r_1+2r_2+\dotsb+mr_=m}}
        (-1)^m \frac{m\cdot(r_1+\dotsb+r_m-1)!}{(r_1)!\dotsm(r_m)!}
        \cdot \prod_{i=1}^m (-\el i l)^{r_i}
        \;.
    \end{align*}
    As a special case for $l=2$ this becomes modulo 2
    \begin{align*}
      t_1^m + t_2^m
      &= \sum_{\mathclap{r_1+2r_2=m}}
        \frac{m\cdot(r_1+r_2-1)!}{(r_1)!(r_2)!}
        \cdot (\el 1 2)^{r_1} \cdot (\el 2 2)^{r_2}
      = \sum_{\mathclap{r_1+2r_2=m}}
        \altbinom{r_1-1}{r_2} (t_1+t_2)^{r_1}(t_1t_2)^{r_2}
    \end{align*}
    where $\frac{(r_1+2r_2)(r_1+r_2-1)!}{(r_1)!(r_2)!}
    = \binom{r_1+r_2-1}{r_2} + 2\binom{r_1+r_2-1}{r_1}$
    and the notation $\altbinom{p}{q}\coloneqq \binom{p+q}{q}$
    was used.
  \end{Lem}
  Before applying this to $t_1=u_i$, $t_2=v_i$, $m=2n+k$, and $l=2$,
  recall the following special properties from
  \autoref{thm:twistedprod:cohomstructure} of the mentioned symbols in
  $\H^*(\Twistedprod{k}{X})$:
  \begin{itemize}
  \item $c^{k+1}=0$,
  \item $c\cdot e(\alpha_i)=0$, and
  \item $e(\alpha_i)^{r_1}d(\alpha_i)^{r_2} = 0$ for $r_1+2r_2>2n$,
    as $\H^{*>2n}(M\times M)=0$.
  \end{itemize}
  Then simplify in terms of $c$, $e(\alpha_i)$, $d(\alpha_i)$:
  \begin{align}\notag
    u_i^{2n+k} + v_i^{2n+k}
    &= \sum_{\mathclap{r_1+2r_2=2n+k}}
      \altbinom{r_1-1}{r_2} (u_i+v_i)^{r_1}(u_iv_i)^{r_2} \\\notag
    &\equalsby{Def.} \sum_{\mathclap{r_1+2r_2=2n+k}}
      \altbinom{r_1-1}{r_2}
      (c+e(\alpha_i))^{r_1}d(\alpha_i)^{r_2} \\\notag
    &\equalsby{$c\cdot e(\alpha_i)=0$}
      \sum_{\mathclap{r_1+2r_2=2n+k}} \altbinom{r_1-1}{r_2}
      \left(c^{r_1}d(\alpha_i)^{r_2}+e(\alpha_i)^{r_1}d(\alpha_i)^{r_2}\right) \\\notag
    &\equalsby{$c^{k+1}=0$, $d(\alpha_i)^{n+s}=0$}
      \altbinom{k-1}{n} c^k d(\alpha_i)^n
      + \sum_{\mathclap{r_1+2r_2=2n+k}} \altbinom{r_1-1}{r_2}
      e(\alpha_i)^{r_1}d(\alpha_i)^{r_2} \\
    \label{eq:lem:twistedprod:indecompcriterion:1}
    &= \altbinom{k-1}{n} c^k d(\alpha_i)^n      
  \end{align}
  Altogether it turns out that
  \begin{align*}
    \pb{\Twistedprod{k}{f}}\s{(2n+k)}(\Twistedprod{k}{M})
    &\cequalsby{\eqref{eq:lem:twistedprod:indecompcriterion:sDkM}}
      (k+1)c^{2n+k} + \sum_{i=1}^n (u_i^{2n+k} + v_i^{2n+k}) \\
    &\equalsby{$c^{k+1}=0$}
      \sum_{i=1}^n (u_i^{2n+k} + v_i^{2n+k}) \\
    &\equalsby{\eqref{eq:lem:twistedprod:indecompcriterion:1}}
      \sum_{i=1}^n \altbinom{k-1}{n} c^k d(\alpha_i)^n \\
    &= \altbinom{k-1}{n} c^k d(\sum_{i=1}^n \alpha_i^n) \\
    &\equalsby{\eqref{eq:lem:twistedprod:indecompcriterion:sM}}
      \altbinom{k-1}{n} c^k d(\pb f \s{(n)}(M))
  \end{align*}
  as was stated.
\end{proof}

\subsection
[A Convenient Generating Set of the Cobordism Ring]
{A Convenient Generating Set of $\c_*$}
Now one can define candidates for the desired basis using the twisted
product construction to finally prove
Brown's theorem~\autoref{thm:brown}. 

Beforehands, recall from \autoref{thm:twistedprod:indecompcriterion}
that one has to both choose the dimension cobination, as well as the
factor of a twisted product correctly, in order to obtain an
indecomposable class representative.
For the correct dimension choice note, that any integer $m$ with
binary representation
\begin{gather*}
  m=2^{i_1}+\dotsb+2^{i_l}
  \;,\qquad
  0 \leq i_1<\dotsb<i_r
\end{gather*}
can be written as $m=2^{i_1}+2n\eqqcolon k+2n$, where:
\begin{itemize}
\item $n=0$ for $l=1$, else $n=2^{i_2-1}+\dotsb+2^{i_l-1}$
\item $\alpha(m)=\alpha(n)+1$.
\item For $m$ not of the form $2^s-1$, $n$ will also not be of the
  form $2^s-1$.
\end{itemize}
This is favourable because the binomial coefficient
$\binom{k-1}{n}$ will never be zero modulo two by the following Lemma,
as is required by the indecomposability criterion for twisted products
in \autoref{thm:twistedprod:indecompcriterion}.
\begin{Lem}
  For $a,b\in\Nat$, $\binom{a}{b}$ will be non-zero modulo 2 if and only
  if $\alpha(a+b)=\alpha(a)+\alpha(b)$.
  \begin{proof}
    This is a direct consequence from Lucas' Theorem % TODO: ref
    stating
    \begin{gather*}
      \binom{a+b}{b} \equiv \prod_{i=0}^s \binom{a_i+b_i}{b_i} \mod 2
    \end{gather*}
    where $a=\sum_{i=1}^s a_i2^{i}$, $b=\sum_{i=1}^s b_i2^{i}$ are the
    binary expansions of $a$ and $b$, and $\binom{0}{1}\coloneqq 0$.
    But this expression will only be non-zero, if $b_i$ is one for any
    $i$ where $a_i+b_i$ is one. In other words, if and only if
    $\alpha(a+b)=\sum_{i=1}^s(a_i+b_i)$.    
  \end{proof}
\end{Lem}
This can be applied to the combination $k-1$, $n$ from above, as one
simply calculates
\begin{align*}
  \alpha(k-1+n)
  &= \alpha\left((2^{i_1}-1)+ (2^{i_2-1}+\dotsb+2^{i_l-1})\right) \\
  &= \alpha\left(1+2^1+\dotsb 2^{i_1-1} + 2^{i_2-1}+\dotsb+2^{i_l-1}\right) \\
  &\equalsby{$i_1<i_2$}
  \alpha\left(1+2^1+\dotsb 2^{i_1-1}) + \alpha(2^{i_2-1}+\dotsb+2^{i_l-1}\right)\\
  &= \alpha(k-1) + \alpha(n)
\end{align*}

Now define the basis representatives as follows.
\begin{Thm}\label{thm:brown}
  Consider the manifolds $\G m$ for $m\neq 2^s-1\in\Nat$ that are
  defined by
  \begin{align*}
    \G 0 &\coloneqq \pt \;, \\
    \G m &\coloneqq \Twistedprod{2^{i_1}}{\G n}
        &\text{
          where
          $\textstyle m=\sum_{r=1}^l 2^{i_r}=2^{i_1}+2n$
          with $0\leq i_1<\dotsb<i_l$.
          }
          \;.
  \end{align*}
  Then:
  \begin{enumerate}
  \item\label{item:brownimmersionproperty} $\G m$ immerses into $\R^{2m-\alpha(m)}$.
  \item The cobordism classes of the $\G m$, $m\neq 2^s-1$, are
    indecomposable, and thus % TODO: ref
    \begin{gather*}
      \c_*\cong \left( [\G m] \,\big|\, m\neq 2^s-1 \right)_{\Zmod2}
      = \Zmod2\left[[\G m] \,\big|\, m\neq 2^s-1\right]
    \end{gather*}
  \end{enumerate}
\end{Thm}
The proof of the two parts is splitted into the subsequent sections.

\subsubsection{Immersion Property}

\begin{proof}[proof of
  \itemref{thm:brown}{item:brownimmersionproperty}]
  \optcite[p.~84, proof Thm.~1.30]{immersionconj}
  In order to proof the immersion property of the manifolds $\G m$,
  conduct an induction over $\alpha(m)$.
  \begin{description}
  \item[$\alpha(m)=1$:]
    As an induction start note that in the case of $m=2^i$ one has
    \begin{gather*}
      \G m
      =\Twistedprod{2^i}{\G 0}
      =\Twistedprod{2^i}{\pt}
      =\RP{2^i}
    \end{gather*}
    which immerses into $\R^{2m-1}=\R^{2m-\alpha(m)}$ by
    Whitney's immersion theorem. % TODO: ref
  \item[$\alpha(m)\geq2$:]
    For the induction step write $m=2^{i_1}+2n$ as above, and assume
    there is an immersion
    \begin{gather*}
      \iota\colon\G n \immlongto \R^{2n-\alpha(n)}
      \;.
    \end{gather*}
  \end{description}
  As described in
  \itemref{rem:twistedprodproperties}{item:twistedprodpreservesimmersions}
  this induces an immersion of twisted products
  \begin{gather*}
    1\times\iota\times\iota\colon
    \G m
    = \Sphere{2^{i_1}} \times \G n\times \G n/\sim
    \immlongto
    \Sphere{2^{i_1}} \times \R^{2n-\alpha(n)} \times \R^{2n-\alpha(n)}/\sim
    = \Twistedprod{2^{i_1}}{\R^{2n-\alpha(n)}}
    \;.
  \end{gather*}
  The latter twisted product however is a fibre bundle over
  $\RP{2^i_1}$ as in
  \itemref{rem:twistedprodproperties}{item:twistedprodfibrebdl},
  more precisely a vector bundle of dimension $4n-2\alpha(n)$.
  The needed immersion will be given by the following Lemma.
  \begin{Lem}
    \optcite[compare Prop. 4.3]{brown}
    The total space of a vector bundle $E\to M^n$ of rank $k$ over a
    closed manifold of dimension $n$ can be immersed into $\R^{2n+k}$.
  \end{Lem}
  One calculates
  \begin{align*}
    2\cdot 2^{i_1} + \left(4n-2\alpha(n)\right)
    &= 2\cdot (2^{i_1}+2n) - 2\alpha(n) \\
    &= 2m - 2\alpha(n) \\
    &= 2m - 2\alpha(m)+2 \\
    &\leqby{m\geq 2}
    2m - \alpha(m)
    \;,
  \end{align*}
  so altogether one gets the immersion
  \begin{gather*}
    \G m
    \immlongto \Twistedprod{2^{i_1}}{\R^{2n-\alpha(n)}}
    \immlongto \R^{2m - \alpha(m)}
    \;.
  \end{gather*}
  % TODO: proof Lemma
\end{proof}

\subsubsection{Generating Property}
% TODO

%%% Local Variables:
%%% mode: latex
%%% TeX-master: "thesis"
%%% End:



% Literature
\printnomenclature
\nocite{*}
\printbibliography


% Erklärung
%%%%%%%%%%%%%%%%%%%%%%%%%%%%%%%%%%
%  Master Thesis in Mathematics
% "Immersions and Stiefel-Whitney classes of Manifolds"
% -- Legal Declaration --
% 
% Author: Gesina Schwalbe
% Supervisor: Georgios Raptis
% University of Regensburg 2018
%%%%%%%%%%%%%%%%%%%%%%%%%%%%%%%%%%


% TODO: Translate? May be in English or German.
\selectlanguage{ngerman}
\chapter*{Eigenständigkeitserklärung}
\thispagestyle{empty}
Hiermit erkläre ich, Gesina Schwalbe, geboren am 29.01.1996 in Aachen,
dass
\begin{itemize}
\item die vorgelegten Druckexemplare und die vorgelegte elektronische
  Version der Arbeit identisch sind,
\item ich die Arbeit selbständig verfasst und keine anderen als die
  angegebenen Quellen und Hilfsmittel benutzt habe und
\item ich die Arbeit nicht bereits an einer anderen Hochschule zur
  Erlangung eines akademischen Grades eingereicht habe.
\end{itemize}
Desweiteren bestätige ich hiermit, dass ich von den in
§~26 Abs.~6 Prüfungsordnung%
\footnote{
  Prüfungs- und Studienordnung für den Masterstudiengang Mathematik an
  der Universität Regensburg (Fassung vom 14.10.2015)
}
vorgesehenen Rechtsfolgen Kenntnis habe.


\vspace*{5em}
Regensburg, den \handindate
\hfill
Gesina Schwalbe


%%% Local Variables:
%%% mode: latex
%%% TeX-master: "thesis"
%%% End:


\end{document}
