%%%%%%%%%%%%%%%%%%%%%%%%%%%%%%%%%%
%  Master Thesis in Mathematics
% "Immersions and Stiefel-Whitney classes of Manifolds"
% -- Chapter 2: Preliminaries --
% 
% Author: Gesina Schwalbe
% Supervisor: Georgios Raptis
% University of Regensburg 2018
%%%%%%%%%%%%%%%%%%%%%%%%%%%%%%%%%%

\chapter{Preliminaries}
\section{Steenrod Squares}

\begin{Def}[Steenrod Squares] % TODO: Check
  The Steenrod Squares $\Sq i$ for $i\in\Nat$ are each a family of
  cohomology operations of the form
  \begin{gather*}
    \left(
      \Sq i\colon H^n(X;\Zmod2) \to H^{n+i}(X;\Zmod2)
      \;\middle|\;
      n\in\Nat
    \right)
  \end{gather*}
  that satisfy the following relations for any space $X$, and any map of
  spaces $f\colon X\to Y$:
  \begin{description}
  \item[(Naturality)] $\pb f\circ\Sq i = \Sq i\circ\pb f$
  \item[(Stability)] $\susp\circ\Sq i = \Sq i\circ\susp$ % TODO: suspension
  \item[(Cartan formula)] For any $n\in\Nat$, and $x,y\in H^n(X)$ holds
    \begin{gather}\label{cartan}\tag{Cartan's formula}
      \Sq i(x\cup y) = \sum_{r+s=i}\Sq r(x)\cup\Sq s(y)
    \end{gather}
  \item[(Adem relations)] For $\alpha<2\beta$ holds
    \begin{gather}\label{adem}\tag{Adem's formula}
      \Sq\alpha \circ \Sq\beta =
      \sum_{j=0}^{\lfloor \frac \alpha 2 \rfloor}
      \binom{\beta-j-1}{\alpha-2j}
      \Sq{\alpha+\beta+j}\Sq{j}
    \end{gather}
  \item[(Fixed values)] The following values are fixed for $x\in H^n(X)$:
    \begin{gather}\label{sqlowerbound}
      \Sq i(x) = \begin{cases}
        0 & n<i \\
        x^2 & n=i
      \end{cases}
      \qquad\text{and}\qquad
      \Sq i = \begin{cases}
        \Id & i=0\\
        \beta & i=1
      \end{cases}
    \end{gather}
     where $\beta$ denotes the Bockstein homomorphism. % TODO: ref
  \end{description}
  $\Sq{}\coloneqq \sum_{j\in\Nat}\Sq j$ is the formal sum of all
  Steenrod squares called the \emph{the total Steenrod square}.
  Note that for any degree $n\in\Nat$ the total Steenrod square
  $\Sq{}\colon H^n(X)\to H^*(X)$ is well-defined since the sum is
  finite by \eqref{sqlowerbound}.
  Also \ref{cartan} can be reformulated to
  $\Sq{}(x\cup y) = \Sq{}(x)\cup\Sq{}(y)$, \idest $\Sq{}$ is a group
  homomorphism with respect to the cup-product.

  Analoguesly define Steenrod squares for a pair of spaces $(X,A)$.
\end{Def}

\begin{Thm}
  The Steenrod squares exist and are uniquely determined by the above
  properties.
  \begin{proof} % TODO: Do proof of existence&uniqueness of Steenrod sq?
    Existence: \cite[Chapter 2]{mosher};
    Uniqueness: \cite[VIII §3]{steenrodepstein}
  \end{proof}
\end{Thm}

\begin{Def}[Steenrod Algebra]% TODO
\end{Def}

\begin{Def}
  The antipode $\antipode\colon \A\to\A$ of the Steenrod algebra is
  inductively defined by the relation
  \begin{gather*}
    1 = \Sq 0
    = \Sq{} \Sqcup \antipode(\Sq{})
    = \sum_{k\geq0}\sum_{r+s=k} \Sq r \Sqcup \antipode(\Sq s)
  \end{gather*}
\end{Def}

\section{Characteristic Classes}
\begin{Def}[Universal Bundle] % TODO
  Example: Universal line bundle $\gamma_1\colon \RP\cong\BO(1)\xrightarrow{\incl} \BO$
\end{Def}

\begin{Def}[Classifying Spaces] 
  \optcite[Chapter~14.4]{tomdieck} % TODO: correct def. classifying spaces
  \begin{itemize}
  \item Any topological group $G$ admits a contractible space $\EG$ with a
    free $G$-action, and a corresponding principal $G$-bundle
    $\gamma^G\colon \EG\to\BG\coloneqq \EG/G$ called the
    \emph{universal $G$-bundle}
    where $\gamma_G$, $\EG$, and $\BG$ are all unique up to
    homotopy. % TODO: Uniqueness of universal bundle up to htpy
    $\BG$ is called the \emph{classifying space} for principal
    $G$-bundles.
  \item $\gamma^G$ fulfills the following universal property:
    For any space $X$ admitting the homotopy type of a CW-complex
    there is a bijection beetween $[X,\BG]$ and the isomorphism classes of
    principal $G$-bundles over $X$, given by
    \begin{gather*}
      \left(f\colon X\to\BG \right) \longmapsto \pb f \gamma^G
      \;
      \text{\optcite[Theorem 1.4]{immersionconj}}.
    \end{gather*}
    This correspondence is natural in $X$.
  \item There is a natural equivalence between the category of
    $n$-dimensional vector bundles $\Vect$ and that of principal
    $\Orth(n)$- respectively $\GL n$-bundles.
  \end{itemize}
\end{Def}

\begin{Def}[Characteristic Class]
  A characteristic class
  \begin{itemize}
  \item of degree $i$
  \item with coefficients in a ring $R$
  \item for principal $G$-bundles for a group $G$
  \end{itemize}
  is a natural transformation
  \begin{gather*}
    \Cl\colon [-, \BG] \Longrightarrow H^i(-; R)\;.
  \end{gather*}
\end{Def}
  
\begin{Rem}
  By Brown's representation theorem % TODO: cite
  $H^i(-;R)$ is a represantable functor represented by $K(i,R)$.
  Thus, by the Yoneda lemma, a characteristic class is
  represented by a morphism
  \begin{gather*}
    \cl\colon \BG \longto K(i, R)
  \end{gather*}
  in $\Top$, i.e. by a cohomology class $\cl$ of $\BG$.
  Thus, application of $\Cl$ to a principal $G$-bundle over a
  space $X$ which is represented by a morphism $\eta\colon X\to\BG$ % TODO: cite
  is
  \begin{gather*}
    \Cl(X) = \pb{\eta} \cl \in H^i(X;R)
    \;.
  \end{gather*}
  This describes a one-to-one correspondence between
  characteristic classes as above and cohomology classes in
  $H^i(BG;R)$, and in the following any characteristic class will be
  identified with its corresponding cohomology class.
\end{Rem}

\begin{Rem}
  % TODO: Why are char classes nice idea?
\end{Rem}

\subsection{Thom Classes}
Let $B$ be a paracompact space, \forexample a manifold,
$\xi\colon E\xrightarrow{p} B$ a vector bundle over $B$ of rank $n>0$,
and $R$ be a unital commutative ring.
\begin{Def}
  A \emph{Thom class} of $\xi$ in $R$-coefficients is a
  cohomology class $\u(\xi)\in H^n(\spherepair{E}; R)$,
  such that for all points $b\in B$ and fiber inclusions
  $i_b\colon (\spherepair{E_b}) \to (\thomspacepair{E})$
  the restriction $\u(\xi)|_{E_b} = \pb i_b (\u(\xi))$ is a
  multiplicative generator,
  \idest a unit, % TODO: any generator is unit?
  of the ring
  $H^n(\spherepair{E_b};R)\cong H^n(\spherepair{\R}; R)\cong R$.
  \optcite[p.~441]{hatcher}
\end{Def}

\begin{Cor}
  The Thom class construction is natural with respect to the pullback
  of vector bundles over paracompact spaces.
  \Idest given any map of paracompact spaces $f\colon A\to B$, and a
  vector bundle $\xi\colon E\to B$,
  the pullback $\pb f \u$ of a Thom class $\u$ of $\xi$ will
  be a Thom class of $\pb f \xi$.
  \begin{proof}
    Let $\u$ be a Thom class of $\xi$ and $a\in A$ any point.
    Consider the restriction $\pb i_a(\pb f \u)$
    of the pullback of $\u$ to the fiber over $a$. To show that this
    is a generator of $H^n(\spherepair{E_a};R)$ first use that
    pullbacks commute with restriction:
    \begin{gather*}
      \pb i_a(\pb f \u)
      = \pb {(f\circ i_a)} \u
      = \pb {(i_{f(a)}\circ f)} \u
      = \pb f (\pb i_{f(a)} \u)
    \end{gather*}
    $\pb i_{f(a)} \u$ is a generator by definition of $\u$.
    Now the restriction of $f$
    \begin{gather*}
      f\colon (\spherepair{(\pb f E)_a}) \to (\spherepair{E_{f(a)}})
    \end{gather*}
    is an isomorphism, and thus
    $\pb f\colon H^k(\spherepair{E_{f(a)}})
    \cong H^k(\spherepair{(\pb f E)_a})$
    sends generators to generators for all $k\in\Nat$.
  \end{proof}
\end{Cor}

\begin{Rem}
  Let $\xi$, $\eta$ be vector bundles over a space $B$
  of rank $i$ and $j$.
  \begin{itemize}
  \item There is a canonical isomorphism
    $\E{(\xi\oplus\eta)}\cong\E\xi\oplus\E\eta$
    with the canonical projections
    $\pi_\xi\colon \E{(\xi\oplus\eta)}\to \E\xi$
    and
    $\pi_\eta\colon\E{(\xi\oplus\eta)}\to \E\eta$.
  \item There is a corresponding Künneth isomorphism
    defining the \emph{cross-product} \optcite[p.~214]{hatcher}
  \begin{align*}
    H^*(\thomspacepair{\E\xi})
      \otimes
      H^*(\thomspacepair{\E\eta})
    &\overset{\sim}\longto
      H^*(\thomspacepair{\E{(\xi\oplus\eta)}})\\
    x\otimes y
    &\longmapsto
      \pb \pi_\xi x \cup \pb \pi_\eta y
      \eqqcolon x\times y
      \;.
  \end{align*}
  \cite[Theorem~3.18]{hatcher}
  \end{itemize}
\end{Rem}
\begin{Cor}\label{thomclassmultiplicative} % TODO: Lemma instead? Join with above Cor?
  The Thom class construction for coefficients in a field $R$ is
  multiplicative in the following sense:
  For vector bundles $\xi\colon E\to B$, $\eta\colon E'\to B$
  of rank $i$ respectively $j$ over a paracompact space $B$, and Thom
  classes
  $\u(\xi)\in H^i(\thomspacepair{\E\xi}; R)$,
  $\u(\eta)\in H^i(\thomspacepair{\E\eta}; R)$
  the class
  \begin{gather*}
    \u(\xi)\times\u(\eta)
    \coloneqq \pb \pi_\xi\u(\xi) \cup \pb \pi_\eta\u(\eta)
    \in H^{i+j}(\thomspacepair{\E{(\xi\oplus\eta)}})
  \end{gather*}
  is a Thom class of $\xi\oplus\eta$.
  \begin{proof}
    Consider a fiber $b\in B$. As cup product and pullback commute
    with restriction, the cross-product also commutes with
    restriction, \idest one has to show that
    \begin{gather*}
      \pb i_a(\u(\xi)\times\u(\eta))
      = \left(\pb i_a\u(\xi)\right)
      \times \left(\pb i_a\u(\eta)\right)
    \end{gather*}
    is a generator of
    $H^{i+j}(\thomspacepair{\E{(\xi\oplus\eta)}};R)$.
    % There are homotopies making the following diagram commute
    % \begin{center}
    %   \begin{tikzcd}
    %     (\spherepair{\E{(\xi)}_b})
    %     \ar[d, dash, "\sim"{above,rotate=90}]
    %     &(\spherepair{\E{(\xi\oplus\eta)}_b})
    %     \ar[l,"\pi_\xi"above] \ar[r,"\pi_\eta"]
    %     \ar[d, dash, "\sim"{above,rotate=90}]
    %     &(\spherepair{\E{(\eta)}_b})
    %     \ar[d, dash, "\sim"{above,rotate=90}]\\
    %     (\spherepair{\R^i})
    %     &(\spherepair{\R^{i+j}})
    %     \ar[l,"\proj"above] \ar[r,"\proj"]
    %     &(\spherepair{\R^{j}})
    %   \end{tikzcd}
    % \end{center}
    By the naturality of the Künneth isomorphism there is the
    following commutative diagram that translates this problem to one
    on the cohomology of spheres:
    \begin{center}
      \begin{tikzcd}
        H^*(\spherepair{\E\xi_b};R)
        \otimes H^*(\spherepair{\E\eta_b};R)
        \ar[r, "\cong"]
        \ar[d, dash, "\cong"{above,rotate=90}]
        & H^*(\spherepair{\E{(\xi\oplus\eta)}_b}; R)
        \ar[d, dash, "\cong"{below,rotate=90}]
        \\
        H^*(\spherepair{\R^i}; R)\otimes H^*(\spherepair{\R^j}; R)
        \ar[r, "\cong"]
        \ar[d, dash, "\cong"{above,rotate=90}]
        & H^*(\spherepair{\R^{i+j}}; R)
        \ar[d, dash, "\cong"{below,rotate=90}]
        \\
        H^*(I^i,\Boundary{I^i}; R)\otimes H^*(I^j, \Boundary{I^j}; R)
        \ar[r, "\cong"]
        \ar[d, dash, "\cong"{above,rotate=90}]
        & H^*(I^{i+j}, \Boundary{I^{i+j}}; R)
        \ar[d, dash, "\cong"{below,rotate=90}]
        \\
        H^*(\Sphere{i}; R)\otimes H^*(\Sphere{j}; R)
        \ar[r, "\cong"]
        & H^*(\Sphere{i+j}; R)
      \end{tikzcd}
    \end{center}
    Furthermore, the simple structure of the cohomology of spheres
    yields for the Künneth isomorphism in the desired degree
    \begin{align*}
      H^{i+j}(\Sphere{i+j};R)
      &\cong
        \left(H^*(\Sphere{i}; R)\otimes H^*(\Sphere{j}; R)\right)_{i+j}\\
      &\coloneqq
        \bigoplus_{\mathclap{r+s=i+j}}
        H^r(\Sphere{i};R)\otimes H^s(\Sphere{j};R)
        =
        H^i(\Sphere{i};R)\otimes H^j(\Sphere{j};R)
    \end{align*}
    by leaving out zero-summands for the last equality.
    Thus, a generator of
    $H^i(\Sphere{i}; R)\otimes H^j(\Sphere{j}; R)$,
    which is the tensor product $\iota_i\otimes\iota_j$ of a generator
    in each factor,
    is mapped to a generator $\iota_{i+j}=\iota_i\times\iota_j$ of
    $H^{i+j}(\Sphere{i+j}; R)$.
    Using the isomorphisms above prooves the claim.
    \optcite[compare proof of Thm~3.19, p.~221]{hatcher}
    % TODO: Maybe put into separate Lemma:
    % Now one can use the fact, that for any two generators
    % $\iota_i$ of $H^i(\spherepair{\R^i})$ and
    % $\iota_j$ of $H^i(\spherepair{\R^j})$,
    % the cross-product
    % $\iota_i\times\iota_j$ is a generator of
    % $H^{i+j}(\spherepair{\R^{i+j}})$.
  \end{proof}
\end{Cor}

\begin{Thm}
  Every vector bundle $\xi$ has a unique Thom class $\u(\xi)$ in
  $\Zmod2$-coefficients.
  Furthermore, for any map of paracompact spaces $f\colon A\to B$ and
  vector bundle $\xi\colon E\to B$ holds $\u(\pb f \xi) = \pb f \u(\xi)$.
  \begin{proof}
    \begin{description}
    \item[Existence:] See \cite[Theorem~4D.10]{hatcher} or use
      \cite[Proposition~17.9.3]{tomdieck}.
    \item[Uniqueness:] % TODO: (GEORGE) Ref for Uniqueness of Thom class 
      Using a suitable Mayer-Vietoris sequence for glueing, and an
      inductive argument starting with the trivial bundle case, one can show:
      Any two classes in $H^n(\thomspacepair{E};R)$ whose
      restrictions coincide on all fibers will coincide.
      However, for $R=\Zmod2$ there is exactly one possible choice for
      a unit $\u(\xi)|_{E_b}\in
      H^n(\spherepair{E_b})^\times\cong\Zmod2^\times=\{1\}$
      over each point $b$.
    \item[Naturality:] Clear from uniqueness and the naturality of Thom classes.
    \end{description}
  \end{proof}
\end{Thm}

\begin{Rem}
  Using paracompactness of $B$ and
  \cite[Proposition~17.9.6]{tomdieck}, one concludes that
  $\u(\xi)\in H^n(\thomspacepair{E};R)$ has to be a unit.
\end{Rem}

\begin{Thm}[Thom isomorphism]
  For any Thom class $\u(\xi)$ of $\xi$, and any degree $k$ there are
  isomorphisms natural with respect to pullback of vector bundles
  % TODO: Ref/proof naturality Thom iso
  \begin{align*}
    \thomiso\colon
    H^k(B;R) &\longrightarrow H^{k+n}(\thomspacepair{E}; R)
    & \thomiso\colon
      H_{k+n}(\thomspacepair{E}; R) &\longrightarrow H^k(B;R)
    \\
    x &\longrightarrow \pb p (x) \cup \u(\xi)
    & \alpha &\longrightarrow \pf p (\u(\xi) \cap \alpha)
  \end{align*}
  called the \emph{Thom isomorphisms}.
  \begin{proof}
    Naturality directly follows from the naturality of the Thom class,
    of pullbacks, and of the cup- respectively cap-product.
    The cohomology part then is a direct application of Leray's Theorem
    \cite[Theorem~4D.8]{hatcher}.
    For the homology part see \forexample \cite[Theorem~14.6]{switzer}.
  \end{proof}
\end{Thm}

\begin{Lem}
  The Thom isomorphisms fulfill for $k\in\Nat$, $x\in H^k(B)$, and $\alpha\in H_k(E)$
  \begin{gather*}
    \capped{t(x)}{\alpha} = \capped{x}{t(\alpha)} \in\Zmod2
  \end{gather*}
  \begin{proof}
    First recall the following properties of the cap product
    $\capped{-}{-}$:
    \begin{align}
      \label{capprod1}
      \capped{a\cup b}{\beta} &= \capped{b}{a\cap\beta}
      \in H_0(X,X'')\\
      \label{capprod2}
      \capped{a}{\pf f \beta} &= \pf f \capped{\pb f a}{\beta}
      \in H_0(X,X'')
    \end{align}
    for
    triples of spaces $(X, X'', X')$, $(Y, Y'', Y')$,
    $f\colon (Y,Y'',Y')\to (X,X'',X')$,
    $a\in H^i(X,X')$, $b\in H^j(X,X')$,
    $\gamma\in H_{i+j}(X, X'\cup X'')$,
    and
    $\beta\in H_j(Y, Y'\cup Y'')$.
    Also note
    \begin{gather}
      \label{capprod3}
      \pf p = \Id \colon
      \Zmod2\cong H_0(\thomspacepair{E})\to H_0(B)\cong\Zmod2
    \end{gather}
    Then with $\u\coloneqq\u(\xi)$
    \begin{align*}
      \capped{t(x)}{\alpha}
      &= \capped{\pb p x \cup \u}{\alpha} \\
      &\overset{\mathllap{\eqref{capprod1}}}=
        \capped{\pb p x}{\u\cap\alpha} \\
      &\overset{\mathllap{\eqref{capprod3}}}=
        \pf p\capped{\pb p x}{\u\cap\alpha} \\
      &\overset{\mathllap{\eqref{capprod2}}}=
        \capped{x}{\pf p(u\cap\alpha)}
        = \capped{x}{\thomiso (\alpha)} \in\Zmod2
    \end{align*}
  \end{proof}
\end{Lem}


\subsection{Stiefel-Whitney Classes}
\begin{Def}[Stiefel-Whitney Classes]\label{def:swclasses}
  The Stiefel-Whitney classes are
  characteristic classes for principal $\Orth$-bundles
  respectively vector bundles,
  \idest cohomology classes
  $w_i\in H^i(\BO;\Zmod2)$, $i\in\Z\geq0$,
  fulfilling the following properties for any vector bundles $\xi$ and
  $\eta$ over a space $B$, and any map $f\colon A\to B$ of spaces:
  \begin{enumerate}
  \item \emph{(Naturality)} $\pb f\w_i(\xi) = \w_i(\pb f \xi)$,
  \item $\w_0(\xi)=1$,
  \item $\w(\gamma_1) = 1 + x$,
  \item \emph{(Multiplicativity)} $\w(\xi \oplus \eta) = \w(\xi)\cup \w(\eta)$
    \\\idest in degree $n$ we have
    $\w_n(\xi\oplus\eta) = \sum_{i+j=n}\w_i(\xi) \cup \w_j(\eta)$,
  \end{enumerate}
  where the \emph{total Stiefel-Whitney class}
  $\w\coloneqq\sum_{i\geq 0}\w_i$ is the formal sum of all
  Stiefel-Whitney classes,
  $\gamma_1$ is the universal line bundle $\RP\cong\BO(1)\to\BO$,
  and $x$ is the%
  \footnote{
    % TODO: maybe shorten reasoning about x well-defined
    This is well-defined, as a ring $R$ of the form $\Zmod2[x]$
    with $x$ of degree 1 only admits two elements, $0$ and a
    generator, in degree 1. Therefore there exists exactly one ring
    isomorphism, and that is defined by sending the unique generator in
    degree 1 to $x$.
  }
  generator of $H^*(\RP;\Zmod2)\cong\Zmod2[x]$.
  \optcite[compare §4, p.~37]{milnor}
\end{Def}

\begin{Thm}[Existence and Uniqueness of Stiefel-Whitney Classes] % TODO: proof
Stiefel-Whitney classes exist and are uniquely defined by the above
properties. Furthermore, they generate $H^*(\BO;\Zmod2)\cong \Zmod2[\w_i|i\geq1]$.
\begin{proof}[proof (existence)]
  Construction utilizes the Euler class.
\end{proof}
\begin{proof}[proof (uniqueness)]
  Use the splitting theorem and the knowledge about $\w(\gamma_1)$.
\end{proof}
\begin{proof}[proof (generators)]
  Inductively show $H^*(\BO(n);\Zmod2)\cong \Zmod2[\w_i|n\geq i\geq1]$
\end{proof}
\end{Thm}

\begin{Rem}[Further Properties of the Stiefel-Whitney Classes] % TODO
  \label{propswclasses}
  Let $\xi$, $\eta$ be vector bundles over a space $X$.
  \begin{enumerate} 
  \item $\w_i(\eta) = 0$
    for any vector bundle $\eta$ with $\rk\eta < i$.
    This follows directly from the construction for the existence of
    the Stiefel-Whitney classes.
    Therefore, the total Stiefel-Whitney class $\w(\xi)$ is
    well-defined (\idest the sum is finite)
    for any vector bundle $\xi$ of finite rank.
  \item $\w_i(\trivbdl)=0$ for $i>0$, and one immediately concludes:
    \begin{enumerate}
    \item The Stiefel-Whitney classes are stable, \idest
      $\w_i(\xi\oplus\trivbdl) = \w_i(\xi)$.
    \item If $\xi\oplus\eta = \trivbdl$, $\w_i(\xi)\cup\w_i(\eta)=1$.
      Especially, for any choice of embedding with normal bundle $\N M$
      of a manifold $M$ we have $\T M \oplus \N M = \trivbdl$ and
      therefore $1 = \w(\T M) \cup \w(\N M)$.
    \end{enumerate}
    \begin{proof} % TODO: make proof of w(\trivbdl)=0 nicer
      This follows from the naturality of the Stiefel-Whitney classes
      and the definition of the trivial rank $n$ bundle:
      $\w(\trivbdl) = \pb\pi \w(\trivbdl)$ where $\pi\colon X\to\pt$
      and $H^i(\pt;\Zmod2) = 0$ for $i>0$.
    \end{proof}
  \end{enumerate}
\end{Rem}

\begin{Def}
  For any space $X$ define the \emph{dual Stiefel-Whitney classes}
  $\dualw_i$ in degree $i$ of $X$ inductively by
  \begin{align*}
    1 &= \dualw_0\cup\w_0 = \dualw_0    &&\text{in degree 0}\\
    0 &= \sum_{i+j=n} \dualw_i\cup\w_j  &&\text{in degree n>0}
  \end{align*}
  With the notation $\dualw\coloneqq \sum_{i\geq0} \dualw_i$ as above
  for the formal sum this can be reformulated as
  \begin{gather*}
    1 = \w\cup\dualw
  \end{gather*}
  For a manifold $M$ define
  $\w(M) \coloneqq \w(\T M)$ and
  $\dualw(M) \coloneqq \dualw(\T M) = \w(\N M)$
  where the last equality is the one from
  Proposition~\ref{propswclasses} above.
\end{Def}

\begin{Thm}
  The Stiefel-Whitney classes can be given as
  \begin{gather*}
    \Sq i(\u(\xi)) = \thomiso(\w_i(\xi)) = \pb p w_i(\xi) \cup \u(\xi)
  \end{gather*}
  for any vector bundle $\xi\colon E\to B$ over a paracompact space
  $X$.
  The representing cohomology classes can be constructed using the
  embeddings $\BO(n)\subset\BO$ for $n\geq i$ and the fact that
  $\w_i=\pb\incl\w_i(\gamma_n)\in H^i(\BO)$ % TODO: ref
  .
  \begin{proof}
    Check naturality of the expression and all further
    definining properties from \ref{def:swclasses}.
    \begin{description}
    \item[Naturality:] Both $\Sq i$ and $t$ respectively also $t^{-1}$
      are natural.
    \item[$\w_0=1$:]
      $H^0(\thomspacepair{\E{\gamma_0}}) = \Zmod2$, thus 1 is the only
      candidate for a Thom class, $\Sq0(1) = \Id(1) = 1$, and the Thom
      isomorphism sends 1 to 1 in this degree.
    \item[$\w(\gamma_1)=1+x$:] % TODO: alt def SW-classes
    \item[Multiplicativity:] % TODO: Multiplicativity alt def SW-classes
      Consider vector bundles $\xi$, $\eta$ over a paracompact space
      $B$. With $\u(\xi\oplus\eta)=\u(\xi)\cup\u(\eta)$ get
      \begin{align*} % TODO: shorten
        \thomiso(\w_i(\xi\oplus\eta))
        &= \Sq i(\u(\xi\oplus\eta)) \\
        &\overset{\mathllap{\text{\autoref{thomclassmultiplicative}}}}=
          \Sq i(\pb\pi_\xi\u(\xi) \cup \pb\pi_\eta\u(\eta))\\
        &\overset{\mathllap{\text{\ref{cartan}}}}=
          \sum_{r+s=i}
          \Sq r(\pb\pi_\xi\u(\xi)) \cup \Sq s(\pb \pi_\eta\u(\eta)) \\
        &\overset{\mathllap{\text{Naturality}}}=
          \sum_{r+s=i}
          \pb\pi_\xi\Sq r(\u(\xi)) \cup \pb \pi_\eta\Sq s(\u(\eta)) \\
        &\overset{\mathllap{\text{Definition}}}=
          \sum_{r+s=i}
          \pb\pi_\xi\thomiso(\w_r(\xi))
          \cup \pb\pi_\eta\thomiso(\w_s(\eta)) \\
        &\overset{\mathllap{\text{Definition}}}=
          \sum_{r+s=i}
          \pb\pi_\xi \left(\pb p_\xi  \w_r(\xi)  \cup \u(\xi) \right)
          \cup
          \pb\pi_\eta\left(\pb p_\eta \w_s(\eta) \cup \u(\eta)\right) \\
        &= \sum_{r+s=i}
          \left(
          \pb\pi_\xi \pb p_\xi \w_r(\xi) \cup
          \pb\pi_\eta\pb p_\eta\w_s(\eta)\right)
          \cup
          \left(\pb\pi_\xi\u(\xi) \cup \pb\pi_\eta\u(\eta)\right) \\
        &\overset{\mathllap{\eqref{projectionscommute}}} =
          \left(\sum_{r+s=i}
          \pb p_{\xi\oplus\eta} \w_r(\xi) \cup \pb p_{\xi\oplus\eta} \w_s(\eta)
          \right)
          \cup
          \u(\xi) \times \u(\eta) \\
        &\overset{\mathllap{\text{Group Hom./Definition}}}=
          \pb p_{\xi\oplus\eta}
          \left(\sum_{r+s=i}\w_r(\xi)\cup\w_s(\eta)\right)
          \cup
          \u(\xi\oplus\eta)\\
        &\overset{\mathllap{\text{Definition}}}=
          % TODO: Intermediate steps
          \thomiso\left(\sum_{r+s=i}\w_r(\xi)\cup\w_s(\eta)\right)
      \end{align*}
      where it is used that
      \begin{gather}\label{projectionscommute}
        p_\xi\circ\pi_\xi = p_{\xi\oplus\eta} = p_\eta\circ\pi_\eta
      \end{gather}
      Applying $\thomiso^{-1}$ yields the result.
      \qedhere
    \end{description}
  \end{proof}
\end{Thm}

\subsection{Wu Classes}
% TODO: What is interesting about Wu classes? -> Conn. to SW cl.
Let $M$ be a compact, $n$-dimensinoal manifold.

\begin{Def}\label{def:wuclasses}
  The $i$th \emph{Wu class} $\v_i(M)$ of $M$ for $0\leq i\leq n$ is defined
  as the cohomology class in $H^i(M)$ that is uniquely determined by
  \begin{center}
    \begin{tikzcd}[row sep=0pt, column sep=small]
      H^i(M) \ar[r, equal, "\sim"]
      & H_{n-1}(M) \ar[r, equal, "\sim"]
      &\Hom{\Zmod2}(H^{n-i}(M), \Zmod2) \ar[r, equal, "\sim"]
      &\Zmod2
      \\
      y \ar[rr, mapsto] &&\capped{ x\cup y }{ \fundcl M }\\
      \v_i(M) \ar[rr, mapsto] &&\capped{\Sq i(x)}{\fundcl M}
    \end{tikzcd}
  \end{center}
  where the first isomorphism from the left is Poincaré duality % TODO: ref
  and the second is the universal coefficient theorem % TODO: ref
  for the field $\Zmod2$.
  Equivalently, for any cohomology class $x\in H^{n-i}(M)$ of fixed
  degree $n-i$ holds
  \begin{gather*}
    x\cup \v_i = \Sq i(x) \in H^n(M) \cong \Zmod2
  \end{gather*}
  Mind the fixed degree of $x$---the above will not be true for other
  degree cohomology classes in general!
\end{Def}

\begin{Rem}
  Some immediate consequences of the definitions of the Wu classes of
  $M$ are
  \begin{itemize}
  \item $\v_0 = 1$
  \item $\v_i = 0$ for $i>\frac n 2$, because $\Sq i(x) = 0$ if the
    degree of x is lower than $i$.
  \end{itemize}
\end{Rem}


\begin{Def}
  The \emph{total Wu class} of $M$ is defined as the sum
  $\sum_{i\geq0}\v_i(M)$. The \emph{total dual Wu class}
  $\dualv(M)\eqqcolon \sum_{i\geq 0}\dualv_i(M)$
  and the dual Wu classes $\dualv_i$
  of $M$ are defined by
  \begin{gather*}
    \v(M) \cup \dualv(M) = 1
  \end{gather*}
  or equivalently
  \begin{align*}
    1 &= \dualv_0(M) \cup \v_0(M) = \dualv_0(M) \\
    0 &= \sum_{r+s=i}\v_r(M)\cup\dualv_s(M)
    &&\text{in degree $0\leq i\leq n$}
  \end{align*}
\end{Def}

The following alternative definition of the Wu classes of a manifold
will be shown to be equivalent to the one above in \ref{def:wuclasses}
\begin{Def}
% TODO: Alternative definition Wu class  
\end{Def}

%%% Local Variables:
%%% mode: latex
%%% TeX-master: "thesis"
%%% End:
