%%%%%%%%%%%%%%%%%%%%%%%%%%%%%%%%%%
%  Master Thesis in Mathematics
% "Immersions and Stiefel-Whitney classes of Manifolds"
% -- Chapter 2: Preliminaries --
% 
% Author: Gesina Schwalbe
% Supervisor: Georgios Raptis
% University of Regensburg 2018
%%%%%%%%%%%%%%%%%%%%%%%%%%%%%%%%%%

\chapter{Preliminaries}
\section{Definitions}
\subsection{Steenrod Squares}
% TODO: Steenrod Squares definition

\subsection{Stiefel-Whitney Classes}
\begin{Def}[Classifying Spaces] % TODO
  There is a natural equivalence between the category of vector
  bundles $\Vect$ and that of principal $\Orth$-bundles.
\end{Def}

\begin{Def}[Characteristic Class]
  A characteristic class
  \begin{itemize}
  \item of degree $i$
  \item with coefficients in a ring $R$
  \item for principal $G$-bundles for a group $G$
  \end{itemize}
  is a natural transformation
  \begin{gather*}
    \cl\colon [-, \BG] \Longleftarrow H^i(-; R)\;.
  \end{gather*}
  By Brown's representation theorem % TODO: cite
  there is a natural equivalence $H^i(-;R)$ is represantable by
  $K(i,R)$. Thus, by the Yoneda lemma, a characteristic number is
  represented by a morphism
  \begin{gather*}
    \BG \longto K(i, R)
  \end{gather*}
  in $\Top$, i.e. by a cohomology class of $\BG$.
  The application of $\cl$ to a principal $G$-bundle over a
  space $X$ represented by a morphism $\eta\colon X\to\BG$ % TODO: cite
  can then also be described as the pullback of this cohomology class
  by $\eta$:
  \begin{gather*}
    \cl(X) = \pb{\eta} \cl \in H^i(X;R)
  \end{gather*}
\end{Def}

\begin{Rem}
  % TODO: Why are char classes nice idea?
\end{Rem}

\begin{Def}[Stiefel-Whitney Classes]
  The $i$th Stiefel-Whitney class $\w_i$ is a generator of
  $H^i(\BO; \Z/2\Z)$, \idest a characteristic class for principal
  $\Orth$-bundles respectively vector bundles, fulfilling
  \begin{enumerate}
  \item $\w_i(\trivbdl)=\delta_{i,0}$
  \item $\w_i(\eta) = 0$
    for any vector bundle $\eta$ with $\rk\eta < i$
  \item $\w_1(\gamma_1) = x$
    where $\gamma_1$ is the universal line bundle $\RP\to\BO$,
    and $x$ is the generator of $H^*(\RP;\Z/2\Z)\cong\Z/2\Z[x]$
    % TODO: How is THE generator of H^*(\RP) defined?
  \item $\w_i(\xi\times\eta)=\w_i(\xi)\times\w_i(\eta)
    \coloneqq \pb{\proj_X}\w_i(\xi) \cup \pb{\proj_Y}\w_i(\eta)
    \in H^i(X\times Y; \Z/2\Z)$
    for vector bundles $\xi\colon X\to\BO$, $\eta\colon Y\to\BO$
  \end{enumerate}
\end{Def}

\begin{Rem}[Uniqueness of Stiefel-Whitney Classes] % TODO
The Stiefel-Whitney classes can be constructed using the splitting
theorem.
\end{Rem}



%%% Local Variables:
%%% mode: latex
%%% TeX-master: "thesis"
%%% End:
