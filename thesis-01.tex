%%%%%%%%%%%%%%%%%%%%%%%%%%%%%%%%%% 
% Master Thesis in Mathematics
% "Immersions and Stiefel-Whitney classes of Manifolds"
% -- Chapter 1: Formulation of the Immersion Conjecture --
% 
% Author: Gesina Schwalbe
% Supervisor: Georgios Raptis
% University of Regensburg 2018
%%%%%%%%%%%%%%%%%%%%%%%%%%%%%%%%%% 

\chapter{Formulation of the Immersion Conjecture}
% Explain carefully how the immersion problem can be reformulated purely
% in terms of homotopy theory. [immersionconj] (The necessary results from differential
% topology and Hirsch-Smale theory should be stated clearly but may be
% presented without proofs.)
This chapter is dedicated to reviewing the concepts and results needed
to formulate and work with the immersion conjecture, and show its
connection to the theory of characteristic classes.

The reformulation and introduction of obstruction theory finally
happens in \autoref{sec:reformulation}.
Beforehand, the definitions and main properties of immersions, and
selected characteristic classes are revised.
Most importantly, the major properties of the Stiefel-Whitney
characteristic classes of vector bundles are given in
\autoref{sec:swclasses}, together with an outline in which ways these
will be particularly useful throughout this paper.

Mind that this chapter is meant as revision and outline, wherefore a
couple of preliminary results are merely referenced without proof.

\section{Immersions and Normal Bundles}\label{sec:immersions}
This section recapitulates the definition and major needed properties
of immersions, \idest local embeddings of manifolds.

More precisely, immersions are merely a special case of monomorphisms
of vector bundles. So, recall that a morphism
$(\xi_1\colon\E_1\to X_1)\to(\xi_2\colon E_2\to X_2)$ 
of vector bundles over different spaces is a map $F\colon E_1\to E_2$
that covers its restriction to the zero section, \idest it makes the
following diagram commute
\begin{center}
  \begin{tikzcd}
    E_1 \ar[r,"F"]\ar[d, "\xi_1"]
    & E_2 \ar[d, "\xi_2"]
    \\
    X_1 \ar[r, "F|_{\zerosec{\xi_1}}"]
    & X_2
  \end{tikzcd}.
\end{center}
Further remember the fact that such a morphism is a monomorphism in
the category of vector bundles $\Vect$, if and only if its restriction
to each fibre is injective.
\begin{Def}
  A smooth map $f\colon M\to N$ of smooth manifolds is called
  an \emph{immersion}, written $M\immto N$, if its differential
  $\Diff f\colon\T M\to\T N$ is a monomorphism of vector
  bundles.
  A homotopy which is an immersion in each stage is called regular.
\end{Def}
\begin{Rem}
  Let $M$ and $N$ be a smooth manifolds.
  \begin{enumerate}
  \item
    Immersions are local embeddings, \idest for an immersion
    $f\colon M\immto N$ there is around every point an open neighbourhood
    on which $f$ is an embedding.
    More descriptive, immersions are mappings that do not allow creases,
    respectively sharp bends, or puncturing.
    However, globally, immersions need not even be injective since
    \forexample self-intersections of the image are allowed.
    \begin{proof}
      Local embeddings are obviously immersions by definition, and the
      other direction is a conclusion from the implicit function theorem.
    \end{proof}
  \item Embeddings are exactly the injective immersions that are
    topological embeddings.
    If $M$ is compact, any injective immersion $f\colon M\to N$ is an
    embedding.
  \end{enumerate}
\end{Rem}
It is easy to see that in general not all vector bundle monomorphisms
between tangent bundles of smooth manifolds need to be the
differential of an immersion. However, taking the differential gives a
canonical inclusion of the set of immersions into the set of vector
bundle monomorphisms.
Later, an important result due to Hirsch and Smale will be needed that
relates spaces of immersions and vector bundle monomorphisms homotopy
theoretically.

Another major property of immersions is that every immersion gives
rise to a normal bundle as follows. This will finally make it
possible to translate the existence of an immersion into the existence
of a vector bundle, respectively later the existence of a homotopy
invariant lifting property.
\begin{LemDef}
  Let $\imm\colon M^n\immto N^{n+r}$ be an immersion of smooth
  manifolds.
  The \emph{normal bundle $\N{\emb}$ of $\imm$}
  is the $r$-dimensional quotient bundle $\pb{\imm}\T N/\T M$,
  respectively the one fulfilling
  $\N{\imm}\oplus\T M\cong\pb{\imm}\T N$.
  This is well-defined, as by definition the fibre-wise rank of the
  immersion's differential is constantly $\dim M$.
\end{LemDef}
Note that unlike the tangent bundle, normal bundles do
not come up with a canonical choice for a fixed manifold.
However, there is one up to a notion of stable equivalence for vector
bundles.
This choice is particularly handy when dealing with
characteristic classes of vector bundles, as these will turn out to be
invariant under the mentioned equivalence relation.
\begin{Def}
  Call two vector bundles $\xi_1$, $\xi_2$ over the same space 
  stably equivalent in case there are $s_1, s_2\in\Nat$ such that
  $\xi_1\oplus\trivbdl^{s_1}\cong\xi_2\oplus\trivbdl^{s_2}$.
\end{Def}
Now the promised notion of a canonical normal bundle can be clarified.
\begin{LemDef}
  Let $M^n$ be a closed, smooth manifold.
  Then all normal bundles of embeddings and immersions of $M$ into
  real space are stably equivalent.
  The resulting equivalence class is called the
  \emph{stable normal bundle of $M$}, written $\N M$.
  When working with vector bundles in a context that is stable in the
  above sense, like \forexample characteristic classes, the stable
  normal bundle of $M$ may be identified with an arbitrary
  representative of itself.
  \begin{proof}[proof (sketch)]
    In order to show stable equivalence, first show that every
    normal bundle of an immersion is stably equivalent to the
    normal bundle of \emph{some} embedding (\idest some injective
    immersion). Then ensure stable equivalence of all normal bundles
    of embeddings.
    \begin{description}
    \item[Immersions]
      Any immersion $\imm\colon M\immto\R^{n+r}$ can be raised to
      higher codimension by concatenation with the linear embedding
      $l\colon\R^{n+r}\immto\R^{n+r+s}$ into the first components.
      As the normal bundles $\N{\imm}$ and
      $\N{l\circ\imm}\cong\N{\imm}\oplus\trivbdl^s$ are stably
      equivalent, raising the codimension does not change the stable
      equivalence class.
      Furthermore, since a regular homotopy yields an
      isomorphism on the normal bundles, it suffices to show:
      \begin{claim}
        For $r>n$ every immersion $M\immto\R^{n+r}$ is regularly
        homotopic to an embedding.
      \end{claim}
      For the claim use bumping techniques as needed for the
      Thom transversality theorem to show that for $r>n$ every
      immersion $\imm\colon M\immto\R^{n+r}$ is regularly homotopic to
      an injective immersion
      (see \forexample \cite[Chap.~II, Lemma~2.5]{adachi}).
      However, as $M$ is chosen to be compact, injective immersions
      are embeddings.
    \item[Embeddings]
      By Whitney's embedding theorem
      (see \forexample \cite[Chap.~II.2]{adachi})
      it is known that every manifold admits an embedding into some
      real space.
      Further, by \forexample the General Position theorem
      (compare \cite[Chap.~2]{embeddingsummary})
      or Haefliger's theorem (see \forexample \cite[Chap.~II.1]{adachi}),
      it is known that for sufficiently large $k\in\Nat$ all embeddings
      $M\immto\R^{n+k}$ are isotopic, \idest homotopic through embeddings,
      and hence their normal bundles are isomorphic.
      Therefore, all normal bundles of embeddings of a manifold are
      stably equivalent.
    \end{description}
  \end{proof}
\end{LemDef}

\section{Characteristic Classes of Vector Bundles}
As the title suggests, the concept of characteristic classes provides the
main tooling throughout the rest of the paper. Therefore, this section
revises the needed results, and recalls in detail the needed properties
of Stiefel-Whitney classes. The latter are of special importance as
they generate all characteristic classes of vector bundles.

\subsection{General Definition and Properties}
Before starting off with the definition of characteristic classes,
recall universal bundles and Steenrod's classification theorem.
\begin{LemDef}\label{def:charcls}
  \optcite[Chapter~14.4]{tomdieck}
  \begin{enumerate}
  \item Any topological group $G$ admits a contractible space $\EG$ with a
    free $G$-action, and a corresponding principal $G$-bundle
    $\gamma^G\colon \EG\to\BG\coloneqq \EG/G$, called the
    \emph{universal $G$-bundle},
    where $\gamma_G$, $\EG$, and $\BG$ are all unique up to
    homotopy.
    $\BG$ is called the \emph{classifying space} for principal
    $G$-bundles.
    For construction and uniqueness see \cite[Example~1B.7~ff.]{hatcher},
    respectively note that universal coverings are unique up to homotopy.
  \item\label{item:classificationthm}
    $\gamma^G$ fulfils the following universal property:
    For any space $X$ admitting the homotopy type of a CW-complex
    there is a bijection between $[X,\BG]$, which denotes the homotopy
    classes of maps from $X$ to $\BG$, and the isomorphism classes of
    principal $G$-bundles over $X$, given by
    \begin{gather*}
      \left(f\colon X\to\BG \right) \longmapsto \pb f \gamma^G
      \;
      \text{\optcite[Theorem 1.4]{immersionconj}}.
    \end{gather*}
    This correspondence is natural in $X$, and is a version of
    Steenrod's classification theorem
    \cite[Theorem~14.4.1]{tomdieck}.
    \optcite[Theorem~1.4, p.~75]{immersionconj}
  \end{enumerate}
\end{LemDef}

Now one can introduce the general concept of characteristic
classes. The particular case of vector bundles, namely the
Stiefel-Whitney classes, will be discussed in detail in
\autoref{sec:swclasses}.
\begin{Def}
  A \emph{characteristic class}
  \begin{itemize}
  \item of degree $i$
  \item with coefficients in a ring $R$
  \item for principal $G$-bundles for a group $G$
  \end{itemize}
  is a natural transformation
  \begin{gather*}
    \Cl\colon [-, \BG] \Longrightarrow \H^i(-; R)\;.
  \end{gather*}
\end{Def}
\begin{Rem}
  By Brown's representation theorem \optcite[Chap.~4.E]{hatcher},
  $\H^i(-;R)$ is a representable functor represented by $K(i,R)$.
  Thus, by the Yoneda lemma, a characteristic class is
  represented by a morphism
  \begin{gather*}
    \cl\colon \BG \longto K(i, R)
  \end{gather*}
  in $\Top$, \idest by a cohomology class $\cl$ of $\BG$.
  Thus, applying $\Cl$ to a principal $G$-bundle over a
  space $X$, which admits the homotopy type of a CW-complex, that is
  represented by a morphism $\eta\colon X\to\BG$
  as in \itemref{def:charcls}{item:classificationthm}
  yields
  \begin{gather*}
    \Cl(X) = \pb{\eta} \cl \in \H^i(X;R)
    \;.
  \end{gather*}
  This describes a one-to-one correspondence between
  characteristic classes as above and cohomology classes in
  $\H^i(BG;R)$, and in the following any characteristic class will be
  identified with its corresponding cohomology class.
\end{Rem}

As this paper is mainly interested in vector bundles, have a closer
look at the significance of classifying spaces in that context,
especially at their stability property and its implications on normal
bundles.
\begin{Lem}\label{lem:classificationvb}
  Let $X$ be a space, $r,s\in\Nat$.
  \begin{enumerate}
  \item\label{item:vbcharacterisation}
    There is a natural equivalence between the category of
    $n$-dimensional vector bundles $\Vect$ and that of principal
    $\Orth(n)$- respectively $\GL n$-bundles.
  \item
    On vector bundles, the inclusion $\B\Orth(r)\to\B\Orth(r+s)$
    represents the direct sum with the trivial bundle $\trivbdl^s$.
    \Idest for vector bundles $\xi_1$ and $\xi_2$ over $X$ with
    classifying maps $f_1$ respectively $f_2$, there is a lift up to
    homotopy of the form
    \begin{center}
      \begin{tikzcd}
        X \ar[r, "f_1"]
        \ar[dr, bend right, "f_2"{left}]
        & \B\Orth(r)
        \ar[d, "\incl"]
        \\
        & \B\Orth(r+s)
      \end{tikzcd}
    \end{center}
    if and only if $\xi_1\oplus\trivbdl^{s}\cong\xi_2$.
  \item\label{item:charclsstablenormalbundle}
    Taking the limit of all classifying maps of the normal bundles of
    embeddings of $M$ yields a homotopy class in
    $[M,\B\Orth=\varinjlim_{k}\B\Orth(k)]$
    that classifies the stable normal bundle of $M$ uniquely.
    Any lift of it in $[M,\B\Orth(r)]$ represents a vector bundle
    $\N{}$ with the property $\N{}\oplus\T M\cong\trivbdl$ by the
    definition of stable equivalence and normal bundles.
  \end{enumerate}  
  \begin{proof}[proof (sketch)]
    The natural equivalence is given by constructing associated vector
    bundles, and the stability property of the family $(\B\Orth(n))_n$
    becomes clear from $\B\Orth(n)\cong\lim_{k\to\infty}G_n(\R^k)$
    where $G_n(\R^k)$ is the Grassmann manifold of $n$-dimensional
    vector subspaces of $\R^k$.
  \end{proof}
\end{Lem}
\begin{Not}
  Throughout this paper assume $R=\Zmod2$ and $G=\Orth(n)$
  respectively $G=\Orth$ if not stated otherwise.
\end{Not}

With the definition at hand, it becomes clear how characteristic classes
serve in translating bundle theoretic problems into homotopy theoretic
ones. This will be a crucial step in reformulating the immersion
conjecture.

\subsection{Stiefel-Whitney Classes}
\label{sec:swclasses}
Now that the general concept is known, this section reviews the
defining and immediate properties of the so-called Stiefel-Whitney
classes, a subset that serves as generators for all characteristic
classes of vector bundles.
There are two ways in which these specific classes will become
essential:
\begin{itemize}
\item Any property stable under cohomology ring operations that holds
  for the generating set of Stiefel-Whitney classes, will apply to all
  characteristic classes of vector bundles.
\item One can construct new characteristic classes out of
  Stiefel-Whitney classes in several ways:
  \begin{description}[labelindent=1em]
  \item[{Formally invert them:}]
    Used to define the dual Stiefel-Whitney classes for Massey's
    theorem, see below.
  \item[{Algebraically combine Stiefel-Whitney classes using their
      generating property:}]
    Used in \autoref{chap:brown} to construct an indecomposability
    indicator.
  \item[{Apply Steenrod squares to Stiefel-Whitney classes:}]
    Used in \autoref{chap:massey} for the Wu classes.
\end{description}
\end{itemize}

First start with the defining properties of the Stiefel-Whitney
classes.
\begin{Def}\label{def:swclasses}
  The \emph{Stiefel-Whitney classes} are
  characteristic classes for principal $\Orth$-bundles
  respectively vector bundles,
  \idest cohomology classes
  $\ws{i}\in \H^i(\BO;\Zmod2)$, $i\in\Nat$,
  fulfilling the following properties for any vector bundles $\xi$ and
  $\eta$ over a space $B$, and any map $f\colon A\to B$ of spaces:
  \begin{axioms}
  \axiom[Naturality] $\pb f\w{i}{\xi} = \w{i}{\pb f \xi}$,
  \axiom $\w{0}{\xi}=1$,
  \axiom $\W{\gamma_1} = 1 + x$,
  \axiom[Multiplicativity]\label{tag:swclassesmultiplicativity}
  $\W{\xi \oplus \eta} = \W{\xi}\cup \W{\eta}$
    \\\idest in degree $n$ we have
    $\w{n}{\xi\oplus\eta} = \sum_{i+j=n}\w{i}{\xi} \cup \w{j}{\eta}$,
  \end{axioms}
  where the \emph{total Stiefel-Whitney class}
  $\Ws\coloneqq\sum_{i\geq 0}\ws{i}$ is the formal sum of all
  Stiefel-Whitney classes,
  $\gamma_1$ is the universal line bundle $\RPinf\cong\BO(1)\to\BO$,
  and $x$ is the%
  \footnote{
    This is well-defined: A ring $R$ of the form $\Zmod2[x]$
    with $\deg(x)=1$ only admits two elements in degree~1, $0$ and a
    generator. Therefore, there exists exactly one ring
    isomorphism, and this sends the unique generator in
    degree 1 to $x$.
  }
  generator of $\H^*(\RPinf;\Zmod2)\cong\Zmod2[x]$.
  \optcite[compare §4, p.~37]{milnor}
\end{Def}
Note, that naturality is already implied by the requirements for a
characteristic class. However, given only the above axioms:
\begin{Thm}
  Stiefel-Whitney classes exist and are uniquely defined by the above
  properties. Furthermore, they generate $\H^*(\BO;\Zmod2)\cong \Zmod2[\ws{i}|i\geq1]$.
  \begin{proof}[proof (sketch)]
    \begin{description}
    \item[Existence]
      A possible concrete construction utilises the Euler
      class. Another definition via Steenrod squares can be 
      found later in \autoref{thm:altdefswclasses}.
    \item[Uniqueness]
      Here one can use the splitting principle and the knowledge about
      $\W{\gamma_1}$
      \cite[Uniqueness Theorem~7.3]{milnor}.
    \item[Generators]
      Either inductively show
      $\H^*(\BO(n);\Zmod2)\cong \Zmod2[\ws{i}|n\geq i\geq1]$
      \cite[Theorem~7.1~ff.]{milnor}.
      Or find generators of $\H^*(\BO)$ that fulfil the axioms for
      the Stiefel-Whitney classes, then apply uniqueness
      \cite[Chap.~7.6]{may}.
      \qedhere
    \end{description}
  \end{proof}
\end{Thm}

As already mentioned, the above generating property means that every
characteristic class of vector bundles of a fixed dimension can be
represented as a certain combination of Stiefel-Whitney classes.
Moreover, they behave extremely well concerning vector bundle
operations as emphasised below.
\begin{Rem}
  \label{rem:propswclasses}
  Let $\xi$, $\eta$ be vector bundles over a space $X$.
  \begin{enumerate} 
  \item\label{item:propswclasses:dimesioncut} $\w{i}{\eta} = 0$
    for any vector bundle $\eta$ with $\rk\eta < i$.
    Therefore, the total Stiefel-Whitney class $\W{\xi}$ is
    well-defined (\idest the sum is finite)
    for any vector bundle $\xi$ of finite rank.
    \begin{proof}
      This follows directly from the splitting principle and
      the multiplicativity axiom.
    \end{proof}
  \item\label{item:swoftrivbdl} $\w{i}{\trivbdl}=0$ for $i>0$, and one immediately concludes
    from multiplicativity:
    \begin{enumerate}
    \item\label{item:swclassesstable}
      The Stiefel-Whitney classes are stable, \idest
      $\w{i}{\xi\oplus\trivbdl} = \w{i}{\xi}$, which once more proves
      the the stability property of characteristic classes of vector bundles.
      Thus, for a manifold $M^n$, all normal bundles $\N{\emb}$ of
      embeddings $\emb\colon M\to\R^{n+k+r}$ share the same
      Stiefel-Whitney classes. These are also the ones obtained by
      pullback along the classifying map of the stable normal bundle
      of $M$, written $\W{\N M}$ accordingly.
    \item\label{item:wuclassmfdinverse}
      If $\xi\oplus\eta = \trivbdl$, $\w{i}{\xi}\cup\w{i}{\eta}=1$.
      Especially, for any choice of embedding $\emb\colon M^n\to\R^{n+k}$
      with normal bundle $\N{\emb}$ of a smooth manifold $M$ we have
      $\T M\oplus\N{\emb} = \trivbdl$ and therefore
      $1 = \W{\T M} \cup \W{\N{\emb}} = \W{\T M}\cup\W{\N M}$.
    \end{enumerate}
    \begin{proof}
      The trivial rank $n$ bundle over $X$ is defined as the pullback
      $\pb \pi \trivbdl$ of the rank $n$ bundle
      $\trivbdl\colon \R^n\to\pt$ over the point by the trivial map
      $\pi\colon X\to\pt$. The naturality of the Stiefel-Whitney
      classes gives $\W{\trivbdl} = \pb\pi \W{\trivbdl}
      \in\pb\pi \left(\H^i(\pt;\Zmod2)\right)$,
      and the result follows from $\H^i(\pt;\Zmod2) = 0$ for $i>0$.
    \end{proof}
  \end{enumerate}
\end{Rem}

In order to algebraically work with the Stiefel-Whitney classes, the
formal inverse is often handy. Especially, since it is well-known for
manifolds as explained below.
\begin{Def}
  Define the \emph{dual Stiefel-Whitney (characteristic) classes}
  $\dualws{i}$ in degree $i$ inductively by
  \begin{align*}
    1 &= \dualws{0}\cup\ws{0} = \dualws{0}    &&\text{in degree 0}\\
    0 &= \sum_{i+j=n} \dualws{i}\cup\ws{j}  &&\text{in degree n>0}
  \end{align*}
  With the notation $\dualWs\coloneqq \sum_{i\geq0} \dualws{i}$ as above
  for the formal sum this can be reformulated as
  \begin{gather*}
    1 = \Ws\cup\dualWs
  \end{gather*}
  in the completion of the polynomial ring $\H^*(\BO)\cong\Zmod2[\ws{i}|i\in\Nat]$.
\end{Def}
By
Remark~\itemref{rem:propswclasses}{item:swoftrivbdl}\ref{item:wuclassmfdinverse},
a first example of dual Stiefel-Whitney classes is given by the
canonical tangent and normal bundle of a manifold, which makes them
especially handy in the context relevant for the immersion conjecture.
\begin{Def}
  For a manifold $M$ use the following abbreviation
  \begin{align*}
    \W{M} &\coloneqq \W{\T M}
            \;,
    &\text{and thus}&
    &\dualW{M} &\coloneqq \dualW{\T M} = \W{\N M}
    \;.
  \end{align*}
\end{Def}

\section{Reformulation of the Immersion Conjecture}
\label{sec:reformulation}
% TODO: section overview

The problem which this paper investigates can finally be clearly
stated with the definitions from
\autoref{sec:immersions}.
The goal of this section is to reformulate the immersion conjecture to
a statement that can be analysed with means of homotopy theory of
vector bundles, and show how characteristic classes relate to this by
finding a powerful obstruction.
The latter will be followed up in the subsequent chapter.

Before reformulating, recall the actual immersion conjecture.
\begin{Def}
  For $n\in\Nat$ consider the unique minimal binary expansion
  \begin{gather*}
    n=2^{i_1}+\dotsb+2^{i_{l_n}},
    \quad\text{with}\quad
    i_1<\dotsb<i_{l_n}
    \;.
  \end{gather*}
  Define $\alpha(n)\coloneqq l_n$, \idest $\alpha(n)$ is the number of
  ones in the binary notation of $n$.
\end{Def}
\begin{Thm}\label{thm:immersionconj}
  For $n\in\Nat$, every closed, smooth, $n$-dimensional manifold
  immerses into $\R^{2n-\alpha(n)}$.
\end{Thm}
In the style of this conjecture, an $n$-manifold that immerses into
some $\R^{2n-\alpha(n)}$ will be said to have the
\emph{immersion property}. % TODO: check, whether defs are needed
And the question, whether a particular manifold does have the
immersion property, will be referred to as the \emph{immersion problem} for
this manifold.

An essential step for the reformulation of the immersion conjecture is
a theorem of Hirsch and Smale, which gives a homotopy theoretical
relation between the spaces of immersions of two manifolds $\Imm M N$
and of the vector bundle monomorphisms $\Mono{\T M}{\T N}$ between
their tangent bundles. For the formulation, one has to equip the
respective sets with a topology as follows.
\begin{Def}
  Let $M$, $N$ be closed smooth manifolds of dimensions $\dim M<\dim N$.
  \begin{enumerate}
  \item
    Equip the set of all vector bundle monomorphisms from $\xi_1$ to
    $\xi_2$ with the compact-open topology (see \forexample
    \cite{hatcher}), and denote that space by $\Mono{\xi_1}{\xi_2}$.
    Note that a path between monomorphisms $F_1$ and $F_2$ in the
    space $\Mono \xi \eta$ is an homotopy from $F_1$ to $F_2$ which is
    a vector bundle monomorphism in each stage.
  \item
    % TODO: (GEORGE) check topology of \Imm(M, N); compare Lecture_Notes_on_Immersions_of_Surfaces_in_3-Space--Nowik.ps
    % TODO: Whitney $C^r$-topology; see [Hirsch, Differential Topology, Chap 2, p.35]
    The set $\Imm M N$ of all immersions from $M$ to $N$ injects
    into $\Mono{\T M}{\T N}$ by taking the differential
    $f\mapsto\Diff f$.
    Equip $\Imm M N$ in the following with the subspace topology.
    This results in the weak topology described in
    \cite[Section~2.1]{hirsch}, which equals the Whitney
    $C^1$-topology since $M$ was chosen compact.
    By the way, $\Imm M N$ is open in $C^1(M,N)$ equipped with the
    Whitney $C^1$-topology
    (see \cite[Section~2.1, Theorem~1.1]{hirsch}),
    and thus not a discrete space.
  \end{enumerate}
\end{Def}

Now one can state the following major result in immersion theory by
Hirsch using preliminary work of Smale
\cite[Sections~5 and 6]{hirschimmersions}. The following formulation
is according to
\cite[Theorem~1.2]{immersionconj}. % TODO: better ref modern formulation
\begin{Thm}[Hirsch-Smale]\label{thm:hirschsmale}
  Let $M$, $N$ be closed manifolds with $\dim M<\dim N$.
  Then the differential map
  $\Diff\colon \Imm M N\to \Mono{\T M}{\T N}$
  induces isomorphisms on the homotopy groups.
  Especially,
  \begin{gather*}
    \Diff_*\colon
    \pi_1(\Imm M N) \overset\sim\longto \pi_1(\Mono{\T M}{\T N})
  \end{gather*}
  describes an isomorphism of path-connected components.
  % Original formulation by Hirsch in [hirschimmersions], sec. 5:
  Therefore, every vector bundle monomorphism
  $F\colon\T M\to\T N$ is homotopic (through vector bundle
  monomorphisms) to a monomorphism which is the differential
  $\Diff f$ of a smooth map $f\colon M\to N$, \idest of an
  immersion.
\end{Thm}
Thus, any monomorphism of vector bundles over smooth, closed
manifolds $M$ and $N$ implies the existence of an immersion from $M$
to $N$.

This finally gives rise to the subsequent reformulation of the
immersion problem of a manifold.
\begin{Thm}\label{thm:immersionconj:equivalences}
  Let $n,k\in\Nat$ and $M^n$ be a closed, smooth, $n$-dimensional manifold.
  The following statements are equivalent.
  \begin{enumerate}
  \item\label{item:immersionconj:1}
    $M$ immerses into $\R^{n+k}$.
  \item\label{item:immersionconj:2}
    There is a vector bundle monomorphism $F\colon\T M\to\T{\R^{n+k}}$.
  \item\label{item:immersionconj:3}
    There is a $k$-dimensional vector bundle
    $\N{}\colon\E{\N{}}\to M$ over $M$ with
    \begin{gather*}
      \N{}\oplus\T M\cong\trivbdl^{n+k}
      \;.
    \end{gather*}
  \item\label{item:immersionconj:4}
    For the map $\N M\colon M\to\B\Orth$ characterising the stable
    normal bundle over $M$ there is a lift $\N{}\colon M\to\B\Orth(k)$
    making the following diagram commute up to homotopy
    \begin{center}
      \begin{tikzcd}
        M
        \ar[r, "\N{}"]
        \ar[dr, "\N{M}"{left}, bend right]
        & \BO(k) \ar[d, "\incl", hookrightarrow] \\
        & \BO
      \end{tikzcd}
    \end{center}
  \end{enumerate}
\end{Thm}
From a homotopy theoretical viewpoint, one obviously is most
interested in statement \ref{item:immersionconj:4}.
One major advantage of this approach is that the bundle theoretic
context allows to apply the huge arsenal of characteristic classes of
vector bundles.
For example the value $k=n-\alpha(n)$ of the minimal immersion dimension
is motivated by means of Stiefel-Whitney classes in \autoref{chap:massey}.

\begin{proof}[proof of \autoref{thm:immersionconj:equivalences}]
  The strategy is to show
  \ref{item:immersionconj:1}$\Rightarrow$%
  \ref{item:immersionconj:4}$\Leftrightarrow$%
  \ref{item:immersionconj:3}$\Leftrightarrow$%
  \ref{item:immersionconj:2}$\Leftrightarrow$%
  \ref{item:immersionconj:1}.
  \begin{description}
  \item[\ref{item:immersionconj:1}$\Rightarrow$\ref{item:immersionconj:4}:]
    The classifying map of an immersion's normal bundle
    lifts $\N{M}$ as required,
    using Steenrod's classification theorem
    \itemref{def:charcls}{item:classificationthm} and the properties
    of the stable normal bundle from
    \itemref{lem:classificationvb}{item:charclsstablenormalbundle}.
  \item[\ref{item:immersionconj:4}$\Rightarrow$\ref{item:immersionconj:3}:]
    Also by
    \itemref{lem:classificationvb}{item:charclsstablenormalbundle},
    any rank $k$ vector bundle that is represented by a lift of
    $\N{M}$ to $[M,\B\Orth(k)]$ as in \ref{item:immersionconj:4} has
    the property needed for \ref{item:immersionconj:3}.
  \item[\ref{item:immersionconj:2}$\Leftrightarrow$\ref{item:immersionconj:3}:]
    In order to get from a vector bundle monomorphism
    $F\colon M\to\R^{n+k}$ to a normal bundle, note that
    $\pb f\T{\R^{n+k}}$ for $f=F|_{\zerosec{\T M}}$ is 
    trivial, and that $F\colon \T M\to \pb f\T{\R^{n+k}}$ is not only
    fibre-wise, but globally injective of constant rank
    $\dim M$.
    Thus, as in the definition of normal bundles,
    $\N{}\coloneqq\pb f\T{\R^{n+k}}/\T M$ is a vector bundle over $M$
    that obviously fulfils the required property of
    \ref{item:immersionconj:3}.
    
    Conversely, when starting with some rank $k$ vector bundle $\N{}$
    such that $\N{}\oplus\T M\cong\trivbdl^{n+k}$, there is a vector
    bundle monomorphism $\N{}\to\trivbdl$ over $M$. Then the
    following chain of vector bundle morphisms
    \begin{center}
      \begin{tikzcd}
        \T M \ar[d]
        \ar[r, hookrightarrow]
        & M\times\R^{n+k} \ar[d,"\trivbdl^{n+k}"]
        \ar[r]
        & \pt\times \R^{n+k} \ar[d,"\trivbdl^{n+k}"]
        \ar[r, hookrightarrow]
        & \R^{n+k}\times \R^{n+k} \ar[d,"\trivbdl^{n+k}"]
        \\
        M
        \ar[r,equals]
        & M
        \ar[r]
        & \pt
        \ar[r, hookrightarrow]
        & \R^{n+k}
      \end{tikzcd}
    \end{center}
    is fibre-wise injective in each stage, and hence a monomorphism as
    was needed.
  \item[\ref{item:immersionconj:1}$\Leftrightarrow$\ref{item:immersionconj:2}:]
    The tricky part is to relate \ref{item:immersionconj:1} and
    \ref{item:immersionconj:2}, even though it is easily seen that
    \ref{item:immersionconj:1} implies \ref{item:immersionconj:2} by
    simply taking $F$ to be the differential $\Diff f$ of the
    immersion from \ref{item:immersionconj:1}.
    
    The converse direction is an application of the Hirsch-Smale
    theorem~\autoref{thm:hirschsmale}.
    However, first substitute the non-compact manifold $\R^{n+k}$ with
    the compact sphere $N=\Sphere{n+k}$, to make $M$ and $N$ comply
    with the preliminaries of the theorem. As $\dim M<n+k$ by assumption,
    every immersion $M\to\Sphere{n+k}$ misses a point on
    $\Sphere{n+k}$ and hence factors over an immersion $M\to\R^{n+k}$.
    This then shows that also \ref{item:immersionconj:2} implies
    \ref{item:immersionconj:1} which makes them equivalent.
    \qedhere
  \end{description}
\end{proof}

This now gives rise to involve the powerful obstruction theory of
characteristic classes of vector bundles as follows.
\begin{Cor}\label{cor:obstruction}
  Let $n,k\in\Nat$, and $M^n$ be a smooth, closed manifold.
  If $M$ immerses into $\R^{n+k}$, then $\dualw{i}{M}=0$ for all
  $i>k$.
  \begin{proof}
    By Theorem~\autoref{thm:immersionconj:equivalences} $M$ immerses
    into $\R^{n+k}$ if and only if there is a rank-$k$
    normal bundle $\N{}$ of $M$.
    Since the Stiefel-Whitney classes are stable,
    $\W{\N{}}=\W{\N M}\cequalsby{Def.}\dualW{M}$.
    However, as explained in
    \itemref{rem:propswclasses}{item:propswclasses:dimesioncut},
    all Stiefel-Whitney classes $\w{i}{\N{}}$ of degree $i$ exceeding
    the rank $k$ of $\N{}$ are zero.
  \end{proof}
\end{Cor}
As a result, the immersion conjecture requires that all $n$-manifolds
have vanishing dual Stiefel-Whitney classes in degrees $i>n-\alpha(n)$.
That this is true, is a theorem of Massey which will be proven in
\autoref{chap:massey}. It was an inspiration to state the conjecture
with the value $k=n-\alpha(n)$ in the first place.


%%% Local Variables:
%%% mode: latex
%%% TeX-master: "thesis"
%%% End:
