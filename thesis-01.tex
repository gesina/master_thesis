%%%%%%%%%%%%%%%%%%%%%%%%%%%%%%%%%% 
% Master Thesis in Mathematics
% "Immersions and Stiefel-Whitney classes of Manifolds"
% -- Chapter 1: Formulation of the Immersion Conjecture --
% 
% Author: Gesina Schwalbe
% Supervisor: Georgios Raptis
% University of Regensburg 2018
%%%%%%%%%%%%%%%%%%%%%%%%%%%%%%%%%% 

\chapter{Formulation of the Immersion Conjecture}
% Explain carefully how the immersion problem can be reformulated purely
% in terms of homotopy theory. [immersionconj] (The necessary results from differential
% topology and Hirsch-Smale theory should be stated clearly but may be
% presented without proofs.)
This chapter is dedicated to gathering the concepts and results needed
to formulate and work with the immersion conjecture.

As a first result, the goal of \autoref{sec:reformulation} will be to
reformulate the immersion conjecture to the existence of certain
lifts up to homotopy, and to outline how characteristic 
classes can directly be applied in investigating this problem.

Beforehand, the definitions and main properties of immersions,
Steenrod squares, and selected characteristic classes are
revised. Most importantly, the major properties and a very handy
construction of the Stiefel-Whitney characteristic classes of vector
bundles is given (\autoref{sec:swclasses}), which uses Steenrod
squares (\autoref{sec:steenrodsquares}, as well as 
the Thom isomorphisms and Thom classes (\autoref{sec:thomclasses}).
Both the construction result and this technique of building new
characteristic classes out of old ones with cohomology operations
prove to be invaluable in \autoref{chap:massey} when having a closer
look at the Stiefel-Whitney numbers of normal bundles of manifolds.

Mind that a couple of preliminary results are merely referenced
without proof.

\section{Immersions and Normal Bundles}\label{sec:immersions}
This section recapitulates the definition and major needed properties
of immersions, \idest local embeddings of manifolds.

More precisely, immersions are merely a special case of monomorphisms
of vector bundles. So, recall that a morphism
$(\xi_1\colon\E_1\to X_1)\to(\xi_2\colon E_2\to X_2)$ 
of vector bundles over different spaces is a map $F\colon E_1\to E_2$
that covers its restriction to the zero section, \idest it makes the
following diagram commute
\begin{center}
  \begin{tikzcd}
    E_1 \ar[r,"F"]\ar[d, "\xi_1"]
    & E_2 \ar[d, "\xi_2"]
    \\
    X_1 \ar[r, "F|_{\zerosec{\xi_1}}"]
    & X_2
  \end{tikzcd}.
\end{center}
Further remember the fact that such a morphism is a monomorphism in
the category of vector bundles $\Vect$, if and only if its restriction
to each fibre is injective.
\begin{Def}
  A smooth map $f\colon M\to N$ of smooth manifolds is called
  an \emph{immersion}, written $M\immto N$, if its differential
  $\Diff f\colon\T M\to\T N$ is a monomorphism of vector
  bundles.
  A homotopy which is an immersion in each stage is called regular.
\end{Def}
\begin{Rem}
  Let $M$ and $N$ be a smooth manifolds.
  \begin{enumerate}
  \item
    Immersions are local embeddings, \idest for an immersion
    $f\colon M\immto N$ there is around every point an open neighbourhood
    on which $f$ is an embedding.
    More descriptive, immersions are mappings that do not allow creases,
    respectively sharp bends, or puncturing.
    However, globally, immersions need not even be injective since
    \forexample self-intersections of the image are allowed.
    \begin{proof}
      Local embeddings are obviously immersions by definition, and the
      other direction is a conclusion from the implicit function theorem.
    \end{proof}
  \item Embeddings are exactly the injective immersions that are
    topological embeddings.
    If $M$ is compact, any injective immersion $f\colon M\to N$ is an
    embedding.
  \end{enumerate}
\end{Rem}
It is easy to see that in general not all vector bundle monomorphisms
between tangent bundles of smooth manifolds need to be the
differential of an immersion. However, taking the differential gives a
canonical inclusion of the set of immersions into the set of vector
bundle monomorphisms.
Later, an important result due to Hirsch and Smale will be needed that
relates spaces of immersions and vector bundle monomorphisms homotopy
theoretically.

Another major property of immersions is that every immersion gives
rise to a normal bundle as follows. This will finally make it
possible to translate the existence of an immersion into the existence
of a vector bundle, respectively later the existence of a homotopy
invariant lifting property.
\begin{LemDef}
  Let $\imm\colon M^n\immto N^{n+r}$ be an immersion of smooth
  manifolds.
  The \emph{normal bundle $\N{\emb}$ of $\imm$}
  is the $r$-dimensional quotient bundle $\pb{\imm}\T N/\T M$,
  respectively the one fulfilling
  $\N{\imm}\oplus\T M\cong\pb{\imm}\T N$.
  This is well-defined, as by definition the fibre-wise rank of the
  immersion's differential is constantly $\dim M$.
\end{LemDef}
Note that unlike the tangent bundle, normal bundles do
not come up with a canonical choice for a fixed manifold.
However, there is one up to a notion of stable equivalence for vector
bundles.
This choice is particularly handy when dealing with
characteristic classes of vector bundles, as these will turn out to be
invariant under the mentioned equivalence relation.
\begin{Def}
  Call two vector bundles $\xi_1$, $\xi_2$ over the same space 
  stably equivalent in case there are $s_1, s_2\in\Nat$ such that
  $\xi_1\oplus\trivbdl^{s_1}\cong\xi_2\oplus\trivbdl^{s_2}$.
\end{Def}
Now the promised notion of a canonical normal bundle can be clarified.
\begin{LemDef}
  Let $M^n$ be a closed, smooth manifold.
  Then all normal bundles of embeddings and immersions of $M$ into
  real space are stably equivalent.
  The resulting equivalence class is called the
  \emph{stable normal bundle of $M$}, written $\N M$.
  When working with vector bundles in a context that is stable in the
  above sense, like \forexample characteristic classes, the stable
  normal bundle of $M$ may be identified with an arbitrary
  representative of itself.
  \begin{proof}[proof (sketch)]
    In order to show stable equivalence, first show that every
    normal bundle of an immersion is stably equivalent to the
    normal bundle of \emph{some} embedding (\idest some injective
    immersion). Then ensure stable equivalence of all normal bundles
    of embeddings.
    \begin{description}
    \item[Immersions]
      Any immersion $\imm\colon M\immto\R^{n+r}$ can be raised to
      higher codimension by concatenation with the linear embedding
      $l\colon\R^{n+r}\immto\R^{n+r+s}$ into the first components.
      As the normal bundles $\N{\imm}$ and
      $\N{l\circ\imm}\cong\N{\imm}\oplus\trivbdl^s$ are stably
      equivalent, raising the codimension does not change the stable
      equivalence class.
      Furthermore, since a regular homotopy yields an
      isomorphism on the normal bundles, it suffices to show:
      \begin{claim}
        For $r>n$ every immersion $M\immto\R^{n+r}$ is regularly
        homotopic to an embedding.
      \end{claim}
      For the claim use bumping techniques as needed for the
      Thom transversality theorem to show that for $r>n$ every
      immersion $\imm\colon M\immto\R^{n+r}$ is regularly homotopic to
      an injective immersion
      (see \forexample \cite[Chap.~II, Lemma~2.5]{adachi}).
      However, as $M$ is chosen to be compact, injective immersions
      are embeddings.
    \item[Embeddings]
      By Whitney's embedding theorem
      (see \forexample \cite[Chap.~II.2]{adachi})
      it is known that every manifold admits an embedding into some
      real space.
      Further, by \forexample the General Position theorem
      (compare \cite[Chap.~2]{embeddingsummary})
      or Haefliger's theorem (see \forexample \cite[Chap.~II.1]{adachi}),
      it is known that for sufficiently large $k\in\Nat$ all embeddings
      $M\immto\R^{n+k}$ are isotopic, \idest homotopic through embeddings,
      and hence their normal bundles are isomorphic.
      Therefore, all normal bundles of embeddings of a manifold are
      stably equivalent.
    \end{description}
  \end{proof}
\end{LemDef}

\section{Characteristic Classes of Vector Bundles}
As the title suggests, the concept of characteristic classes provides the
main tooling throughout the rest of the paper. Therefore, this section
revises the needed results, and recalls in detail the needed properties
of Stiefel-Whitney classes. The latter are of special importance as
they generate all characteristic classes of vector bundles.

As preliminary information, several methods of constructing new
characteristic classes of vector bundles out of Stiefel-Whitney
classes will be utilised throughout this paper:
\begin{description}[labelindent=1em]
\item[{Formally invert them:}]
  Used to define the dual Stiefel-Whitney classes for Massey's
  theorem.
\item[{Algebraically combine Stiefel-Whitney classes using their
    generating property:}]
  Used in \autoref{chap:brown} to construct an indecomposability
  indicator.
\item[{Apply Steenrod squares to Stiefel-Whitney classes:}]
  Used in \autoref{chap:brown} for the Wu classes.
\end{description}
For the last method, a handy construction of the Stiefel-Whitney
classes will be presented which uses Steenrod squares, Thom classes,
and the Thom isomorphisms.

Before starting off with the definition of characteristic classes,
recall universal bundles and Steenrod's classification theorem.
\begin{LemDef}\label{def:charcls}
  \optcite[Chapter~14.4]{tomdieck}
  \begin{enumerate}
  \item Any topological group $G$ admits a contractible space $\EG$ with a
    free $G$-action, and a corresponding principal $G$-bundle
    $\gamma^G\colon \EG\to\BG\coloneqq \EG/G$, called the
    \emph{universal $G$-bundle},
    where $\gamma_G$, $\EG$, and $\BG$ are all unique up to
    homotopy.
    $\BG$ is called the \emph{classifying space} for principal
    $G$-bundles.
    For construction and uniqueness see \cite[Example~1B.7~ff.]{hatcher},
    respectively note that universal coverings are unique up to homotopy.
  \item\label{item:classificationthm}
    $\gamma^G$ fulfils the following universal property:
    For any space $X$ admitting the homotopy type of a CW-complex
    there is a bijection between $[X,\BG]$, which denotes the homotopy
    classes of maps from $X$ to $\BG$, and the isomorphism classes of
    principal $G$-bundles over $X$, given by
    \begin{gather*}
      \left(f\colon X\to\BG \right) \longmapsto \pb f \gamma^G
      \;
      \text{\optcite[Theorem 1.4]{immersionconj}}.
    \end{gather*}
    This correspondence is natural in $X$, and is a version of
    Steenrod's classification theorem
    \cite[Theorem~14.4.1]{tomdieck}.
    \optcite[Theorem~1.4, p.~75]{immersionconj}
  \end{enumerate}
\end{LemDef}

As this paper is mainly interested in vector bundles, have a closer
look at the significance of classifying spaces in that context,
especially at their stability property and its implications on normal
bundles.
\begin{Lem}\label{lem:classificationvb}
  Let $X$ be a space, $r,s\in\Nat$.
  \begin{enumerate}
  \item\label{item:vbcharacterisation}
    There is a natural equivalence between the category of
    $n$-dimensional vector bundles $\Vect$ and that of principal
    $\Orth(n)$- respectively $\GL n$-bundles.
  \item
    On vector bundles, the inclusion $\B\Orth(r)\to\B\Orth(r+s)$
    represents the direct sum with the trivial bundle $\trivbdl^s$.
    \Idest for vector bundles $\xi_1$ and $\xi_2$ over $X$ with
    classifying maps $f_1$ respectively $f_2$, there is a lift up to
    homotopy of the form
    \begin{center}
      \begin{tikzcd}
        X \ar[r, "f_1"]
        \ar[dr, bend right, "f_2"{left}]
        & \B\Orth(r)
        \ar[d, "\incl"]
        \\
        & \B\Orth(r+s)
      \end{tikzcd}
    \end{center}
    if and only if $\xi_1\oplus\trivbdl^{s}\cong\xi_2$.
  \item\label{item:charclsstablenormalbundle}
    Taking the limit of all classifying maps of the normal bundles of
    embeddings of $M$ yields a homotopy class in
    $[M,\B\Orth=\varinjlim_{k}\B\Orth(k)]$
    that classifies the stable normal bundle of $M$ uniquely.
    Any lift of it in $[M,\B\Orth(r)]$ represents a vector bundle
    $\N{}$ with the property $\N{}\oplus\T M\cong\trivbdl$ by the
    definition of stable equivalence and normal bundles.
  \end{enumerate}  
  \begin{proof}[proof (sketch)]
    The natural equivalence is given by constructing associated vector
    bundles, and the stability property of the family $(\B\Orth(n))_n$
    becomes clear from $\B\Orth(n)\cong\lim_{k\to\infty}G_n(\R^k)$
    where $G_n(\R^k)$ is the Grassmann manifold of $n$-dimensional
    vector subspaces of $\R^k$.
  \end{proof}
\end{Lem}

Now one can introduce the general concept of characteristic
classes. The particular case of vector bundles, namely the
Stiefel-Whitney classes, will be discussed in detail in
\autoref{sec:swclasses}.
\begin{Def}
  A \emph{characteristic class}
  \begin{itemize}
  \item of degree $i$
  \item with coefficients in a ring $R$
  \item for principal $G$-bundles for a group $G$
  \end{itemize}
  is a natural transformation
  \begin{gather*}
    \Cl\colon [-, \BG] \Longrightarrow \H^i(-; R)\;.
  \end{gather*}
\end{Def}
\begin{Rem}
  By Brown's representation theorem \optcite[Chap.~4.E]{hatcher},
  $\H^i(-;R)$ is a representable functor represented by $K(i,R)$.
  Thus, by the Yoneda lemma, a characteristic class is
  represented by a morphism
  \begin{gather*}
    \cl\colon \BG \longto K(i, R)
  \end{gather*}
  in $\Top$, \idest by a cohomology class $\cl$ of $\BG$.
  Thus, applying $\Cl$ to a principal $G$-bundle over a
  space $X$, which admits the homotopy type of a CW-complex, that is
  represented by a morphism $\eta\colon X\to\BG$
  as in \itemref{def:charcls}{item:classificationthm}
  yields
  \begin{gather*}
    \Cl(X) = \pb{\eta} \cl \in \H^i(X;R)
    \;.
  \end{gather*}
  This describes a one-to-one correspondence between
  characteristic classes as above and cohomology classes in
  $\H^i(BG;R)$, and in the following any characteristic class will be
  identified with its corresponding cohomology class.
\end{Rem}

Now that the definition is given it should be clear how 
characteristic classes serve in translating bundle theoretic problems
into homotopy theoretic ones. This will be a crucial step in
reformulating the immersion conjecture.

The next subsections will prepare a particular construction of
Stiefel-Whitney classes, the characteristic classes of vector
bundles.

\subsection{Steenrod Squares}\label{sec:steenrodsquares}
Similar to the cup and cap product, Steenrod squares add quite a lot
of further structure to a cohomology ring, thus adding more tools to
differentiate spaces by means of cohomology comparison.
As applications, they will serve in constructing two kinds of
characteristic classes, namely the Stiefel-Whitney and the Wu
classes, and give rise to powerful cohomological obstructions
when combined with further special properties like the Poincaré
duality, as will be seen in \label{chap:massey}.

The following definition of the Steenrod squares is due to
Steenrod and Epstein \cite[Chap.~I.1, p.~1]{steenrodepstein}.
\begin{Def}\label{def:sq}
  The \emph{Steenrod squares} $\Sq i$ for $i\in\Nat$ are each a family
  of cohomology operations, \idest families of homomorphisms, of the
  form
  \begin{gather*}
    \left(
      \Sq i\colon \H^n(X, A;\Zmod2) \to \H^{n+i}(X, A;\Zmod2)
      \;\middle|\;
      n\in\Nat
    \right)
  \end{gather*}
  that satisfy the following relations for any pair of spaces $(X,A)$, and any map of
  pairs of spaces $f\colon (X,A)\to (Y,B)$:
  \begin{description}
  \item[(Naturality)]\label{item:sqnaturality} $\pb f\circ\Sq i = \Sq i\circ\pb f$
  \item[(Stability)]\label{item:sqstability} $\susp\circ\Sq i = \Sq i\circ\susp$
  \item[(Cartan formula)] For any $n\in\Nat$, and $x,y\in \H^n(X)$ holds
    \begin{gather}\label{tag:cartan}\tag{Cartan's formula}
      \Sq i(x\cup y) = \sum_{r+s=i}\Sq r(x)\cup\Sq s(y)
    \end{gather}
  \item[(Fixed values)] The following values are fixed for $x\in \H^n(X,A)$:
    % \begin{gather}\label{eq:sqlowerbound}
    %   \Sq i(x) = \begin{cases}
    %     0 & n<i \\
    %     x^2 & n=i
    %   \end{cases}
    %   \qquad\text{and}\qquad
    %   \Sq i = \begin{cases}
    %     \Id & i=0\\
    %     \beta & i=1
    %   \end{cases}
    % \end{gather}
    \begin{alignat}{4}
      \Sq i(x) &= 0     & \qquad\text{for }n<i \label{eq:sqlowerbound}\\
      \Sq i(x) &= x^2   & \qquad\text{for }n=i \label{eq:sqsquared}\\
      \Sq 0    &= \Id   \label{eq:sqidentity}
      % \\\notag
      % \Sq 1    &= \beta \label{eq:sqbockstein}
    \end{alignat}
    % where $\beta$ denotes the Bockstein homomorphism
    % \cite[see \forexample][Chap.~3.E]{hatcher} of the exact
    % coefficient sequence
    % \begin{center}
    %   \begin{tikzcd}
    %     0 \ar[r]
    %     &\Zmod2 \ar[r,"\incl"]
    %     &\Zmod4 \ar[r,"\proj"]
    %     &\Zmod2 \ar[r]
    %     &0
    %   \end{tikzcd}
    % \end{center}
  \item[(Adem relations)] For $\alpha<2\beta$ holds
    \begin{gather}\label{tag:adem}\tag{Adem's formula}
      \Sq\alpha \circ \Sq\beta =
      \sum_{j=0}^{\left\lfloor \frac \alpha 2 \right\rfloor}
      \binom{\beta-j-1}{\alpha-2j}
      \Sq{\alpha+\beta+j}\Sq{j}
    \end{gather}
  \end{description}
  $\SQ\coloneqq \sum_{j\in\Nat}\Sq j$ is the formal sum of all
  Steenrod squares called the \emph{the total Steenrod square}.
  Note that for any degree $n\in\Nat$ the total Steenrod square
  $\SQ\colon \H^n(X)\to \H^*(X)$ is well-defined since the sum is
  finite by \eqref{eq:sqlowerbound}.
  Also \ref{tag:cartan} can be reformulated to
  $\SQ(x\cup y) = \SQ(x)\cup\SQ(y)$, \idest $\SQ$ is a group
  homomorphism with respect to the cup-product.
\end{Def}

\begin{Thm}
  The Steenrod squares exist and are uniquely determined by
  naturality, \ref{tag:cartan}, and the fixed values
  %\eqref{eq:sqlowerbound}, \eqref{eq:sqsquared}, and
  %\eqref{eq:sqidentity}
  from Definition~\autoref{def:sq}.
  \begin{proof}
    For existence see \cite[Chapter 2]{mosher},
    for uniqueness see \cite[VIII §3]{steenrodepstein}.
  \end{proof}
\end{Thm}

The fact that Steenrod squares, or in general cohomology operations,
can be added and concatenated, already gives a hint that they might
form a ring, which was proven by Steenrod. The following notation and
facts are according to \cite[Chap.~6]{mosher}.
\begin{Def}
  The \emph{Steenrod algebra} $\A$ is the quotient
  of the graded $\Zmod2$-polynomial algebra
  $\Zmod2[\Sq i|i\in\Nat]$ with grading $\deg \Sq i\coloneqq i$
  by the two-sided relations of both \ref{tag:adem} and $\Sq 0=1$.
  With the induced grading it is an associative, connected,
  non-commutative graded Hopf algebra over $\Zmod2$.
\end{Def}
\begin{Not}
  In the following, iterated Steenrod squares
  $\Sq{i_1}\cdot\Sq{i_2}\dotsm\Sq{i_l}$ will have the short form
  $\Sq{(i_1,i_2,\dotsc,i_l)}$,
  and will be evaluated on an element $x$ of a cohomology ring as
  $\Sq{i_1}\circ\dotsb\circ\Sq{i_l}(x)$ respecting the
  properties from \autoref{def:sq}.
  Furthermore, for a sequence $I=(i_1,\dotsc,i_l)$,
  respectively $\Sq I$, denote by
  \begin{description}[labelindent=1em]
  \item[$\l(I)\coloneqq l$] the \emph{length} of $I$,
  \item[$\d(I)\coloneqq \sum_{j=0}^{l} i_j$] the \emph{degree} of $I$,
    and by
  \item[$\e(I)\coloneqq 2i_1-\d(I)=\sum_{j=1}^{l-1}(i_j-2i_{j+1})$]
    the \emph{excess} of $I$.
  \end{description}
  The sequence $I$, respectively $\Sq I$, is called \emph{admissible},
  if $\l(I)=1$ or $i_j\geq 2i_{j+1}$ for $0\leq j<\l(I)$.
\end{Not}

\begin{Rem}\label{rem:sq}
  The following properties will be needed for Massey's Theorem:
  \begin{enumerate}
  \item The set of iterated Steenrod squares $\Sq I$ of admissible
    sequences $I$ forms a basis for $A$ as $\Zmod2$-vector space
    \cite[Chap.~6, Theorem~1]{mosher}.
  \end{enumerate}
  Let $I=(i_1,\dotsc,i_l)$ be a sequence in $\Nat$.
  \begin{enumerate}[resume*]
  \item $\deg(\Sq I(x)) = \deg(x) + \d(\Sq I)$.
  \item\label{item:squpperboundgeneral} $\Sq I(x) = 0$ for  $\deg(x)<\e(I)$ if $I$ is admissible.
    \begin{proof}
      This follows by induction over $\l(I)$ using
      \eqref{eq:sqlowerbound}. The case $\l(I)=1$ follows directly 
      from \eqref{eq:sqlowerbound}.
      For $\l(I)>1$ and $J\coloneqq(i_2,\dotsc,i_l)$, the condition
      $\deg(x) < \e(I)=i_1-\d(J)$
      implies
      \begin{gather*}
        \deg(\Sq J(x))
        = \deg(x)+\d(J) < \e(J)+\d(J) = 2i_2
        \overset{\text{adm.}}\leq i_1
      \end{gather*}
      So,
      $\Sq I(x)=\Sq{i_1}(\Sq J(x)) \cequalsby{\eqref{eq:sqlowerbound}} 0$.
    \end{proof}
  \end{enumerate}
\end{Rem}

The subsequent concept of formally inverting a formal sum of elements
in a graded $\Zmod2$-algebra will recur for several characteristic
classes. This particular definition is needed to define the Wu classes
in \autoref{sec:wuclasses}. 
\begin{Def}\label{def:antipode}
  The antipode $\antipode\colon \A\to\A$ of the Steenrod algebra is a
  graded homomorphism inductively defined by the relation
  \begin{gather*}
    1 = \Sq 0
    = \SQ \Sqcup \antipode(\SQ)
    = \sum_{k\geq0}\sum_{r+s=k} \Sq r \Sqcup \antipode(\Sq s)
  \end{gather*}
\end{Def}


\subsection{Thom Classes and the Thom Isomorphisms}\label{sec:thomclasses}
In the following, Thom classes and the Thom isomorphisms will be
revised. From the definition it will be clear that Thom classes are
\emph{no} characteristic classes in the above sense. However, the
tools that will be introduced in this section are an important
ingredient in constructing Stiefel-Whitney characteristic classes from
Steenrod squares.

Throughout the section,
let $B$ be a paracompact space, \forexample a manifold,
$\xi\colon E\xrightarrow{p} B$ a vector bundle over $B$ of rank $\rkk>0$,
and $R$ be principal ideal domain.
\begin{Def}
  A \emph{Thom class} of $\xi$ in $R$-coefficients is a
  cohomology class $\u{\xi}\in \H^{\rkk}(\spherepair{E}; R)$,
  such that for all points $b\in B$ and fibre inclusions
  $i_b\colon (\spherepair{E_b}) \to (\thomspacepair{E})$
  the restriction $\u{\xi}|_{E_b} = \pb i_b (\u{\xi})$ is a
  free generator of the $R$-module $\H^{\rkk}(\spherepair{E_b}; R)$,
  \idest a unit
  of the ring
  $\H^{\rkk}(\spherepair{E_b};R)\cong \H^{\rkk}(\spherepair{\R}; R)\cong R$.
  \optcite[p.~441]{hatcher}
\end{Def}

The following corollaries will deduce notions of naturality,
multiplicativity, and uniqueness for Thom classes quite directly from
their above definition.

\begin{Cor}\label{cor:thomclsnatural}
  The Thom class construction is natural with respect to the pullback
  of vector bundles over paracompact spaces.
  \Idest given any map of paracompact spaces $f\colon A\to B$, and a
  vector bundle $\xi\colon E\to B$,
  the pullback $\pb f \U$ of a Thom class $\U$ of $\xi$ will
  be a Thom class of $\pb f \xi$.
  \begin{proof}
    Let $\U$ be a Thom class of $\xi$ and $a\in A$ any point.
    Consider the restriction $\pb i_a(\pb f \U)$
    of the pullback of $\U$ to the fibre over $a$. To show that this
    is a generator of $\H^{\rkk}(\spherepair{E_a};R)$ first use that
    pullbacks commute with restriction:
    \begin{gather*}
      \pb i_a(\pb f \U)
      = \pb {(f\circ i_a)} \U
      = \pb {(i_{f(a)}\circ f)} \U
      = \pb f (\pb i_{f(a)} \U)
    \end{gather*}
    $\pb i_{f(a)} \U$ is a generator by definition of $\U$.
    Now the restriction of $f$
    \begin{gather*}
      f\colon (\spherepair{(\pb f E)_a}) \to (\spherepair{E_{f(a)}})
    \end{gather*}
    is an isomorphism, and thus
    $\pb f\colon \H^r(\spherepair{E_{f(a)}})
    \cong \H^r(\spherepair{(\pb f E)_a})$
    sends generators to generators for all $r\in\Nat$.
  \end{proof}
\end{Cor}

\begin{Rem}
  Let $\xi$, $\eta$ be vector bundles over a space $B$.
  \begin{itemize}
  \item There is a canonical isomorphism
    $\E{(\xi\oplus\eta)}\cong\E\xi\oplus\E\eta$
    with the canonical projections
    $\pi_\xi\colon \E{(\xi\oplus\eta)}\to \E\xi$
    and
    $\pi_\eta\colon\E{(\xi\oplus\eta)}\to \E\eta$.
  \item There is a corresponding Künneth isomorphism
    defining the \emph{cross-product} \optcite[p.~214]{hatcher}
    \begin{align*}
      \H^*(\thomspacepair{\E\xi})
      \otimes
      \H^*(\thomspacepair{\E\eta})
      &\longisoto
        \H^*(\thomspacepair{\E{(\xi\oplus\eta)}})\\
      x\otimes y
      &\longmapsto
        \pb \pi_\xi x \cup \pb \pi_\eta y
        \eqqcolon x\times y
        \;.
    \end{align*}
    \cite[Theorem~3.18]{hatcher}
  \end{itemize}
\end{Rem}
\begin{Cor}\label{cor:thomclassmultiplicative}
  The Thom class construction for coefficients in a field $R$ is
  multiplicative in the following sense:
  For vector bundles $\xi\colon E\to B$, $\eta\colon E'\to B$
  of rank $\rkk$ respectively $\rkl$ over a paracompact space $B$, and Thom
  classes
  $\u{\xi}\in \H^{\rkk}(\thomspacepair{\E\xi}; R)$,
  $\u{\eta}\in \H^{\rkl}(\thomspacepair{\E\eta}; R)$
  the class
  \begin{gather*}
    \u{\xi}\times\u{\eta}
    \coloneqq \pb \pi_\xi\u{\xi} \cup \pb \pi_\eta\u{\eta}
    \in \H^{\rkk+\rkl}(\thomspacepair{\E{(\xi\oplus\eta)}})
  \end{gather*}
  is a Thom class of $\xi\oplus\eta$.
  \begin{proof}
    Consider a fibre $b\in B$. As cup product and pullback commute
    with restriction, the cross-product also commutes with
    restriction, \idest one has to show that
    \begin{gather*}
      \pb i_a(\u{\xi}\times\u{\eta})
      = \left(\pb i_a\u{\xi}\right)
      \times \left(\pb i_a\u{\eta}\right)
    \end{gather*}
    is a generator of
    $\H^{\rkk+\rkl}(\thomspacepair{\E{(\xi\oplus\eta)}};R)$.
    % There are homotopies making the following diagram commute
    % \begin{center}
    %   \begin{tikzcd}
    %     (\spherepair{\E{(\xi)}_b})
    %     \ar[d, dash, "\isosymb"{above,rotate=90}]
    %     &(\spherepair{\E{(\xi\oplus\eta)}_b})
    %     \ar[l,"\pi_\xi"above] \ar[r,"\pi_\eta"]
    %     \ar[d, dash, "\isosymb"{above,rotate=90}]
    %     &(\spherepair{\E{(\eta)}_b})
    %     \ar[d, dash, "\isosymb"{above,rotate=90}]\\
    %     (\spherepair{\R^i})
    %     &(\spherepair{\R^{i+j}})
    %     \ar[l,"\proj"above] \ar[r,"\proj"]
    %     &(\spherepair{\R^{j}})
    %   \end{tikzcd}
    % \end{center}
    By the naturality of the Künneth isomorphism there is the
    following commutative diagram that translates this problem to one
    on the cohomology of spheres:
    \begin{center}
      \begin{tikzcd}
        \H^*(\spherepair{\E\xi_b};R)
        \otimes \H^*(\spherepair{\E\eta_b};R)
        \ar[r, "\cong"]
        \ar[d, dash, "\cong"{above,rotate=90}]
        & \H^*(\spherepair{\E{(\xi\oplus\eta)}_b}; R)
        \ar[d, dash, "\cong"{below,rotate=90}]
        \\
        \H^*(\spherepair{\R^{\rkk}}; R)\otimes \H^*(\spherepair{\R^{\rkl}}; R)
        \ar[r, "\cong"]
        \ar[d, dash, "\cong"{above,rotate=90}]
        & \H^*(\spherepair{\R^{\rkk+\rkl}}; R)
        \ar[d, dash, "\cong"{below,rotate=90}]
        \\
        \H^*(I^{\rkk},\Boundary{I^{\rkk}}; R)\otimes \H^*(I^{\rkl}, \Boundary{I^{\rkl}}; R)
        \ar[r, "\cong"]
        \ar[d, dash, "\cong"{above,rotate=90}]
        & \H^*(I^{\rkk+\rkl}, \Boundary{I^{\rkk+\rkl}}; R)
        \ar[d, dash, "\cong"{below,rotate=90}]
        \\
        \H^*(\Sphere{\rkk}; R)\otimes \H^*(\Sphere{\rkl}; R)
        \ar[r, "\cong"]
        & \H^*(\Sphere{\rkk+\rkl}; R)
      \end{tikzcd}
    \end{center}
    Furthermore, the simple structure of the cohomology of spheres
    yields for the Künneth isomorphism in the desired degree
    \begin{align*}
      \H^{\rkk+\rkl}(\Sphere{\rkk+\rkl};R)
      &\cong
        \left(\H^*(\Sphere{\rkk}; R)\otimes \H^*(\Sphere{\rkl}; R)\right)_{\rkk+\rkl}\\
      &\coloneqq
        \bigoplus_{\mathclap{r+s=\rkk+\rkl}}
        \H^r(\Sphere{\rkk};R)\otimes \H^s(\Sphere{\rkl};R)
        =
        \H^{\rkk}(\Sphere{\rkk};R)\otimes \H^{\rkl}(\Sphere{\rkl};R)
    \end{align*}
    by leaving out zero-summands for the last equality.
    Thus, a generator of
    $\H^{\rkk}(\Sphere{\rkk}; R)\otimes \H^{\rkl}(\Sphere{\rkl}; R)$,
    which is the tensor product $\iota_{\rkk}\otimes\iota_{\rkl}$ of a generator
    in each factor,
    is mapped to a generator $\iota_{\rkk+\rkl}=\iota_{\rkk}\times\iota_{\rkl}$ of
    $\H^{\rkk+\rkl}(\Sphere{\rkl+\rkl}; R)$.
    Using the isomorphisms above proves the claim.
    \optcite[compare proof of Theorem~3.19, p.~221]{hatcher}
    % % Maybe put into separate Lemma:
    % Now one can use the fact, that for any two generators
    % $\iota_{\rkk}$ of $\H^{\rkk}(\spherepair{\R^{\rkk}})$ and
    % $\iota_{\rkl}$ of $\H^{\rkl}(\spherepair{\R^{\rkl}})$,
    % the cross-product
    % $\iota_{\rkk}\times\iota_{\rkl}$ is a generator of
    % $\H^{\rkk+\rkl}(\spherepair{\R^{\rkk+\rkl}})$.
  \end{proof}
\end{Cor}

\begin{Cor}
  Every vector bundle $\xi$ has a unique Thom class $\u{\xi}$ in
  $\Zmod2$-coefficients.
  Furthermore, for any map of paracompact spaces $f\colon A\to B$ and
  vector bundle $\xi\colon E\to B$ holds $\u{\pb f \xi} = \pb f \u{\xi}$.
  \begin{proof}[proof (sketch)]
    \begin{description}
    \item[Existence:] See \cite[Theorem~4D.10]{hatcher} or use
      \cite[Proposition~17.9.3]{tomdieck}.
    \item[Uniqueness:] % TODO: (GEORGE) Ref for Uniqueness of Thom class 
      Using a suitable Mayer-Vietoris sequence for gluing, and an
      inductive argument starting with the trivial bundle case, one can show:
      Any two classes in $\H^{\rkk}(\thomspacepair{E};R)$ whose
      restrictions coincide on all fibres will coincide.
      However, for $R=\Zmod2$ there is exactly one possible choice for
      a unit $\u{\xi}|_{E_b}\in
      \H^{\rkk}(\spherepair{E_b})^\times\cong\Zmod2^\times=\{1\}$
      over each point $b$.
    \item[Naturality:] Clear from uniqueness and the naturality of Thom classes.
    \end{description}
  \end{proof}
\end{Cor}

\begin{Rem}
  Using paracompactness of $B$ and
  \cite[Proposition~17.9.6]{tomdieck}, one concludes that
  $\u{\xi}\in \H^{\rkk}(\thomspacepair{E};R)$ has to be a unit.
\end{Rem}

Having a good notion of Thom classes by now, please recall the Thom
isomorphisms relating the cohomology of a vector bundle's total
space to that of its base space.
\begin{Thm}
  For any Thom class $\u{\xi}$ of $\xi$, and any degree $r$ there are
  the following isomorphisms, called the \emph{Thom isomorphisms},
  that are natural with respect to pullbacks of vector bundles over
  paracompact spaces:
  \begin{align*}
    \thomiso\colon
    \H^r(B;R) &\longrightarrow \H^{r+\rkk}(\thomspacepair{E}; R)
    & \thomiso\colon
      \H_{r+\rkk}(\thomspacepair{E}; R) &\longrightarrow \H^r(B;R)
    \\
    x &\longrightarrow \pb p (x) \cup \u{\xi}
    & \alpha &\longrightarrow \pf p (\u{\xi} \cap \alpha)
               \;.
  \end{align*}
  \begin{proof}
    Naturality directly follows from the naturality of the Thom class
    in \autoref{cor:thomclsnatural}, and naturality of the cup-
    respectively cap-product.
    The cohomology part then is a direct application of Leray's theorem
    \cite[Theorem~4D.8]{hatcher}.
    For the homology part see \forexample \cite[Theorem~14.6]{switzer}.
  \end{proof}
\end{Thm}

The relation amongst these isomorphisms will be crucial in
calculations for Wu's theorem in \autoref{sec:wuclassesmain}.
\begin{Lem}\label{lem:thomisoself-adjoint}
  If $B$ is connected, the Thom isomorphisms are adjoint in the
  following sense:
  For $r\in\Nat$, $x\in \H^r(B)$,
  and $\alpha\in \H_{r+\rkk}(\thomspacepair{E})$  holds 
  \begin{gather*}
    \capped{t(x)}{\alpha} = \capped{x}{t(\alpha)} \in\Zmod2
  \end{gather*}
\end{Lem}
In order to proof Lemma~\autoref{lem:thomisoself-adjoint},
first recall the following properties of the cap product.
\begin{Rem}
  For
  a map of triples of spaces
  $f\colon (Y,Y'',Y')\to (X,X'',X')$,
  cohomology classes
  $a\in \H^i(X,X')$ and $b\in \H^j(X,X')$,
  homology classes
  $\gamma\in \H_{i+j}(X, X'\cup X'')$
  and
  $\beta\in \H_j(Y, Y'\cup Y'')$,
  and a vector bundle $E\xrightarrow{p}B$
  holds
  \begin{align}
    \label{eq:capprod1}
    \capped{a\cup b}{\beta} &= \capped{b}{a\cap\beta}
                              \in \H_0(X,X'')\\
    \label{eq:capprod2}
    \capped{a}{\pf f \beta} &= \pf f \capped{\pb f a}{\beta}
                              \in \H_0(X,X'')
                            &&\text{\cite[Chap.~3.3.2, p.\,241]{hatcher}}\\
    \label{eq:capprod3}
    \pf p &= \Id \colon
    \Zmod2\cong \H_0(\thomspacepair{E})\to \H_0(B)\cong\Zmod2
  \end{align}
\end{Rem}
\begin{proof}[proof of Lemma~\autoref{lem:thomisoself-adjoint}]
  With $\U\coloneqq\u{\xi}$ one calculates
  \begin{align*}
    \capped{t(x)}{\alpha}
    &= \capped{\pb p x \cup \U}{\alpha} \\
    &\equalsby{\eqref{eq:capprod1}}
      \capped{\pb p x}{\U\cap\alpha} \\
    &\equalsby{\eqref{eq:capprod3}}
      \pf p\capped{\pb p x}{\U\cap\alpha} \\
    &\equalsby{\eqref{eq:capprod2}}
      \capped{x}{\pf p(u\cap\alpha)}
      = \capped{x}{\thomiso (\alpha)} \in\Zmod2
      \qedhere
  \end{align*}
\end{proof}

% TODO: Better motivation for relationship Thom isomorphism <-> fundamental cl
As this paper is rather interested in manifolds than general spaces,
have a look at the homology Thom isomorphism of a manifold, or more
precisely the fundamental class of a manifold. The following gives a
more geometrical notion of the latter with the help of normal bundles,
and---more importantly---will enable to work on the simpler
cohomology of spheres when needed.
\begin{Lem}\label{lem:thomisofundcl}
  Let $M$ be an $n$-dimensional compact manifold, and
  $\emb\colon M\to\R^{n+\rkk}\subset\Sphere{n+\rkk}$ be an embedding with
  normal bundle $\N{\emb}$ of rank $\rkk>0$.
  The normal bundle gives rise to an embedding
  $e\colon \E{\N{\emb}}\to\Sphere{n+\rkk}$ of its total space
  as a tubular neighbourhood $e(\E{\N{\emb}})$ of $i(M)$ into the
  $(n+\rkk)$-sphere.
  The quotient map
  \begin{gather*}
    \collapse\colon
    \Sphere{n+\rkk}
    \to \Sphere{n+\rkk}/\left( \Sphere{n+\rkk}\setminus e(\E{\N{\emb}}) \right)
    \cong \Discbdl{\N{\emb}} / \Spherebdl{\N{\emb}}
    \cong \Thomspace{\N{\emb}}
  \end{gather*}
  that collapses every point outside of $e(\E{\N{\emb}})$ to the infinity
  point fulfils
  \begin{center}
    \begin{tikzcd}[row sep=0pt]
      \H_{n+\rkk}(\Sphere{n+\rkk}) \ar[r, "\pf c"]
      & \H_{n+\rkk}(\Thomspace{\N{\emb}}) \ar [r, "\pf \incl"]
      & \relH_{n+\rkk}(\Thomspace{\N{\emb}})
      %\coloneqq \H_{n+\rkk}(\Thomspace{\N{\emb}}, \infty)
      \ar [r, "\pf t"{above}, "\cong"{below}]
      & \H_{n}(M)% \cong \Zmod2
      \\
      \fundcl{\Sphere{n+\rkk}} \ar[rrr, mapsto]
      &&&\thomiso(\pf \incl\pf\collapse\fundcl{\Sphere{n+\rkk}}) = \fundcl M
    \end{tikzcd}
  \end{center}
  where $\incl\colon(\Thomspace{\N{\emb}},\emptyset)\to(\Thomspace{\N{\emb}}, \{\infty\})$ is the
  canonical inclusion of pairs of spaces.
  This holds for any choice of embeddings $i$ and $e$.
  \begin{proof}
    First note that by the long exact sequence of the pair 
     $(\Thomspace{\N{\emb}}, \{\infty\})$ the map
     $\pf \incl\colon \H_r(\Thomspace{\N{\emb}})\to \relH(\Thomspace{\N{\emb}})$
    is an isomorphism in every degree $r>0$. The proof will be
    conducted in two steps, first proving the connected case, then
    the general one.
    
    \begin{description}
    \item[Connected case:]
      Assume that $M$ is connected.
      Then
      $\H_{n+k}(\Thomspace{\N{\emb}})
      \cong\relH_r(\Thomspace{\N{\emb}})
      \cong \H_n(M) \cong \Zmod2=\{\fundcl M, 0\}$ by
      the Thom isomorphism, and by connectedness of $M$.
      Thus, one only has to show that
      $\pf\collapse\fundcl{\Sphere{n+\rkk}}$
      is non-zero.    
      The trick now is to reduce once again to the homology of the
      sphere:
      Locally around any point $p\in\Thomspace{\N{\emb}}\setminus\{\infty\}$ the
      collapse map $\collapse$ is by definition a
      homeomorphism. Therefore, on homology there is the following 
      commutative diagram
      \begin{center}
        \begin{tikzcd}
          \fundcl{\Sphere{n+\rkk}}
          \ar[d,mapsto]\ar[r,phantom,"\in"{near start}]
          &\H_{n+\rkk}(\Sphere{n+\rkk})
          \ar[r, "\pf\incl\pf\collapse"]\ar[d,"\pf\incl"]
          &\relH_{n+\rkk}(\Thomspace{\N{\emb}})
          \ar[d,"\pf\incl"]\\
          \left.\fundcl{\Sphere{n+\rkk}}\right|_{\collapse^{-1}(p)}
          \ar[r,phantom,"\in"{near start}]
          &\H_{n+k}(\Sphere{n+\rkk},\Sphere{n+\rkk}\setminus\collapse^{-1}(p))
          \ar[r,"\pf\collapse"{above},"\cong"{below}]
          &\H_{n+\rkk}(\Thomspace{\N{\emb}}, \Thomspace{\N{\emb}}\setminus p)
        \end{tikzcd}
      \end{center}
      By definition of the fundamental class $\fundcl{\Sphere{n+\rkk}}$,
      the class $\fundcl{\Sphere{n+\rkk}}|_{\collapse^{-1}(p)}$ in the
      diagram is a generator, and thus also is
      $\pf\collapse\left(\fundcl{\Sphere{n+\rkk}}|_{\collapse^{-1}(p)}\right)
      = \left(\pf\incl\pf\collapse\fundcl{\Sphere{n+\rkk}}\right)|_p$.
      However, then $\pf\incl\pf\collapse\fundcl{\Sphere{n+\rkk}}\in
      \H_{n+\rkk}(\Thomspace{\N{\emb}})$
      cannot be zero as was to be shown.

    \item[General case] In case $M=\coprod_i M_i$ is the disjoint sum of its connected
      components $M_i$, $i\in I$ for some index set $I$, note the following:
      \begin{itemize}
      \item $\E{\N{\emb}} = \coprod_i\E{\N{\emb_i}}$,
        where $\emb_i\coloneqq\emb|_{M_i}$.
        % where $\N{M_i}\coloneqq\N{\emb}|_{M_i}$.
      \item Thus, $\Thomspace{\N{\emb}} = \bigvee_i\Thomspace{\N{\emb_i}}$ using the
        collapse maps
        \begin{gather*}
          \collapse_i\colon
          \Sphere{n+\rkk}
          \overset{\collapse}\longto
          \Sphere{n+\rkk}/\left(\Sphere{n+\rkk}\setminus e(\E{\N{\emb}})\right)
          \overset{\proj}\longto
          \Sphere{n+\rkk}/\left(\Sphere{n+\rkk}\setminus e(\E{\N{\emb_i}})\right)
        \end{gather*}
        for the disjoint parts,
        and $\collapse=\bigvee_i\collapse_i$.
      \item Thus, $\H_r(\Thomspace{\N{\emb}}) = \prod_i \H_r(\Thomspace{\N{\emb_i}})$ for
        all degrees $r$, $\fundcl M = (\fundcl{M_i})_i$, and
        $\pf\collapse = \prod_i\pf{\collapse_i}$.
        % \item Thus, $\H_{n+\rkk}(\Thomspace{\N{\emb}})
        %   \cong \H_{n+\rkk}(M)\cong\prod_i \Zmod2$, and $\fundcl{M} = (1)_i$.
      \end{itemize}
      With this one sees directly from the definition of the fundamental
      class of a manifold that
      $\pf\incl\pf\collapse\fundcl{\Sphere{n+\rkk}} = \fundcl M$
      if and only if for all connected component manifolds $M_i$ holds
      $\pf\incl\pf{\collapse_i}\fundcl{\Sphere{n+\rkk}} = \fundcl{M_i}$
      which is true by the first case.
      \qedhere
    \end{description}
  \end{proof}
\end{Lem}


\subsection{Definition and Construction of Stiefel-Whitney Classes}
\label{sec:swclasses}
This section is going to recall the defining axioms, as well as some
immediate properties of Stiefel-Whitney classes which are generators
for characteristic classes of vector bundles.
Furthermore, a handy construction of them is introduced based on the
results prepared in the previous subsections.

% TODO: section overview
\begin{Def}\label{def:swclasses}
  The \emph{Stiefel-Whitney classes} are
  characteristic classes for principal $\Orth$-bundles
  respectively vector bundles,
  \idest cohomology classes
  $\ws{i}\in \H^i(\BO;\Zmod2)$, $i\in\Z\geq0$,
  fulfilling the following properties for any vector bundles $\xi$ and
  $\eta$ over a space $B$, and any map $f\colon A\to B$ of spaces:
  \begin{axioms}
  \axiom[Naturality] $\pb f\w{i}{\xi} = \w{i}{\pb f \xi}$,
  \axiom $\w{0}{\xi}=1$,
  \axiom $\W{\gamma_1} = 1 + x$,
  \axiom[Multiplicativity]\label{tag:swclassesmultiplicativity}
  $\W{\xi \oplus \eta} = \W{\xi}\cup \W{\eta}$
    \\\idest in degree $n$ we have
    $\w{n}{\xi\oplus\eta} = \sum_{i+j=n}\w{i}{\xi} \cup \w{j}{\eta}$,
  \end{axioms}
  where the \emph{total Stiefel-Whitney class}
  $\Ws\coloneqq\sum_{i\geq 0}\ws{i}$ is the formal sum of all
  Stiefel-Whitney classes,
  $\gamma_1$ is the universal line bundle $\RPinf\cong\BO(1)\to\BO$,
  and $x$ is the%
  \footnote{
    This is well-defined: A ring $R$ of the form $\Zmod2[x]$
    with $\deg(x)=1$ only admits two elements in degree~1, $0$ and a
    generator. Therefore, there exists exactly one ring
    isomorphism, and this sends the unique generator in
    degree 1 to $x$.
  }
  generator of $\H^*(\RPinf;\Zmod2)\cong\Zmod2[x]$.
  \optcite[compare §4, p.~37]{milnor}
\end{Def}

\begin{Thm}
  Stiefel-Whitney classes exist and are uniquely defined by the above
  properties. Furthermore, they generate $\H^*(\BO;\Zmod2)\cong \Zmod2[\ws{i}|i\geq1]$.
  \begin{proof}[proof (sketch)]
    \begin{description}
    \item[Existence]
      A possible concrete construction utilises the Euler
      class. Another definition via Steenrod squares can be 
      found below in \autoref{thm:altdefswclasses}.
    \item[Uniqueness]
      Here one can use the splitting principle and the knowledge about
      $\W{\gamma_1}$
      \cite[Uniqueness Theorem~7.3]{milnor}.
    \item[Generators]
      Either inductively show
      $\H^*(\BO(n);\Zmod2)\cong \Zmod2[\ws{i}|n\geq i\geq1]$
      \cite[Theorem~7.1~ff.]{milnor}.
      Or find generators of $\H^*(\BO)$ that fulfil the axioms for
      the Stiefel-Whitney classes, then apply uniqueness
      \cite[Chap.~7.6]{may}.
    \end{description}
  \end{proof}
\end{Thm}

As already mentioned, the above generating property means that every
characteristic class of vector bundles of a fixed dimension can be
represented as a certain combination of Stiefel-Whitney classes.
Moreover, they behave extremely well concerning vector bundle
operations as the below immediate properties further emphasise.
\begin{Rem}
  \label{rem:propswclasses}
  Let $\xi$, $\eta$ be vector bundles over a space $X$.
  \begin{enumerate} 
  \item\label{item:propswclasses:dimesioncut} $\w{i}{\eta} = 0$
    for any vector bundle $\eta$ with $\rk\eta < i$.
    Therefore, the total Stiefel-Whitney class $\W{\xi}$ is
    well-defined (\idest the sum is finite)
    for any vector bundle $\xi$ of finite rank.
    \begin{proof}
      This follows directly from the splitting principle and
      the multiplicativity axiom.
    \end{proof}
  \item\label{item:swoftrivbdl} $\w{i}{\trivbdl}=0$ for $i>0$, and one immediately concludes
    from multiplicativity:
    \begin{enumerate}
    \item\label{item:swclassesstable}
      The Stiefel-Whitney classes are stable, \idest
      $\w{i}{\xi\oplus\trivbdl} = \w{i}{\xi}$, which once more proves
      the the stability property of characteristic classes of vector bundles.
      Thus, for a manifold $M^n$, all normal bundles $\N{\emb}$ of
      embeddings $\emb\colon M\to\R^{n+k+r}$ share the same
      Stiefel-Whitney classes. These are also the ones obtained by
      pullback along the classifying map of the stable normal bundle
      of $M$, written $\W{\N M}$ accordingly.
    \item\label{item:wuclassmfdinverse}
      If $\xi\oplus\eta = \trivbdl$, $\w{i}{\xi}\cup\w{i}{\eta}=1$.
      Especially, for any choice of embedding $\emb\colon M^n\to\R^{n+k}$
      with normal bundle $\N{\emb}$ of a smooth manifold $M$ we have
      $\T M\oplus\N{\emb} = \trivbdl$ and therefore
      $1 = \W{\T M} \cup \W{\N{\emb}} = \W{\T M}\cup\W{\N M}$.
    \end{enumerate}
    \begin{proof}
      The trivial rank $n$ bundle over $X$ is defined as the pullback
      $\pb \pi \trivbdl$ of the rank $n$ bundle
      $\trivbdl\colon \R^n\to\pt$ over the point by the trivial map
      $\pi\colon X\to\pt$. The naturality of the Stiefel-Whitney
      classes gives $\W{\trivbdl} = \pb\pi \W{\trivbdl}
      \in\pb\pi \left(\H^i(\pt;\Zmod2)\right)$,
      and the result follows from $\H^i(\pt;\Zmod2) = 0$ for $i>0$.
    \end{proof}
  \end{enumerate}
\end{Rem}

In order to algebraically work with the Stiefel-Whitney classes, the
formal inverse is often handy. Especially, since it is well-known for
manifolds as explained below.
\begin{Def}
  Define the \emph{dual Stiefel-Whitney (characteristic) classes}
  $\dualws{i}$ in degree $i$ inductively by
  \begin{align*}
    1 &= \dualws{0}\cup\ws{0} = \dualws{0}    &&\text{in degree 0}\\
    0 &= \sum_{i+j=n} \dualws{i}\cup\ws{j}  &&\text{in degree n>0}
  \end{align*}
  With the notation $\dualWs\coloneqq \sum_{i\geq0} \dualws{i}$ as above
  for the formal sum this can be reformulated as
  \begin{gather*}
    1 = \Ws\cup\dualWs
  \end{gather*}
  in the completion of the polynomial ring $\H^*(\BO)\cong\Zmod2[\ws{i}|i\in\Nat]$.
  For a manifold $M$ define
  $\W{M} \coloneqq \W{\T M}$ and
  $\dualW{M} \coloneqq \dualW{\T M} = \W{\N M}$
  where the last equality is the one from
  Remark~\itemref{rem:propswclasses}{item:swoftrivbdl}\ref{item:wuclassmfdinverse}
  above.
\end{Def}

Now, the promised construction of the Stiefel-Whitney classes can be presented.
\begin{Thm}\label{thm:altdefswclasses}
  The Stiefel-Whitney classes can be given as
  \begin{gather*}
    \Sq i(\u{\xi}) = \thomiso(\w{i}{\xi}) = \pb p \w{i}{\xi} \cup \u{\xi}
  \end{gather*}
  for any vector bundle $\xi\colon E\to B$ over a paracompact space
  $X$. As the Thom isomorphism is a group homomorphism, one can
  formulate the above as
  \begin{gather*}
    \SQ(\u{\xi}) = \thomiso(\W{\xi}) = \pb p \W{\xi} \cup \u{\xi}
  \end{gather*}
  The representing cohomology classes can be constructed using the
  embeddings $\BO(n)\subset\BO$ for $n\geq i$ and the fact that
  $\ws{i}=\pb\incl\w{i}{\gamma_n}\in \H^i(\BO)$
  \cite[see \forexample][Theorem~7.1~ff.]{milnor}.
  \begin{proof}
    Check naturality of the expression and all further
    defining properties from \autoref{def:swclasses}.
    \begin{description}
    \item[Naturality:] Both $\Sq i$ and $t$ respectively also $t^{-1}$
      are natural.
    \item[$\ws{0}=1$:]
      $\H^0(\thomspacepair{\E{\gamma_0}}) = \Zmod2$, thus 1 is the only
      candidate for a Thom class, $\Sq0(1) = \Id(1) = 1$, and the Thom
      isomorphism sends 1 to 1 in this degree.
    \item[$\W{\gamma_1}=1+x$:]
      $\Sq 0(\u{\gamma_1})\cequalsby{\eqref{eq:sqidentity}}\u{\gamma_1}
      = \pb p\w{0}{\gamma_1}\cup\u{\gamma_1}$
      directly gives
      $\pb p\w{0}{\gamma_1}=1$, and thus $\w{0}{\gamma_1}=1$.
      With
      $\Sq 1(\u{\gamma_1})\cequalsby{\eqref{eq:sqsquared}}\u{\gamma_1}^2
      =\pb p\w{1}{\gamma_1}\cup x$,
      the class $\pb p\w{1}{\gamma_1}$ cannot be zero, thus also
      $\w{1}{\gamma_1}$ cannot be zero, but it is the generator $x$.
    \item[Multiplicativity:]
      Consider vector bundles $\xi$, $\eta$ over a paracompact space
      $B$. With $\u{\xi\oplus\eta}=\u{\xi}\cup\u{\eta}$ and the fact
      \begin{gather}\label{eq:projectionscommute}
        p_\xi\circ\pi_\xi = p_{\xi\oplus\eta} = p_\eta\circ\pi_\eta
      \end{gather}
      get
      \begin{align*}
        \thomiso(\w{i}{\xi\oplus\eta})
        &= \Sq i(\u{\xi\oplus\eta}) \\
        &\equalsby{\autoref{cor:thomclassmultiplicative}}
          \Sq i(\pb\pi_\xi\u{\xi} \cup \pb\pi_\eta\u{\eta})\\
        &\equalsby{\ref{tag:cartan}}
          \sum_{r+s=i}
          \Sq r(\pb\pi_\xi\u{\xi}) \cup \Sq s(\pb \pi_\eta\u{\eta}) \\
        &\equalsby{Naturality}
          \sum_{r+s=i}
          \pb\pi_\xi\Sq r(\u{\xi}) \cup \pb \pi_\eta\Sq s(\u{\eta}) \\
        &\equalsby{Definition}
          \sum_{r+s=i}
          \pb\pi_\xi\thomiso(\w{r}{\xi})
          \cup \pb\pi_\eta\thomiso(\w{s}{\eta}) \\
        &\equalsby{Definition}
          \sum_{r+s=i}
          \pb\pi_\xi \left(\pb p_\xi  \w{r}{\xi}  \cup \u{\xi} \right)
          \cup
          \pb\pi_\eta\left(\pb p_\eta \w{s}{\eta} \cup \u{\eta}\right) \\
        &= \left(
          \sum_{r+s=i}
          \pb\pi_\xi \pb p_\xi \w{r}{\xi} \cup
          \pb\pi_\eta\pb p_\eta\w{s}{\eta}
          \right)
          \cup
          \left(\pb\pi_\xi\u{\xi} \cup \pb\pi_\eta\u{\eta}\right) \\
        &\equalsby{\eqref{eq:projectionscommute}, Definition}
          \left(\sum_{r+s=i}
          \pb p_{\xi\oplus\eta} \w{r}{\xi} \cup \pb p_{\xi\oplus\eta} \w{s}{\eta}
          \right)
          \cup
          \u{\xi} \times \u{\eta} \\
        &\equalsby{Group Hom., \autoref{cor:thomclassmultiplicative}}
          \pb p_{\xi\oplus\eta}
          \left(\sum_{r+s=i}\w{r}{\xi}\cup\w{s}{\eta}\right)
          \cup
          \u{\xi\oplus\eta}\\
        &\equalsby{Definition}
          \thomiso\left(\sum_{r+s=i}\w{r}{\xi}\cup\w{s}{\eta}\right)
          \;.
      \end{align*}
      Applying $\thomiso^{-1}$ yields the result.
      \qedhere
    \end{description}
  \end{proof}
\end{Thm}



\section{Reformulation of the Immersion Conjecture}
\label{sec:reformulation}
% TODO: section overview

The problem which this paper investigates can finally be clearly
stated with the definitions from
\autoref{sec:immersions}.
The goal of this section is to complete this chapter with a
reformulation of the immersion conjecture to a statement that can be
analysed with means of homotopy theory of vector bundles. This
then makes it possible to directly involve the Stiefel-Whitney classes
from \autoref{sec:swclasses}. The short elaborations on how
obstructions arise that way will be needed in the subsequent
chapter.

Before reformulating, recall the actual immersion conjecture.
\begin{Def}
  For $n\in\Nat$ consider the unique minimal binary expansion
  \begin{gather*}
    n=2^{i_1}+\dotsb+2^{i_{l_n}},
    \quad\text{with}\quad
    i_1<\dotsb<i_{l_n}
    \;.
  \end{gather*}
  Define $\alpha(n)\coloneqq l_n$, \idest $\alpha(n)$ is the number of
  ones in the binary notation of $n$.
\end{Def}
\begin{Thm}\label{thm:immersionconj}
  For $n\in\Nat$, every closed, smooth, $n$-dimensional manifold
  immerses into $\R^{2n-\alpha(n)}$.
\end{Thm}
In the style of this conjecture, an $n$-manifold that immerses into
some $\R^{2n-\alpha(n)}$ will be said to have the
\emph{immersion property}. % TODO: check, whether defs are needed
And the question, whether a particular manifold does have the
immersion property, will be referred to as the \emph{immersion problem} for
this manifold.

An essential step for the reformulation of the immersion conjecture is
a theorem of Hirsch and Smale, which gives a homotopy theoretical
relation between the spaces of immersions of two manifolds $\Imm M N$
and of the vector bundle monomorphisms $\Mono{\T M}{\T N}$ between
their tangent bundles. For the formulation, one has to equip the
respective sets with a topology as follows.
\begin{Def}
  Let $M$, $N$ be closed smooth manifolds of dimensions $\dim M<\dim N$.
  \begin{enumerate}
  \item
    Equip the set of all vector bundle monomorphisms from $\xi_1$ to
    $\xi_2$ with the compact-open topology (see \forexample
    \cite{hatcher}), and denote that space by $\Mono{\xi_1}{\xi_2}$.
    Note that a path between monomorphisms $F_1$ and $F_2$ in the
    space $\Mono \xi \eta$ is an homotopy from $F_1$ to $F_2$ which is
    a vector bundle monomorphism in each stage.
  \item
    % TODO: (GEORGE) check topology of \Imm(M, N); compare Lecture_Notes_on_Immersions_of_Surfaces_in_3-Space--Nowik.ps
    % TODO: Whitney $C^r$-topology; see [Hirsch, Differential Topology, Chap 2, p.35]
    The set $\Imm M N$ of all immersions from $M$ to $N$ injects
    into $\Mono{\T M}{\T N}$ by taking the differential
    $f\mapsto\Diff f$.
    Equip $\Imm M N$ in the following with the subspace topology.
    This results in the weak topology described in
    \cite[Section~2.1]{hirsch}, which equals the Whitney
    $C^1$-topology since $M$ was chosen compact.
    By the way, $\Imm M N$ is open in $C^1(M,N)$ equipped with the
    Whitney $C^1$-topology
    (see \cite[Section~2.1, Theorem~1.1]{hirsch}),
    and thus not a discrete space.
  \end{enumerate}
\end{Def}

Now one can state the following major result in immersion theory by
Hirsch using preliminary work of Smale
\cite[Sections~5 and 6]{hirschimmersions}. The following formulation
is according to
\cite[Theorem~1.2]{immersionconj}. % TODO: better ref modern formulation
\begin{Thm}[Hirsch-Smale]\label{thm:hirschsmale}
  Let $M$, $N$ be closed manifolds with $\dim M<\dim N$.
  Then the differential map
  $\Diff\colon \Imm M N\to \Mono{\T M}{\T N}$
  induces isomorphisms on the homotopy groups.
  Especially,
  \begin{gather*}
    \Diff_*\colon
    \pi_1(\Imm M N) \overset\sim\longto \pi_1(\Mono{\T M}{\T N})
  \end{gather*}
  describes an isomorphism of path-connected components.
  % Original formulation by Hirsch in [hirschimmersions], sec. 5:
  Therefore, every vector bundle monomorphism
  $F\colon\T M\to\T N$ is homotopic (through vector bundle
  monomorphisms) to a monomorphism which is the differential
  $\Diff f$ of a smooth map $f\colon M\to N$, \idest of an
  immersion.
\end{Thm}
Thus, any monomorphism of vector bundles over smooth, closed
manifolds $M$ and $N$ implies the existence of an immersion from $M$
to $N$.

This finally gives rise to the subsequent reformulation of the
immersion problem of a manifold in \autoref{thm:immersionconj}.
\begin{Thm}\label{thm:immersionconj:equivalences}
  Let $n,k\in\Nat$ and $M^n$ be a closed, smooth, $n$-dimensional manifold.
  The following statements are equivalent.
  \begin{enumerate}
  \item\label{item:immersionconj:1}
    $M$ immerses into $\R^{n+k}$.
  \item\label{item:immersionconj:2}
    There is a vector bundle monomorphism $F\colon\T M\to\T{\R^{n+k}}$.
  \item\label{item:immersionconj:3}
    There is a $k$-dimensional vector bundle
    $\N{}\colon\E{\N{}}\to M$ over $M$ with
    \begin{gather*}
      \N{}\oplus\T M\cong\trivbdl^{n+k}
      \;.
    \end{gather*}
  \item\label{item:immersionconj:4}
    For the map $\N M\colon M\to\B\Orth$ characterising the stable
    normal bundle over $M$ there is a lift $\N{}\colon M\to\B\Orth(k)$
    making the following diagram commute up to homotopy
    \begin{center}
      \begin{tikzcd}
        M
        \ar[r, "\N{}"]
        \ar[dr, "\N{M}"{left}, bend right]
        & \BO(k) \ar[d, "\incl", hookrightarrow] \\
        & \BO
      \end{tikzcd}
    \end{center}
  \end{enumerate}
\end{Thm}
From a homotopy theoretical viewpoint, one obviously is most
interested in statement \ref{item:immersionconj:4}.
One major advantage of this approach is that the bundle theoretic
context allows to apply the huge arsenal of characteristic classes of
vector bundles.
For example the value $k=n-\alpha(n)$ of the minimal immersion dimension
is motivated by means of Stiefel-Whitney classes in \autoref{chap:massey}.

\begin{proof}[proof of \autoref{thm:immersionconj:equivalences}]
  \begin{description}
  \item[\ref{item:immersionconj:2}$\Leftrightarrow$\ref{item:immersionconj:3}:]
    In order to get from a vector bundle monomorphism
    $F\colon M\to\R^{n+k}$ to a normal bundle, note that
    $\pb f\T{\R^{n+k}}$ for $f=F|_{\zerosec{\T M}}$ is 
    trivial, and that $F\colon \T M\to \pb f\T{\R^{n+k}}$ is not only
    fibre-wise, but globally injective of constant rank
    $\dim M$.
    Thus, as in the definition of normal bundles,
    $\N{}\coloneqq\pb f\T{\R^{n+k}}/\T M$ is a vector bundle over $M$
    that obviously fulfils the required property of
    \ref{item:immersionconj:3}.
    
    Conversely, when starting with some rank $k$ vector bundle $\N{}$
    such that $\N{}\oplus\T M\cong\trivbdl^{n+k}$, there is a vector
    bundle monomorphism $\N{}\to\trivbdl$ over $M$. Then the
    following chain of vector bundle morphisms
    \begin{center}
      \begin{tikzcd}
        \T M \ar[d]
        \ar[r, hookrightarrow]
        & M\times\R^{n+k} \ar[d,"\trivbdl^{n+k}"]
        \ar[r]
        & \pt\times \R^{n+k} \ar[d,"\trivbdl^{n+k}"]
        \ar[r, hookrightarrow]
        & \R^{n+k}\times \R^{n+k} \ar[d,"\trivbdl^{n+k}"]
        \\
        M
        \ar[r,equals]
        & M
        \ar[r]
        & \pt
        \ar[r, hookrightarrow]
        & \R^{n+k}
      \end{tikzcd}
    \end{center}
    is fibre-wise injective in each stage, and hence a monomorphism as
    was needed.
  \item[\ref{item:immersionconj:1}$\Leftrightarrow$\ref{item:immersionconj:2}:]
    The tricky part is to relate \ref{item:immersionconj:1} and
    \ref{item:immersionconj:2}, even though it is easily seen that
    \ref{item:immersionconj:1} implies \ref{item:immersionconj:2} by
    simply taking $F$ to be the differential $\Diff f$ of the
    immersion from \ref{item:immersionconj:1}.
    
    The converse direction is an application of the Hirsch-Smale
    theorem~\autoref{thm:hirschsmale}.
    However, first substitute the non-compact manifold $\R^{n+k}$ with
    the compact sphere $N=\Sphere{n+k}$, to make $M$ and $N$ comply
    with the preliminaries of the theorem. As $\dim M<n+k$ by assumption,
    every immersion $M\to\Sphere{n+k}$ misses a point on
    $\Sphere{n+k}$ and hence factors over an immersion $M\to\R^{n+k}$.
    This then shows that also \ref{item:immersionconj:2} implies
    \ref{item:immersionconj:1} which makes them equivalent.
  \item[\ref{item:immersionconj:3}$\Leftrightarrow$\ref{item:immersionconj:4}:]
    For showing
    \enquote{\ref{item:immersionconj:3}$\Rightarrow$\ref{item:immersionconj:4}}
    it is convenient to remember the equivalence
    \enquote{\ref{item:immersionconj:3}$\Leftrightarrow$\ref{item:immersionconj:1}},
    whereby the existence of a vector bundle $\N{}$ over $M$ with
    $\N{}\oplus\T M\cong\trivbdl$ implies the existence of an
    immersion $M\immto\R^{n+k}$.
    The classifying map of that immersion's normal bundle
    then lifts $\N{M}$ as required,
    using Steenrod's classification theorem
    \itemref{def:charcls}{item:classificationthm} and the properties
    of the stable normal bundle from
    \itemref{lem:classificationvb}{item:charclsstablenormalbundle}.
    
    On the other hand, also by
    \itemref{lem:classificationvb}{item:charclsstablenormalbundle},
    any rank $k$ vector bundle that is represented by a lift of
    $\N{M}$ to $[M,\B\Orth(k)]$ has the property needed for
    \ref{item:immersionconj:4}.
    \qedhere
  \end{description}
\end{proof}

This now gives rise to involve the powerful obstruction theory of
Stiefel-Whitney classes as follows.
\begin{Cor}\label{cor:obstruction}
  Let $n,k\in\Nat$, and $M^n$ be a smooth, closed manifold.
  If $M$ immerses into $\R^{n+k}$, then $\dualw{i}{M}=0$ for all
  $i>k$.
  \begin{proof}
    By Theorem~\autoref{thm:immersionconj:equivalences} $M$ immerses
    into $\R^{n+k}$ if and only if there is a rank-$k$
    normal bundle $\N{}$ of $M$.
    Since the Stiefel-Whitney classes are stable,
    $\W{\N{}}=\W{\N M}\cequalsby{Def.}\dualW{M}$.
    However, as explained in
    \itemref{rem:propswclasses}{item:propswclasses:dimesioncut},
    all Stiefel-Whitney classes $\w{i}{\N{}}$ of degree $i$ exceeding
    the rank $k$ of $\N{}$ are zero.
  \end{proof}
\end{Cor}
As a result, the immersion conjecture requires that all $n$-manifolds
have vanishing dual Stiefel-Whitney classes in degrees $i>n-\alpha(n)$.
That this is true, is a theorem of Massey which will be proven in
\autoref{chap:massey}. It was an inspiration to state the conjecture
with the value $k=n-\alpha(n)$ in the first place.


%%% Local Variables:
%%% mode: latex
%%% TeX-master: "thesis"
%%% End:
