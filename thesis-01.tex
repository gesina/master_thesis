%%%%%%%%%%%%%%%%%%%%%%%%%%%%%%%%%%
%  Master Thesis in Mathematics
% "Immersions and Stiefel-Whitney classes of Manifolds"
% -- Chapter 2: Preliminaries --
% 
% Author: Gesina Schwalbe
% Supervisor: Georgios Raptis
% University of Regensburg 2018
%%%%%%%%%%%%%%%%%%%%%%%%%%%%%%%%%%

\chapter{Preliminaries}
\section{Definitions}
\subsection{Steenrod Squares}
\begin{Def}[Cohomology Operations] % TODO
\end{Def}
\begin{Def}[Steenrod Squares] % TODO
\end{Def}


\subsection{Stiefel-Whitney Classes}
\begin{Def}[Universal Bundle] % TODO
  Example: Universal line bundle $\gamma_1\colon \RP\cong\BO(1) \overset\incl\to \BO$
\end{Def}

\begin{Def}[Classifying Spaces] % TODO
  \begin{itemize}
  \item Any topological group $G$ admits a contractible space $\EG$ with a
    free $G$-action, and a corresponding principal $G$-bundle
    $\gamma^G\colon \EG\to\BG\coloneqq \EG/G$ called the
    \emph{universal $G$-bundle}
    all unique up to homotopy.
    $\BG$ is called the \emph{classifying space} for principal
    $G$-bundles.
  \item $\gamma^G$ fulfills the following universal property:
    For any space $X$ admitting the homotopy type of a CW-complex
    there is a bijection beetween $[X,\BG]$ and the isomorphism classes of
    principal $G$-bundles over $X$, given by
    \begin{gather*}
      \left(f\colon X\to\BG \right) \longmapsto \pb f \gamma^G
      \;
      \text{\optcite[Theorem 1.4]{immersionconj}}.
    \end{gather*}
    This correspondence is natural in $X$.
  \item There is a natural equivalence between the category of
    $n$-dimensional vector bundles $\Vect$ and that of principal
    $\Orth(n)$- respectively $\GL n$-bundles.
  \end{itemize}
\end{Def}

\begin{Def}[Characteristic Class]
  A characteristic class
  \begin{itemize}
  \item of degree $i$
  \item with coefficients in a ring $R$
  \item for principal $G$-bundles for a group $G$
  \end{itemize}
  is a natural transformation
  \begin{gather*}
    \Cl\colon [-, \BG] \Longrightarrow H^i(-; R)\;.
  \end{gather*}
\end{Def}
  
\begin{Rem}
  By Brown's representation theorem % TODO: cite
  $H^i(-;R)$ is a represantable functor represented by $K(i,R)$.
  Thus, by the Yoneda lemma, a characteristic class is
  represented by a morphism
  \begin{gather*}
    \cl\colon \BG \longto K(i, R)
  \end{gather*}
  in $\Top$, i.e. by a cohomology class $\cl$ of $\BG$.
  Thus, application of $\Cl$ to a principal $G$-bundle over a
  space $X$ which is represented by a morphism $\eta\colon X\to\BG$ % TODO: cite
  is
  \begin{gather*}
    \Cl(X) = \pb{\eta} \cl \in H^i(X;R)
    \;.
  \end{gather*}
  This describes a one-to-one correspondence between
  characteristic classes as above and cohomology classes in
  $H^i(BG;R)$, and in the following any characteristic class will be
  identified with its corresponding cohomology class.
\end{Rem}

\begin{Rem}
  % TODO: Why are char classes nice idea?
\end{Rem}

\begin{Def}[Stiefel-Whitney Classes]
  The Stiefel-Whitney classes are cohomology classes
  $w_i\in H^i(\BO;\Zmod2)$, $i\in\Z\geq0$,
  \idest characteristic classes for principal $\Orth$-bundles
  respectively vector bundles, fulfilling
  for any vector bundles $\xi$ and $\eta$ over a space $X$
  \begin{enumerate}
  \item $\w_0%(\xi)
    = 1$,
  \item $\w(\gamma_1) = 1 + x$,
    % \item (naturality) $\w(\pb f \xi) = \pf \w(\xi)$
  \item (additivity) $\w(\xi \oplus \eta) = \w(\xi)\cup \w(\eta)$
    \\\idest in degree $n$ we have
    $\w_n(\xi\oplus\eta) = \sum_{i+j=n}\w_i(\xi) \cup \w_j(\eta)$,
  \end{enumerate}
  where
  $\w\coloneqq\sum_{i\geq 0}\w_i$ is the formal sum of all
  Stiefel-Whitney classes,
  $\gamma_1$ is the universal line bundle $\RP\cong\BO(1)\to\BO$,
  and $x$ is the%
  \footnote{
    % TODO: maybe shorten reasoning about x well-defined
    This is well-defined, as a ring $R$ of the form $\Zmod2[x]$
    with $x$ of degree 1 only admits two elements, $0$ and a
    generator, in degree 1. Therefore there exists exactly one ring
    isomorphism, and that is defined by sending the unique generator in
    degree 1 to $x$.
  }
  generator of $H^*(\RP;\Zmod2)\cong\Zmod2[x]$.
\end{Def}

\begin{Thm}[Existence and Uniqueness of Stiefel-Whitney Classes] % TODO: proof
Stiefel-Whitney classes exist and are uniquely defined by the above
properties. Furthermore, they generate $H^*(\BO;\Zmod2)\cong \Zmod2[\w_i|i\geq1]$.
\begin{proof}[proof (existence)]
  Construction utilizes the Euler class.
\end{proof}
\begin{proof}[proof (uniqueness)]
  Use the splitting theorem and the knowledge about $\w(\gamma_1)$.
\end{proof}
\begin{proof}[proof (generators)]
  Inductively show $H^*(\BO(n);\Zmod2)\cong \Zmod2[\w_i|n\geq i\geq1]$
\end{proof}
\end{Thm}

\begin{Rem}[Further Properties of the Stiefel-Whitney Classes] % TODO
  \label{propswclasses}
  Let $\xi$, $\eta$ be vector bundles over a space $X$.
  \begin{enumerate} 
  \item $\w_i(\eta) = 0$
    for any vector bundle $\eta$ with $\rk\eta < i$.
    This follows directly from the construction for the existence of
    the Stiefel-Whitney classes.
    Therefore, the total Stiefel-Whitney class $\w(\xi)$ is
    well-defined (\idest the sum is finite)
    for any vector bundle $\xi$ of finite rank.
  \item $\w_i(\trivbdl)=0$ for $i>0$, and one immediately concludes:
    \begin{enumerate}
    \item The Stiefel-Whitney classes are stable, \idest
      $\w_i(\xi\oplus\trivbdl) = \w_i(\xi)$.
    \item If $\xi\oplus\eta = \trivbdl$, $\w_i(\xi)\cup\w_i(\eta)=1$.
      Especially, for any choice of embedding with normal bundle $\N M$
      of a manifold $M$ we have $\T M \oplus \N M = \trivbdl$ and
      therefore $1 = \w(\T M) \cup \w(\N M)$.
    \end{enumerate}
    \begin{proof} % TODO: make proof of w(\trivbdl)=0 nicer
      This follows from the naturality of the Stiefel-Whitney classes
      and the definition of the trivial rank $n$ bundle:
      $\w(\trivbdl) = \pb\pi \w(\trivbdl)$ where $\pi\colon X\to\pt$
      and $H^i(\pt;\Zmod2) = 0$ for $i>0$.
    \end{proof}
  \end{enumerate}
\end{Rem}

\begin{Def}
  For any space $X$ define the \emph{dual Stiefel-Whitney classes}
  $\dualw_i$ in degree $i$ of $X$ inductively by
  \begin{align*}
    1 &= \dualw_0\cup\w_0 = \dualw_0    &&\text{in degree 0}\\
    0 &= \sum_{i+j=n} \dualw_i\cup\w_j  &&\text{in degree n>0}
  \end{align*}
  With the notation $\dualw\coloneqq \sum_{i\geq0} \dualw_i$ as above
  for the formal sum this can be reformulated as
  \begin{gather*}
    1 = \w\cup\dualw
  \end{gather*}
  For a manifold $M$ define
  $\w(M) \coloneqq \w(\T M)$ and
  $\dualw(M) \coloneqq \dualw(\T M) = \w(\N M)$
  where the last equality is the one from
  Proposition~\ref{propswclasses} above.
\end{Def}


%%% Local Variables:
%%% mode: latex
%%% TeX-master: "thesis"
%%% End:
