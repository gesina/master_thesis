%%%%%%%%%%%%%%%%%%%%%%%%%%%%%%%%%% 
% Master Thesis in Mathematics
% "Immersions and Stiefel-Whitney classes of Manifolds"
% -- Chapter 1: Preliminaries --
% 
% Author: Gesina Schwalbe
% Supervisor: Georgios Raptis
% University of Regensburg 2018
%%%%%%%%%%%%%%%%%%%%%%%%%%%%%%%%%% 

\chapter{Preliminaries}
\section{Steenrod Squares}

The following definition of the Steenrod squares is due to \cite[Chap.~I.1, p.~1]{steenrodepstein}.
\begin{Def}[Steenrod Squares]\label{def:sq}
  The Steenrod Squares $\Sq i$ for $i\in\Nat$ are each a family of
  cohomology operations, \idest families of homomorphisms, of the form
  \begin{gather*}
    \left(
      \Sq i\colon \H^n(X, A;\Zmod2) \to \H^{n+i}(X, A;\Zmod2)
      \;\middle|\;
      n\in\Nat
    \right)
  \end{gather*}
  that satisfy the following relations for any pair of spaces $(X,A)$, and any map of
  pairs of spaces $f\colon (X,A)\to (Y,B)$:
  \begin{description}
  \item[(Naturality)]\label{item:sqnaturality} $\pb f\circ\Sq i = \Sq i\circ\pb f$
  \item[(Stability)]\label{item:sqstability} $\susp\circ\Sq i = \Sq i\circ\susp$
  \item[(Cartan formula)] For any $n\in\Nat$, and $x,y\in \H^n(X)$ holds
    \begin{gather}\label{tag:cartan}\tag{Cartan's formula}
      \Sq i(x\cup y) = \sum_{r+s=i}\Sq r(x)\cup\Sq s(y)
    \end{gather}
  \item[(Fixed values)] The following values are fixed for $x\in \H^n(X,A)$:
    % \begin{gather}\label{eq:sqlowerbound}
    %   \Sq i(x) = \begin{cases}
    %     0 & n<i \\
    %     x^2 & n=i
    %   \end{cases}
    %   \qquad\text{and}\qquad
    %   \Sq i = \begin{cases}
    %     \Id & i=0\\
    %     \beta & i=1
    %   \end{cases}
    % \end{gather}
    \begin{alignat}{4}
      \Sq i(x) &= 0     & \qquad\text{for }n<i \label{eq:sqlowerbound}\\
      \Sq i(x) &= x^2   & \qquad\text{for }n=i \label{eq:sqsquared}\\
      \Sq 0    &= \Id   \label{eq:sqidentity}
      % \\\notag
      % \Sq 1    &= \beta \label{eq:sqbockstein}
    \end{alignat}
    % where $\beta$ denotes the Bockstein homomorphism
    % \cite[see \forexample][Chap.~3.E]{hatcher} of the exact
    % coefficient sequence
    % \begin{center}
    %   \begin{tikzcd}
    %     0 \ar[r]
    %     &\Zmod2 \ar[r,"\incl"]
    %     &\Zmod4 \ar[r,"\proj"]
    %     &\Zmod2 \ar[r]
    %     &0
    %   \end{tikzcd}
    % \end{center}
  \item[(Adem relations)] For $\alpha<2\beta$ holds
    \begin{gather}\label{tag:adem}\tag{Adem's formula}
      \Sq\alpha \circ \Sq\beta =
      \sum_{j=0}^{\left\lfloor \frac \alpha 2 \right\rfloor}
      \binom{\beta-j-1}{\alpha-2j}
      \Sq{\alpha+\beta+j}\Sq{j}
    \end{gather}
  \end{description}
  $\SQ\coloneqq \sum_{j\in\Nat}\Sq j$ is the formal sum of all
  Steenrod squares called the \emph{the total Steenrod square}.
  Note that for any degree $n\in\Nat$ the total Steenrod square
  $\SQ\colon \H^n(X)\to \H^*(X)$ is well-defined since the sum is
  finite by \eqref{eq:sqlowerbound}.
  Also \ref{tag:cartan} can be reformulated to
  $\SQ(x\cup y) = \SQ(x)\cup\SQ(y)$, \idest $\SQ$ is a group
  homomorphism with respect to the cup-product.
\end{Def}

\begin{Thm}
  The Steenrod squares exist and are uniquely determined by
  naturality, \ref{tag:cartan}, and the fixed values
  %\eqref{eq:sqlowerbound}, \eqref{eq:sqsquared}, and
  %\eqref{eq:sqidentity}
  from Definition~\autoref{def:sq}.
  \begin{proof}
    For existence see \cite[Chapter 2]{mosher},
    for uniqueness see \cite[VIII §3]{steenrodepstein}.
  \end{proof}
\end{Thm}

\begin{Def}[Steenrod Algebra]
  The Steenrod algebra $\A$ is the quotient
  of the graded $\Zmod2$-polynomial algebra
  $\Zmod2[\Sq i|i\in\Nat]$ with grading $\deg \Sq i\coloneqq i$
  by the two-sided relations of both \ref{tag:adem} and $\Sq 0=1$.
  With the induced grading it is an associative, connected,
  non-commutative graded Hopf algebra over $\Zmod2$
  \cite[Chap.~6]{mosher}.
\end{Def}
\begin{Not}
  In the following, iterated Steenrod squares
  $\Sq{i_1}\cdot\Sq{i_2}\dotsm\Sq{i_l}$ will have the short form
  $\Sq{(i_1,i_2,\dotsc,i_l)}$,
  and will be evaluated on an element $x$ of a cohomology ring as
  $\Sq{i_1}\circ\dotsb\circ\Sq{i_l}(x)$ respecting the
  properties from \autoref{def:sq}.
  Furthermore, for a sequence $I=(i_1,\dotsc,i_l)$,
  respectively $\Sq I$, denote by
  \begin{description}[labelindent=1em]
  \item[$\l(I)\coloneqq l$] the \emph{length} of $I$,
  \item[$\d(I)\coloneqq \sum_{j=0}^{l} i_j$] the \emph{degree} of $I$,
    and by
  \item[$\e(I)\coloneqq 2i_1-\d(I)=\sum_{j=1}^{l-1}(i_j-2i_{j+1})$]
    the \emph{excess} of $I$.
  \end{description}
  The sequence $I$, respectively $\Sq I$, is called \emph{admissible},
  if $\l(I)=1$ or $i_j\geq 2i_{j+1}$ for $0\leq j<\l(I)$.
\end{Not}

\begin{Rem}\label{rem:sq}
  The following properties will be needed for Massey's Theorem:
  \begin{enumerate}
  \item The set of iterated Steenrod squares $\Sq I$ of admissible
    sequences $I$ forms a basis for $A$ as $\Zmod2$-vector space
    \cite[Chap.~6, Theorem~1]{mosher}.
  \end{enumerate}
  Let $I=(i_1,\dotsc,i_l)$ be a sequence in $\Nat$.
  \begin{enumerate}[resume*]
  \item $\deg(\Sq I(x)) = \deg(x) + \d(\Sq I)$.
  \item\label{item:squpperboundgeneral} $\Sq I(x) = 0$ for  $\deg(x)<\e(I)$ if $I$ is admissible.
    \begin{proof}
      This follows by induction over $\l(I)$ using
      \eqref{eq:sqlowerbound}. The case $\l(I)=1$ follows directly 
      from \eqref{eq:sqlowerbound}.
      For $\l(I)>1$ and $J\coloneqq(i_2,\dotsc,i_l)$, the condition
      $\deg(x) < \e(I)=i_1-\d(J)$
      implies
      \begin{gather*}
        \deg(\Sq J(x))
        = \deg(x)+\d(J) < \e(J)+\d(J) = 2i_2
        \overset{\text{adm.}}\leq i_1
      \end{gather*}
      So,
      $\Sq I(x)=\Sq{i_1}(\Sq J(x)) \cequalsby{\eqref{eq:sqlowerbound}} 0$.
    \end{proof}
  \end{enumerate}
\end{Rem}

\begin{Def}\label{def:antipode}
  The antipode $\antipode\colon \A\to\A$ of the Steenrod algebra is a
  graded homomorphism inductively defined by the relation
  \begin{gather*}
    1 = \Sq 0
    = \SQ \Sqcup \antipode(\SQ)
    = \sum_{k\geq0}\sum_{r+s=k} \Sq r \Sqcup \antipode(\Sq s)
  \end{gather*}
\end{Def}

\section{Characteristic Classes}
\begin{Def}[Classifying Spaces]\label{def:charcls}
  \optcite[Chapter~14.4]{tomdieck}
  \begin{enumerate}
  \item Any topological group $G$ admits a contractible space $\EG$ with a
    free $G$-action, and a corresponding principal $G$-bundle
    $\gamma^G\colon \EG\to\BG\coloneqq \EG/G$, called the
    \emph{universal $G$-bundle},
    where $\gamma_G$, $\EG$, and $\BG$ are all unique up to
    homotopy.
    $\BG$ is called the \emph{classifying space} for principal
    $G$-bundles.
    For construction and uniqueness see \cite[Example~1B.7~ff.]{hatcher},
    respectively note that universal coverings are unique up to homotopy.
  \item\label{item:classificationthm}
    $\gamma^G$ fulfills the following universal property:
    For any space $X$ admitting the homotopy type of a CW-complex
    there is a bijection beetween $[X,\BG]$ and the isomorphism classes of
    principal $G$-bundles over $X$, given by
    \begin{gather*}
      \left(f\colon X\to\BG \right) \longmapsto \pb f \gamma^G
      \;
      \text{\optcite[Theorem 1.4]{immersionconj}}.
    \end{gather*}
    This correspondence is natural in $X$, and is a version of
    Steenrod's classification theorem
    \cite[Theorem~14.4.1]{tomdieck}.
    \optcite[Theorem~1.4, p.~75]{immersionconj}
  \item There is a natural equivalence between the category of
    $n$-dimensional vector bundles $\Vect$ and that of principal
    $\Orth(n)$- respectively $\GL n$-bundles.
  \end{enumerate}
\end{Def}

\begin{Def}[Characteristic Class]
  A characteristic class
  \begin{itemize}
  \item of degree $i$
  \item with coefficients in a ring $R$
  \item for principal $G$-bundles for a group $G$
  \end{itemize}
  is a natural transformation
  \begin{gather*}
    \Cl\colon [-, \BG] \Longrightarrow \H^i(-; R)\;.
  \end{gather*}
\end{Def}

\begin{Rem}
  By Brown's representation theorem \optcite[Chap.~4.E]{hatcher}
  $\H^i(-;R)$ is a representable functor represented by $K(i,R)$.
  Thus, by the Yoneda lemma, a characteristic class is
  represented by a morphism
  \begin{gather*}
    \cl\colon \BG \longto K(i, R)
  \end{gather*}
  in $\Top$, \idest by a cohomology class $\cl$ of $\BG$.
  Thus, applying $\Cl$ to a principal $G$-bundle over a
  space $X$, which admits the homotopy type of a CW-complex, that is
  represented by a morphism $\eta\colon X\to\BG$
  as in \itemref{def:charcls}{item:classificationthm}
  yields
  \begin{gather*}
    \Cl(X) = \pb{\eta} \cl \in \H^i(X;R)
    \;.
  \end{gather*}
  This describes a one-to-one correspondence between
  characteristic classes as above and cohomology classes in
  $\H^i(BG;R)$, and in the following any characteristic class will be
  identified with its corresponding cohomology class.
\end{Rem}

\begin{Rem}
  % TODO: Why are char classes nice idea?
\end{Rem}

\subsection{Thom Classes}
Let $B$ be a paracompact space, \forexample a manifold,
$\xi\colon E\xrightarrow{p} B$ a vector bundle over $B$ of rank $\rkk>0$,
and $R$ be principal ideal domain.
\begin{Def}
  A \emph{Thom class} of $\xi$ in $R$-coefficients is a
  cohomology class $\u{\xi}\in \H^{\rkk}(\spherepair{E}; R)$,
  such that for all points $b\in B$ and fiber inclusions
  $i_b\colon (\spherepair{E_b}) \to (\thomspacepair{E})$
  the restriction $\u{\xi}|_{E_b} = \pb i_b (\u{\xi})$ is a
  free generator of the $R$-module $\H^{\rkk}(\spherepair{E_b}; R)$,
  \idest a unit
  of the ring
  $\H^{\rkk}(\spherepair{E_b};R)\cong \H^{\rkk}(\spherepair{\R}; R)\cong R$.
  \optcite[p.~441]{hatcher}
\end{Def}

The following corollaries will quite directly deduce notions of
naturality, multiplicativity, and uniqueness for Thom classes from
their above definition.

\begin{Cor}\label{cor:thomclsnatural}
  The Thom class construction is natural with respect to the pullback
  of vector bundles over paracompact spaces.
  \Idest given any map of paracompact spaces $f\colon A\to B$, and a
  vector bundle $\xi\colon E\to B$,
  the pullback $\pb f \U$ of a Thom class $\U$ of $\xi$ will
  be a Thom class of $\pb f \xi$.
  \begin{proof}
    Let $\U$ be a Thom class of $\xi$ and $a\in A$ any point.
    Consider the restriction $\pb i_a(\pb f \U)$
    of the pullback of $\U$ to the fiber over $a$. To show that this
    is a generator of $\H^{\rkk}(\spherepair{E_a};R)$ first use that
    pullbacks commute with restriction:
    \begin{gather*}
      \pb i_a(\pb f \U)
      = \pb {(f\circ i_a)} \U
      = \pb {(i_{f(a)}\circ f)} \U
      = \pb f (\pb i_{f(a)} \U)
    \end{gather*}
    $\pb i_{f(a)} \U$ is a generator by definition of $\U$.
    Now the restriction of $f$
    \begin{gather*}
      f\colon (\spherepair{(\pb f E)_a}) \to (\spherepair{E_{f(a)}})
    \end{gather*}
    is an isomorphism, and thus
    $\pb f\colon \H^r(\spherepair{E_{f(a)}})
    \cong \H^r(\spherepair{(\pb f E)_a})$
    sends generators to generators for all $r\in\Nat$.
  \end{proof}
\end{Cor}

\begin{Rem}
  Let $\xi$, $\eta$ be vector bundles over a space $B$.
  \begin{itemize}
  \item There is a canonical isomorphism
    $\E{(\xi\oplus\eta)}\cong\E\xi\oplus\E\eta$
    with the canonical projections
    $\pi_\xi\colon \E{(\xi\oplus\eta)}\to \E\xi$
    and
    $\pi_\eta\colon\E{(\xi\oplus\eta)}\to \E\eta$.
  \item There is a corresponding Künneth isomorphism
    defining the \emph{cross-product} \optcite[p.~214]{hatcher}
    \begin{align*}
      \H^*(\thomspacepair{\E\xi})
      \otimes
      \H^*(\thomspacepair{\E\eta})
      &\longisoto
        \H^*(\thomspacepair{\E{(\xi\oplus\eta)}})\\
      x\otimes y
      &\longmapsto
        \pb \pi_\xi x \cup \pb \pi_\eta y
        \eqqcolon x\times y
        \;.
    \end{align*}
    \cite[Theorem~3.18]{hatcher}
  \end{itemize}
\end{Rem}
\begin{Cor}\label{cor:thomclassmultiplicative}
  The Thom class construction for coefficients in a field $R$ is
  multiplicative in the following sense:
  For vector bundles $\xi\colon E\to B$, $\eta\colon E'\to B$
  of rank $\rkk$ respectively $\rkl$ over a paracompact space $B$, and Thom
  classes
  $\u{\xi}\in \H^{\rkk}(\thomspacepair{\E\xi}; R)$,
  $\u{\eta}\in \H^{\rkl}(\thomspacepair{\E\eta}; R)$
  the class
  \begin{gather*}
    \u{\xi}\times\u{\eta}
    \coloneqq \pb \pi_\xi\u{\xi} \cup \pb \pi_\eta\u{\eta}
    \in \H^{\rkk+\rkl}(\thomspacepair{\E{(\xi\oplus\eta)}})
  \end{gather*}
  is a Thom class of $\xi\oplus\eta$.
  \begin{proof}
    Consider a fiber $b\in B$. As cup product and pullback commute
    with restriction, the cross-product also commutes with
    restriction, \idest one has to show that
    \begin{gather*}
      \pb i_a(\u{\xi}\times\u{\eta})
      = \left(\pb i_a\u{\xi}\right)
      \times \left(\pb i_a\u{\eta}\right)
    \end{gather*}
    is a generator of
    $\H^{\rkk+\rkl}(\thomspacepair{\E{(\xi\oplus\eta)}};R)$.
    % There are homotopies making the following diagram commute
    % \begin{center}
    %   \begin{tikzcd}
    %     (\spherepair{\E{(\xi)}_b})
    %     \ar[d, dash, "\isosymb"{above,rotate=90}]
    %     &(\spherepair{\E{(\xi\oplus\eta)}_b})
    %     \ar[l,"\pi_\xi"above] \ar[r,"\pi_\eta"]
    %     \ar[d, dash, "\isosymb"{above,rotate=90}]
    %     &(\spherepair{\E{(\eta)}_b})
    %     \ar[d, dash, "\isosymb"{above,rotate=90}]\\
    %     (\spherepair{\R^i})
    %     &(\spherepair{\R^{i+j}})
    %     \ar[l,"\proj"above] \ar[r,"\proj"]
    %     &(\spherepair{\R^{j}})
    %   \end{tikzcd}
    % \end{center}
    By the naturality of the Künneth isomorphism there is the
    following commutative diagram that translates this problem to one
    on the cohomology of spheres:
    \begin{center}
      \begin{tikzcd}
        \H^*(\spherepair{\E\xi_b};R)
        \otimes \H^*(\spherepair{\E\eta_b};R)
        \ar[r, "\cong"]
        \ar[d, dash, "\cong"{above,rotate=90}]
        & \H^*(\spherepair{\E{(\xi\oplus\eta)}_b}; R)
        \ar[d, dash, "\cong"{below,rotate=90}]
        \\
        \H^*(\spherepair{\R^{\rkk}}; R)\otimes \H^*(\spherepair{\R^{\rkl}}; R)
        \ar[r, "\cong"]
        \ar[d, dash, "\cong"{above,rotate=90}]
        & \H^*(\spherepair{\R^{\rkk+\rkl}}; R)
        \ar[d, dash, "\cong"{below,rotate=90}]
        \\
        \H^*(I^{\rkk},\Boundary{I^{\rkk}}; R)\otimes \H^*(I^{\rkl}, \Boundary{I^{\rkl}}; R)
        \ar[r, "\cong"]
        \ar[d, dash, "\cong"{above,rotate=90}]
        & \H^*(I^{\rkk+\rkl}, \Boundary{I^{\rkk+\rkl}}; R)
        \ar[d, dash, "\cong"{below,rotate=90}]
        \\
        \H^*(\Sphere{\rkk}; R)\otimes \H^*(\Sphere{\rkl}; R)
        \ar[r, "\cong"]
        & \H^*(\Sphere{\rkk+\rkl}; R)
      \end{tikzcd}
    \end{center}
    Furthermore, the simple structure of the cohomology of spheres
    yields for the Künneth isomorphism in the desired degree
    \begin{align*}
      \H^{\rkk+\rkl}(\Sphere{\rkk+\rkl};R)
      &\cong
        \left(\H^*(\Sphere{\rkk}; R)\otimes \H^*(\Sphere{\rkl}; R)\right)_{\rkk+\rkl}\\
      &\coloneqq
        \bigoplus_{\mathclap{r+s=\rkk+\rkl}}
        \H^r(\Sphere{\rkk};R)\otimes \H^s(\Sphere{\rkl};R)
        =
        \H^{\rkk}(\Sphere{\rkk};R)\otimes \H^{\rkl}(\Sphere{\rkl};R)
    \end{align*}
    by leaving out zero-summands for the last equality.
    Thus, a generator of
    $\H^{\rkk}(\Sphere{\rkk}; R)\otimes \H^{\rkl}(\Sphere{\rkl}; R)$,
    which is the tensor product $\iota_{\rkk}\otimes\iota_{\rkl}$ of a generator
    in each factor,
    is mapped to a generator $\iota_{\rkk+\rkl}=\iota_{\rkk}\times\iota_{\rkl}$ of
    $\H^{\rkk+\rkl}(\Sphere{\rkl+\rkl}; R)$.
    Using the isomorphisms above proves the claim.
    \optcite[compare proof of Thm~3.19, p.~221]{hatcher}
    % % Maybe put into separate Lemma:
    % Now one can use the fact, that for any two generators
    % $\iota_{\rkk}$ of $\H^{\rkk}(\spherepair{\R^{\rkk}})$ and
    % $\iota_{\rkl}$ of $\H^{\rkl}(\spherepair{\R^{\rkl}})$,
    % the cross-product
    % $\iota_{\rkk}\times\iota_{\rkl}$ is a generator of
    % $\H^{\rkk+\rkl}(\spherepair{\R^{\rkk+\rkl}})$.
  \end{proof}
\end{Cor}

\begin{Cor}
  Every vector bundle $\xi$ has a unique Thom class $\u{\xi}$ in
  $\Zmod2$-coefficients.
  Furthermore, for any map of paracompact spaces $f\colon A\to B$ and
  vector bundle $\xi\colon E\to B$ holds $\u{\pb f \xi} = \pb f \u{\xi}$.
  \begin{proof}[proof (sketch)]
    \begin{description}
    \item[Existence:] See \cite[Theorem~4D.10]{hatcher} or use
      \cite[Proposition~17.9.3]{tomdieck}.
    \item[Uniqueness:] % TODO: (GEORGE) Ref for Uniqueness of Thom class 
      Using a suitable Mayer-Vietoris sequence for glueing, and an
      inductive argument starting with the trivial bundle case, one can show:
      Any two classes in $\H^{\rkk}(\thomspacepair{E};R)$ whose
      restrictions coincide on all fibers will coincide.
      However, for $R=\Zmod2$ there is exactly one possible choice for
      a unit $\u{\xi}|_{E_b}\in
      \H^{\rkk}(\spherepair{E_b})^\times\cong\Zmod2^\times=\{1\}$
      over each point $b$.
    \item[Naturality:] Clear from uniqueness and the naturality of Thom classes.
    \end{description}
  \end{proof}
\end{Cor}

\begin{Rem}
  Using paracompactness of $B$ and
  \cite[Proposition~17.9.6]{tomdieck}, one concludes that
  $\u{\xi}\in \H^{\rkk}(\thomspacepair{E};R)$ has to be a unit.
\end{Rem}

\begin{Thm}[Thom isomorphism]
  For any Thom class $\u{\xi}$ of $\xi$, and any degree $r$ there are
  isomorphisms that are natural with respect to pullbacks of vector
  bundles over paracompact spaces
  \begin{align*}
    \thomiso\colon
    \H^r(B;R) &\longrightarrow \H^{r+\rkk}(\thomspacepair{E}; R)
    & \thomiso\colon
      \H_{r+\rkk}(\thomspacepair{E}; R) &\longrightarrow \H^r(B;R)
    \\
    x &\longrightarrow \pb p (x) \cup \u{\xi}
    & \alpha &\longrightarrow \pf p (\u{\xi} \cap \alpha)
  \end{align*}
  called the \emph{Thom isomorphisms}.
  \begin{proof}
    Naturality directly follows from the naturality of the Thom class
    in \autoref{cor:thomclsnatural}, and naturality of the cup-
    respectively cap-product.
    The cohomology part then is a direct application of Leray's theorem
    \cite[Theorem~4D.8]{hatcher}.
    For the homology part see \forexample \cite[Theorem~14.6]{switzer}.
  \end{proof}
\end{Thm}

\begin{Lem}\label{lem:thomisoself-adjoint}
  If $B$ is connected, the Thom isomorphisms are adjoint in the
  following sense:
  For $r\in\Nat$, $x\in \H^r(B)$,
  and $\alpha\in \H_{r+\rkk}(\thomspacepair{E})$  holds 
  \begin{gather*}
    \capped{t(x)}{\alpha} = \capped{x}{t(\alpha)} \in\Zmod2
  \end{gather*}
\end{Lem}
In order to proof Lemma~\autoref{lem:thomisoself-adjoint},
first recall the following properties of the cap product.
\begin{Rem}
  For
  a map of triples of spaces
  $f\colon (Y,Y'',Y')\to (X,X'',X')$,
  cohomology classes
  $a\in \H^i(X,X')$ and $b\in \H^j(X,X')$,
  homology classes
  $\gamma\in \H_{i+j}(X, X'\cup X'')$
  and
  $\beta\in \H_j(Y, Y'\cup Y'')$,
  and a vector bundle $E\xrightarrow{p}B$
  holds
  \begin{align}
    \label{eq:capprod1}
    \capped{a\cup b}{\beta} &= \capped{b}{a\cap\beta}
                              \in \H_0(X,X'')\\
    \label{eq:capprod2}
    \capped{a}{\pf f \beta} &= \pf f \capped{\pb f a}{\beta}
                              \in \H_0(X,X'')
                            &&\text{\cite[Chap.~3.3.2, p.\,241]{hatcher}}\\
    \label{eq:capprod3}
    \pf p &= \Id \colon
    \Zmod2\cong \H_0(\thomspacepair{E})\to \H_0(B)\cong\Zmod2
  \end{align}
\end{Rem}
\begin{proof}[proof of Lemma~\autoref{lem:thomisoself-adjoint}]
  With $\U\coloneqq\u{\xi}$ one calculates
  \begin{align*}
    \capped{t(x)}{\alpha}
    &= \capped{\pb p x \cup \U}{\alpha} \\
    &\equalsby{\eqref{eq:capprod1}}
      \capped{\pb p x}{\U\cap\alpha} \\
    &\equalsby{\eqref{eq:capprod3}}
      \pf p\capped{\pb p x}{\U\cap\alpha} \\
    &\equalsby{\eqref{eq:capprod2}}
      \capped{x}{\pf p(u\cap\alpha)}
      = \capped{x}{\thomiso (\alpha)} \in\Zmod2
      \qedhere
  \end{align*}
\end{proof}

% TODO: Better motivation for relationship thom iso <-> fundamental cl
Normal bundles give an interesting, more geometrical
description of the fundamental class of a manifold:
\begin{Lem}\label{lem:thomisofundcl}
  Let $M$ be an $n$-dimensional compact manifold, and
  $i\colon M\to\R^{n+\rkk}\subset\Sphere{n+\rkk}$ be an embedding with
  normal bundle $\N M$ of rank $\rkk>0$.
  The normal bundle gives rise to an embedding
  $e\colon \E{\N M}\to\Sphere{n+\rkk}$ of its total space
  as a tubular neighbourhood $e(\E{\N M})$ of $i(M)$ into the
  $(n+\rkk)$-sphere.
  The quotient map
  \begin{gather*}
    \collapse\colon
    \Sphere{n+\rkk}
    \to \Sphere{n+\rkk}/\left( \Sphere{n+\rkk}\setminus e(\E{\N M}) \right)
    \cong \Discbdl{\N M} / \Spherebdl{\N M}
    \cong \Thomspace{\N M}
  \end{gather*}
  that collapses every point outside of $e(\E{\N M})$ to the infinity
  point fulfills
  \begin{center}
    \begin{tikzcd}[row sep=0pt]
      \H_{n+\rkk}(\Sphere{n+\rkk}) \ar[r, "\pf c"]
      & \H_{n+\rkk}(\Thomspace{\N M}) \ar [r, "\pf \incl"]
      & \relH_{n+\rkk}(\Thomspace{\N M})
      %\coloneqq \H_{n+\rkk}(\Thomspace{\N M}, \infty)
      \ar [r, "\pf t"{above}, "\cong"{below}]
      & \H_{n}(M)% \cong \Zmod2
      \\
      \fundcl{\Sphere{n+\rkk}} \ar[rrr, mapsto]
      &&&\thomiso(\pf \incl\pf\collapse\fundcl{\Sphere{n+\rkk}}) = \fundcl M
    \end{tikzcd}
  \end{center}
  where $\incl\colon(\Thomspace{\N M},\emptyset)\to(\Thomspace{\N M}, \{\infty\})$ is the
  canonical inclusion of pairs of spaces.
  This holds for any choice of embeddings $i$ and $e$.
  \begin{proof}
    First note that by the long exact sequence of the pair 
     $(\Thomspace{\N M}, \{\infty\})$ the map
     $\pf \incl\colon \H_r(\Thomspace{\N M})\to \relH(\Thomspace{\N M})$
    is an isomorphism in every degree $r>0$. The proof will be
    conducted in two steps, first proving the connected case, then
    the general one.
    
    \begin{description}
    \item[Connected case:]
      Assume that $M$ is connected.
      Then
      $\H_{n+k}(\Thomspace{\N M})
      \cong\relH_r(\Thomspace{\N M})
      \cong \H_n(M) \cong \Zmod2=\{\fundcl M, 0\}$ by
      the Thom isomorphism, and by connectedness of $M$.
      Thus, one only has to show that
      $\pf\collapse\fundcl{\Sphere{n+\rkk}}$
      is non-zero.    
      The trick now is to reduce once again to the homology of the
      sphere:
      Locally around any point $p\in\Thomspace{\N M}\setminus\{\infty\}$ the
      collapse map $\collapse$ is by definition a
      homeomorphism. Therefore, on homology there is the following 
      commutative diagram
      \begin{center}
        \begin{tikzcd}
          \fundcl{\Sphere{n+\rkk}}
          \ar[d,mapsto]\ar[r,phantom,"\in"{near start}]
          &\H_{n+\rkk}(\Sphere{n+\rkk})
          \ar[r, "\pf\incl\pf\collapse"]\ar[d,"\pf\incl"]
          &\relH_{n+\rkk}(\Thomspace{\N M})
          \ar[d,"\pf\incl"]\\
          \left.\fundcl{\Sphere{n+\rkk}}\right|_{\collapse^{-1}(p)}
          \ar[r,phantom,"\in"{near start}]
          &\H_{n+k}(\Sphere{n+\rkk},\Sphere{n+\rkk}\setminus\collapse^{-1}(p))
          \ar[r,"\pf\collapse"{above},"\cong"{below}]
          &\H_{n+\rkk}(\Thomspace{\N M}, \Thomspace{\N M}\setminus p)
        \end{tikzcd}
      \end{center}
      By definition of the fundamental class $\fundcl{\Sphere{n+\rkk}}$,
      the class $\fundcl{\Sphere{n+\rkk}}|_{\collapse^{-1}(p)}$ in the
      diagram is a generator, and thus also is
      $\pf\collapse\left(\fundcl{\Sphere{n+\rkk}}|_{\collapse^{-1}(p)}\right)
      = \left(\pf\incl\pf\collapse\fundcl{\Sphere{n+\rkk}}\right)|_p$.
      However, then $\pf\incl\pf\collapse\fundcl{\Sphere{n+\rkk}}\in
      \H_{n+\rkk}(\Thomspace{\N M})$
      cannot be zero as was to be shown.

    \item[General case] In case $M=\coprod_i M_i$ is the disjoint sum of its connected
      components $M_i$, $i\in I$ for some index set $I$, note the following:
      \begin{itemize}
      \item $\E{\N M} = \coprod_i\E{\N{M_i}}$,
        where $\N{M_i}\coloneqq\N M|_{M_i}$.
      \item Thus, $\Thomspace{\N M} = \bigvee_i\Thomspace{\N{M_i}}$ using the
        collapse maps
        \begin{gather*}
          \collapse_i\colon
          \Sphere{n+\rkk}
          \overset{\collapse}\longto
          \Sphere{n+\rkk}/\left(\Sphere{n+\rkk}\setminus e(\E{\N{M}})\right)
          \overset{\proj}\longto
          \Sphere{n+\rkk}/\left(\Sphere{n+\rkk}\setminus e(\E{\N{M_i}})\right)
        \end{gather*}
        for the disjoint parts,
        and $\collapse=\bigvee_i\collapse_i$.
      \item Thus, $\H_r(\Thomspace{\N M}) = \prod_i \H_r(\Thomspace{\N{M_i}})$ for
        all degrees $r$, $\fundcl M = (\fundcl{M_i})_i$, and
        $\pf\collapse = \prod_i\pf{\collapse_i}$.
        % \item Thus, $\H_{n+\rkk}(\Thomspace{\N M})
        %   \cong \H_{n+\rkk}(M)\cong\prod_i \Zmod2$, and $\fundcl{M} = (1)_i$.
      \end{itemize}
      With this one sees directly from the definition of the fundamental
      class of a manifold that
      $\pf\incl\pf\collapse\fundcl{\Sphere{n+\rkk}} = \fundcl M$
      if and only if for all connected component manifolds $M_i$ holds
      $\pf\incl\pf{\collapse_i}\fundcl{\Sphere{n+\rkk}} = \fundcl M_i$
      which is true by the first case.
      \qedhere
    \end{description}
  \end{proof}
\end{Lem}


\subsection{Stiefel-Whitney Classes}
\begin{Def}[Stiefel-Whitney Classes]\label{def:swclasses}
  The Stiefel-Whitney classes are
  characteristic classes for principal $\Orth$-bundles
  respectively vector bundles,
  \idest cohomology classes
  $\ws{i}\in \H^i(\BO;\Zmod2)$, $i\in\Z\geq0$,
  fulfilling the following properties for any vector bundles $\xi$ and
  $\eta$ over a space $B$, and any map $f\colon A\to B$ of spaces:
  \begin{axioms}
  \axiom[Naturality] $\pb f\w{i}{\xi} = \w{i}{\pb f \xi}$,
  \axiom $\w{0}{\xi}=1$,
  \axiom $\W{\gamma_1} = 1 + x$,
  \axiom[Multiplicativity]\label{tag:swclassesmultiplicativity}
  $\W{\xi \oplus \eta} = \W{\xi}\cup \W{\eta}$
    \\\idest in degree $n$ we have
    $\w{n}{\xi\oplus\eta} = \sum_{i+j=n}\w{i}{\xi} \cup \w{j}{\eta}$,
  \end{axioms}
  where the \emph{total Stiefel-Whitney class}
  $\Ws\coloneqq\sum_{i\geq 0}\ws{i}$ is the formal sum of all
  Stiefel-Whitney classes,
  $\gamma_1$ is the universal line bundle $\RPinf\cong\BO(1)\to\BO$,
  and $x$ is the%
  \footnote{
    This is well-defined: A a ring $R$ of the form $\Zmod2[x]$
    with $\deg(x)=1$ only admits two elements in degree~1, $0$ and a
    generator. Therefore, there exists exactly one ring
    isomorphism, and this sends the unique generator in
    degree 1 to $x$.
  }
  generator of $\H^*(\RPinf;\Zmod2)\cong\Zmod2[x]$.
  \optcite[compare §4, p.~37]{milnor}
\end{Def}

\begin{Thm}[Existence and Uniqueness of Stiefel-Whitney Classes]
  Stiefel-Whitney classes exist and are uniquely defined by the above
  properties. Furthermore, they generate $\H^*(\BO;\Zmod2)\cong \Zmod2[\ws{i}|i\geq1]$.
  \begin{proof}[proof (sketch)]
    \begin{description}
    \item[Existence]
      A possible concrete construction utilizes the Euler
      class. Another definition via Steenrod squares can be 
      found below in \autoref{thm:altdefswclasses}.
    \item[Uniqueness]
      Here one can use the splitting principle and the knowledge about
      $\W{\gamma_1}$
      \cite[Uniqueness Theorem~7.3]{milnor}.
    \item[Generators]
      Either inductively show
      $\H^*(\BO(n);\Zmod2)\cong \Zmod2[\ws{i}|n\geq i\geq1]$
      \cite[Theorem~7.1~ff.]{milnor}.
      Or find generators of $\H^*(\BO)$ that fulfill the axioms for
      the Stiefel-Whitney classes, then apply uniqueness
      \cite[Chap.~7.6]{may}.
    \end{description}
  \end{proof}
\end{Thm}

\begin{Rem}[Further Properties of the Stiefel-Whitney Classes]
  \label{rem:propswclasses}
  Let $\xi$, $\eta$ be vector bundles over a space $X$.
  \begin{enumerate} 
  \item $\w{i}{\eta} = 0$
    for any vector bundle $\eta$ with $\rk\eta < i$.
    Therefore, the total Stiefel-Whitney class $\W{\xi}$ is
    well-defined (\idest the sum is finite)
    for any vector bundle $\xi$ of finite rank.
    \begin{proof}
      This follows directly from the construction for the existence of
      the Stiefel-Whitney classes.
    \end{proof}
  \item $\w{i}{\trivbdl}=0$ for $i>0$, and one immediately concludes
    from multiplicativity:
    \begin{enumerate}
    \item The Stiefel-Whitney classes are stable, \idest
      $\w{i}{\xi\oplus\trivbdl} = \w{i}{\xi}$.
    \item\label{item:wuclassmfdinverse}
      If $\xi\oplus\eta = \trivbdl$, $\w{i}{\xi}\cup\w{i}{\eta}=1$.
      Especially, for any choice of embedding with normal bundle $\N M$
      of a manifold $M$ we have $\T M \oplus \N M = \trivbdl$ and
      therefore $1 = \W{\T M} \cup \W{\N M}$.
    \end{enumerate}
    \begin{proof}
      The trivial rank $n$ bundle over $X$ is defined as the pullback
      $\pb \pi \trivbdl$ of the rank $n$ bundle
      $\trivbdl\colon \R^n\to\pt$ over the point by the trivial map
      $\pi\colon X\to\pt$. The naturality of the Stiefel-Whitney
      classes gives $\W{\trivbdl} = \pb\pi \W{\trivbdl}
      \in\pb\pi \left(\H^i(\pt;\Zmod2)\right)$,
      and the result follows from $\H^i(\pt;\Zmod2) = 0$ for $i>0$.
    \end{proof}
  \end{enumerate}
\end{Rem}

\begin{Def}
  Define the \emph{dual Stiefel-Whitney (characteristic) classes}
  $\dualws{i}$ in degree $i$ inductively by
  \begin{align*}
    1 &= \dualws{0}\cup\ws{0} = \dualws{0}    &&\text{in degree 0}\\
    0 &= \sum_{i+j=n} \dualws{i}\cup\ws{j}  &&\text{in degree n>0}
  \end{align*}
  With the notation $\dualWs\coloneqq \sum_{i\geq0} \dualws{i}$ as above
  for the formal sum this can be reformulated as
  \begin{gather*}
    1 = \Ws\cup\dualWs
  \end{gather*}
  in the completion of the polynomial ring $\H^*(\BO)\cong\Zmod2[\ws{i}|i\in\Nat]$.
  For a manifold $M$ define
  $\W{M} \coloneqq \W{\T M}$ and
  $\dualW{M} \coloneqq \dualW{\T M} = \W{\N M}$
  where the last equality is the one from
  Remark~\autoref{rem:propswclasses} above.
\end{Def}

\begin{Thm}\label{thm:altdefswclasses}
  The Stiefel-Whitney classes can be given as
  \begin{gather*}
    \Sq i(\u{\xi}) = \thomiso(\w{i}{\xi}) = \pb p \w{i}{\xi} \cup \u{\xi}
  \end{gather*}
  for any vector bundle $\xi\colon E\to B$ over a paracompact space
  $X$. As the Thom isomorphism is a group homomorphism, one can
  formulate the above as
  \begin{gather*}
    \SQ(\u{\xi}) = \thomiso(\W{\xi}) = \pb p \W{\xi} \cup \u{\xi}
  \end{gather*}
  The representing cohomology classes can be constructed using the
  embeddings $\BO(n)\subset\BO$ for $n\geq i$ and the fact that
  $\ws{i}=\pb\incl\w{i}{\gamma_n}\in \H^i(\BO)$
  \cite[see \forexample][Theorem~7.1~ff.]{milnor}.
  \begin{proof}
    Check naturality of the expression and all further
    definining properties from \autoref{def:swclasses}.
    \begin{description}
    \item[Naturality:] Both $\Sq i$ and $t$ respectively also $t^{-1}$
      are natural.
    \item[$\ws{0}=1$:]
      $\H^0(\thomspacepair{\E{\gamma_0}}) = \Zmod2$, thus 1 is the only
      candidate for a Thom class, $\Sq0(1) = \Id(1) = 1$, and the Thom
      isomorphism sends 1 to 1 in this degree.
    \item[$\W{\gamma_1}=1+x$:]
      $\Sq 0(\u{\gamma_1})\cequalsby{\eqref{eq:sqidentity}}\u{\gamma_1}
      = \pb p\w{0}{\gamma_1}\cup\u{\gamma_1}$
      directly gives
      $\pb p\w{0}{\gamma_1}=1$, and thus $\w{0}{\gamma_1}=1$.
      With
      $\Sq 1(\u{\gamma_1})\cequalsby{\eqref{eq:sqsquared}}\u{\gamma_1}^2
      =\pb p\w{1}{\gamma_1}\cup x$,
      the class $\pb p\w{1}{\gamma_1}$ cannot be zero, thus also
      $\w{1}{\gamma_1}$ cannot be zero, but it is the generator $x$.
    \item[Multiplicativity:]
      Consider vector bundles $\xi$, $\eta$ over a paracompact space
      $B$. With $\u{\xi\oplus\eta}=\u{\xi}\cup\u{\eta}$ and the fact
      \begin{gather}\label{eq:projectionscommute}
        p_\xi\circ\pi_\xi = p_{\xi\oplus\eta} = p_\eta\circ\pi_\eta
      \end{gather}
      get
      \begin{align*}
        \thomiso(\w{i}{\xi\oplus\eta})
        &= \Sq i(\u{\xi\oplus\eta}) \\
        &\equalsby{\autoref{cor:thomclassmultiplicative}}
          \Sq i(\pb\pi_\xi\u{\xi} \cup \pb\pi_\eta\u{\eta})\\
        &\equalsby{\ref{tag:cartan}}
          \sum_{r+s=i}
          \Sq r(\pb\pi_\xi\u{\xi}) \cup \Sq s(\pb \pi_\eta\u{\eta}) \\
        &\equalsby{Naturality}
          \sum_{r+s=i}
          \pb\pi_\xi\Sq r(\u{\xi}) \cup \pb \pi_\eta\Sq s(\u{\eta}) \\
        &\equalsby{Definition}
          \sum_{r+s=i}
          \pb\pi_\xi\thomiso(\w{r}{\xi})
          \cup \pb\pi_\eta\thomiso(\w{s}{\eta}) \\
        &\equalsby{Definition}
          \sum_{r+s=i}
          \pb\pi_\xi \left(\pb p_\xi  \w{r}{\xi}  \cup \u{\xi} \right)
          \cup
          \pb\pi_\eta\left(\pb p_\eta \w{s}{\eta} \cup \u{\eta}\right) \\
        &= \left(
          \sum_{r+s=i}
          \pb\pi_\xi \pb p_\xi \w{r}{\xi} \cup
          \pb\pi_\eta\pb p_\eta\w{s}{\eta}
          \right)
          \cup
          \left(\pb\pi_\xi\u{\xi} \cup \pb\pi_\eta\u{\eta}\right) \\
        &\equalsby{\eqref{eq:projectionscommute}, Definition}
          \left(\sum_{r+s=i}
          \pb p_{\xi\oplus\eta} \w{r}{\xi} \cup \pb p_{\xi\oplus\eta} \w{s}{\eta}
          \right)
          \cup
          \u{\xi} \times \u{\eta} \\
        &\equalsby{Group Hom., \autoref{cor:thomclassmultiplicative}}
          \pb p_{\xi\oplus\eta}
          \left(\sum_{r+s=i}\w{r}{\xi}\cup\w{s}{\eta}\right)
          \cup
          \u{\xi\oplus\eta}\\
        &\equalsby{Definition}
          \thomiso\left(\sum_{r+s=i}\w{r}{\xi}\cup\w{s}{\eta}\right)
          \;.
      \end{align*}
      Applying $\thomiso^{-1}$ yields the result.
      \qedhere
    \end{description}
  \end{proof}
\end{Thm}

\subsection{Wu Classes}
Let $M$ be a compact, $n$-dimensional manifold.
This section introduces a family of cohomology classes
of manifolds, together with several possibilities of how to
define---and eventually work with---them.
Later then, \autoref{sec:wutheorem} reveals their close relation to
the Stiefel-Whitney classes which is utilized in the proof of Massey's
theorem.

\begin{Def}\label{def:wuclasses}
  The $i$th \emph{Wu class} $\v{i}{M}$ of $M$ for $0\leq i\leq n$ is defined
  as the cohomology class in $\H^i(M)$ that is uniquely determined by
  \begin{center}
    \begin{tikzcd}[row sep=0pt, column sep=small]
      \H^i(M) \ar[r, equal, "\isosymb"]
      & \H_{n-1}(M) \ar[r, equal, "\isosymb"]
      &\Hom{\Zmod2}(\H^{n-i}(M), \Zmod2) \ar[r, equal, "\isosymb"]
      &\Zmod2
      \\
      y \ar[rr, mapsto] &&\capped{ x\cup y }{ \fundcl M }\\
      \v{i}{M} \ar[rr, mapsto] &&\capped{\Sq i(x)}{\fundcl M}
    \end{tikzcd}
  \end{center}
  where the first isomorphism from the left is Poincaré duality
  and the second is the universal coefficient theorem
  for the field $\Zmod2$.
  Equivalently, for any cohomology class $x\in \H^{n-i}(M)$ of fixed
  degree $n-i$ holds
  \begin{gather*}
    x\cup \v{i}{M} = \Sq i(x) \in \H^n(M) \cong \Zmod2
  \end{gather*}
  Mind the fixed degree of $x$---the above will not be true for other
  degree cohomology classes in general!
\end{Def}

\begin{Rem}
  Some immediate consequences of the definitions of the Wu classes of
  $M$ are
  \begin{itemize}
  \item $\v{0}{M} = 1$
  \item $\v{i}{M} = 0$ for $i>\frac n 2$, because $\Sq i(x) = 0$ if the
    degree of x is lower than $i$.
  \end{itemize}
\end{Rem}


\begin{Def}\label{def:dualwuclasses}
  The \emph{total Wu class} of $M$ is defined as the sum
  $\sum_{i\geq0}\v{i}{M}$. The \emph{total dual Wu class}
  $\dualV{M}\eqqcolon \sum_{i\geq 0}\dualv{i}{M}$
  and the dual Wu classes $\dualv{i}{M}$
  of $M$ are defined by
  \begin{gather*}
    \V{M} \cup \dualV{M} = 1
  \end{gather*}
  or equivalently
  \begin{align*}
    1 &= \dualv{0}{M} \cup \v{0}{M} = \dualv{0}{M} \\
    0 &= \sum_{r+s=i}\v{r}{M}\cup\dualv{s}{M}
      &&\text{in degree $0\leq i\leq n$}
  \end{align*}
\end{Def}

The following more general definition of Wu classes
will be shown to be equivalent to the one above in
Definition~\autoref{def:wuclasses}.
It uses the antipode of the Steenrod algebra (see \autoref{def:antipode}). 
\begin{Def}\label{def:altwuclasses}
  Let $\xi\colon E\xrightarrow{p} B$ be a vector bundle over a
  paracompact space $B$.
  The $i$th \emph{Wu class} $\v{i}{\xi}$ of $\xi$ for $0\leq i\leq n$
  is defined as the cohomology class in $\H^i(B)$ that is uniquely
  determined by
  \begin{gather*}
    \antipode(\Sq i)(\u{\xi}) = \thomiso(\v{i}{\xi}) = \pb p \v{i}{\xi} \cup \u{\xi}
  \end{gather*}
  The \emph{total Wu class} of $\xi$ is defined as usual as
  $\V{\xi}\coloneqq\sum_{i\geq0}\v{i}{\xi}$ and satisfies accordingly 
  $\antipode(\SQ)(\u{\xi})=\V{\xi}$.
\end{Def}
\begin{Rem}
  Compare this to the possible definition of the Stiefel-Whitney
  classes in \autoref{thm:altdefswclasses}.
\end{Rem}

\begin{Thm}\label{thm:altdefwuclasses}
  Let $M$ be a compact manifold of dimension $n$, and let
  $\N M\colon \E{\N M} \xrightarrow{p} M$ be
  any normal bundle of rank $k$ of an embedding of $M$ into
  $\R^{n+k}$. Then
  \begin{gather*}
    \v{i}{M} = \v{i}{\N M}\in \H^i(M)
  \end{gather*}
  \begin{proof}
    To proof \autoref{thm:altdefwuclasses}, the defining property of
    $\v{i}{M}$ will be checked on $\v{i}{\N M}$, \idest one has to show
    that for any $x\in \H^{n-i}(M)$ holds
    \begin{gather*}
      \capped{x\cup \v{i}{\N M}}{\fundcl M}
      = \capped{\Sq i(x)}{\fundcl M}
    \end{gather*}
    Note, that this is simply the $i$th degree of the equation
    \begin{gather}\label{eq:proofaltdefwuclasses:claim}
      \capped{x\cup \V{\N M}}{\fundcl M}
      = \capped{\SQ(x)}{\fundcl M}
    \end{gather}
    which will be proven below.
    Beforehands, recall the following:
    \begin{description}
    \item[By \autoref{lem:thomisoself-adjoint}:]
      $\capped{\thomiso(z)}{\alpha} = \capped{z}{\thomiso(\alpha)}$
      for $r\in\Nat$, $z\in \H^r(M)$,
      $\alpha\in \H_{r+\rkk}(\thomspacepair{\E{\N M}})$.
    \item[By \autoref{lem:thomisofundcl}:]
      $\thomiso(\pf\incl\pf\collapse\fundcl{\Sphere{n+\rkk}}) = \fundcl M$
      where $\collapse\colon \Sphere{n+\rkk}\to \Thomspace{\N M}$ is
      the collapse map of a tubular embedding of the normal bundle $\N
      M$.
    \item[By \eqref{eq:sqlowerbound} and \eqref{eq:sqidentity}:]
      The total Steenrod square
      $\SQ\colon \H^m(\Sphere m)
      \to \H^*(\Sphere m)$
      is the identity on $\H^m(\Sphere m)$, \idest
      $\Sq i\colon \H^m(\Sphere{m})\to \H^{m+i}(\Sphere{m})$ is zero for
      $i\neq0$.
    \item[By Definition \autoref{def:sq}:] The total Steenrod Square
      $\SQ$ is natural and a ring homomorphism.
    \item[By \eqref{eq:capprod2}:]
      For any map of spaces $f\colon X\to Y$ and co-/homology classes
      $a$ and $\beta$ in the corresponding co-/homology groups holds
      $\capped{a}{\pf f \beta} = \pf f \capped{\pb f a}{\beta}$
    \end{description}
    For the proof fix some $i\leq n$, some arbitrary $x\in
    \H^{n-i}(M)$, as well as a collapse map
    $\collapse\colon\Sphere{n+k}\to\Thomspace{\N M}$ as in
    \autoref{lem:thomisofundcl}.
    For simplicity, denote
    $\Vu\coloneqq \V{\N M}$,
    $\Sph\coloneqq \Sphere{n+\rkk}$,
    $\U\coloneqq \u{\N M}$ and
    $\pf\collapse\coloneqq \pf{(\incl\circ\collapse)}$
    respectively
    $\pb\collapse\coloneqq \pb{(\incl\circ\collapse)}$.
    
    
    In a first calculation reformulate
    $\capped{x\cup \V{\N M}}{\fundcl M}$ using the cohomology of the
    $(n+k)$-sphere:
    \begin{align}\notag
      \capped{x\cup \Vu}{\fundcl M}
      &\equalsby{\autoref{lem:thomisofundcl}}
        \capped
        {x\cup\Vu}
        {\thomiso\left(\pf\collapse\fundcl{\Sph}\right)}
      \\\notag
      &\equalsby{\autoref{lem:thomisoself-adjoint}}
        \capped
        {\thomiso\left(x\cup\Vu\right)}
        {\pf\collapse\fundcl{\Sph}}
      \\\notag
      &\equalsby{\eqref{eq:capprod2}}
        \pf\collapse\capped
        {\pb\collapse\thomiso\left(x\cup\Vu\right)}
        {\fundcl{\Sph}}
      \\\notag
      &\equalsby{\eqref{eq:sqlowerbound} and \eqref{eq:sqidentity}}
        \pf\collapse\capped
        {\SQ\left(\pb\collapse\thomiso\left(x\cup\Vu\right)\right)}
        {\fundcl{\Sph}}
      \\\notag
      &\equalsby{Naturality}
        \pf\collapse\capped
        {\pb\collapse\SQ\left(\thomiso\left(x\cup\Vu\right)\right)}
        {\fundcl{\Sph}}
      \\\label{eq:proofaltdefwuclasses:eq1}
      &\equalsby{\eqref{eq:capprod2}}
        \capped
        {\SQ\left(\thomiso\left(x\cup\Vu\right)\right)}
        {\pf\collapse\fundcl{\Sph}}
    \end{align}
    Having introduced $\SQ$ on the left hand side, one observes a
    certain commutativity of the Thom isomorphism, and the total
    Steenrod square:
    \begin{align}\notag
      \SQ\left(\thomiso\left(x\cup\Vu\right)\right)
      &=
        \SQ\left(\left(\pb p x\cup \pb p\Vu\right) \cup \U\right)
      \\\notag
      &=
        \SQ\left(\pb p x\cup \left(\pb p\Vu \cup \U\right)\right)
      \\\notag
      &\equalsby{Def. $\thomiso$}
       \SQ\left(\pb p x\cup \thomiso(\Vu)\right)
      \\\notag
      &\equalsby{Naturality, Ring Hom.}
        \pb p\SQ(x) \cup \SQ(\thomiso(\Vu))
      \\\notag
      &\equalsby{Def. $\V{M}$}
        \pb p\SQ(x) \cup \SQ(\antipode(\SQ)(u))
      \\\notag
      &\equalsby{Def. $\antipode$}
        \pb p\SQ(x) \cup u
      \\\label{eq:proofaltdefwuclasses:eq2}
      &\equalsby{Def. $\thomiso$}
        \thomiso(\SQ(x))
    \end{align}
    Inserting \eqref{eq:proofaltdefwuclasses:eq2} into
    \eqref{eq:proofaltdefwuclasses:eq1} from above easily yields the
    claim in \eqref{eq:proofaltdefwuclasses:claim} that proves the theorem:
    \begin{align*}
      \capped{x\cup \Vu}{\fundcl M}
      &\cequalsby{\eqref{eq:proofaltdefwuclasses:eq1}}
        \capped
        {\SQ\left(\thomiso\left(x\cup\Vu\right)\right)}
        {\pf\collapse\fundcl{\Sph}}
      \\
      &\cequalsby{\eqref{eq:proofaltdefwuclasses:eq2}}
        \capped
        {\thomiso(\SQ(x))}
        {\pf\collapse\fundcl{\Sph}}
      \\
      &\equalsby{\autoref{lem:thomisoself-adjoint}}
        \capped
        {\SQ(x)}
        {\thomiso\left(\pf\collapse\fundcl{\Sph}\right)}
      \\
      &\equalsby{\autoref{lem:thomisofundcl}}
        \capped
        {\SQ(x)}
        {\fundcl M}
        \qedhere
    \end{align*}
  \end{proof}
\end{Thm}


\section[Wu's Theorem]{Wu's Theorem}\label{sec:wutheorem}
This section is meant to show a direct connection between the
Stiefel-Whitney classes and the Wu classes, which can be shown using
the alternative definition of the Wu classes from
\autoref{thm:altdefwuclasses}.

\begin{Thm}[Wu]\label{thm:wu}
  A closed manifold gives rise to the following two equalities
  \begin{align}\notag
    \dualW{M} &= \SQ\left(\dualV{M}\right)
    &\text{respectively}&
    &\dualw{k}{M} &= \sum_{i\geq0} \Sq i\left(\dualv{k-i}{M}\right)
    &\text{and}
    \\
    \W{M} &= \SQ\left(\V{M}\right)
    &\text{respectively}&
    &\w{k}{M} &= \sum_{i\geq0} \Sq i\left(\v{k-i}{M}\right)
    \label{tag:wuformula}\tag{Wu's formula}
  \end{align}
  that are equivalent using
  $\dualW{M}\cup\W{M} = 1$ and
  \begin{gather*}
    \SQ(\dualV{M})\cup\SQ(\V{M})
    = \SQ(\dualV{M}\cup\V{M})
    = \SQ(1)
    = 1
    \;.
  \end{gather*}
\end{Thm}
% \begin{Rem}
%   The following is equivalent to Wu's formula
%   \begin{gather*}
%     \dualW{M} = \SQ\left(\dualV{M}\right)
%     \qquad\text{respectively}\qquad
%     \dualw{k}{M} = \sum_{i\geq0} \Sq i\left(\dualv{k-i}{M}\right)
%   \end{gather*}
%   using the inverses $\dualW{M}\cup\W{M} = 1$ and
%   \begin{gather*}
%     \SQ(\dualV{M})\cup\SQ(\V{M})
%     = \SQ(\dualV{M}\cup\V{M})
%     = \SQ(1)
%     = 1
%     \;.
%   \end{gather*}
% \end{Rem}
The proof uses the alternative characterization
\begin{gather*}
  \antipode(\SQ)(\u{\N M}) = \thomiso(\V{\N M})
\end{gather*}
of the Wu classes from \autoref{thm:altdefwuclasses}, and quite directly
follows from the following Lemma:
\begin{Lem}\label{lem:wu}
  For a vector bundle $\xi\colon \E\xi\xrightarrow{p}\B\xi$ over a
  paracompact space holds
  \begin{gather*}
    \SQ(\V{\xi}) \cup \W{\xi}= 1
    \;.
  \end{gather*}
\end{Lem}
\begin{proof}[proof of \ref{tag:wuformula}]
  Lemma \autoref{lem:wu} states for a closed manifold $M$
  \begin{gather*}
    \SQ(\V{M}) \cup \W{\N M}
    = \SQ(\V{\N M}) \cup \W{\N M}
    \cequalsby{\autoref{lem:wu}} 1
    \;.
  \end{gather*}
  Cupping with $\W{M}=\W{\T M}$ on both sides yields the claim
  as $\W{\T M} \cup \W{\N M} = 1$ by
  \itemref{rem:propswclasses}{item:wuclassmfdinverse}.
\end{proof}
\begin{proof}[proof of Lemma~\autoref{lem:wu}]
  For simplicity use the shortenings
  $\U\coloneqq \u{\xi}$,
  $\Vu\coloneqq \V{\xi}$, and
  $\Ws\coloneqq \W{\xi}$.
  Then calculate
  \begin{align*}
    \thomiso(1)
    &= \U
    \\
    &\equalsby{Def.~\autoref{def:antipode}}
      \SQ\left(\antipode(\SQ)(\U)\right)
    \\
    &\equalsby{Def.~\autoref{def:altwuclasses}}
      \SQ\left(\thomiso(\Vu)\right)
    \\
    &\equalsby{Def.~$\thomiso$}
      \SQ\left(\pb p\Vu \cup \U\right)
    \\
    &\equalsby{\ref{tag:cartan}, Naturality}
      \pb p\SQ(\Vu) \cup \SQ(\U)
    \\
    &\equalsby{\autoref{thm:altdefswclasses}}
      \pb p\SQ(\Vu) \cup \thomiso(\Ws)
    \\
    &\equalsby{Def.~$\thomiso$}
      \pb p\SQ(\Vu) \cup \left(\pb p \Ws \cup \U\right)
    \\
    &=
      \pb p\left(\SQ(\Vu) \cup \Ws\right) \cup \U
    \\
    &\equalsby{Def.~$\thomiso$}
      \thomiso\left(\SQ(\Vu) \cup \Ws\right)
  \end{align*}
  Applying the inverse of the Thom isomorphism to both sides
  yields the equality which was to be shown.
\end{proof}


%%% Local Variables:
%%% mode: latex
%%% TeX-master: "thesis"
%%% End:
