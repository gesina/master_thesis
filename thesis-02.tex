%%%%%%%%%%%%%%%%%%%%%%%%%%%%%%%%%% 
% Master Thesis in Mathematics
% "Immersions and Stiefel-Whitney classes of Manifolds"
% -- Chapter 2: Homotopy Theoretical Formulation of the Immersion Problem --
% 
% Author: Gesina Schwalbe
% Supervisor: Georgios Raptis
% University of Regensburg 2018
%%%%%%%%%%%%%%%%%%%%%%%%%%%%%%%%%%


\chapter{Homotopy Theoretical Formulation of the Immersion Problem}
% Explain carefully how the immersion problem can be reformulated purely
% in terms of homotopy theory. [immersionconj] (The necessary results from differential
% topology and Hirsch-Smale theory should be stated clearly but may be
% presented without proofs.)

% TODO: chapter overview

\begin{Thm}\label{thm:immersionconj}
  For $n\in\Nat$, every closed, smooth, $n$-dimensional manifold
  immerses into $\R^{2n-\alpha(n)}$, where $\alpha(n)$ is defined as
  the minimal number of summands in any binary representation of $n$.
\end{Thm}

An essential step for the reformulation of the immersion conjecture is
a theorem of Hirsch and Smale, which gives a homotopy theoretical
relation between the spaces of immersions of two manifolds $\Imm M N$
and of the vectorbundle monomorphisms $\Mono{\T M}{\T N}$ between
their tangent bundles. For the formulation, one has to equip the
respective sets with a topology as follows.
\begin{Def}
  Let $M$, $N$ be closed smooth manifolds of dimensions $\dim M < \dim N$.
  \begin{enumerate}
  \item % TODO: Whitney $C^r$-topology; see [Hirsch, Differential Topology, Chap 2, p.35]
  \item
    Equip the set of all vector bundle monomorphisms from $\xi_1$ to
    $\xi_2$ with the compact-open topology, and denote that space by
    $\Mono{\xi_1}{\xi_2}$.
    Note that a path between monomorphisms $F_1$ and $F_2$ in the
    space $\Mono \xi \eta$ is an homotopy from $F_1$ to $F_2$ which is
    a vector bundle monomorphism in each stage.
  \item % TODO: (GEORGE) check topology of \Imm(M, N); compare Lecture_Notes_on_Immersions_of_Surfaces_in_3-Space--Nowik.ps
    The set $\Imm M N$ of all immersions from $M$ to $N$ injects
    into $\Mono{\T M}{\T N}$ by taking the differential
    $f\mapsto\Diff f$.
    Equip $\Imm M N$ in the following with the subspace topology.
    This results in the weak topology described in
    \cite[Section~2.1]{hirsch}, which equals the Whitney
    $C^1$-topology since $M$ was chosen compact.
    By the way, $\Imm M N$ is open in $C^1(M,N)$ equipped with the
    Whitney $C^1$-topology
    (see \cite[Section~2.1, Theorem~1.1]{hirsch}),
    and thus not a discrete space.
  \end{enumerate}
\end{Def}

Now one can state the following major result in immersion theory by
Hirsch using preliminary work of Smale
\cite[Sections~5 and 6]{hirschimmersions}.
\begin{Thm}[Hirsch-Smale]\label{thm:hirschsmale}
  \optcite[Theorem~1.2]{immersionconj} % TODO: better ref for modern formulation
  Let $M$, $N$ be closed manifolds with $\dim M<\dim N$.
  Then the differential map
  $\Diff\colon \Imm M N\to \Mono{\T M}{\T N}$
  induces isomorphisms on the homotopy groups.
  Especially,
  \begin{gather*}
    \Diff_*\colon
    \pi_1(\Imm M N) \overset\sim\longto \pi_1(\Mono{\T M}{\T N})
  \end{gather*}
  describes an isomorphism of path-connected components.
  % Original formulation by Hirsch in [hirschimmersions], sec. 5:
  Therefore, every vector bundle monomorphism
  $F\colon\T M\to\T N$ is homotopic (through vector bundle
  monomorphisms) to a monomorphism which is the differential
  $\Diff f$ of a smooth map $f\colon M\to N$, \idest of an
  immersion.
\end{Thm}
Thus, any monomorphism of vector bundles over smooth, closed
manifolds $M$ and $N$ implies the existence of an immersion from $M$
to $N$.

This gives rise to the subsequent reformulation of the immersion
conjecture in \autoref{thm:immersionconj}.
\begin{Thm}\label{thm:immersionconj:equivalences}
  Let $n,k\in\Nat$ and $M^n$ be a closed, smooth, $n$-dimensional manifold.
  The following statements are equivalent.
  \begin{enumerate}
  \item\label{item:immersionconj:1}
    $M$ immerses into $\R^{n+k}$.
  \item\label{item:immersionconj:2}
    There is a vector bundle monomorphism $F\colon\T M\to\T{\R^{n+k}}$.
  \item\label{item:immersionconj:3}
    There is a $k$-dimensional vector bundle
    $\N{}\colon\E{\N{}}\to M$ over $M$ with
    \begin{gather*}
      \N{}\oplus\T M\cong\trivbdl^{n+k}
      \;.
    \end{gather*}
  \item\label{item:immersionconj:4}
    For the map $\N M\colon M\to\B\Orth$ characterising the stable
    normal bundle over $M$ there is a lift $\N{}\colon M\to\B\Orth(k)$
    making the following diagram commute up to homotopy
    \begin{center}
      \begin{tikzcd}
        M
        \ar[r, "\N{}"]
        \ar[dr, "\N{M}"{left}, bend right]
        & \BO(k) \ar[d, "\incl", hookrightarrow] \\
        & \BO
      \end{tikzcd}
    \end{center}
  \end{enumerate}
\end{Thm}
From a homotopy theoretical viewpoint, one obviously is most
interested in statement \ref{item:immersionconj:4}.
So, before passing over to the proof, have a look at some first
advantages of this approach.

For one, this formulation points out a certain homotopy invariance of
the immersion problem, namely an $n$-manifold already satisfies the
immersion property if it is \emph{homotopic} to one doing so. 
The interesting intermediate result that every $n$-manifold is
at least \emph{cobordant} to one fulfilling the conjecture's property
will be discussed in \autoref{chap:brown}.

Moreover as a second major advantage, statement
\ref{item:immersionconj:4} directly puts the immersion problem in the
setting of characteristic classes, more precisely Stiefel-Whitney
classes, which provides a formidable further arsenal of tooling.
As an example, the motivation for the exact value of the immersion
dimension will be deduced in \autoref{chap:massey}.

\begin{proof}[proof of \autoref{thm:immersionconj:equivalences}]
  \begin{description}
  \item[\ref{item:immersionconj:3}$\Leftrightarrow$\ref{item:immersionconj:4}:]
    Using Steenrod's classification theorem
    \itemref{def:charcls}{item:classificationthm} 
    and the properties of the stable normal bundle of a manifold
    directly yields the equivalence of \ref{item:immersionconj:3} and
    \ref{item:immersionconj:4}.
  \item[\ref{item:immersionconj:2}$\Leftrightarrow$\ref{item:immersionconj:3}:]
    In order to get from a vector bundle monomorphism
    $F\colon M\to\R^{n+k}$ to a normal bundle, first note that the
    trivial bundle $\T{\R^{n+k}}$ is easily equipped with an
    orthonormal frame. Thus, one can consider the fibre-wise orthogonal
    complement $\N{}$ of the image $F(\T M)$ which is a vector bundle over
    $F(M)\subset\R^{n+k}$. Its pullback to $M$ along
    $F|_{\zerosec{\T M}}$ obviously fulfils the required property of
    \ref{item:immersionconj:3}.
    
    Conversely, when starting with some rank $k$ vector bundle $\N{}$
    such that $\N{}\oplus\T M\cong\trivbdl^{n+k}$, there is a vector
    bundle monomorphism $\N{}\to\trivbdl$ over $M$. Then the
    following chain of vector bundle morphisms
    \begin{center}
      \begin{tikzcd}
        \T M \ar[d]
        \ar[r, hookrightarrow]
        & M\times\R^{n+k} \ar[d,"\trivbdl^{n+k}"]
        \ar[r]
        & \pt\times \R^{n+k} \ar[d,"\trivbdl^{n+k}"]
        \ar[r, hookrightarrow]
        & \R^{n+k}\times \R^{n+k} \ar[d,"\trivbdl^{n+k}"]
        \\
        M
        \ar[r,equals]
        & M
        \ar[r]
        & \pt
        \ar[r, hookrightarrow]
        & \R^{n+k}
      \end{tikzcd}
    \end{center}
    is fibre-wise injective in each stage, and hence a monomorphism as
    was needed.
  \item[\ref{item:immersionconj:1}$\Leftrightarrow$\ref{item:immersionconj:2}:]
    The tricky part is to relate \ref{item:immersionconj:1} and
    \ref{item:immersionconj:2}, even though it is easily seen that
    \ref{item:immersionconj:1} implies \ref{item:immersionconj:2} by
    simply taking $F$ to be the differential $\Diff f$ of the
    immersion from \ref{item:immersionconj:1}.
    
    The converse direction is an application of the Hirsch-Smale
    theorem~\autoref{thm:hirschsmale}.
    However, first substitute the non-compact manifold $\R^{n+k}$ with
    the compact sphere $N=\Sphere{n+k}$, to make $M$ and $N$ comply
    with the preliminaries of the theorem. As $\dim M<n+k$ by assumption,
    every immersion $M\to\Sphere{n+k}$ misses a point on
    $\Sphere{n+k}$ and hence factors over an immersion $M\to\R^{n+k}$.
    This then shows that also \ref{item:immersionconj:2} implies
    \ref{item:immersionconj:1} which makes them equivalent.
  \end{description}
\end{proof}


%%% Local Variables:
%%% mode: latex
%%% TeX-master: "thesis"
%%% End:
