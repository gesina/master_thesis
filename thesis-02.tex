%%%%%%%%%%%%%%%%%%%%%%%%%%%%%%%%%% 
% Master Thesis in Mathematics
% "Immersions and Stiefel-Whitney classes of Manifolds"
% -- Chapter 2: General Obstructions by Characteristic Classes --
% 
% Author: Gesina Schwalbe
% Supervisor: Georgios Raptis
% University of Regensburg 2018
%%%%%%%%%%%%%%%%%%%%%%%%%%%%%%%%%% 

\chapter{General Obstructions by Characteristic Classes}
\label{chap:massey}
% Review the theory of Stiefel-Whitney classes and the properties of the
% Steenrod algebra (as needed). Give a careful and detailed exposition of
% the proof of Massey’s theorem. Explain that the estimate in Massey’s
% theorem is the best possible in general.
% [massey], [immersionconj]

% TODO: Chapter overview
\begin{itemize}
\item Preliminaries: new Characteristic classes from SW classes
\item Massey's Theorem: no obstruction through char cls up to $n-\alpha(n)$
\item Best possible result for general bound on obstacles
\item generalisation?
\end{itemize}

\section{Wu Classes}\label{sec:wuclassesmain}
\subsection{Equivalent Definitions}\label{sec:wuclasses}
Let $M$ be a compact, $n$-dimensional manifold.
This section introduces a family of cohomology classes
of manifolds, together with several possibilities of how to
define---and eventually work with---them.
Later then, \autoref{sec:wutheorem} reveals their close relation to
the Stiefel-Whitney classes which is utilised in the proof of Massey's
theorem.

\begin{Def}\label{def:wuclasses}
  The $i$th \emph{Wu class} $\v{i}{M}$ of $M$ for $0\leq i\leq n$ is defined
  as the cohomology class in $\H^i(M)$ that is uniquely determined by
  \begin{center}
    \begin{tikzcd}[row sep=0pt, column sep=small]
      \H^i(M) \ar[r, equal, "\isosymb"]
      & \H_{n-1}(M) \ar[r, equal, "\isosymb"]
      &\Hom{\Zmod2}(\H^{n-i}(M), \Zmod2) \ar[r, equal, "\isosymb"]
      &\Zmod2
      \\
      y \ar[rr, mapsto] &&\capped{ x\cup y }{ \fundcl M }\\
      \v{i}{M} \ar[rr, mapsto] &&\capped{\Sq i(x)}{\fundcl M}
    \end{tikzcd}
  \end{center}
  where the first isomorphism from the left is Poincaré duality
  and the second is the universal coefficient theorem
  for the field $\Zmod2$.
  Equivalently, for any cohomology class $x\in \H^{n-i}(M)$ of fixed
  degree $n-i$ holds
  \begin{gather*}
    x\cup \v{i}{M} = \Sq i(x) \in \H^n(M) \cong \Zmod2
  \end{gather*}
  Mind the fixed degree of $x$---the above will not be true for other
  degree cohomology classes in general!
\end{Def}

\begin{Rem}
  Some immediate consequences of the definitions of the Wu classes of
  $M$ are
  \begin{itemize}
  \item $\v{0}{M} = 1$
  \item $\v{i}{M} = 0$ for $i>\frac n 2$, because $\Sq i(x) = 0$ if the
    degree of x is lower than $i$.
  \end{itemize}
\end{Rem}


\begin{Def}\label{def:dualwuclasses}
  The \emph{total Wu class} of $M$ is defined as the sum
  $\sum_{i\geq0}\v{i}{M}$. The \emph{total dual Wu class}
  $\dualV{M}\eqqcolon \sum_{i\geq 0}\dualv{i}{M}$
  and the dual Wu classes $\dualv{i}{M}$
  of $M$ are defined by
  \begin{gather*}
    \V{M} \cup \dualV{M} = 1
  \end{gather*}
  or equivalently
  \begin{align*}
    1 &= \dualv{0}{M} \cup \v{0}{M} = \dualv{0}{M} \\
    0 &= \sum_{r+s=i}\v{r}{M}\cup\dualv{s}{M}
      &&\text{in degree $0\leq i\leq n$}
  \end{align*}
\end{Def}

The following more general definition of Wu classes
will be shown to be equivalent to the one above in
Definition~\autoref{def:wuclasses}.
It uses the antipode of the Steenrod algebra (see \autoref{def:antipode}). 
\begin{Def}\label{def:altwuclasses}
  Let $\xi\colon E\xrightarrow{p} B$ be a vector bundle over a
  paracompact space $B$.
  The $i$th \emph{Wu class} $\v{i}{\xi}$ of $\xi$ for $0\leq i\leq n$
  is defined as the cohomology class in $\H^i(B)$ that is uniquely
  determined by
  \begin{gather*}
    \antipode(\Sq i)(\u{\xi}) = \thomiso(\v{i}{\xi}) = \pb p \v{i}{\xi} \cup \u{\xi}
  \end{gather*}
  The \emph{total Wu class} of $\xi$ is defined as usual as
  $\V{\xi}\coloneqq\sum_{i\geq0}\v{i}{\xi}$ and satisfies accordingly 
  $\antipode(\SQ)(\u{\xi})=\V{\xi}$.
\end{Def}
\begin{Rem}
  Compare this to the possible definition of the Stiefel-Whitney
  classes in Theorem~\autoref{thm:altdefswclasses}.
\end{Rem}

\begin{Thm}\label{thm:altdefwuclasses}
  Let $M$ be a compact manifold of dimension $n$, and let
  $\N{\emb}\colon \E{\N{\emb}} \xrightarrow{p} M$ be
  any normal bundle of rank $k$ of an embedding
  $\emb\colon M\immto\R^{n+k}$. Then
  \begin{gather*}
    \v{i}{M} = \v{i}{\N{\emb}}\in \H^i(M)
  \end{gather*}
  \begin{proof}
    To proof \autoref{thm:altdefwuclasses}, the defining property of
    $\v{i}{M}$ will be checked on $\v{i}{\N{\emb}}$, \idest one has to show
    that for any $x\in \H^{n-i}(M)$ holds
    \begin{gather*}
      \capped{x\cup \v{i}{\N{\emb}}}{\fundcl M}
      = \capped{\Sq i(x)}{\fundcl M}
    \end{gather*}
    Note, that this is simply the $i$th degree of the equation
    \begin{gather}\label{eq:proofaltdefwuclasses:claim}
      \capped{x\cup \V{\N{\emb}}}{\fundcl M}
      = \capped{\SQ(x)}{\fundcl M}
    \end{gather}
    which will be proven below.
    Beforehand, recall the following:
    \begin{description}
    \item[By \autoref{lem:thomisoself-adjoint}:]
      $\capped{\thomiso(z)}{\alpha} = \capped{z}{\thomiso(\alpha)}$
      for $r\in\Nat$, $z\in \H^r(M)$,
      $\alpha\in \H_{r+\rkk}(\thomspacepair{\E{\N{\emb}}})$.
    \item[By \autoref{lem:thomisofundcl}:]
      $\thomiso(\pf\incl\pf\collapse\fundcl{\Sphere{n+\rkk}}) = \fundcl M$
      where $\collapse\colon \Sphere{n+\rkk}\to \Thomspace{\N{\emb}}$ is
      the collapse map of a tubular embedding of the normal bundle $\N
      M$.
    \item[By \eqref{eq:sqlowerbound} and \eqref{eq:sqidentity}:]
      The total Steenrod square
      $\SQ\colon \H^m(\Sphere m)
      \to \H^*(\Sphere m)$
      is the identity on $\H^m(\Sphere m)$, \idest
      $\Sq i\colon \H^m(\Sphere{m})\to \H^{m+i}(\Sphere{m})$ is zero for
      $i\neq0$.
    \item[By Definition \autoref{def:sq}:] The total Steenrod Square
      $\SQ$ is natural and a ring homomorphism.
    \item[By \eqref{eq:capprod2}:]
      For any map of spaces $f\colon X\to Y$ and co-/homology classes
      $a$ and $\beta$ in the corresponding co-/homology groups holds
      $\capped{a}{\pf f \beta} = \pf f \capped{\pb f a}{\beta}$
    \end{description}
    For the proof fix some $i\leq n$, some arbitrary $x\in
    \H^{n-i}(M)$, as well as a collapse map
    $\collapse\colon\Sphere{n+k}\to\Thomspace{\N{\emb}}$ as in
    \autoref{lem:thomisofundcl}.
    For simplicity, denote
    $\Vu\coloneqq \V{\N{\emb}}$,
    $\Sph\coloneqq \Sphere{n+\rkk}$,
    $\U\coloneqq \u{\N{\emb}}$ and
    $\pf\collapse\coloneqq \pf{(\incl\circ\collapse)}$
    respectively
    $\pb\collapse\coloneqq \pb{(\incl\circ\collapse)}$.
    
    
    In a first calculation reformulate
    $\capped{x\cup \V{\N{\emb}}}{\fundcl M}$ using the cohomology of the
    $(n+k)$-sphere:
    \begin{align}\notag
      \capped{x\cup \Vu}{\fundcl M}
      &\equalsby{\autoref{lem:thomisofundcl}}
        \capped
        {x\cup\Vu}
        {\thomiso\left(\pf\collapse\fundcl{\Sph}\right)}
      \\\notag
      &\equalsby{\autoref{lem:thomisoself-adjoint}}
        \capped
        {\thomiso\left(x\cup\Vu\right)}
        {\pf\collapse\fundcl{\Sph}}
      \\\notag
      &\equalsby{\eqref{eq:capprod2}}
        \pf\collapse\capped
        {\pb\collapse\thomiso\left(x\cup\Vu\right)}
        {\fundcl{\Sph}}
      \\\notag
      &\equalsby{\eqref{eq:sqlowerbound} and \eqref{eq:sqidentity}}
        \pf\collapse\capped
        {\SQ\left(\pb\collapse\thomiso\left(x\cup\Vu\right)\right)}
        {\fundcl{\Sph}}
      \\\notag
      &\equalsby{Naturality}
        \pf\collapse\capped
        {\pb\collapse\SQ\left(\thomiso\left(x\cup\Vu\right)\right)}
        {\fundcl{\Sph}}
      \\\label{eq:proofaltdefwuclasses:eq1}
      &\equalsby{\eqref{eq:capprod2}}
        \capped
        {\SQ\left(\thomiso\left(x\cup\Vu\right)\right)}
        {\pf\collapse\fundcl{\Sph}}
    \end{align}
    Having introduced $\SQ$ on the left hand side, one observes a
    certain commutativity of the Thom isomorphism, and the total
    Steenrod square:
    \begin{align}\notag
      \SQ\left(\thomiso\left(x\cup\Vu\right)\right)
      &=
        \SQ\left(\left(\pb p x\cup \pb p\Vu\right) \cup \U\right)
      \\\notag
      &=
        \SQ\left(\pb p x\cup \left(\pb p\Vu \cup \U\right)\right)
      \\\notag
      &\equalsby{Def. $\thomiso$}
       \SQ\left(\pb p x\cup \thomiso(\Vu)\right)
      \\\notag
      &\equalsby{Naturality, Ring Hom.}
        \pb p\SQ(x) \cup \SQ(\thomiso(\Vu))
      \\\notag
      &\equalsby{Def. $\V{M}$}
        \pb p\SQ(x) \cup \SQ(\antipode(\SQ)(u))
      \\\notag
      &\equalsby{Def. $\antipode$}
        \pb p\SQ(x) \cup u
      \\\label{eq:proofaltdefwuclasses:eq2}
      &\equalsby{Def. $\thomiso$}
        \thomiso(\SQ(x))
    \end{align}
    Inserting \eqref{eq:proofaltdefwuclasses:eq2} into
    \eqref{eq:proofaltdefwuclasses:eq1} from above easily yields the
    claim in \eqref{eq:proofaltdefwuclasses:claim} that proves the theorem:
    \begin{align*}
      \capped{x\cup \Vu}{\fundcl M}
      &\cequalsby{\eqref{eq:proofaltdefwuclasses:eq1}}
        \capped
        {\SQ\left(\thomiso\left(x\cup\Vu\right)\right)}
        {\pf\collapse\fundcl{\Sph}}
      \\
      &\cequalsby{\eqref{eq:proofaltdefwuclasses:eq2}}
        \capped
        {\thomiso(\SQ(x))}
        {\pf\collapse\fundcl{\Sph}}
      \\
      &\equalsby{\autoref{lem:thomisoself-adjoint}}
        \capped
        {\SQ(x)}
        {\thomiso\left(\pf\collapse\fundcl{\Sph}\right)}
      \\
      &\equalsby{\autoref{lem:thomisofundcl}}
        \capped
        {\SQ(x)}
        {\fundcl M}
        \qedhere
    \end{align*}
  \end{proof}
\end{Thm}


\subsection{Wu's Theorem}\label{sec:wutheorem}
This section is meant to show a direct connection between the
Stiefel-Whitney classes and the Wu classes, which can be shown using
the alternative definition of the Wu classes from
\autoref{thm:altdefwuclasses}.

\begin{Thm}[Wu]\label{thm:wu}
  A closed manifold gives rise to the following two equalities
  \begin{align}\notag
    \dualW{M} &= \SQ\left(\dualV{M}\right)
    &\text{respectively}&
    &\dualw{k}{M} &= \sum_{i\geq0} \Sq i\left(\dualv{k-i}{M}\right)
    &\text{and}
    \\
    \W{M} &= \SQ\left(\V{M}\right)
    &\text{respectively}&
    &\w{k}{M} &= \sum_{i\geq0} \Sq i\left(\v{k-i}{M}\right)
    \label{tag:wuformula}\tag{Wu's formula}
  \end{align}
  that are equivalent using
  $\dualW{M}\cup\W{M} = 1$ and
  \begin{gather*}
    \SQ(\dualV{M})\cup\SQ(\V{M})
    = \SQ(\dualV{M}\cup\V{M})
    = \SQ(1)
    = 1
    \;.
  \end{gather*}
\end{Thm}
% \begin{Rem}
%   The following is equivalent to Wu's formula
%   \begin{gather*}
%     \dualW{M} = \SQ\left(\dualV{M}\right)
%     \qquad\text{respectively}\qquad
%     \dualw{k}{M} = \sum_{i\geq0} \Sq i\left(\dualv{k-i}{M}\right)
%   \end{gather*}
%   using the inverses $\dualW{M}\cup\W{M} = 1$ and
%   \begin{gather*}
%     \SQ(\dualV{M})\cup\SQ(\V{M})
%     = \SQ(\dualV{M}\cup\V{M})
%     = \SQ(1)
%     = 1
%     \;.
%   \end{gather*}
% \end{Rem}
The proof uses the alternative characterisation
\begin{gather*}
  \antipode(\SQ)(\u{\N{\emb}}) = \thomiso(\V{\N{\emb}})
\end{gather*}
of the Wu classes from \autoref{thm:altdefwuclasses}, and quite directly
follows from the following Lemma:
\begin{Lem}\label{lem:wu}
  For a vector bundle $\xi\colon \E\xi\xrightarrow{p}\B\xi$ over a
  paracompact space holds
  \begin{gather*}
    \SQ(\V{\xi}) \cup \W{\xi}= 1
    \;.
  \end{gather*}
\end{Lem}
\begin{proof}[proof of \ref{tag:wuformula}]
  Lemma \autoref{lem:wu} states for a closed manifold $M^n$ and any
  embedding $\emb\colon M\to\R^{n+k}$ that
  \begin{gather*}
    \SQ(\V{M}) \cup \dualW{M}
    = \SQ(\V{\N M}) \cup \W{\N{\emb}}
    \cequalsby{\autoref{lem:wu}} 1
    \;.
  \end{gather*}
  Cupping with $\W{M}=\W{\T M}$ on both sides yields the claim
  as $\W{\T M} \cup \W{\N{\emb}} = 1$ by
  \itemref{rem:propswclasses}{item:wuclassmfdinverse}.
\end{proof}
\begin{proof}[proof of Lemma~\autoref{lem:wu}]
  For simplicity use the shortenings
  $\U\coloneqq \u{\xi}$,
  $\Vu\coloneqq \V{\xi}$, and
  $\Ws\coloneqq \W{\xi}$.
  Then calculate
  \begin{align*}
    \thomiso(1)
    &= \U
    \\
    &\equalsby{Def.~\autoref{def:antipode}}
      \SQ\left(\antipode(\SQ)(\U)\right)
    \\
    &\equalsby{Def.~\autoref{def:altwuclasses}}
      \SQ\left(\thomiso(\Vu)\right)
    \\
    &\equalsby{Def.~$\thomiso$}
      \SQ\left(\pb p\Vu \cup \U\right)
    \\
    &\equalsby{\ref{tag:cartan}, Naturality}
      \pb p\SQ(\Vu) \cup \SQ(\U)
    \\
    &\equalsby{\autoref{thm:altdefswclasses}}
      \pb p\SQ(\Vu) \cup \thomiso(\Ws)
    \\
    &\equalsby{Def.~$\thomiso$}
      \pb p\SQ(\Vu) \cup \left(\pb p \Ws \cup \U\right)
    \\
    &=
      \pb p\left(\SQ(\Vu) \cup \Ws\right) \cup \U
    \\
    &\equalsby{Def.~$\thomiso$}
      \thomiso\left(\SQ(\Vu) \cup \Ws\right)
  \end{align*}
  Applying the inverse of the Thom isomorphism to both sides
  yields the equality which was to be shown.
\end{proof}


\section{Massey's Theorem on the Stiefel-Whitney Classes of Manifolds}
Massey's main theorem on the Stiefel-Whitney classes of manifolds
gives a very concrete obstruction on what degrees of the dual
Stiefel-Whitney classes may be non-zero.
More precisely, it formulates a connecting to the existence of binary
representations of the manifold's dimension.
\begin{Thm}[Massey]\label{thm:massey}
  \optcite[Theorem~I.]{massey}
  Let $M$ be a compact, $n$-dimensional manifold.
  Given an integer $q$ with $0<q<n$ such that $\dualw{n-q}{M}\neq0$,
  there is a sequence of integers $h_1\geq\dotsb\geq h_q\geq0$ of
  length $q$ that fulfils
  \begin{gather*}
    n = \sum_{i=1}^{q} 2^{h_i}
  \end{gather*}
\end{Thm}

As an immediate consequence all dual Stiefel-Whitney classes of degree
greater than $n-\alpha(n)$ of any manifold must be zero, because there
cannot be any shorter representation of $n$ by powers of two
than its binary representation of length $\alpha(n)$.

The following sections are dedicated to the proof of Massey's Theorem.
Let $M$ be a compact, $n$-dimensional manifold throughout the proof.
The latter consists of several steps:
\begin{steps}
\item\label{tag:masseystep1} 
  Show that for any $q$, admissible iterated Steenrod square $\Sq I$, and cohomology
  class $x\in\H^q(M)$ of degree $q$ such that $\Sq I(x)$ is non-trivial there exists
  some representation of the form
  \begin{gather*}
    \deg \left(\Sq I(x)\right)
    = 2^k\cdot
    \left( 2^{k_1}+\dotsb+2^{k_{q-1}} + 1 \right)
    \;.
  \end{gather*}
\item\label{tag:masseystep2}
  Find some iterated Steenrod square which is non-trivial in degree
  $\Sq I\colon\H^q(M)\to\H^n(M)$.
\end{steps}
Applying \ref{tag:masseystep1} to the Steenrod square $\Sq I$ from
\ref{tag:masseystep2} and some $x\in\H^q(M)$ with $\Sq I(x)\neq 0$
immediately yields the result as
\begin{gather*}
  n = \deg\left(\Sq I(x)\right) = \underbrace
  {2^{k_1+k}+\dotsb+2^{k_{q-1}+k} + 2^{k}}_{\text{$q$ summands}}
  \;.
\end{gather*}

\subsection[Degree Dissection for Steenrod Squares]
{Step 1: A Degree Dissection for Steenrod Squares}
\ref{tag:masseystep1} requires to proof the following claim.
\begin{Lem}[\ref{tag:masseystep1}]\label{lem:masseystep1}
  Let $q\geq 0$ be an integer,
  and $I\in\Nat^{\l(I)}$ an admissible sequence of integers.
  Further, let $x\in\H^q(M)$ be a cohomology class of degree $q$
  such that $\Sq I(x)$ is non-trivial.
  Then there exists $k\in\Nat$ and a sequence of integers
  $k_1\geq\dotsb\geq k_{q-1}\geq0$ of length $q-1$ such that the
  degree of $\Sq I(x)$ can be represented as the dissection
  \begin{gather*}
    \deg \Sq I(x)
    = \deg x + \d(I)
    = 2^k\cdot
    \left( 2^{k_1}+\dotsb+2^{k_{q-1}} + 1 \right)
    \;.
  \end{gather*}
  % Mind that the maximum length of such a sequence is $\deg\Sq I(x)$.
\end{Lem}

In order to split the proof into several cases, recall that
$\Sq I(x) = 0$ for $\e(I)>\deg x$ by
\itemref{rem:sq}{item:squpperboundgeneral}. This leaves the two cases
$\e(I)<\deg x$ and $\e(I)=\deg x$.
Inductively applying the following Lemma by Serre restricts the
proof of \autoref{lem:masseystep1} to the first case where $\e(I)<q$.
\begin{Lem}[Serre]
  \label{lem:serre}
  Every admissible sequence $I\in\Nat^{\l(I)}$ of excess $\e(I)>0$
  admits an admissible sequence $J$ with $\e(J)<\e(I)$,
  together with some $k\in\Nat$
  such that for any cohomology class $x\in\H^{\e(I)}$ holds
  \begin{gather*}
    \Sq I(x) = \left(\Sq J(x)\right) ^{2^k}
    \qquad\text{respectively}\qquad
    \deg\left(\Sq I(x)\right) = 2^k\cdot \deg\left(\Sq J(x)\right)
  \end{gather*}
\end{Lem}

Before proving \autoref{lem:serre} one can finish the argumentation for
the case $\e(I)<\deg x$.
\begin{proof}[proof of \autoref{lem:masseystep1}]
  Let $q\in\Nat$ and $I=(i_1,\dotsc,i_l)$ be an admissible sequence such that
  $\e(I)<q$.
  Assume there is a cohomology class $x\in\H^q(M)$ such that $\Sq I(x)\neq0$.
  Set
  \begin{alignat*}{4}
    \alpha_0 &= q-1-\e(I)\geq 0 &\qquad&\text{which is positive as $\e(I)<q$,} \\
    \alpha_r &= i_{r}-2i_{r+1}  &&\text{for $1\leq r< \l(I)$, and} \\ 
    \alpha_{\l(I)} &= i_{\l(I)}
    \;.
  \end{alignat*}
  It is an easy exercise that the excess of $I$ can be rewritten as
  $\e(I)=\sum_{r=1}^{\l(I)}\alpha_r$, so
  \begin{gather}\label{eq:alpha0}
    \sum_{r=0}^{\l(I)}\alpha_r = \alpha_0 + \e(I) \cequalsby{Def.} q-1
    \;.
  \end{gather}
  Just as easily one sees
  \begin{align}\label{eq:proofmassey:eq1}
    i_s
    &= \sum_{r=0}^{s}2^r\alpha_{s+r}\\\notag
  \end{align}
  The above definitions then directly imply the following reformulation
  of $\d(I)$ in terms of $\alpha_i$:
  \begin{align}\notag
    \d(I) \coloneqq \sum_{s=1}^{\l(I)}i_s 
    &\cequalsby{\eqref{eq:proofmassey:eq1}}
      \sum_{s=1}^{\l(I)}\sum_{r=0}^{s}2^r\alpha_{s+r}\\\notag
    &\equalsby{Reorder} \sum_{j=1}^{\l(I)}
      \left(\sum_{m=0}^{j-1}2^m\right)\alpha_j
      = \sum_{j=1}^{\l(I)}(2^j-1)\alpha_j 
      = \sum_{j=1}^{\l(I)}2^j\alpha_j
      - \sum_{j=1}^{\l(I)}\alpha_j\\
    \label{eq:proofmassey:eq2}
    &= \sum_{j=1}^{\l(I)}2^j\alpha_j
      - \e(I)
      \;.
  \end{align}
  All put together yields
  \begin{align*}
    \deg\left(\Sq I(x)\right)
    &= \deg(x) + \d(I)\\
    &= 1 + \deg(x) -1 +\d(I) \\
    &\equalsby{\eqref{eq:proofmassey:eq2}}
      1 + q - 1 - \e(I) + \sum_{j=1}^{\l(I)}2^j\alpha_j \\
    &\equalsby{Def.}
      1 + \alpha_0 + \sum_{j=1}^{\l(I)}2^j\alpha_j \\
    &= 1 + \sum_{j=0}^{\l(I)}2^j\alpha_j
    = 1 + \left(
      \underbrace{2^0 +\dotsb+ 2^0}_{\text{$\alpha_0$ times}}
      + \dotsb
      + \underbrace{2^{\l(I)}+\dotsb+2^{\l(I)}}_{\text{$\alpha_{\l(I)}$ times}}
      \right)
  \end{align*}
  which is one plus a sum of exactly
  $\sum_{j=0}^{\l(I)}\alpha_j\cequalsby{\eqref{eq:alpha0}} q-1$
  powers of two as was to be shown.
  The $k_j$ are in this case
  \begin{gather*}
    k_j = \begin{cases}
      0 & 0< j\leq \alpha_0\\
      1 & \alpha_0< j\leq \alpha_1\\
      \vdots\\
      \l(I) & \alpha_{\l(I)-1}<j\leq \alpha_{\l(I)}
    \end{cases}
    \qedhere
  \end{gather*}
\end{proof}

\begin{proof}[proof of \autoref{lem:serre}]
  \optcite[Lemma~1, converse part, p.~159]{serre}
  First note that any admissible sequence $I\coloneqq(i_1,\dotsc,i_l)\in\Nat^l$
  of excess $\e(I)>0$ can be written as
  $I=(2^{k-1}i_k,\dotsc,2i_k,i_k,i_{k+1},\dotsc,i_l)$
  with $l>k\geq 1$ chosen maximal, \idest $i_k>2i_{k+1}$. If $I$ is admissible, the subsequence
  $J\coloneqq(i_{k+1},\dotsc,i_l)$ will be admissible as well.
  In order to see that such $J$ and $k$ fulfil the requirements from
  \autoref{lem:serre} one only needs to show
  \begin{claim}
    For $I$ and $J$ as above, and $x\in\H^{\e(I)}(X)$ a cohomology
    class of a space $X$ holds
    \begin{enumerate}
    \item $\Sq I(x) = \left(\Sq J(x)\right)^{2^k}$, and
    \item $\e(I)<\e(J)$.
    \end{enumerate}
  \end{claim}
  For the first part one simply has to check that
  $\deg(\Sq J(x))=i_k$, as the statement then inductively follows from
  property \eqref{eq:sqsquared} that $\Sq i(y)=y^2$ for any cohomology
  class with $i=\deg(y)$.
  So calculate
  \begin{align*}
    \deg(\Sq J(x))
    &= \d(J) + \deg(x)\\
    &= \d(J) + \e(I)\\
    &= \d(J) 
      + \left(\sum_{r=1}^{\l(I)-1}(i_r-2i_{r+1})\right)
      + i_{\l(I)}\\
    &= \d(J)
      + \left(\sum_{r=1}^{k-1}\underbrace{(i_r-2i_{r+1})}_{=0}\right)
      + (i_k-2i_{k+1})
      +\left(\sum_{r=k+1}^{\l(I)-1}(i_r-2i_{r+1})\right) + i_{\l(I)}\\
    &= \d(J)
      + (i_k-2i_{k+1})
      + \e(J) \\
    &= \d(J) + i_k - 2i_{k+1} + 2i_{k+1} - \d(J) \\
    &= i_k
  \end{align*}
  Comparing the excesses yields the second part:
  \begin{gather*}
    \e(I) - \e(J)
    = \left(\sum_{r=1}^{k-1}\underbrace{(i_r-2i_{r+1})}_{=0}\right)
    + \underbrace{(i_k - 2i_{k+1})}_{\text{$>0$ by def. of k}}
    > 0
    \qedhere
  \end{gather*}
\end{proof}


\subsection[Non-trivial Iterated Steenrod Squares for Manifolds]
{Step 2: Existence of Non-trivial Iterated Steenrod Squares for Manifolds}

This section is dedicated to finding an iterated Steenrod square
$\Sq I$ of degree $n-q$ that is non-trivial in degree
$\Sq I\colon\H^q(M)\to\H^n(M)$ in order to complete
\ref{tag:masseystep2}
and thus prove Theorem~\autoref{thm:massey}.

The candidate is the multiplication map
\begin{gather*}
  \H^q(M) \longto \H^n(M)
  \qquad
  x\longmapsto x\cup\dualw{n-q}{M}
\end{gather*}
which is non-trivial as the cup-product is non-degenerated
by Poincaré duality \cite[Proposition~3.38]{hatcher}.
It remains to see that this map actually comes from a (sum of)
iterated Steenrod square(s) which then must be non-trivial.
But this easily follows from the next main Lemma,
the proof of which essentially uses Wu's Theorem and the properties of
the Wu classes.

\begin{Lem}\label{lem:masseystep2}
  Let $0<k<n$ and $x\in\H^k(M)$ be a cohomology class.
  Then
  \begin{gather*}
    x\cdot\dualw{n-k}{M} = \sum_{i>0}\Sq i(x)\cdot\dualw{n-k-i}{M}
  \end{gather*}
\end{Lem}
Applying this inductively with the induction step
\begin{gather*}
  \Sq I(x)\cdot\dualw{n-k-j}
  = \sum_{i>0}\Sq i\circ\Sq I(x)\cdot\dualw{n-k-(i+j)}{M}
\end{gather*}
yields the needed result.

\begin{proof}[proof of \autoref{lem:masseystep2}]
  As above let $0<k<n$ and $x\in\H^k(M)$.
  For simplicity write
  $\ws i\coloneqq\w i M$, $\dualws i\coloneqq\dualw i M$,
  $\vu i = \v i M$ and $\dualvu i = \dualv i M$.
  In order to translate from Stiefel-Whitney classes to Steenrod
  squares and back again recall the following results.
  \begin{description}
  \item[By Theorem \autoref{thm:wu} (Wu):]
    $\sum_{j=0}^{s} \Sq j(\dualvu{s-j}) = \dualws{s-j}$
    for any $s\leq n$.
  \item[By Definition~\autoref{def:wuclasses} of $\Vu$:]
    $\vu i \cdot y = \Sq i(y)$
    for any $i\leq n$ and $y\in\H^{n-i}(M)$.
  \item[By Definition~\autoref{def:dualwuclasses} of $\dualVu$:]
    $\dualvu d = \sum_{i=1}^{d} \vu i\cdot \dualvu{d-i}$ for $d>0$.
  \end{description}
  The full calculation is then
  \begin{align*}
    x\cdot \dualws{n-k}
    &\cequalsby{\autoref{thm:wu}}
      x\cdot \sum_{i=0}^{n-k} \Sq i(\dualvu{n-k-i}) \\
    &\equalsby{Def. $\Sq 0$}
      x\cdot\left(
      \dualvu{n-k} + \sum_{i=1}^{n-k} \Sq i(\dualvu{n-k-i})
      \right) \\
    &\equalsby{Def. $\dualVu$}
      x\cdot\left(
      \left(\sum_{i=1}^{n-k} \vu{i}\cdot\dualvu{n-k-i}\right)
      + \left(\sum_{i=1}^{n-k} \Sq i(\dualvu{n-k-i})\right)
      \right) \\
    &=
      \sum_{i=1}^{n-k} \left(
      \vu{i}\cdot x\cdot \dualvu{n-k-i}
      + x\cdot \Sq i(\dualvu{n-k-i})
      \right) \\
    &\equalsby{Def. $\Vu$}
      \sum_{i=1}^{n-k} \left(
      \Sq i (x\cdot \dualvu{n-k-i})
      + x\cdot \Sq i(\dualvu{n-k-i})
      \right) \\
    &\equalsby{\ref{tag:cartan}, Def. $\Sq 0$}
      \sum_{i=1}^{n-k} \left(
      \left( \sum_{r=0}^{i} \Sq r(x)\cdot \Sq{i-r}(\dualvu{n-k-i}) \right)
      + \Sq 0(x)\cdot \Sq i(\dualvu{n-k-i})
      \right) \\
    &=
      \sum_{i=1}^{n-k} \left(
      \sum_{r=1}^{i} \Sq r(x)\cdot \Sq{i-r}(\dualvu{n-k-i})
      \right) \\
    &\equalsby{Reorder}
      \sum_{r=1}^{n-k} \Sq r(x) \cdot
      \left( \sum_{j=0}^{n-k-r} \Sq j(\dualvu{n-k-(j+r)}) \right) \\
    &\equalsby{\autoref{thm:wu}}
      \sum_{r=1}^{n-k} \Sq r(x) \cdot\dualws{n-k-r}
      \qedhere
  \end{align*}
\end{proof}

\section{Best Possible Result}
Recall that a closed $n$-manifold can only immerse into
$\R^{2n-a}$ if all Stiefel-Whitney classes $\dualw{i}{\N M}$ of its
normal bundle are zero in degrees greater than $n-a$.
Massey's Theorem now states that this condition is met for
$a=\alpha(n)$, and thus suggests the conjecture that every
manifold immerses into $\R^{2n-\alpha(n)}$.

Naturally, there occurs the question whether Massey's $\alpha(n)$ is
the best possible (\idest largest) result for arbitrary closed
$n$-manifolds.
The answer is yes, proved by the following counterexamples of manifolds
not immersing into $\R^{2n-(\alpha(n)+1)}$.

\begin{Thm}
  \optcite[p.~87]{immersionconj}
  Denote by $\RP{i}$ the $i$th real projective space.
  \begin{enumerate}
  \item For $n=2^i$, the Stiefel-Whitney class
    $\dualw{n-\alpha(n)}{\RP{2^i}}$ is not zero.
  \item For $n\in\Nat$ with binary expansion
    $n=\sum_{r=1}^{q}2^{i_r}$, $i_1>\dotsb>i_q$, the Stiefel-Whitney class
    $\dualw{n-\alpha(n)}{\prod_{r=1}^{q}\RP{2^r}}$ is not zero.
  \end{enumerate}
  \begin{proof}
    The total Stiefel-Whitney class of $\RP{n}$ for arbitrary
    $n\in\Nat$ is
    \begin{gather*}
      \W{\RP n} = (1+x)^{n+1}\in\H^*(\RP n)\cong \Zmod2[x]/(x^{n+1})
    \end{gather*}
    (see \forexample \cite[Example~(19.4.1)]{tomdieck}).
    For $n=2^i$ this takes the form
    \begin{align*}
      \W{\RP{2^i}}
      &= (x+1)^{2^i+1}
      = 1 + x + x^{2^i}
        \text{ , thus} \\
      \dualW{\RP{2^i}}
      &= \sum_{r=0}^{2^i-1} x^r
        \text{ , especially with $\alpha(2^i)=1$} \\
      \dualw{n-\alpha(n)}{\RP{2^i}}
      &= x^{2^i-1}\neq 0
      %   \quad\text{, as} \\
      % \W{\RP{2^i}}\cdot\dualW{\RP{2^i}}
      % &= \sum_{r=0}^{2^i-1}x^r
      % + \sum_{r=1}^{2^i}x^r
      % + \sum_{r=0}^{2^i-1}x^{2^i}\cdot x^r \\
      % &= \left( 1 + \sum_{r=1}^{2^i-1}x^r \right)
      %   + \left( \sum_{r=1}^{2^i-1}x^r + x^{2^i} \right)
      %   + x^{2^i}
      %   = 1
    \end{align*}
    which proves the first statement.
    For the second statement where $n=\sum_{r=1}^q 2^{i_q}$ note that
    \begin{gather}\label{eq:bestresultproof1}
      n-\alpha(n)
      = \left( \sum_{r=1}^q 2^{i_r} \right) - q
      = \sum_{r=1}^q \left( 2^{i_r} - 1 \right)
      \;.
    \end{gather}
    Using multiplicativity %(\ref{tag:swclassesmultiplicativity})
    of the Stiefel-Whitney classes one gets
    \begin{align*}
      \dualW{\prod_{r=1}^q \RP{i_r}}
      &= \prod_{r=1}^q \dualW{\RP{i_r}}
        \text{ , and especially by \eqref{eq:bestresultproof1}} \\
      \dualw{n-\alpha(n)}{\prod_{r=1}^{q}\RP{i_r}}
      &= \prod_{r=1}^q x^{2^r-1}
        = x^{n-\alpha(n)} \neq 0
        \qedhere
    \end{align*}
  \end{proof}
\end{Thm}


\section{Outlook: Generalisation of Massey's Theorem} % TODO
\optcite[Section~II.1]{immersionconj}


%%% Local Variables:
%%% mode: latex
%%% TeX-master: "thesis"
%%% End:

