%%%%%%%%%%%%%%%%%%%%%%%%%%%%%%%%%% 
% Master Thesis in Mathematics
% "Immersions and Stiefel-Whitney classes of Manifolds"
% -- Chapter 2: Homotopy Theoretical Formulation of the Immersion Problem --
% 
% Author: Gesina Schwalbe
% Supervisor: Georgios Raptis
% University of Regensburg 2018
%%%%%%%%%%%%%%%%%%%%%%%%%%%%%%%%%%


\chapter{Homotopy Theoretical Formulation of the Immersion Problem}
% Explain carefully how the immersion problem can be reformulated purely
% in terms of homotopy theory. [immersionconj] (The necessary results from differential
% topology and Hirsch-Smale theory should be stated clearly but may be
% presented without proofs.)

  
Immersions are merely a special case of monomorphisms of vector
bundles. So, recall that
a morphism $(\xi_1\colon\E_1\to X_1)\to(\xi_2\colon E_2\to X_2)$
of vector bundles over different spaces is a map $F\colon E_1\to E_2$
that covers its restriction to the zero section, \idest it makes the
following diagram commute
\begin{center}
  \begin{tikzcd}
    E_1 \ar[r,"F"]\ar[d, "\xi_1"]
    & E_2 \ar[d, "\xi_2"]
    \\
    X_1 \ar[r, "F|_{\zerosec{\xi_1}}"]
    & X_2
  \end{tikzcd}.
\end{center}
Further remember the fact that such a morphism is a monomorphism in
the category of vector bundles $\Vect$, if and only if its restriction
to each fibre is injective.
\begin{Def}
  A (smooth) map $f\colon M\to N$ of smooth manifolds is called
  an \emph{immersion} if its differential
  $\Diff f\colon\T M\to\T N$ is a monomorphism of vector
  bundles.
\end{Def}
Note that not all vector bundle monomorphisms between tangent bundles
of smooth manifolds need to be 

Later, an important result due to Hirsch and Smale will be needed that
relates spaces of immersions and vector bundle monomorphisms.
For the formulation, one has to equip the respective sets with a
topology as follows.
\begin{Def}
  Let $M$, $N$ be closed smooth manifolds of dimensions $\dim M < \dim N$.
  \begin{enumerate}
  \item % TODO: Whitney $C^r$-topology; see [Hirsch, Differential Topology, Chap 2, p.35]
  \item
    Equip the set of all vector bundle monomorphisms from $\xi_1$ to
    $\xi_2$ with the compact-open topology, and denote that space by
    $\Mono{\xi_1}{\xi_2}$.
  \item % TODO: (GEORGE) check topology of \Imm(M, N); compare Lecture_Notes_on_Immersions_of_Surfaces_in_3-Space--Nowik.ps
    The set $\Imm M N$ of all immersions from $M$ to $N$ injects
    into $\Mono{\T M}{\T N}$ by taking the differential
    $f\mapsto\Diff f$.
    Equip $\Imm M N$ in the following with the subspace topology.
    This results in the weak topology described in
    \cite[Section~2.1]{hirsch}, which equals the Whitney
    $C^1$-topology since $M$ was chosen compact.
    By the way, $\Imm M N$ is open in $C^1(M,N)$ equipped with the
    Whitney $C^1$-topology
    (see \cite[Section~2.1, Theorem~1.1]{hirsch}),
    and thus not a discrete space.
  \end{enumerate}
\end{Def}


\begin{Thm}\label{thm:immersionconj}
  For $n\in\Nat$, every closed, smooth, $n$-dimensional manifold
  immerses into $\R^{2n-\alpha(n)}$, where $\alpha(n)$ is defined as
  the minimal number of summands in any binary representation of $n$.
\end{Thm}

\begin{Rem}[Stable Normal Bundle] % TODO
  
\end{Rem}

\begin{Thm}\label{thm:immersionconj:equalities}
  Let $n,k\in\Nat$ and $M^n$ be a closed, smooth, $n$-dimensional manifold.
  The following statements are equal.
  \begin{enumerate}
  \item\label{item:immersionconj:1} $M$ immerses into $\R^{n+k}$.
  \item\label{item:immersionconj:2} There is a (normal) $k$-dimensional vector bundle
    $\N{}\colon\E{\N{}}\to M$ over $M$ with
    \begin{gather*}
      \N{}\oplus\T M\cong\trivbdl^{n+k}
      \;.
    \end{gather*}
  \item\label{item:immersionconj:3}
    For the map $\N M\colon M\to\B\Orth$ characterising the stable
    normal vector bundle over $M$
    there is a lift $\N{}\colon M\to\B\Orth(k)$ making the following
    diagram commute up to homotopy
    \begin{center}
      \begin{tikzcd}
        M
        \ar[r, "\N{}"]
        \ar[dr, "\N{M}"{left}, bend right]
        & \BO(k) \ar[d, "\incl"] \\
        & \BO
      \end{tikzcd}
    \end{center}
  \end{enumerate}
\end{Thm}
From a homotopy theoretical viewpoint, one obviously is most
interested in statement \ref{item:immersionconj:3}.
So, before passing over to the proof, have a look at some first
advantages of this approach.

For one, this formulation points out a certain homotopy invariance of
the immersion problem, namely an $n$-manifold already satisfies the
immersion property if it is \emph{homotopic} to one doing so. 
The interesting intermediate result that every $n$-manifold is
at least \emph{cobordant} to one fulfilling the conjecture's property
will be discussed in \autoref{chap:brown}.

Moreover as a second major advantage, statement
\ref{item:immersionconj:3} directly puts the immersion problem in the
setting of characteristic classes, more precisely Stiefel-Whitney
classes, which provides a formidable further arsenal of tooling.
As an example, the motivation for the exact value of the immersion
dimension will be deduced in \autoref{chap:massey}.

\begin{proof}[proof of \autoref{thm:immersionconj:equalities}]
  Using Steenrod's classification theorem
  \itemref{def:charcls}{item:classificationthm} 
  and the properties of the stable normal bundle of a manifold
  directly yields the equality of \ref{item:immersionconj:2} and
  \ref{item:immersionconj:3}.

  The trickier part is to relate \ref{item:immersionconj:1} and
  \ref{item:immersionconj:2}, even though it is easily seen that
  \ref{item:immersionconj:1} implies \ref{item:immersionconj:2} by
  simply taking $\N{}$ to be the normal bundle of the immersion from
  \ref{item:immersionconj:1}. The converse direction however heavily
  relies on the following major result in immersion theory by Hirsch
  using preliminary work of Smale
  \cite[Sections~5 and 6]{hirschimmersions}.
  
  \begin{Thm}[Hirsch-Smale]
    \optcite[Theorem~1.2]{immersionconj} % TODO: better ref for modern formulation
    Let $M$, $N$ be closed manifolds with $\dim M<\dim N$.
    Then the differential map
    $\Diff\colon \Imm M N\to \Mono{\T M}{\T N}$
    induces isomorphisms on the homotopy groups.
    Especially,
    \begin{gather*}
      \Diff_*\colon
      \pi_*(\Imm M N) \overset\sim\longto \pi_*(\Mono{\T M}{\T N})
    \end{gather*}
    describes an isomorphism of path-connected components.
    % Original formulation by Hirsch in [hirschimmersions], sec. 5:
    Therefore, every vector bundle monomorphism
    $F\colon\T M\to\T N$ is homotopic (through vector bundle
    monomorphisms) to a monomorphism which is the differential
    $\Diff f$ of a smooth map $f\colon M\to N$, \idest of an
    immersion.
  \end{Thm}
\end{proof}


%%% Local Variables:
%%% mode: latex
%%% TeX-master: "thesis"
%%% End:
