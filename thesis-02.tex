%%%%%%%%%%%%%%%%%%%%%%%%%%%%%%%%%% 
% Master Thesis in Mathematics
% "Immersions and Stiefel-Whitney classes of Manifolds"
% -- Chapter 2: General Obstructions by Characteristic Classes --
% 
% Author: Gesina Schwalbe
% Supervisor: Georgios Raptis
% University of Regensburg 2018
%%%%%%%%%%%%%%%%%%%%%%%%%%%%%%%%%% 

\chapter{Vanishing Obstructions by Characteristic Classes} % TODO: nicer title
\label{chap:massey}
% Review the theory of Stiefel-Whitney classes and the properties of the
% Steenrod algebra (as needed). Give a careful and detailed exposition of
% the proof of Massey’s theorem. Explain that the estimate in Massey’s
% theorem is the best possible in general.
% [massey], [immersionconj]
Recall from Corollary~\ref{cor:obstruction} that a manifold can
only immerse in codimension $k$ if its dual Stiefel-Whitney classes in
degrees higher than $k$ are zero.
This chapter is dedicated to the proof of a theorem by Massey
\cite{massey}, which says that this obstruction vanishes in general for
$k=n-\alpha(n)$ (\autoref{sec:massey}), and that this is the best
possible general result (\autoref{sec:bestpossibleresult}), \idest for
$k<n-\alpha(n)$ there are examples of manifolds not immersing into
$\R^{n+k}$.

The preliminary work for Massey's theorem encompasses a review on
Steenrod squares in \autoref{sec:steenrodsquares}, and the
investigation of Wu characteristic classes in
\autoref{sec:wuclassesmain}.
The latter especially involves relating the Wu classes to
Stiefel-Whitney classes using a formula by Wu, which is proven in
\autoref{sec:wutheorem}.

\section{Steenrod Squares}\label{sec:steenrodsquares}
Similar to the cup and cap product Steenrod squares add further
structure to the cohomology ring, thus adding more tools to
differentiate spaces by means of cohomology.
As applications, they will serve in constructing two kinds of
characteristic classes, the Stiefel-Whitney and the Wu classes, and
are a key ingredient for the proof of Massey's theorem on obstructions
for normal bundles of manifolds.

The following definition of the Steenrod squares is due to
Steenrod and Epstein \cite[Chap.~I.1, p.~1]{steenrodepstein}.
\begin{Def}\label{def:sq}
  The \emph{Steenrod squares} $\Sq i$ for $i\in\Nat$ are each a family
  of cohomology operations, \idest families of homomorphisms, of the
  form
  \begin{gather*}
    \left(
      \Sq i\colon \H^n(X, A;\Zmod2) \to \H^{n+i}(X, A;\Zmod2)
      \;\middle|\;
      n\in\Nat
    \right)
  \end{gather*}
  that satisfy the following relations for any pair of spaces $(X,A)$, and any map of
  pairs of spaces $f\colon (X,A)\to (Y,B)$:
  \begin{description}
  \item[(Naturality)]\label{item:sqnaturality}
    $\pb f\circ\Sq i = \Sq i\circ\pb f$.
  \item[(Stability)]\label{item:sqstability}
    $\susp\circ\Sq i = \Sq i\circ\susp$
    where $\susp$ is the suspension isomorphism on cohomology.
  \item[(Cartan formula)] For any $n\in\Nat$, and $x,y\in \H^n(X)$ holds
    \begin{gather}\label{tag:cartan}\tag{Cartan's formula}
      \Sq i(x\cup y) = \sum_{r+s=i}\Sq r(x)\cup\Sq s(y)
    \end{gather}
  \item[(Fixed values)] The following values are fixed for $x\in \H^n(X,A)$:
    % \begin{gather}\label{eq:sqlowerbound}
    %   \Sq i(x) = \begin{cases}
    %     0 & n<i \\
    %     x^2 & n=i
    %   \end{cases}
    %   \qquad\text{and}\qquad
    %   \Sq i = \begin{cases}
    %     \Id & i=0\\
    %     \beta & i=1
    %   \end{cases}
    % \end{gather}
    \begin{alignat}{4}
      \Sq i(x) &= 0     & \qquad\text{for }n<i \label{eq:sqlowerbound}\\
      \Sq i(x) &= x^2   & \qquad\text{for }n=i \label{eq:sqsquared}\\
      \Sq 0    &= \Id   \label{eq:sqidentity}
      % \\\notag
      % \Sq 1    &= \beta \label{eq:sqbockstein}
    \end{alignat}
    % where $\beta$ denotes the Bockstein homomorphism
    % \cite[see \forexample][Chap.~3.E]{hatcher} of the exact
    % coefficient sequence
    % \begin{center}
    %   \begin{tikzcd}
    %     0 \ar[r]
    %     &\Zmod2 \ar[r,"\incl"]
    %     &\Zmod4 \ar[r,"\proj"]
    %     &\Zmod2 \ar[r]
    %     &0
    %   \end{tikzcd}
    % \end{center}
  \item[(Adem relations)] For $\alpha<2\beta$ holds
    \begin{gather}\label{tag:adem}\tag{Adem's formula}
      \Sq\alpha \circ \Sq\beta =
      \sum_{j=0}^{\left\lfloor \frac \alpha 2 \right\rfloor}
      \binom{\beta-j-1}{\alpha-2j}
      \Sq{\alpha+\beta+j}\Sq{j}
    \end{gather}
  \end{description}
  $\SQ\coloneqq \sum_{j\in\Nat}\Sq j$ is the formal sum of all
  Steenrod squares called the \emph{the total Steenrod square}.
  Note that for any degree $n\in\Nat$ the total Steenrod square
  $\SQ\colon \H^n(X)\to \H^*(X)$ is well-defined since the sum is
  finite by \eqref{eq:sqlowerbound}.
  Also \ref{tag:cartan} can be reformulated to
  $\SQ(x\cup y) = \SQ(x)\cup\SQ(y)$, \idest $\SQ$ is a group
  homomorphism with respect to the cup-product.
\end{Def}

\begin{Thm}
  The Steenrod squares exist and are uniquely determined by
  naturality, \ref{tag:cartan}, and the fixed values
  %\eqref{eq:sqlowerbound}, \eqref{eq:sqsquared}, and
  %\eqref{eq:sqidentity}
  from Definition~\ref{def:sq}.
  \begin{proof}
    For existence see \cite[Chapter 2]{mosher},
    for uniqueness see \cite[VIII §3]{steenrodepstein}.
  \end{proof}
\end{Thm}

The fact that Steenrod squares, or in general cohomology operations,
can be added and concatenated, already gives a hint that they might
form a ring, which was proven by Steenrod. The following notation and
facts are according to \cite[Chap.~6]{mosher}.
\begin{Def}
  The \emph{Steenrod algebra} $\A$ is the quotient
  of the graded $\Zmod2$-polynomial algebra
  $\Zmod2[\Sq i|i\in\Nat]$ with grading $\deg \Sq i\coloneqq i$
  by the two-sided relations of both \ref{tag:adem} and $\Sq 0=1$.
  With the induced grading, multiplication, and the diagonal defined
  by $\Sq i\mapsto\sum_{r+s=i}\Sq r\otimes\Sq s$, it is an
  associative, connected, non-commutative graded Hopf algebra over
  $\Zmod2$.
\end{Def}
\begin{Not}
  In the following, iterated Steenrod squares
  $\Sq{i_1}\cdot\Sq{i_2}\dotsm\Sq{i_l}$ will have the short form
  $\Sq{(i_1,i_2,\dotsc,i_l)}$,
  and will be evaluated on an element $x$ of a cohomology ring as
  $\Sq{i_1}\circ\dotsb\circ\Sq{i_l}(x)$ respecting the
  properties from Definition~\ref{def:sq}.
  Furthermore, for a sequence $I=(i_1,\dotsc,i_l)$,
  respectively $\Sq I$, denote by
  \begin{description}[labelindent=1em]
  \item[$\l(I)\coloneqq l$] the \emph{length} of $I$,
  \item[$\d(I)\coloneqq \sum_{j=1}^{l} i_j$] the \emph{degree} of $I$,
    and by
  \item[$\e(I)\coloneqq 2i_1-\d(I)=\sum_{j=1}^{l-1}(i_j-2i_{j+1})$]
    the \emph{excess} of $I$.
  \end{description}
  The sequence $I$, respectively $\Sq I$, is called \emph{admissible},
  if $\l(I)=1$ or $i_j\geq 2i_{j+1}$ for $0\leq j<\l(I)$.
\end{Not}

\begin{Rem}\label{rem:sq}
  The following properties will be needed for Massey's Theorem:
  \begin{enumerate}
  \item The set of iterated Steenrod squares $\Sq I$ of admissible
    sequences $I$ forms a basis for $\A$ as $\Zmod2$-vector space
    \cite[Chap.~6, Theorem~1]{mosher}.
  \end{enumerate}
  Let $I=(i_1,\dotsc,i_l)$ be a sequence in $\Nat$.
  \begin{enumerate}[resume]
  \item $\deg(\Sq I(x)) = \deg(x) + \d(\Sq I)$.
  \item\label{item:squpperboundgeneral} $\Sq I(x) = 0$ for  $\deg(x)<\e(I)$ if $I$ is admissible.
    \begin{proof}
      This follows by induction over $\l(I)$ using
      \eqref{eq:sqlowerbound}. The case $\l(I)=1$ follows directly 
      from \eqref{eq:sqlowerbound}.
      For $\l(I)>1$ and $J\coloneqq(i_2,\dotsc,i_l)$, the condition
      $\deg(x) < \e(I)=i_1-\d(J)$
      implies
      \begin{gather*}
        \deg(\Sq J(x))
        = \deg(x)+\d(J) < \e(J)+\d(J) = 2i_2
        \overset{\text{adm.}}\leq i_1
      \end{gather*}
      So,
      $\Sq I(x)=\Sq{i_1}(\Sq J(x)) \cequalsby{\eqref{eq:sqlowerbound}} 0$.
    \end{proof}
  \end{enumerate}
\end{Rem}

The subsequent concept of formally inverting a formal sum of elements
in a graded $\Zmod2$-algebra will recur for several characteristic
classes. This particular definition is needed to define the Wu classes
in \autoref{sec:wuclasses}. 
\begin{Def}\label{def:antipode}
  The antipode $\antipode\colon \A\to\A$ of the Steenrod algebra is a
  graded homomorphism inductively defined by the relation
  \begin{gather*}
    1 = \Sq 0
    = \SQ \Sqcup \antipode(\SQ)
    = \sum_{k\geq0}\sum_{r+s=k} \Sq r \Sqcup \antipode(\Sq s)
  \end{gather*}
\end{Def}

\section{Wu Classes and the Wu Formula}\label{sec:wuclassesmain}
A crucial part in the proof of Massey's theorem is to cleverly involve
a new type of characteristic classes with nice properties that are
directly connected to Stiefel-Whitney classes through Steenrod
squares.
The Wu classes were developed by Wu~Wen-Tsün, and will be introduced in
\autoref{sec:wuclasses}.
The proof of Wu's theorem on their relation to Stiefel-Whitney
classes in \autoref{sec:wutheorem} utilises some properties of Thom
classes and the Thom isomorphism as reviewed in
\autoref{sec:thomclasses}, as well as a certain construction of
Stiefel-Whitney classes as introduced in
\autoref{sec:swclsconstruction}.


\subsection{Thom Classes and the Thom Isomorphisms}\label{sec:thomclasses}
In this subsection, Thom classes and the Thom isomorphisms will be
reviewed. The tools to be recalled in this section are an important
ingredient for the proof of Wu's theorem, both directly as well as for
constructing the Stiefel-Whitney characteristic classes from Steenrod
squares.

Let $B$ be a paracompact space, \forexample a manifold,
$\xi\colon E\xrightarrow{p} B$ a vector bundle over $B$ of rank $\rkk>0$,
and let $R$ be a principal ideal domain.
\begin{Def}
  A \emph{Thom class} of $\xi$ in $R$-coefficients is a
  cohomology class $\u{\xi}\in \H^{\rkk}(\spherepair{E}; R)$,
  such that for all points $b\in B$ and fibre inclusions
  $i_b\colon (\spherepair{E_b}) \to (\thomspacepair{E})$
  the restriction $\u{\xi}|_{E_b} = \pb i_b (\u{\xi})$ is a
  free generator of the $R$-module $\H^{\rkk}(\spherepair{E_b}; R)$,
  \idest a unit
  of the ring
  $\H^{\rkk}(\spherepair{E_b};R)\cong \H^{\rkk}(\spherepair{\R}; R)\cong R$.
  \optcite[p.~441]{hatcher}
\end{Def}

The following corollaries will deduce notions of naturality,
multiplicativity, and uniqueness for Thom classes quite directly from
their above definition.

\begin{Cor}\label{cor:thomclsnatural}
  The Thom class construction is natural with respect to the pullback
  of vector bundles over paracompact spaces.
  \Idest given any map of paracompact spaces $f\colon A\to B$, and a
  vector bundle $\xi\colon E\to B$ with pullback map
  $F\colon\pb f\xi\to\xi$,
  then $\pb F \U$ of a Thom class $\U$ of $\xi$ will
  be a Thom class of $\pb f \xi$.
  \begin{proof}
    Let $\U$ be a Thom class of $\xi$ and $a\in A$ any point.
    Consider the restriction $\pb i_a(\pb f \U)$
    of the pullback of $\U$ to the fibre over $a$. To show that this
    is a generator of $\H^{\rkk}(\spherepair{E_a};R)$ first use that
    pullbacks commute with restriction:
    \begin{gather*}
      \pb i_a(\pb f \U)
      = \pb {(f\circ i_a)} \U
      = \pb {(i_{f(a)}\circ f)} \U
      = \pb f (\pb i_{f(a)} \U)
    \end{gather*}
    $\pb i_{f(a)} \U$ is a generator by definition of $\U$.
    Now the restriction of $f$
    \begin{gather*}
      f\colon (\spherepair{(\pb f E)_a}) \to (\spherepair{E_{f(a)}})
    \end{gather*}
    is an isomorphism, and thus
    $\pb f\colon \H^r(\spherepair{E_{f(a)}})
    \cong \H^r(\spherepair{(\pb f E)_a})$
    sends generators to generators for all $r\in\Nat$.
  \end{proof}
\end{Cor}

\begin{Def}
  Let $\xi$, $\eta$ be vector bundles over a space $B$.
  Define the \emph{cross-product} as in \cite[p.~214]{hatcher} to be
  the map
    \begin{align*}
      \H^*(\thomspacepair{\E\xi})
      \otimes
      \H^*(\thomspacepair{\E\eta})
      &\longto
        \H^*(\thomspacepair{\E{(\xi\oplus\eta)}})\\
      x\otimes y
      &\longmapsto
        \pb \pi_\xi x \cup \pb \pi_\eta y
        \eqqcolon x\times y
        \;.
    \end{align*}
\end{Def}
\begin{Cor}\label{cor:thomclassmultiplicative}
  The Thom class construction for coefficients in a field $R$ is
  multiplicative in the following sense:
  For vector bundles $\xi\colon E\to B$, $\eta\colon E'\to B$
  of rank $\rkk$ respectively $\rkl$ over a paracompact space $B$, and Thom
  classes
  $\u{\xi}\in \H^{\rkk}(\thomspacepair{\E\xi}; R)$,
  $\u{\eta}\in \H^{\rkl}(\thomspacepair{\E\eta}; R)$
  the class
  \begin{gather*}
    \u{\xi}\times\u{\eta}
    \coloneqq \pb \pi_\xi\u{\xi} \cup \pb \pi_\eta\u{\eta}
    \in \H^{\rkk+\rkl}(\thomspacepair{\E{(\xi\oplus\eta)}})
  \end{gather*}
  is a Thom class of $\xi\oplus\eta$.
  \begin{proof}
    Consider a fibre $b\in B$. As cup product and pullback commute
    with restriction, the cross-product also commutes with
    restriction, \idest one has to show that
    \begin{gather*}
      \pb i_b(\u{\xi}\times\u{\eta})
      = \left(\pb i_b\u{\xi}\right)
      \times \left(\pb i_b\u{\eta}\right)
    \end{gather*}
    is a generator of
    $\H^{\rkk+\rkl}(\spherepair{\E{(\xi\oplus\eta)}_b};R)$.
    % There are homotopies making the following diagram commute
    % \begin{center}
    %   \begin{tikzcd}
    %     (\spherepair{\E{(\xi)}_b})
    %     \ar[d, dash, "\isosymb"{above,rotate=90}]
    %     &(\spherepair{\E{(\xi\oplus\eta)}_b})
    %     \ar[l,"\pi_\xi"above] \ar[r,"\pi_\eta"]
    %     \ar[d, dash, "\isosymb"{above,rotate=90}]
    %     &(\spherepair{\E{(\eta)}_b})
    %     \ar[d, dash, "\isosymb"{above,rotate=90}]\\
    %     (\spherepair{\R^i})
    %     &(\spherepair{\R^{i+j}})
    %     \ar[l,"\proj"above] \ar[r,"\proj"]
    %     &(\spherepair{\R^{j}})
    %   \end{tikzcd}
    % \end{center}
    On fibres one has that
    \begin{align*}
      (\spherepair{\E{(\xi\oplus\eta)}_b})
      &= \left(
        \E{\xi}_b\times\E{\eta}_b,
        \left(\E{\xi}_b\times(\minuszero{\E{\eta}_b})\right)
        \cup
        \left((\minuszero{\E{\xi}_b})\times\E{\eta}_b\right)
        \right)
        \\
      &= (\spherepair{\E{\xi}_b})\times(\spherepair{\E{\eta}_b}
    \end{align*}
    which makes the relative Künneth isomorphism theorem applicable.
    See \forexample \cite[Theorem~3.18]{hatcher}.
    By the naturality of that isomorphism there is the
    following commutative diagram that translates this problem to one
    on the cohomology of spheres (all cohomology rings with
    $R$-coefficients),
    similar to \cite[proof of Theorem~3.19, p.~221]{hatcher}:
    \begin{center}
      \begin{tikzcd}
        \H^*(\spherepair{\E\xi_b})
        \otimes \H^*(\spherepair{\E\eta_b})
        \ar[r, "\cong"]
        \ar[d, dash, "\cong"{above,rotate=90}]
        & \H^*(\spherepair{\E{(\xi\oplus\eta)}_b})
        \ar[d, dash, "\cong"{below,rotate=90}]
        \\
        \H^*(\spherepair{\R^{\rkk}})
        \otimes \H^*(\spherepair{\R^{\rkl}})
        \ar[r, "\cong"]
        \ar[d, dash, "\cong"{above,rotate=90}]
        & \H^*(\spherepair{\R^{\rkk+\rkl}})
        \ar[d, dash, "\cong"{below,rotate=90}]
        \\
        \H^*(I^{\rkk},\Boundary{I^{\rkk}})
        \otimes \H^*(I^{\rkl}, \Boundary{I^{\rkl}})
        \ar[r, "\cong"]
        \ar[d, dash, "\cong"{above,rotate=90}]
        & \H^*(I^{\rkk+\rkl}, \Boundary{I^{\rkk+\rkl}})
        \ar[d, dash, "\cong"{below,rotate=90}]
        \\
        \H^*(\Sphere{\rkk})
        \otimes \H^*(\Sphere{\rkl})
        \ar[r, "\cong"]
        & \H^*(\Sphere{\rkk+\rkl})
      \end{tikzcd}
    \end{center}
    Furthermore, the simple structure of the cohomology of spheres
    yields for the Künneth isomorphism in the desired degree
    \begin{align*}
      \H^{\rkk+\rkl}(\Sphere{\rkk+\rkl};R)
      &\cong
        \left(\H^*(\Sphere{\rkk}; R)\otimes \H^*(\Sphere{\rkl}; R)\right)_{\rkk+\rkl}\\
      &\coloneqq
        \bigoplus_{\mathclap{r+s=\rkk+\rkl}}
        \H^r(\Sphere{\rkk};R)\otimes \H^s(\Sphere{\rkl};R)
        =
        \H^{\rkk}(\Sphere{\rkk};R)\otimes \H^{\rkl}(\Sphere{\rkl};R)
    \end{align*}
    by leaving out zero-summands for the last equality.
    Thus, a generator of
    $\H^{\rkk}(\Sphere{\rkk}; R)\otimes \H^{\rkl}(\Sphere{\rkl}; R)$,
    which is the tensor product $\iota_{\rkk}\otimes\iota_{\rkl}$ of a generator
    in each factor,
    is mapped to a generator $\iota_{\rkk+\rkl}=\iota_{\rkk}\times\iota_{\rkl}$ of
    $\H^{\rkk+\rkl}(\Sphere{\rkl+\rkl}; R)$.
    Using the isomorphisms above proves the claim.
    % % Maybe put into separate Lemma:
    % Now one can use the fact, that for any two generators
    % $\iota_{\rkk}$ of $\H^{\rkk}(\spherepair{\R^{\rkk}})$ and
    % $\iota_{\rkl}$ of $\H^{\rkl}(\spherepair{\R^{\rkl}})$,
    % the cross-product
    % $\iota_{\rkk}\times\iota_{\rkl}$ is a generator of
    % $\H^{\rkk+\rkl}(\spherepair{\R^{\rkk+\rkl}})$.
  \end{proof}
\end{Cor}

\begin{Cor}
  Every vector bundle $\xi$ has a unique Thom class $\u{\xi}$ in
  $\Zmod2$-coefficients.
  Furthermore, for any map of paracompact spaces $f\colon A\to B$ and
  vector bundle $\xi\colon E\to B$ holds $\u{\pb f \xi} = \pb f \u{\xi}$.
  \begin{proof}[proof (sketch)]
    \begin{description}
    \item[Existence:] See \cite[Theorem~4D.10]{hatcher} or use
      \cite[Prop.~17.9.3]{tomdieck}.
    \item[Uniqueness:]
      Using a suitable Mayer-Vietoris sequence for gluing, and an
      inductive argument starting with the trivial bundle case, one can show:
      Any two classes in $\H^{\rkk}(\thomspacepair{E};R)$ whose
      restrictions coincide on all fibres will coincide.
      However, for $R=\Zmod2$ there is exactly one possible choice for
      a unit $\u{\xi}|_{E_b}\in
      \H^{\rkk}(\spherepair{E_b})^\times\cong\Zmod2^\times=\{1\}$
      over each point $b$.
      Compare \forexample \cite[Theorem~(17.9.4)]{tomdieck}.
    \item[Naturality:]
      Clear from uniqueness and the naturality of Thom classes.
    \end{description}
  \end{proof}
\end{Cor}

\begin{Rem}
  Using paracompactness of $B$ and
  \cite[Proposition~17.9.6]{tomdieck}, one concludes that
  $\u{\xi}\in \H^{\rkk}(\thomspacepair{E};R)$ has to be a unit.
\end{Rem}

Having a good notion of Thom classes by now, we recall the Thom
isomorphism relating the (co)homology of the total space with that of the base
space of a vector bundle.
\begin{Thm}
  For any Thom class $\u{\xi}$ of $\xi$, and any degree $r$ there are
  the following isomorphisms, called the \emph{Thom isomorphism} for
  cohomology respectively homology,
  that are natural with respect to pullbacks of vector bundles over
  paracompact spaces:
  \begin{align*}
    \thomiso\colon
    \H^r(B;R) &\longrightarrow \H^{r+\rkk}(\thomspacepair{E}; R)
    & \thomiso\colon
      \H_{r+\rkk}(\thomspacepair{E}; R) &\longrightarrow \H_r(B;R)
    \\
    x &\longrightarrow \pb p (x) \cup \u{\xi}
    & \alpha &\longrightarrow \pf p (\u{\xi} \cap \alpha)
               \;.
  \end{align*}
  \begin{proof}
    Naturality directly follows from the naturality of the Thom class
    in Corollary~\ref{cor:thomclsnatural}, and naturality of the cup-
    respectively cap-product.
    The cohomology part then is a direct application of Leray's theorem
    \cite[Theorem~4D.8]{hatcher}.
    For the homology part see \forexample \cite[Theorem~14.6]{switzer}.
  \end{proof}
\end{Thm}
\begin{Lem}\label{lem:thomisoself-adjoint}
  If $B$ is connected, the Thom isomorphisms are adjoint in the
  following sense:
  For $r\in\Nat$, $x\in \H^r(B)$,
  and $\alpha\in \H_{r+\rkk}(\thomspacepair{E})$  holds 
  \begin{gather*}
    \pf p\capped{t(x)}{\alpha} = \capped{x}{t(\alpha)}
    \in\H_0(B)\cong \Zmod2
  \end{gather*}
\end{Lem}
In order to proof Lemma~\ref{lem:thomisoself-adjoint},
first recall the following properties of the cap product.
\begin{Rem}
  For
  a map of triples of spaces
  $f\colon (Y,Y'',Y')\to (X,X'',X')$,
  cohomology classes
  $a\in \H^i(X,X')$ and $b\in \H^j(X,X')$,
  homology classes
  $\gamma\in \H_{i+j}(X, X'\cup X'')$
  and
  $\beta\in \H_j(Y, Y'\cup Y'')$,
  and a vector bundle $E\xrightarrow{p}B$
  holds
  \begin{align}
    \label{eq:capprod1}
    \capped{a\cup b}{\beta} &= \capped{b}{a\cap\beta}
                              \in \H_0(X,X'')\\
    \label{eq:capprod2}
    \capped{a}{\pf f \beta} &= \pf f \capped{\pb f a}{\beta}
                              \in \H_0(X,X'')
                            &&\text{\cite[Chap.~3.3.2, p.\,241]{hatcher}}
  \end{align}
\end{Rem}
\begin{proof}[proof of Lemma~\ref{lem:thomisoself-adjoint}]
  With $\U\coloneqq\u{\xi}$ one calculates
  \begin{align*}
    \pf p\capped{t(x)}{\alpha}
    &= \pf p\capped{\pb p x \cup \U}{\alpha} \\
    &\equalsby{\eqref{eq:capprod1}}
      \pf p\capped{\pb p x}{\U\cap\alpha} \\
    &\equalsby{\eqref{eq:capprod2}}
      \capped{x}{\pf p(u\cap\alpha)}
      = \capped{x}{\thomiso (\alpha)} \in\Zmod2
      \qedhere
  \end{align*}
\end{proof}

Now, we can focus on the specific case of normal bundles of manifolds.
The following is a well-known result from intersection theory, which
links the fundamental class of a manifold with that of a submanifold
using the Thom isomorphism corresponding to the normal bundle of the
embedding.
\begin{Lem}\label{lem:thomisofundcl}
  Let $M$ be an $n$-dimensional compact manifold, and
  $\emb\colon M\to\R^{n+\rkk}\subset\Sphere{n+\rkk}$ be an embedding with
  normal bundle $\N{\emb}$ of rank $\rkk>0$.
  The normal bundle gives rise to an embedding
  $e\colon \E{\N{\emb}}\to\Sphere{n+\rkk}$ of its total space
  as a tubular neighbourhood $e(\E{\N{\emb}})$ of $i(M)$ into the
  $(n+\rkk)$-sphere.
  The quotient map
  \begin{gather*}
    \collapse\colon
    \Sphere{n+\rkk}
    \to \Sphere{n+\rkk}/\left( \Sphere{n+\rkk}\setminus e(\E{\N{\emb}}) \right)
    \cong \Discbdl{\N{\emb}} / \Spherebdl{\N{\emb}}
    \cong \Thomspace{\N{\emb}}
  \end{gather*}
  that collapses every point outside of $e(\E{\N{\emb}})$ to the infinity
  point fulfils
  \begin{center}
    \begin{tikzcd}[row sep=0pt]
      \H_{n+\rkk}(\Sphere{n+\rkk}) \ar[r, "\pf c"]
      & \H_{n+\rkk}(\Thomspace{\N{\emb}}) \ar [r, "\pf \incl"]
      & \relH_{n+\rkk}(\Thomspace{\N{\emb}})
      %\coloneqq \H_{n+\rkk}(\Thomspace{\N{\emb}}, \infty)
      \ar [r, "\pf t"{above}, "\cong"{below}]
      & \H_{n}(M)% \cong \Zmod2
      \\
      \fundcl{\Sphere{n+\rkk}} \ar[rrr, mapsto]
      &&&\thomiso(\pf \incl\pf\collapse\fundcl{\Sphere{n+\rkk}})
      = \fundcl M
    \end{tikzcd}
  \end{center}
  where
  $\incl\colon
  (\Thomspace{\N{\emb}},\emptyset)
  \to(\Thomspace{\N{\emb}}, \{\infty\})$
  is the canonical inclusion of pairs of spaces.
  This holds for any choice of embeddings $i$ and $e$.
  \begin{proof}
    First note that by the long exact sequence of the pair 
    $(\Thomspace{\N{\emb}}, \{\infty\})$ the map
    $\pf \incl\colon
    \H_r(\Thomspace{\N{\emb}})
    \to \relH_r(\Thomspace{\N{\emb}})$
    is an isomorphism in every degree $r>0$. The proof will be
    conducted in two steps, first proving the connected case, then
    the general one.
    \begin{description}
    \item[Connected case:]
      Assume that $M$ is connected.
      Then
      \begin{gather*}
        \H_{n+k}(\Thomspace{\N{\emb}})
        \cong\relH_{n+k}(\Thomspace{\N{\emb}})
        \cong \H_n(M) \cong \Zmod2=\{\fundcl M, 0\}
      \end{gather*}
      by the Thom isomorphism, and since $M$ is connected and closed.
      Thus, one only has to show that
      $\pf\collapse\fundcl{\Sphere{n+\rkk}}$
      is non-zero.
      The trick now is to reduce once again to the homology of the
      sphere:
      Restricted to $e(\E{\N{\emb}})$ the collapse map looks like
      \begin{gather*}
        c|_{e(\E{\N{\emb}})}\colon
        e(\E{\N{\emb}})
        \overset{e}{\underset{\sim}\longto} \E{\N{\emb}}
        \cong \Discbdl{\N{\emb}}\setminus\Spherebdl{\N{\iota}}
        \cong \Thomspace{\N{\emb}}\setminus \{\infty\}
      \end{gather*}
      \idest it is a homeomorphism onto
      $\Thomspace{\N{\emb}}\setminus\{\infty\}$.
      Therefore, for any point
      $p\in\Thomspace{\N{\emb}}\setminus\{\infty\}$
      on homology there is the following commutative diagram
      \begin{center}
        \begin{tikzcd}
          \fundcl{\Sphere{n+\rkk}}
          \ar[d,mapsto]\ar[r,phantom,"\in"{near start}]
          &\H_{n+\rkk}(\Sphere{n+\rkk})
          \ar[r, "\pf\incl\pf\collapse"]\ar[d,"\pf\incl"]
          &\relH_{n+\rkk}(\Thomspace{\N{\emb}})
          \ar[d,"\pf\incl"]\\
          \left.\fundcl{\Sphere{n+\rkk}}\right|_{\collapse^{-1}(p)}
          \ar[r,phantom,"\in"{near start}]
          &\H_{n+k}(\Sphere{n+\rkk},\Sphere{n+\rkk}\setminus\collapse^{-1}(p))
          \ar[r,"\pf\collapse"{above},"\cong"{below}]
          &\H_{n+\rkk}(\Thomspace{\N{\emb}}, \Thomspace{\N{\emb}}\setminus p)
        \end{tikzcd}
      \end{center}
      By definition of the fundamental class $\fundcl{\Sphere{n+\rkk}}$,
      the class $\fundcl{\Sphere{n+\rkk}}|_{\collapse^{-1}(p)}$ in the
      diagram is a generator, and thus also is
      $\pf\collapse\left(\fundcl{\Sphere{n+\rkk}}|_{\collapse^{-1}(p)}\right)
      = \left(\pf\incl\pf\collapse\fundcl{\Sphere{n+\rkk}}\right)|_p$.
      However, then $\pf\incl\pf\collapse\fundcl{\Sphere{n+\rkk}}\in
      \H_{n+\rkk}(\Thomspace{\N{\emb}})$
      cannot be zero as was to be shown.

    \item[General case] In case $M=\coprod_i M_i$ is the disjoint sum of its connected
      components $M_i$, $i\in I$ for some index set $I$, note the following:
      \begin{itemize}
      \item $\E{\N{\emb}} = \coprod_i\E{\N{\emb_i}}$,
        where $\emb_i\coloneqq\emb|_{M_i}$.
        % where $\N{M_i}\coloneqq\N{\emb}|_{M_i}$.
      \item Thus, $\Thomspace{\N{\emb}} = \bigvee_i\Thomspace{\N{\emb_i}}$ using the
        collapse maps
        \begin{gather*}
          \collapse_i\colon
          \Sphere{n+\rkk}
          \overset{\collapse}\longto
          \Sphere{n+\rkk}/\left(\Sphere{n+\rkk}\setminus e(\E{\N{\emb}})\right)
          \overset{\proj}\longto
          \Sphere{n+\rkk}/\left(\Sphere{n+\rkk}\setminus e(\E{\N{\emb_i}})\right)
        \end{gather*}
        for the disjoint parts,
        and $\collapse=\bigvee_i\collapse_i$.
      \item Thus,
        $\relH_r(\Thomspace{\N{\emb}})
        = \bigoplus_i \relH_r(\Thomspace{\N{\emb_i}})$ for
        all degrees $r$, $\fundcl M = (\fundcl{M_i})_i$, and
        $\pf\collapse = \prod_i\pf{\collapse_i}$.
        % \item Thus, $\H_{n+\rkk}(\Thomspace{\N{\emb}})
        %   \cong \H_{n+\rkk}(M)\cong\prod_i \Zmod2$, and $\fundcl{M} = (1)_i$.
      \end{itemize}
      With this one sees directly from the definition of the fundamental
      class of a manifold that
      $\pf\incl\pf\collapse\fundcl{\Sphere{n+\rkk}} = \fundcl M$
      if and only if for all connected component manifolds $M_i$ holds
      $\pf\incl\pf{\collapse_i}\fundcl{\Sphere{n+\rkk}} = \fundcl{M_i}$,
      which is true by the first case.
      \qedhere
    \end{description}
  \end{proof}
\end{Lem}

\subsection{A Construction of Stiefel-Whitney Classes}
\label{sec:swclsconstruction}
Now, the promised construction of the Stiefel-Whitney classes from
Thom classes and Steenrod squares can be presented.

\begin{Thm}\label{thm:altdefswclasses}
  The Stiefel-Whitney classes can be given as
  \begin{gather*}
    \Sq i(\u{\xi}) = \thomiso(\w{i}{\xi}) = \pb p \w{i}{\xi} \cup \u{\xi}
  \end{gather*}
  for any vector bundle $\xi\colon E\to B$ over a paracompact space
  $X$. As the Thom isomorphism is a group homomorphism, one can
  formulate the above as
  \begin{gather*}
    \SQ(\u{\xi}) = \thomiso(\W{\xi}) = \pb p \W{\xi} \cup \u{\xi}
  \end{gather*}
  \begin{proof}
    One has to check naturality of the expression and all further
    defining properties from Definition~\ref{def:swclasses}.
    \begin{description}
    \item[Naturality:] Both $\Sq i$ and $t$ respectively also $t^{-1}$
      are natural.
    \item[$\ws{0}=1$:]
      $\H^0(\thomspacepair{\E{\gamma_0}}) = \Zmod2$, thus 1 is the only
      candidate for a Thom class, $\Sq0(1) = \Id(1) = 1$, and the Thom
      isomorphism sends 1 to 1 in this degree.
    \item[$\W{\gamma_1}=1+x$:]
      For $\w0{\gamma_1}$ the defining relation
      \begin{gather*}
        \pb p\w{0}{\gamma_1}\cup\u{\gamma_1}
        = \Sq 0(\u{\gamma_1})
        \cequalsby{\eqref{eq:sqidentity}}\u{\gamma_1} 
      \end{gather*}
      directly gives
      $\pb p\w{0}{\gamma_1}=1$, and thus $\w{0}{\gamma_1}=1$.
      For $\w1{\gamma_1}$ first note that $\H^*(\RP1)=\Zmod2[x]/(x^2)$
      and that for both $i=0,1$, one has the Thom isomorphism
      $t\colon\Zmod2\cong\H^i(\RP1)\to\relH^{i+1}(\Thomspace{\gamma_1})$.
      Hence, 
      \begin{enumerate}[1.]
      \item $\relH^1(\Thomspace{\gamma_1})\cong\Zmod2$ has a unique
        non-zero element,
      \item $t(x)=\pb p x\cup\u{\gamma_1}$ is non-zero, thus
        \begin{itemize}
        \item $\u{\gamma_1}\in\relH^1(\Thomspace{\gamma_1})$
          is non-zero, and
        \item $\pb p x\in\relH^1(\Thomspace{\gamma_1})$ is non-zero.
        \end{itemize}
      \end{enumerate}
      So $\pb p x=\u{\gamma_1}\in\relH^1(\Thomspace{\gamma_1})$
      and the defining relation of $\w1{\gamma_1}$ gives:
      \begin{gather*}
        t(\w{1}{\gamma_1})
        = \pb p\w{1}{\gamma_1}\cup x
        = \Sq 1(\u{\gamma_1})
        \cequalsby{\eqref{eq:sqsquared}}\u{\gamma_1}^2
        = \pb p x \cup \u{\gamma_1}
        = t(x)
        \;,
      \end{gather*}
      which is equivalent to $\w{1}{\gamma_1}=x$, since $t$ is an
      isomorphism.
    \item[Multiplicativity:]
      Consider vector bundles $\xi$, $\eta$ over a paracompact space
      $B$. With $\u{\xi\oplus\eta}=\u{\xi}\cup\u{\eta}$ and the fact that
      \begin{gather}\label{eq:projectionscommute}
        p_\xi\circ\pi_\xi = p_{\xi\oplus\eta} = p_\eta\circ\pi_\eta
      \end{gather}
      one gets
      \begin{align*}
        \thomiso(\w{i}{\xi\oplus\eta})
        &= \Sq i(\u{\xi\oplus\eta}) \\
        &\equalsby{\ref{cor:thomclassmultiplicative}}
          \Sq i(\pb\pi_\xi\u{\xi} \cup \pb\pi_\eta\u{\eta})\\
        &\equalsby{\ref{tag:cartan}}
          \sum_{r+s=i}
          \Sq r(\pb\pi_\xi\u{\xi}) \cup \Sq s(\pb \pi_\eta\u{\eta}) \\
        &\equalsby{Naturality}
          \sum_{r+s=i}
          \pb\pi_\xi\Sq r(\u{\xi}) \cup \pb \pi_\eta\Sq s(\u{\eta}) \\
        &\equalsby{Definition}
          \sum_{r+s=i}
          \pb\pi_\xi\thomiso(\w{r}{\xi})
          \cup \pb\pi_\eta\thomiso(\w{s}{\eta}) \\
        &\equalsby{Definition}
          \sum_{r+s=i}
          \pb\pi_\xi \left(\pb p_\xi  \w{r}{\xi}  \cup \u{\xi} \right)
          \cup
          \pb\pi_\eta\left(\pb p_\eta \w{s}{\eta} \cup \u{\eta}\right) \\
        &= \left(
          \sum_{r+s=i}
          \pb\pi_\xi \pb p_\xi \w{r}{\xi} \cup
          \pb\pi_\eta\pb p_\eta\w{s}{\eta}
          \right)
          \cup
          \left(\pb\pi_\xi\u{\xi} \cup \pb\pi_\eta\u{\eta}\right) \\
        &\equalsby{\eqref{eq:projectionscommute}, Definition}
          \left(\sum_{r+s=i}
          \pb p_{\xi\oplus\eta} \w{r}{\xi} \cup \pb p_{\xi\oplus\eta} \w{s}{\eta}
          \right)
          \cup
          \u{\xi} \times \u{\eta} \\
        &\equalsby{Group Hom., \ref{cor:thomclassmultiplicative}}
          \pb p_{\xi\oplus\eta}
          \left(\sum_{r+s=i}\w{r}{\xi}\cup\w{s}{\eta}\right)
          \cup
          \u{\xi\oplus\eta}\\
        &\equalsby{Definition}
          \thomiso\left(\sum_{r+s=i}\w{r}{\xi}\cup\w{s}{\eta}\right)
          \;.
      \end{align*}
      Applying $\thomiso^{-1}$ yields the result.
      \qedhere
    \end{description}
  \end{proof}
\end{Thm}


\subsection{Equivalent Definitions of Wu Classes}\label{sec:wuclasses}
Let $M$ be a compact, $n$-dimensional manifold.
In the following, two approaches to define the Wu characteristic
classes are pursued: an explicit one specially for normal bundles of
manifolds using Poincaré duality, and a more general one, which makes
clear that the Wu classes indeed are characteristic classes of vector
bundles (\idest natural).

\begin{Def}\label{def:wuclasses}
  The $i$th \emph{Wu class} $\v{i}{M}$ of $M$ for $0\leq i\leq n$ is defined
  as the cohomology class in $\H^i(M)$ that is uniquely determined by
  \begin{center}
    \begin{tikzcd}[row sep=0pt, column sep=small]
      \H^i(M) \ar[r, equal, "\isosymb"]
      & \H_{n-i}(M) \ar[r, equal, "\isosymb"]
      &\Hom{\Zmod2}(\H^{n-i}(M), \Zmod2) \ar[r, equal, "\isosymb"]
      &\Zmod2
      \\
      y \ar[rr, mapsto] &&\capped{ x\cup y }{ \fundcl M }\\
      \v{i}{M} \ar[rr, mapsto] &&\capped{\Sq i(x)}{\fundcl M}
    \end{tikzcd}
  \end{center}
  where the first isomorphism from the left is Poincaré duality
  and the second is the universal coefficient theorem
  for the field $\Zmod2$.
  Equivalently, for any cohomology class $x\in \H^{n-i}(M)$ of fixed
  degree $n-i$ holds
  \begin{gather*}
    x\cup \v{i}{M} = \Sq i(x) \in \H^n(M) \cong \Zmod2
  \end{gather*}
  Mind the fixed degree of $x$---the above will not be true for other
  degree cohomology classes in general!
\end{Def}

\begin{Rem}
  Some immediate consequences of the definitions of the Wu classes of
  $M$ are
  \begin{itemize}
  \item $\v{0}{M} = 1$
  \item $\v{i}{M} = 0$ for $i>\frac n 2$, because $\Sq i(x) = 0$ if the
    degree of x is lower than $i$.
  \end{itemize}
\end{Rem}

As for the Stiefel-Whitney classes, one has a notion of a total class.
\begin{Def}\label{def:dualwuclasses}
  The \emph{total Wu class} of $M$ is defined as the sum
  $\sum_{i\geq0}\v{i}{M}$. The \emph{total dual Wu class}
  $\dualV{M}\eqqcolon \sum_{i\geq 0}\dualv{i}{M}$
  and the dual Wu classes $\dualv{i}{M}$
  of $M$ are defined by
  \begin{gather*}
    \V{M} \cup \dualV{M} = 1
  \end{gather*}
  or equivalently
  \begin{align*}
    1 &= \dualv{0}{M} \cup \v{0}{M} = \dualv{0}{M} \\
    0 &= \sum_{r+s=i}\v{r}{M}\cup\dualv{s}{M}
      &&\text{in degree $0\leq i\leq n$}
  \end{align*}
\end{Def}

The following more general definition of Wu classes
will turn out to be equivalent in a certain sense to the one above in
Definition~\ref{def:wuclasses}.
For the formulation recall the antipode of the Steenrod algebra
(see Definition~\ref{def:antipode}).
\begin{Def}\label{def:altwuclasses}
  Let $\xi\colon E\xrightarrow{p} B$ be a vector bundle over a
  paracompact space $B$.
  The $i$th \emph{Wu class} $\v{i}{\xi}$ of $\xi$ for $0\leq i\leq n$
  is defined as the cohomology class in $\H^i(B)$ that is uniquely
  determined by
  \begin{gather*}
    \antipode(\Sq i)(\u{\xi}) = \thomiso(\v{i}{\xi}) = \pb p \v{i}{\xi} \cup \u{\xi}
  \end{gather*}
  The \emph{total Wu class} of $\xi$ is defined as usual as
  $\V{\xi}\coloneqq\sum_{i\geq0}\v{i}{\xi}$ and satisfies accordingly 
  $\antipode(\SQ)(\u{\xi})=t(\V{\xi})$.
\end{Def}
\begin{Rem}
  Compare this to the possible definition of the Stiefel-Whitney
  classes in Theorem~\ref{thm:altdefswclasses}.
  Similarly, naturality with respect to pullbacks of vector bundles
  follows immediately from the definition, making the Wu classes
  characteristic classes.
\end{Rem}

Recall that the definition of Stiefel-Whitney classes of manifolds
utilises the canonical tangent bundle structure.
In contrast to that, the Wu classes of manifolds utilise normal
bundles, as will be clear from the promised equivalence below.
\begin{Thm}\label{thm:altdefwuclasses}
  Let $M$ be a compact manifold of dimension $n$, and let
  $\N{\emb}\colon \E{\N{\emb}} \xrightarrow{p} M$ be
  any normal bundle of rank $k$ of an embedding
  $\emb\colon M\immto\R^{n+k}$. Then
  \begin{gather*}
    \v{i}{M} = \v{i}{\N{\emb}}\in \H^i(M)
  \end{gather*}
  \begin{proof}
    To proof Theorem~\ref{thm:altdefwuclasses}, the defining property of
    $\v{i}{M}$ will be checked on $\v{i}{\N{\emb}}$, \idest one has to show
    that for any $x\in \H^{n-i}(M)$ holds
    \begin{gather*}
      \capped{x\cup \v{i}{\N{\emb}}}{\fundcl M}
      = \capped{\Sq i(x)}{\fundcl M}
    \end{gather*}
    Note, that this is simply the $i$th degree of the equation
    \begin{gather}\label{eq:proofaltdefwuclasses:claim}
      \capped{x\cup \V{\N{\emb}}}{\fundcl M}
      = \capped{\SQ(x)}{\fundcl M}
    \end{gather}
    which will be proven below.
    Beforehand, recall the following:
    \begin{description}
    \item[By \ref{lem:thomisoself-adjoint}:]
      $\pf p\capped{\thomiso(z)}{\alpha} = \capped{z}{\thomiso(\alpha)}$
      for $r\in\Nat$, $z\in \H^r(M)$,
      $\alpha\in \H_{r+\rkk}(\thomspacepair{\E{\N{\emb}}})$.
    \item[By \ref{lem:thomisofundcl}:]
      $\thomiso(\pf\incl\pf\collapse\fundcl{\Sphere{n+\rkk}}) = \fundcl M$
      where $\collapse\colon \Sphere{n+\rkk}\to \Thomspace{\N{\emb}}$ is
      the collapse map of a tubular embedding of the normal bundle $\N
      M$.
    \item[By \eqref{eq:sqlowerbound} and \eqref{eq:sqidentity}:]
      The total Steenrod square
      $\SQ\colon \H^m(\Sphere m)
      \to \H^*(\Sphere m)$
      is the identity on $\H^m(\Sphere m)$, \idest
      $\Sq i\colon \H^m(\Sphere{m})\to \H^{m+i}(\Sphere{m})$ is zero for
      $i\neq0$.
    \item[By Definition~\ref{def:sq}:] The total Steenrod Square
      $\SQ$ is natural and a ring homomorphism.
    \item[By \eqref{eq:capprod2}:]
      For any map of spaces $f\colon X\to Y$ and co-/homology classes
      $a$ and $\beta$ in the corresponding co-/homology groups holds
      $\capped{a}{\pf f \beta} = \pf f \capped{\pb f a}{\beta}$
    \end{description}
    For the proof fix some $i\leq n$, some arbitrary $x\in
    \H^{n-i}(M)$, as well as a collapse map
    $\collapse\colon\Sphere{n+k}\to\Thomspace{\N{\emb}}$ as in
    Lemma~\ref{lem:thomisofundcl}.
    For simplicity, denote
    $\Vu\coloneqq \V{\N{\emb}}$,
    $\Sph\coloneqq \Sphere{n+\rkk}$,
    $\U\coloneqq \u{\N{\emb}}$ and
    $\pf\collapse\coloneqq \pf{(\incl\circ\collapse)}$
    respectively
    $\pb\collapse\coloneqq \pb{(\incl\circ\collapse)}$.
    
    
    In a first calculation reformulate
    $\capped{x\cup \V{\N{\emb}}}{\fundcl M}$ using the cohomology of the
    $(n+k)$-sphere:
    \begin{align}\notag
      \capped{x\cup \Vu}{\fundcl M}
      &\equalsby{\ref{lem:thomisofundcl}}
        \capped
        {x\cup\Vu}
        {\thomiso\left(\pf\collapse\fundcl{\Sph}\right)}
      \\\notag
      &\equalsby{\ref{lem:thomisoself-adjoint}}
        \pf p\capped
        {\thomiso\left(x\cup\Vu\right)}
        {\pf\collapse\fundcl{\Sph}}
      \\\notag
      &\equalsby{\eqref{eq:capprod2}}
        \pf p\pf\collapse\capped
        {\pb\collapse\thomiso\left(x\cup\Vu\right)}
        {\fundcl{\Sph}}
      \\\notag
      &\equalsby{\eqref{eq:sqlowerbound} and \eqref{eq:sqidentity}}
        \pf p\pf\collapse\capped
        {\SQ\left(\pb\collapse\thomiso\left(x\cup\Vu\right)\right)}
        {\fundcl{\Sph}}
      \\\notag
      &\equalsby{Naturality}
        \pf p\pf\collapse\capped
        {\pb\collapse\SQ\left(\thomiso\left(x\cup\Vu\right)\right)}
        {\fundcl{\Sph}}
      \\\label{eq:proofaltdefwuclasses:eq1}
      &\equalsby{\eqref{eq:capprod2}}
        \pf p\capped
        {\SQ\left(\thomiso\left(x\cup\Vu\right)\right)}
        {\pf\collapse\fundcl{\Sph}}
    \end{align}
    Having introduced $\SQ$ on the left hand side, one observes a
    certain commutativity of the Thom isomorphism, and the total
    Steenrod square:
    \begin{align}\notag
      \SQ\left(\thomiso\left(x\cup\Vu\right)\right)
      &=
        \SQ\left(\left(\pb p x\cup \pb p\Vu\right) \cup \U\right)
      \\\notag
      &=
        \SQ\left(\pb p x\cup \left(\pb p\Vu \cup \U\right)\right)
      \\\notag
      &\equalsby{Def. $\thomiso$}
       \SQ\left(\pb p x\cup \thomiso(\Vu)\right)
      \\\notag
      &\equalsby{Naturality, Ring Hom.}
        \pb p\SQ(x) \cup \SQ(\thomiso(\Vu))
      \\\notag
      &\equalsby{Def. $\V{M}$}
        \pb p\SQ(x) \cup \SQ(\antipode(\SQ)(u))
      \\\notag
      &\equalsby{Def. $\antipode$}
        \pb p\SQ(x) \cup u
      \\\label{eq:proofaltdefwuclasses:eq2}
      &\equalsby{Def. $\thomiso$}
        \thomiso(\SQ(x))
    \end{align}
    Inserting \eqref{eq:proofaltdefwuclasses:eq2} into
    \eqref{eq:proofaltdefwuclasses:eq1} from above easily yields the
    claim in \eqref{eq:proofaltdefwuclasses:claim} that proves the theorem:
    \begin{align*}
      \capped{x\cup \Vu}{\fundcl M}
      &\cequalsby{\eqref{eq:proofaltdefwuclasses:eq1}}
        \pf p\capped
        {\SQ\left(\thomiso\left(x\cup\Vu\right)\right)}
        {\pf\collapse\fundcl{\Sph}}
      \\
      &\equalsby{\eqref{eq:proofaltdefwuclasses:eq2}}
        \pf p\capped
        {\thomiso(\SQ(x))}
        {\pf\collapse\fundcl{\Sph}}
      \\
      &\equalsby{\ref{lem:thomisoself-adjoint}}
        \capped
        {\SQ(x)}
        {\thomiso\left(\pf\collapse\fundcl{\Sph}\right)}
      \\
      &\equalsby{\ref{lem:thomisofundcl}}
        \capped
        {\SQ(x)}
        {\fundcl M}
        \qedhere
    \end{align*}
  \end{proof}
\end{Thm}


\subsection{Wu's Theorem}\label{sec:wutheorem}
Wu's theorem, the goal of this section, gives a close relation of
the Wu classes from above with the Stiefel-Whitney classes using
Steenrod squares. Besides, it also immediately proves that Wu classes
as defined in Theorem~\ref{thm:altdefwuclasses} are actually characteristic
classes.

\begin{Thm}[Wu]\label{thm:wu}
  A closed manifold gives rise to the following two equalities
  \begin{align}\notag
    \dualW{M} &= \SQ\left(\dualV{M}\right)
    &\text{respectively}&
    &\dualw{k}{M} &= \sum_{i\geq0} \Sq i\left(\dualv{k-i}{M}\right)
    &\text{and}
    \\
    \W{M} &= \SQ\left(\V{M}\right)
    &\text{respectively}&
    &\w{k}{M} &= \sum_{i\geq0} \Sq i\left(\v{k-i}{M}\right)
    \label{tag:wuformula}\tag{Wu's formula}
  \end{align}
  that are equivalent using
  $\dualW{M}\cup\W{M} = 1$ and
  \begin{gather*}
    \SQ(\dualV{M})\cup\SQ(\V{M})
    = \SQ(\dualV{M}\cup\V{M})
    = \SQ(1)
    = 1
    \;.
  \end{gather*}
\end{Thm}
% \begin{Rem}
%   The following is equivalent to Wu's formula
%   \begin{gather*}
%     \dualW{M} = \SQ\left(\dualV{M}\right)
%     \qquad\text{respectively}\qquad
%     \dualw{k}{M} = \sum_{i\geq0} \Sq i\left(\dualv{k-i}{M}\right)
%   \end{gather*}
%   using the inverses $\dualW{M}\cup\W{M} = 1$ and
%   \begin{gather*}
%     \SQ(\dualV{M})\cup\SQ(\V{M})
%     = \SQ(\dualV{M}\cup\V{M})
%     = \SQ(1)
%     = 1
%     \;.
%   \end{gather*}
% \end{Rem}
The proof uses the alternative characterisation
\begin{gather*}
  \antipode(\SQ)(\u{\N{\emb}}) = \thomiso(\V{\N{\emb}})
\end{gather*}
of the Wu classes from Theorem~\ref{thm:altdefwuclasses}, and quite directly
follows from the following Lemma:
\begin{Lem}\label{lem:wu}
  For a vector bundle $\xi\colon \E\xi\xrightarrow{p}\B\xi$ over a
  paracompact space holds
  \begin{gather*}
    \SQ(\V{\xi}) \cup \W{\xi}= 1
    \;.
  \end{gather*}
\end{Lem}
\begin{proof}[proof of \ref{tag:wuformula}]
  Lemma~\ref{lem:wu} states for a closed manifold $M^n$ and any
  embedding $\emb\colon M\to\R^{n+k}$ that
  \begin{gather*}
    \SQ(\V{M}) \cup \dualW{M}
    = \SQ(\V{\N M}) \cup \W{\N{\emb}}
    \cequalsby{\ref{lem:wu}} 1
    \;.
  \end{gather*}
  Cupping with $\W{M}=\W{\T M}$ on both sides yields the claim
  as $\W{\T M} \cup \W{\N{\emb}} = 1$ by
  Remark~\itemref{rem:propswclasses}{item:wuclassmfdinverse}.
\end{proof}
\begin{proof}[proof of Lemma~\ref{lem:wu}]
  For simplicity use the shortenings
  $\U\coloneqq \u{\xi}$,
  $\Vu\coloneqq \V{\xi}$, and
  $\Ws\coloneqq \W{\xi}$.
  Then calculate
  \begin{align*}
    \thomiso(1)
    &= \U
    \\
    &\equalsby{Def.~\ref{def:antipode}}
      \SQ\left(\antipode(\SQ)(\U)\right)
    \\
    &\equalsby{Def.~\ref{def:altwuclasses}}
      \SQ\left(\thomiso(\Vu)\right)
    \\
    &\equalsby{Def.~$\thomiso$}
      \SQ\left(\pb p\Vu \cup \U\right)
    \\
    &\equalsby{\ref{tag:cartan}, Naturality}
      \pb p\SQ(\Vu) \cup \SQ(\U)
    \\
    &\equalsby{\ref{thm:altdefswclasses}}
      \pb p\SQ(\Vu) \cup \thomiso(\Ws)
    \\
    &\equalsby{Def.~$\thomiso$}
      \pb p\SQ(\Vu) \cup \left(\pb p \Ws \cup \U\right)
    \\
    &=
      \pb p\left(\SQ(\Vu) \cup \Ws\right) \cup \U
    \\
    &\equalsby{Def.~$\thomiso$}
      \thomiso\left(\SQ(\Vu) \cup \Ws\right)
  \end{align*}
  Applying the inverse of the Thom isomorphism to both sides
  yields the equality which was to be shown.
\end{proof}


\section{Massey's Theorem on the Stiefel-Whitney Classes of Manifolds}
\label{sec:massey}
Massey's main theorem on the Stiefel-Whitney classes of manifolds
gives a concrete statement, in which degrees the dual
Stiefel-Whitney classes may be non-zero in general.
As of Corollary~\ref{cor:obstruction} this is directly related to the
immersion problem respectively an important obstruction for it.

\begin{Thm}[Massey]\label{thm:massey}
  \optcite[Theorem~I.]{massey}
  Let $M$ be a compact, $n$-dimensional manifold.
  Given an integer $q$ with $0<q<n$ such that $\dualw{n-q}{M}\neq0$,
  there is a sequence of integers $h_1\geq\dotsb\geq h_q\geq0$ of
  length $q$ that fulfils
  \begin{gather*}
    n = \sum_{i=1}^{q} 2^{h_i}
  \end{gather*}
\end{Thm}
Note that the minimal length of a representation of $n$ by
powers of two is its binary representation of length $\alpha(n)$.
As an immediate consequence:
\begin{Cor}
  For any manifold, all its dual Stiefel-Whitney classes of degree
  greater than $n-\alpha(n)$ must be zero.
\end{Cor}

The following subsections are dedicated to the proof of Massey's Theorem.
Let $M$ be a compact, $n$-dimensional manifold throughout the proof.
The latter consists of several steps:
\begin{steps}
\item\label{tag:masseystep1} 
  Show that for any $q$, admissible iterated Steenrod square $\Sq I$, and cohomology
  class $x\in\H^q(M)$ of degree $q$ such that $\Sq I(x)$ is non-trivial there exists
  some representation of the form
  \begin{gather*}
    \deg \left(\Sq I(x)\right)
    = 2^k\cdot
    \left( 2^{k_1}+\dotsb+2^{k_{q-1}} + 1 \right)
  \end{gather*}
  (all used results in during step hold for any space $X$).
\item\label{tag:masseystep2}
  Find some iterated Steenrod square which is non-trivial in degree
  \begin{gather*}
    \Sq I\colon\H^q(M)\to\H^n(M)
    \;.
  \end{gather*}
\end{steps}
Applying \ref{tag:masseystep1} to the Steenrod square $\Sq I$ from
\ref{tag:masseystep2} and some $x\in\H^q(M)$ with $\Sq I(x)\neq 0$
immediately yields the result as
\begin{gather*}
  n = \deg\left(\Sq I(x)\right) = \underbrace
  {2^{k_1+k}+\dotsb+2^{k_{q-1}+k} + 2^{k}}_{\text{$q$ summands}}
  \;.
\end{gather*}

\subsection[Degree Dissection for Steenrod Squares]
{Step 1: A Degree Dissection for Steenrod Squares}
For this step $M$ may be any space.
\ref{tag:masseystep1} requires to proof the following claim.
\begin{Lem}[\ref{tag:masseystep1}]\label{lem:masseystep1}
  Let $q\geq 0$ be an integer,
  and $I\in\Nat^{\l(I)}$ an admissible sequence of integers.
  Further, let $x\in\H^q(M)$ be a cohomology class of degree $q$
  such that $\Sq I(x)$ is non-trivial.
  Then there exists $k\in\Nat$ and a sequence of integers
  $0\leq k_1\leq\dotsb\leq k_{q-1}$ of length $q-1$ such that the
  degree of $\Sq I(x)$ can be represented as the dissection
  \begin{gather*}
    \deg \Sq I(x)
    = \deg x + \d(I)
    = 2^k\cdot
    \left( 1 + 2^{k_1}+\dotsb+2^{k_{q-1}} \right)
    \;.
  \end{gather*}
  % Mind that the maximum length of such a sequence is $\deg\Sq I(x)$.
\end{Lem}

In order to split the proof into several cases, recall that
$\Sq I(x) = 0$ for $\e(I)>\deg x$ by
Remark~\itemref{rem:sq}{item:squpperboundgeneral}. This leaves the two
cases $\e(I)<\deg x$ and $\e(I)=\deg x$.
Inductively applying the following Lemma by Serre restricts the
proof of Lemma~\ref{lem:masseystep1} to the first case where $\e(I)<q$.
\begin{Lem}[Serre]
  \label{lem:serre}
  Every admissible sequence $I\in\Nat^{\l(I)}$ of excess $\e(I)>0$
  admits an admissible sequence $J$ with $\e(J)<\e(I)$,
  together with some $k\in\Nat$
  such that for any cohomology class $x\in\H^{\e(I)}(M)$ holds
  \begin{gather*}
    \Sq I(x) = \left(\Sq J(x)\right) ^{2^k}
    \qquad\text{respectively}\qquad
    \deg\left(\Sq I(x)\right) = 2^k\cdot \deg\left(\Sq J(x)\right)
  \end{gather*}
\end{Lem}

Before proving Lemma~\ref{lem:serre} one can finish the argumentation for
the case $\e(I)<\deg x$.
\begin{proof}[proof of Lemma~\ref{lem:masseystep1}]
  Let $q\in\Nat$ and $I=(i_1,\dotsc,i_l)$ be an admissible sequence such that
  $\e(I)<q$.
  Assume there is a cohomology class $x\in\H^q(M)$ such that $\Sq I(x)\neq0$.
  Set
  \begin{alignat*}{4}
    \alpha_0 &= q-1-\e(I)\geq 0 &\qquad&\text{which is positive as $\e(I)<q$,} \\
    \alpha_r &= i_{r}-2i_{r+1}  &&\text{for $1\leq r< \l(I)$, and} \\ 
    \alpha_{\l(I)} &= i_{\l(I)}
    \;.
  \end{alignat*}
  It is an easy exercise that the excess of $I$ can be rewritten as
  $\e(I)=\sum_{r=1}^{\l(I)}\alpha_r$, so
  \begin{gather}\label{eq:alpha0}
    \sum_{r=0}^{\l(I)}\alpha_r = \alpha_0 + \e(I) \cequalsby{Def.} q-1
    \;.
  \end{gather}
  Just as easily one sees
  \begin{align}\label{eq:proofmassey:eq1}
    i_s
    &= \sum_{r=0}^{s}2^r\alpha_{s+r}\\\notag
  \end{align}
  The above definitions then directly imply the following reformulation
  of $\d(I)$ in terms of $\alpha_i$:
  \begin{align}\notag
    \d(I)
    &\coloneqq \sum_{s=1}^{\l(I)}i_s  \\
    &\equalsby{\eqref{eq:proofmassey:eq1}}
      \sum_{s=1}^{\l(I)}\sum_{r=0}^{s}2^r\alpha_{s+r}\\\notag
    &\equalsby{Reorder} \sum_{j=1}^{\l(I)}
      \left(\sum_{m=0}^{j-1}2^m\right)\alpha_j
      = \sum_{j=1}^{\l(I)}(2^j-1)\alpha_j 
      = \sum_{j=1}^{\l(I)}2^j\alpha_j
      - \sum_{j=1}^{\l(I)}\alpha_j\\
    \label{eq:proofmassey:eq2}
    &= \sum_{j=1}^{\l(I)}2^j\alpha_j
      - \e(I)
      \;.
  \end{align}
  All put together yields
  \begin{align*}
    \deg\left(\Sq I(x)\right)
    &= \deg(x) + \d(I)\\
    &= 1 + \deg(x) -1 +\d(I) \\
    &\equalsby{\eqref{eq:proofmassey:eq2}}
      1 + q - 1 - \e(I) + \sum_{j=1}^{\l(I)}2^j\alpha_j \\
    &\equalsby{Def.}
      1 + \alpha_0 + \sum_{j=1}^{\l(I)}2^j\alpha_j \\
    &= 1 + \sum_{j=0}^{\l(I)}2^j\alpha_j
    = 1 + \left(
      \underbrace{2^0 +\dotsb+ 2^0}_{\text{$\alpha_0$ times}}
      + \dotsb
      + \underbrace{2^{\l(I)}+\dotsb+2^{\l(I)}}_{\text{$\alpha_{\l(I)}$ times}}
      \right)
  \end{align*}
  which is one plus a sum of exactly
  $\sum_{j=0}^{\l(I)}\alpha_j\cequalsby{\eqref{eq:alpha0}} q-1$
  powers of two as was to be shown.
  The $k_j$ are in this case
  \begin{gather*}
    k_j = \begin{cases}
      0 & 0< j\leq \alpha_0\\
      1 & \alpha_0< j\leq \alpha_1\\
      \vdots\\
      \l(I) & \alpha_{\l(I)-1}<j\leq \alpha_{\l(I)}
    \end{cases}
    \qedhere
  \end{gather*}
\end{proof}

\begin{proof}[proof of Lemma~\ref{lem:serre}]
  \optcite[Lemma~1, converse part, p.~159]{serre}
  First note that any admissible sequence $I\coloneqq(i_1,\dotsc,i_l)\in\Nat^l$
  of excess $\e(I)>0$ can be written as
  \begin{gather*}
    I=(2^{k-1}i_k,\dotsc,2i_k,i_k,i_{k+1},\dotsc,i_l)
  \end{gather*}
  with $l>k\geq 1$ chosen maximal, \idest $i_k>2i_{k+1}$. If $I$ is admissible, the subsequence
  $J\coloneqq(i_{k+1},\dotsc,i_l)$ will be admissible as well.
  In order to see that such $J$ and $k$ fulfil the requirements from
  Lemma~\ref{lem:serre} one only needs to show
  \begin{claim}
    For $I$ and $J$ as above, and $x\in\H^{\e(I)}(X)$ a cohomology
    class of a space $X$ holds
    \begin{enumerate}
    \item $\Sq I(x) = \left(\Sq J(x)\right)^{2^k}$, and
    \item $\e(I)<\e(J)$.
    \end{enumerate}
  \end{claim}
  For the first part one simply has to check that
  $\deg(\Sq J(x))=i_k$, as the statement then inductively follows from
  property \eqref{eq:sqsquared} that $\Sq i(y)=y^2$ for any cohomology
  class with $i=\deg(y)$.
  So calculate
  \begin{align*}
    \deg(\Sq J(x))
    &= \d(J) + \deg(x)\\
    &= \d(J) + \e(I)\\
    &= \d(J) 
      + \left(\sum_{r=1}^{\l(I)-1}(i_r-2i_{r+1})\right)
      + i_{\l(I)}\\
    &= \d(J)
      + \left(\sum_{r=1}^{k-1}\underbrace{(i_r-2i_{r+1})}_{=0}\right)
      + (i_k-2i_{k+1})
      +\left(\sum_{r=k+1}^{\l(I)-1}(i_r-2i_{r+1})\right) + i_{\l(I)}\\
    &= \d(J)
      + (i_k-2i_{k+1})
      + \e(J) \\
    &= \d(J) + i_k - 2i_{k+1} + 2i_{k+1} - \d(J) \\
    &= i_k
  \end{align*}
  Comparing the excesses yields the second part:
  \begin{gather*}
    \e(I) - \e(J)
    = \left(\sum_{r=1}^{k-1}\underbrace{(i_r-2i_{r+1})}_{=0}\right)
    + \underbrace{(i_k - 2i_{k+1})}_{\text{$>0$ by def. of k}}
    > 0
    \qedhere
  \end{gather*}
\end{proof}

% TODO: better title
\subsection[Non-trivial Iterated Steenrod Squares for Manifolds]
{Step 2: Existence of Non-trivial Iterated Steenrod Squares for Manifolds}

This section is dedicated to the search of an iterated Steenrod square
$\Sq I$ of degree $n-q$ that is non-trivial in degree
$\Sq I\colon\H^q(M)\to\H^n(M)$ in order to complete
\ref{tag:masseystep2} of the prove of Theorem~\ref{thm:massey}.

The candidate is the multiplication map
\begin{gather*}
  \H^q(M) \longto \H^n(M)
  \qquad
  x\longmapsto x\cup\dualw{n-q}{M}
\end{gather*}
which is non-trivial if and only if $\dualw{n-q}{M}\neq 0$, since the
cup-product is non-degenerated by Poincaré duality
\cite[Proposition~3.38]{hatcher}.
It remains to see that this map actually comes from a sum of
iterated Steenrod squares, one of which then must be non-trivial and
of the correct degree.
But this easily follows from the next main Lemma,
the proof of which essentially uses Wu's theorem and the properties of
the Wu classes.

\begin{Lem}\label{lem:masseystep2}
  Let $0<k<n$ and $x\in\H^k(M)$ be a cohomology class.
  Then
  \begin{gather*}
    x\cdot\dualw{n-k}{M} = \sum_{i>0}\Sq i(x)\cdot\dualw{n-k-i}{M}
  \end{gather*}
\end{Lem}
Again, before proving the above lemma, let us see how this contributes
to the final result.
\begin{proof}[proof of Theorem~\ref{thm:massey}]
  The trick to complete \ref{tag:masseystep1} is to show that
  for $\dualw{n-q}{M}\neq 0$, and $k$, $x$ as in the Lemma above
  \begin{gather*}
    x\cdot \dualw{n-q}{M} = \sum_{I\in A}\Sq I(x)
    \;,
  \end{gather*}
  where $A$ is some collection of sequences of degree $(n-q)$.
  Because then for at least one $I\in A$, $\Sq I(x)$ is non-trivial,
  as is needed to apply Lemma~\ref{lem:masseystep1} which then yields
  the desired decomposition of $n=\deg(\Sq I(x))$.

  The above lemma serves for a descending induction on the maximum
  degree $n-q-r$ of the dual Stiefel-Whitney classes occurring in
  the sum describing $x\cdot\dualw{n-q}{M}$.
  More precisely, assume that
  \begin{gather*}
    x\cdot \dualw{n-q}{M}
    = \sum_{I\in A}\Sq I(x)
    + \sum_{j=r}^{n-q-1}\Sq{I_j}(x)\cdot\dualw{n-q-j}{M}
    \;.
  \end{gather*}
  Then Lemma~\ref{lem:masseystep2} reduces the maximum degree $n-q-r$
  of dual Stiefel-Whitney classes with the induction rule
  \begin{gather*}
    \Sq I(x)\cdot\dualw{n-k-j}{M}
    \cequalsby{\ref{lem:masseystep2}}
    \sum_{i>0}\Sq i\circ\Sq I(x)\cdot\dualw{n-k-(i+j)}{M}
  \end{gather*}
  as follows:
  \begin{align*}
    \sum_{j=r}^{n-q-1}\Sq{I_j}(x)\cdot\dualw{n-q-j}{M}
    &= \sum_{j=r}^{n-q-1}\sum_{i>0}
      \Sq i(\Sq{I_j}(x))\cdot \dualw{n-q-j-i}{M}
    \\
    &= \sum_{j=r}^{n-q-1}\sum_{i=1}^{n-q-j}
      \Sq{(i)\concat I_j}(x)\cdot \dualw{n-q-j-i}{M}
    \\
    &= \sum_{\mathrlap{r+1\leq i+j\leq n-q}}
      \Sq{(i)\concat I_j}(x)\cdot \dualw{n-q-j-i}{M}
    \\
    &=\sum_{\mathrlap{i+j=n-q}} \Sq{(i)\concat I_j}(x)\cdot \dualw0{M}
      + \sum_{\mathrlap{r+1\leq i+j\leq n-q-1}}
      \Sq{(i)\concat I_j}(x)\cdot \dualw{n-q-j-i}{M}
    \\
    &=\sum_{\mathrlap{i+j=n-q}} \Sq{(i)\concat I_j}(x)
      + \sum_{\mathrlap{r+1\leq i+j\leq n-q-1}}
      \Sq{(i)\concat I_j}(x)\cdot \dualw{n-q-j-i}{M}
      \;.
  \end{align*}
  Going down respectively up to $r=n-q$, one obtains that
  $x\cdot\dualw{n-q}{M}$ is a sum of iterated Steenrod squares
  evaluated on $x$ as was to be shown.
\end{proof}

\begin{proof}[proof of Lemma~\ref{lem:masseystep2}]
  As above let $0<k<n$ and $x\in\H^k(M)$.
  For simplicity write
  $\ws i\coloneqq\w i M$, $\dualws i\coloneqq\dualw i M$,
  $\vu i = \v i M$ and $\dualvu i = \dualv i M$.
  In order to translate from Stiefel-Whitney classes to Steenrod
  squares and back again recall the following results.
  \begin{description}
  \item[By Theorem~\ref{thm:wu} (Wu):]
    $\sum_{j=0}^{s} \Sq j(\dualvu{s-j}) = \dualws{s-j}$
    for any $s\leq n$.
  \item[By Definition~\ref{def:wuclasses} of $\Vu$:]
    $\vu i \cdot y = \Sq i(y)$
    for any $i\leq n$ and $y\in\H^{n-i}(M)$.
  \item[By Definition~\ref{def:dualwuclasses} of $\dualVu$:]
    $\dualvu d = \sum_{i=1}^{d} \vu i\cdot \dualvu{d-i}$ for $d>0$.
  \end{description}
  The full calculation is then
  \begin{align*}
    x\cdot \dualws{n-k}
    &\cequalsby{\ref{thm:wu}}
      x\cdot \sum_{i=0}^{n-k} \Sq i(\dualvu{n-k-i}) \\
    &\equalsby{Def. $\Sq 0$}
      x\cdot\left(
      \dualvu{n-k} + \sum_{i=1}^{n-k} \Sq i(\dualvu{n-k-i})
      \right) \\
    &\equalsby{Def. $\dualVu$}
      x\cdot\left(
      \left(\sum_{i=1}^{n-k} \vu{i}\cdot\dualvu{n-k-i}\right)
      + \left(\sum_{i=1}^{n-k} \Sq i(\dualvu{n-k-i})\right)
      \right) \\
    &=
      \sum_{i=1}^{n-k} \left(
      \vu{i}\cdot x\cdot \dualvu{n-k-i}
      + x\cdot \Sq i(\dualvu{n-k-i})
      \right) \\
    &\equalsby{Def. $\Vu$}
      \sum_{i=1}^{n-k} \left(
      \Sq i (x\cdot \dualvu{n-k-i})
      + x\cdot \Sq i(\dualvu{n-k-i})
      \right) \\
    &\equalsby{\ref{tag:cartan}, Def. $\Sq 0$}
      \sum_{i=1}^{n-k} \left(
      \left( \sum_{r=0}^{i} \Sq r(x)\cdot \Sq{i-r}(\dualvu{n-k-i}) \right)
      + \Sq 0(x)\cdot \Sq i(\dualvu{n-k-i})
      \right) \\
    &=
      \sum_{i=1}^{n-k} \left(
      \sum_{r=1}^{i} \Sq r(x)\cdot \Sq{i-r}(\dualvu{n-k-i})
      \right) \\
    &\equalsby{Reorder}
      \sum_{r=1}^{n-k} \Sq r(x) \cdot
      \left( \sum_{j=0}^{n-k-r} \Sq j(\dualvu{n-k-(j+r)}) \right) \\
    &\equalsby{\ref{thm:wu}}
      \sum_{r=1}^{n-k} \Sq r(x) \cdot\dualws{n-k-r}
      \qedhere
  \end{align*}
\end{proof}
This finished the proof of Massey's theorem.


\section{Best Possible Result}\label{sec:bestpossibleresult}
Recall that a closed $n$-manifold can only immerse into
$\R^{n+k}$ if all Stiefel-Whitney classes $\dualw{i}{\N M}$ of its
normal bundle are zero in degrees greater than $k$.
Massey's Theorem now states that this condition is met for
$k=n-\alpha(n)$, and thus for all manifolds the above
obstruction to the immersion property vanishes. From this arose the
idea for the immersion conjecture.

Naturally, there occurs the question whether Massey's $n-\alpha(n)$ is
the best possible (\idest smallest) result for arbitrary closed
$n$-manifolds, making the immersion conjecture the best
guess of a general possible immersion codimension.
The answer is yes, proved by the following counterexamples of manifolds
not immersing into $\R^{2n-(\alpha(n)+1)}$.

\begin{Thm}
  Denote by $\RP{i}$ the $i$th real projective space.
  \begin{enumerate}
  \item For $n=2^i$, the Stiefel-Whitney class
    $\dualw{n-\alpha(n)}{\RP{2^i}}$ is not zero.
  \item For $n\in\Nat$ with binary expansion
    $n=\sum_{r=1}^{q}2^{i_r}$, $i_1>\dotsb>i_q$, the Stiefel-Whitney class
    $\dualw{n-\alpha(n)}{\prod_{r=1}^{q}\RP{2^r}}$ is not zero.
  \end{enumerate}
  \begin{proof}
    Compare also \cite[p.~87]{immersionconj}.
    The total Stiefel-Whitney class of $\RP{n}$ for arbitrary
    $n\in\Nat$ is
    \begin{gather*}
      \W{\RP n} = (1+x_n)^{n+1}\in\H^*(\RP n)\cong \Zmod2[x_n]/(x_n^{n+1})
    \end{gather*}
    (see \forexample \cite[Example~(19.4.1)]{tomdieck}).
    For $n=2^i$ this takes the form
    \begin{align*}
      \W{\RP{2^i}}
      &= (x+1)^{2^i+1}
      = 1 + x + x^{2^i}\;,
        &&\text{thus} \\
      \dualW{\RP{2^i}}
      &= \sum_{r=0}^{2^i-1} x^r\;
        &&\text{especially with $\alpha(2^i)=1$} \\
      \dualw{n-\alpha(n)}{\RP{2^i}}
      &= x^{2^i-1}\neq 0
      %   \quad\text{, as} \\
      % \W{\RP{2^i}}\cdot\dualW{\RP{2^i}}
      % &= \sum_{r=0}^{2^i-1}x^r
      % + \sum_{r=1}^{2^i}x^r
      % + \sum_{r=0}^{2^i-1}x^{2^i}\cdot x^r \\
      % &= \left( 1 + \sum_{r=1}^{2^i-1}x^r \right)
      %   + \left( \sum_{r=1}^{2^i-1}x^r + x^{2^i} \right)
      %   + x^{2^i}
      %   = 1
    \end{align*}
    which proves the first statement.
    For the second statement where $n=\sum_{r=1}^q 2^{i_r}$ note that
    \begin{gather}\label{eq:bestresultproof1}
      n-\alpha(n)
      = \left( \sum_{r=1}^q 2^{i_r} \right) - q
      = \sum_{r=1}^q \left( 2^{i_r} - 1 \right)
      \;.
    \end{gather}
    Using multiplicativity %(\ref{tag:swclassesmultiplicativity})
    of the Stiefel-Whitney classes one gets
    \begin{align*}
      \dualW{\prod_{r=1}^q \RP{2^{i_r}}}
      &= \prod_{r=1}^q \dualW{\RP{2^{i_r}}}
        = \prod_{r=1}^q \left(\sum_{i=1}^{2^{i_r}-1} x_{2^{i_r}}^{i}\right)
        &\text{and by combinatorics} \\
      \dualw{n-\alpha(n)}{\prod_{r=1}^{q}\RP{2^{i_r}}}
      &= \prod_{r=1}^q x_{2^{i_r}}^{2^{i_r}-1}
        \neq 0
    \end{align*}
    in
    $\Zmod2\left[x_{2^{i_1}},\dotsc,x_{2^{i_q}}\right]/
    \left(x_{2^{i_1}}^{2^{i_1}+1},\dotsc,x_{2^{i_q}}^{2^{i_q}+1}\right)
    \cong \H^*\left(\prod_{r=1}^q \RP{2^{i_r}}\right)$.
  \end{proof}
\end{Thm}

%%% Local Variables:
%%% mode: latex
%%% TeX-master: "thesis"
%%% End:

