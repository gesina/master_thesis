%%%%%%%%%%%%%%%%%%%%%%%%%%%%%%%%%% 
% Master Thesis in Mathematics
% "Immersions and Stiefel-Whitney classes of Manifolds"
% -- Chapter 3: Massey's Theorem --
% 
% Author: Gesina Schwalbe
% Supervisor: Georgios Raptis
% University of Regensburg 2018
%%%%%%%%%%%%%%%%%%%%%%%%%%%%%%%%%% 

\chapter{Massey's Theorem}

Massey's main theorem on the Stiefel-Whitney classes of manifolds
gives a very concrete obstruction on what degrees of the dual
Stiefel-Whitney classes may be non-zero
by connecting this condition with the existence of binary
representations of the manifold's dimension.
\begin{Thm}[Massey]\label{thm:massey}
  \optcite[Theorem~I.]{massey}
  Let $M$ be a compact, $n$-dimensional manifold.
  Given an integer $q$ with $0<q<n$ such that $\dualw{n-q}{M}\neq0$,
  there is a sequence of integers $h_1\geq\dotsb\geq h_q\geq0$ of
  length $q$ that fulfills
  \begin{gather*}
    n = \sum_{i=1}^{q} 2^{h_i}
  \end{gather*}
\end{Thm}

As an immediate consequence all dual Stiefel-Whitney classes of degree
greater than $n-\alpha(n)$ of any manifold must be zero, because there
cannot be any shorter representation of $n$ by powers of two
than its binary representation of length $\alpha(n)$.

The following sections are dedicated to the proof of Massey's Theorem.
Let $M$ be a compact, $n$-dimensional manifold throughout the proof.
The latter consists of several steps:
\begin{steps}
\item\label{tag:masseystep1}
  Show that for any $q$, admissible iterated Steenrod square $\Sq I$, and cohomology
  class $x\in\H^q(M)$ of degree $q$ such that $\Sq I(x)$ is non-trivial there exists
  some representation of the form
  \begin{gather*}
    \deg \left(\Sq I(x)\right)
    = 2^k\cdot
    \left( 2^{k_1}+\dotsb+2^{k_{q-1}} + 1 \right)
    \;.
  \end{gather*}
\item\label{tag:masseystep2}
  Find some iterated Steenrod square which is non-trivial in degree
  $\Sq I\colon\H^q(M)\to\H^n(M)$.
\end{steps}
Applying \ref{tag:masseystep1} to the Steenrod square $\Sq I$ from
\ref{tag:masseystep2} and some $x\in\H^q(M)$ with $\Sq I(x)\neq 0$
immediately yields the result as
\begin{gather*}
  n = \deg\left(\Sq I(x)\right) = \underbrace
  {2^{k_1+k}+\dotsb+2^{k_{q-1}+k} + 2^{k}}_{\text{$q$ summands}}
  \;.
\end{gather*}

\section{Step 1: Disecting Degrees of Iterated Steenrod Squares}% TODO: titel
\ref{tag:masseystep1} requires to proof the following claim.
\begin{Lem}[\ref{tag:masseystep1}]\label{lem:masseystep1}
  Let $q\geq 0$ be an integer,
  and $I\in\Nat^{\l(I)}$ an admissible sequence of integers.
  Further, let $x\in\H^q(M)$ be a cohomology class of degree $q$
  such that $\Sq I(x)$ is non-trivial.
  Then there exists $k\in\Nat$ and a sequence of integers
  $k_1\geq\dotsb\geq k_{q-1}\geq0$ of length $q-1$ such that the
  degree of $\Sq I(x)$ can be represented as
  \begin{gather*}
    \deg \Sq I(x)
    = \deg x + \d(I)
    = 2^k\cdot
    \left( 2^{k_1}+\dotsb+2^{k_{q-1}} + 1 \right)
    \;.
  \end{gather*}
  % Mind that the maximum length of such a sequence is $\deg\Sq I(x)$.
\end{Lem}

In order to split the proof into several cases, recall that
$\Sq I(x) = 0$ for $\e(I)>\deg x$ by
\itemref{rem:sq}{item:squpperboundgeneral}. This leaves the two cases
$\e(I)<\deg x$ and $\e(I)=\deg x$ to be considered.
Applying the following Lemma by Serre inductively, enables one to restrict the
proof of \autoref{lem:masseystep1} to the first case where $\e(I)<q$.
\begin{Lem}[Serre] % TODO: proof 
  \label{lem:serre}
  Every admissible sequence $I\in\Nat^{\l(I)}$ admits an admissible
  sequence $J$ with $\e(J)<\e(I)$,
  together with a power of two $2^k$
  such that for any cohomology class $x\in\H^{\e(I)}$ holds
  \begin{gather*}
    \Sq I(x) = \left(\Sq J(x)\right) ^{2^k}
    \qquad\text{respectively}\qquad
    \deg\left(\Sq I(x)\right) = 2^k\cdot \deg\left(\Sq J(x)\right)
  \end{gather*}
\end{Lem}

Before proving \autoref{lem:serre} one can finish the argumentation for
the case $\e(I)<\deg x$.
\begin{proof}[proof of \autoref{lem:masseystep1}]
  Let $q\in\Nat$ and $I=(i_1,\dotsc,i_l)$ be an admissible sequence such that
  $\e(I)<q$.
  Assume there is a cohomology class $x\in\H^q(M)$ in degree $q$ such
  that $\Sq I(x)\neq0$.
  Set
  \begin{alignat*}{4}
    \alpha_0 &= q-1-\e(I) &\qquad&\text{which is greater 0 as $\e(I)<q$,} \\
    \alpha_r &= i_{r}-2i_{r+1} &&\text{for $1\leq r< \l(I)$, and} \\ 
    \alpha_{\l(I)} &= i_{\l(I)}
    \;.
  \end{alignat*}
  It is an easy excercise that the excess of $I$ can be rewritten as
  $\e(I)=\sum_{r=1}^{\l(I)}\alpha_r$, so
  \begin{gather}\label{eq:alpha0}
    \sum_{r=0}^{\l(I)}\alpha_r = \alpha_0 + \e(I) \cequalsby{Def.} q-1
    \;.
  \end{gather}
  Just as easily one sees
  \begin{align}\label{eq:proofmassey:eq1}
    i_s
    &= \sum_{r=0}^{s}2^r\alpha_{s+r}\\\notag
  \end{align}
  This then directly implies the following reformulation of $\d(I)$ in
  terms of $\alpha_i$ values
  by recursively applying the definitions from above:
  \begin{align}\notag
    \d(I) \coloneqq \sum_{s=1}^{\l(I)}i_s 
    &\cequalsby{\eqref{eq:proofmassey:eq1}}
      \sum_{s=1}^{\l(I)}\sum_{r=0}^{s}2^r\alpha_{s+r}\\\notag
    &\equalsby{Reorder} \sum_{j=1}^{\l(I)}
      \left(\sum_{m=0}^{j-1}2^m\right)\alpha_j
      = \sum_{j=1}^{\l(I)}(2^j-1)\alpha_j 
      = \sum_{j=1}^{\l(I)}2^j\alpha_j
      - \sum_{j=1}^{\l(I)}\alpha_j\\
    \label{eq:proofmassey:eq2}
    &= \sum_{j=1}^{\l(I)}2^j\alpha_j
      - \e(I)
      \;.
  \end{align}
  All put together yields
  \begin{align*}
    \deg\left(\Sq I(x)\right)
    &= \deg(x) + \d(I)\\
    &= 1 + \deg(x) -1 +\d(I) \\
    &\equalsby{\eqref{eq:proofmassey:eq2}}
      1 + q - 1 - \e(I) + \sum_{j=1}^{\l(I)}2^j\alpha_j \\
    &\equalsby{Def.}
      1 + \alpha_0 + \sum_{j=1}^{\l(I)}2^j\alpha_j \\
    &= 1 + \sum_{j=0}^{\l(I)}2^j\alpha_j
    = 1 + \left(
      \underbrace{2^0 +\dotsb+ 2^0}_{\text{$\alpha_0$ times}}
      + \dotsb
      + \underbrace{2^{\l(I)}+\dotsb+2^{\l(I)}}_{\text{$\alpha_{\l(I)}$ times}}
      \right)
  \end{align*}
  which is one plus a sum of exactly
  $\sum_{j=0}^{\l(I)}\alpha_j\cequalsby{\eqref{eq:alpha0}} q-1$
  powers of two as was to be shown.
  The $k_j$ are in this case
  \begin{gather*}
    k_j = \begin{cases}
      0 & 0< j\leq \alpha_0\\
      1 & \alpha_0< j\leq \alpha_1\\
      \vdots\\
      \l(I) & \alpha_{\l(I)-1}<j\leq \alpha_{\l(I)}
    \end{cases}
    \qedhere
  \end{gather*}
\end{proof}

\begin{proof}[proof of \autoref{lem:serre}]
  \optcite[Lemma~1, converse part, p.~159]{serre}
  First note that any sequence $I\coloneqq(i_1,\dotsc,i_l)\in\Nat^l$
  of integers can be written as
  $I=(2^{r-1}i_r,\dotsc,2i_r,i_r,i_{r+1},\dotsc,i_l)$
  for some $r\geq 1$. If $I$ is admissible, the subsequence
  $J\coloneqq(i_{r+1},\dotsc,i_l)$ will be admissible as well.
  Then \autoref{lem:serre} follows directly from
  \begin{claim}
    For $I$ and $J$ as above, and $x\in\H^{\e(I)}(X)$ a cohomology
    class of degree $\e(I)$ of some space $X$, holds
    \begin{enumerate}
    \item $\Sq I(x) = \left(\Sq J(x)\right)^{2^r}$, and
    \item $\e(I)<\e(J)$.
    \end{enumerate}
  \end{claim}
  For the first part one only has to show
  $\deg(\Sq J(x))=\deg(x)+\d(J)=\e(I)+\d(J)=i_r$
  as the statement then inductively follows from property
  \eqref{eq:sqsquared} that $\Sq i(y)=y^2$ for any cohomology class
  with $i=\deg(y)$.
  So calculate
  \begin{align*}
    \d(J) + \e(I)
    &= \d(J) 
      + \left(\sum_{j=1}^{\l(I)-1}(i_j-2i_{j+1})\right)
      + i_{\l(I)}\\
    &= \d(J)
      + \left(\sum_{j=1}^{r-1}\underbrace{(i_j-2i_{j+1})}_{=0}\right)
      + (i_r-2i_{r+1})
      +\left(\sum_{j=r+1}^{\l(I)-1}(i_j-2i_{j+1})\right) + i_{\l(I)}\\
    &= \d(J)
      + (i_r-2i_{r+1})
      + \e(J) \\
    &= \d(J) + i_r - 2i_{r+1} + 2i_{r+1} - \d(J) \\
    &= i_r
  \end{align*}
  Comparing the excesses
  \begin{gather*}
    \e(I) - \e(J)
    = \left(\sum_{j=1}^{r-1}\underbrace{(i_j-2i_{j+1})}_{=0}\right)
    + \underbrace{(i_r - 2i_{r+1})}_{\text{$>0$ by Def. of r}}
    > 0
  \end{gather*}
  yields the second part.
\end{proof}

%%% Local Variables:
%%% mode: latex
%%% TeX-master: "thesis"
%%% End:


