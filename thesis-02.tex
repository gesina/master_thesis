%%%%%%%%%%%%%%%%%%%%%%%%%%%%%%%%%% 
% Master Thesis in Mathematics
% "Immersions and Stiefel-Whitney classes of Manifolds"
% -- Chapter 3: Massey's Theorem --
% 
% Author: Gesina Schwalbe
% Supervisor: Georgios Raptis
% University of Regensburg 2018
%%%%%%%%%%%%%%%%%%%%%%%%%%%%%%%%%% 

\chapter{Massey's Theorem}

\section{Wu's Theorem}
\begin{Thm}[Wu]\label{wu}
  For a closed $n$-dimensional manifold holds:
  \begin{gather*}
    \w(M) = \Sq{}\left(\v(M)\right)
    \qquad\text{respectively}\qquad
    \w_k = \sum_{i\geq0} \Sq i\left(\v_{k-i}(M)\right)
  \end{gather*}
\end{Thm}
\begin{Rem}
  The following is equivalent to the Wu formula
  \begin{gather*}
    \dualw(M) = \Sq{}\left(\dualv(M)\right)
    \qquad\text{respectively}\qquad
    \dualw_k = \sum_{i\geq0} \Sq i\left(\dualv_{k-i}(M)\right)
  \end{gather*}
  using the inverses $\dualw(M)\cup\w(M) = 1$ and
  \begin{gather*}
    \Sq{}(\dualv(M))\cup\Sq{}(\v(M))
    = \Sq{}(\dualv(M)\cup\v(M))
    = \Sq{}(1)
    = 1
    \;.
  \end{gather*}
\end{Rem}
\begin{proof}[proof of \autoref{wu}]
  The following subsections are dedicated to the proof of
  Wu's theorem. The steps conducted for the proof
  are:
  \begin{enumerate} % TODO: Shorten proof of Wu's Theorem?
  \item Give an alternative definition of the Wu classes.
  \item Proof the theorem using this alternative definition.
  \item Show that both definitions are equivalent.
  \end{enumerate}
\end{proof}



%%% Local Variables:
%%% mode: latex
%%% TeX-master: "thesis"
%%% End:


