%%%%%%%%%%%%%%%%%%%%%%%%%%%%%%%%%% 
% Master Thesis in Mathematics
% "Immersions and Stiefel-Whitney classes of Manifolds"
% -- Chapter 3: Massey's Theorem --
% 
% Author: Gesina Schwalbe
% Supervisor: Georgios Raptis
% University of Regensburg 2018
%%%%%%%%%%%%%%%%%%%%%%%%%%%%%%%%%% 

\chapter{Massey's Theorem}

Massey's main theorem on the Stiefel-Whitney classes of manifolds
gives a very concrete obstruction on what degrees of the dual
Stiefel-Whitney classes may be non-zero
by connecting this condition with the existence of binary
representations of the manifold's dimension.
\begin{Thm}[Massey]\label{thm:massey}
  \optcite[Theorem~I.]{massey}
  Let $M$ be a compact, $n$-dimensional manifold.
  Given an integer $q$ with $0<q<n$ such that $\dualw{n-q}{M}\neq0$,
  there is a sequence of integers $h_1\geq\dotsb\geq h_q\geq0$ of
  length $q$ that fulfills
  \begin{gather*}
    n = \sum_{i=1}^{q} 2^{h_i}
  \end{gather*}
\end{Thm}

As an immediate consequence all dual Stiefel-Whitney classes of degree
greater than $n-\alpha(n)$ of any manifold must be zero, because there
cannot be any shorter representation of $n$ by powers of two
than its binary representation of length $\alpha(n)$.

The following sections are dedicated to the proof of Massey's Theorem.
It consists of several steps:
\begin{steps}
\item\label{item:masseystep1}
  Show that for any $q$, iterated Steenrod square $\Sq I$, and cohomology
  class $x\in\H^q(M)$ of degree $q$ such that $\Sq I(x)$ is non-trivial there exists
  some representation of the form
  \begin{gather*}
    \deg \Sq I(x)
    = 2^k\cdot
    \left( 2^{k_1}+\dotsb+2^{k_{q-1}} + 1 \right)
    \;.
  \end{gather*}
\item\label{item:masseystep2}
  Find some iterated Steenrod square which is non-trivial in degree
  $\Sq I\colon\H^q(M)\to\H^n(M)$.
\end{steps}
Applying \ref{item:masseystep1} to the Steenrod square $\Sq I$ from
\ref{item:masseystep2} and some $x\in\H^q(M)$ with $\Sq I(x)\neq 0$
immediately yields the result as
\begin{gather*}
  n = \Sq I(x) = \underbrace
  {2^{k_1+k}+\dotsb+2^{k_{q-1}+k} + 2^{k}}_{\text{$q$ summands}}
  \;.
\end{gather*}




  
%%% Local Variables:
%%% mode: latex
%%% TeX-master: "thesis"
%%% End:


