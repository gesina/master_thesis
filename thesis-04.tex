%%%%%%%%%%%%%%%%%%%%%%%%%%%%%%%%%% 
% Master Thesis in Mathematics
% "Immersions and Stiefel-Whitney classes of Manifolds"
% -- Chapter 4: Outlook --
% 
% Author: Gesina Schwalbe
% Supervisor: Georgios Raptis
% University of Regensburg 2018
%%%%%%%%%%%%%%%%%%%%%%%%%%%%%%%%%% 

\chapter{Outlook}\label{chap:outlook}
This chapter means to give an overview of the main steps towards a
proof of the immersion conjecture, and an outline of the role of
the theorems of Massey and Brown for the development.
The proof discussed will be the one finalised by R.~L.~Cohen in
his dissertation \cite{cohen}, and is geared to his highly
recommendable lecture notes \cite{immersionconj} on this topic.
Proofs of the following statements are omitted. For details the reader
is advised to consult the latter reference if no other is given.

Normal bundle will be considered to arise from immersions into
Euclidean space if not stated otherwise. 
Vector bundles will be identified with their classifying maps.
% TODO: The reader is assumed to have sound knowledge of ....

Recall from \autoref{chap:reformulation} that the statement in
question is:
\begin{Thm*}
  For $n\in\Nat$, every closed smooth $n$-manifold immerses into
  $\R^{2n-\alpha(n)}$.
\end{Thm*}
Furthermore, it was shown that this is equivalent to:
\begin{Thm*}
  For $n\in\Nat$, and any closed smooth $n$-manifold,
  the classifying map $\N M\colon M\to\BO$ factors over the
  inclusion $\BO(n-\alpha(n))\to\BO$.
\end{Thm*}
This theorem essentially says that universal $\O(n-\alpha(n))$ vector
bundle is the better choice for classifying normal bundles of
$n$-manifolds, rather than that of $\BO$. Here better is meant in
terms of partial order by factorisation.
This is not astonishing, as $\BO$ is a 
universal space of \emph{all} vector bundles up to stable
equivalence over \emph{all} (sufficiently nice) spaces.
Due to Whitney's immersion theorem, a first improvement of $\BO$ is
the universal bundle over $\BO(n-1)$.
The trick of the proof of the immersion conjecture now is to
find a universal vector bundle $\BOmodI n\to\BO$ for normal
bundles of $n$-manifolds, that is even better than
$\BO(n-\alpha(n))$. In other words a space and a map
\begin{gather*}
  \BOmodI n\longto\BO(n-\alpha(n))\longto\BO
\end{gather*}
over which  all classifying maps of stable normal bundles factor.

More precisely, for a fixed $n\in\Nat$ the proof
encompasses the following main steps:
\begin{enumerate}[1.]
\item
  Get an idea of the statement on cohomology, and determine up to a
  Thom isomorphism the ideal $\I n\subset\H^*(\BO)$ of characteristic
  classes that vanish on all normal bundles of $n$-manifolds.
  This turns out to be a generalisation of Massey's theorem on
  characteristic classes of normal bundles.
\item
  Find a spectrum $\MOmodI n$ with a map
  $\MOmodI n\to\MOspec$ which acts as a universal
  spectrum for Thom spectra of normal bundles of $n$-manifolds,
  \idest all maps $\Tspec{\N{M^n}}\to \MOspec$ of a Thom spectrum of a
  normal bundle factor over the above map.
  Furthermore, the map $\MOmodI n\to\MOspec$ should factor
  over $\MOspec(n-\alpha(n))\to\MOspec$, where $\MOspec(n-\alpha(n))$
  is the Thom spectrum of the universal vector bundle
  over $\BO(n-\alpha(n))$.
\item
  Find a sufficient criterion when a map of a spectrum into the Thom
  spectrum of a bundle \emph{de-Thom-ifies}, \idest sort of
  comes from a map of Thom spectra which is induced by a map
  of vector bundles.
\item
  Construct a space $\BOmodI n$ by de-Thom-ifying a suitable resolution
  of the map $\MOmodI n\to\MOspec$, and taking the
  $n$-skeleton of the inverse limit of the resulting resolution.
  This can then be shown to act as a universal space for normal
  bundles of manifolds together with the map $\BOmodI n\to\BO$ from its
  definition.
\item
  Show that there is a space $X^n$ and a map
  $X^n\to\BO(n-\alpha(n))\to\BO$,
  which factors over $\BOmodI n$, and admits some nice splitting and
  commutativity properties on the Thom spectra corresponding to the
  classifying maps.
\item
  Show that the latter implies that
  the map $\BOmodI n\to\BO$ factors over the universal bundle
  $\BO(n-\alpha(n))\to\BO$.
\end{enumerate}
In the end, one has a the space $\BOmodI n$ which admits for
any $n$-manifold a commutative diagram
\begin{center}
  \begin{tikzcd}[column sep=small]
    M \ar[r]
    \ar[drr, "\N M", bend right=15]
    &\BOmodI n
    \ar[r]
    &\BO(n-\alpha(n))
    \ar[d, "\incl"]\\
    &&\BO
  \end{tikzcd}
\end{center}
as was to be shown for the immersion conjecture.

The last two steps turn out to be quite tricky and are the topic of
Cohen's dissertation \cite{cohen}. This is also the point where the
R.~L.~Brown's theorem from \autoref{chap:brown} comes in, saying
that the immersion conjecture is true up to cobordism.
It is needed for some choices in the construction of $X^n$.
The other results were investigated and found by E.~H.~Brown~Jr. and
F.~P.~Peterson.

In the following a couple of aspects of the above steps are filled
with more details.
For this fix some $n\in\Nat$.

\section*{A Generalisation of Massey's Theorem}
In search of properties of such a universal bundle
$\rho\colon X\to\BO$ for stable normal bundles of $n$-manifolds,
one idea is that the \enquote{best} such one should fulfil
\begin{gather*}
  \ker(\pb\rho)
  = \bigcap_{\text{$M$ mfd.}}
  \ker(\pb{\N M}\colon\H^*(\BO)\to \H^*(M)
  \eqqcolon \I n
  \;,
\end{gather*}
\idest all characteristic classes from $\H^*(\BO)$ that vanish along
normal bundle pullbacks, already vanish along $\pb\rho$.
Note that by Massey's theorem the ideal
$\left(\ws{i}|i>n-\alpha(n)\right)$ lies within $\I n$.

\begin{Not*}
  For $k\in\Nat$ let
  \begin{itemize}
  \item $\univbdl{k}$ be the universal bundle of $\Orth(k)$, 
  \item $\MO(k)\cong\Thomspace{\univbdl{}}$ the corresponding Thom space,
  \item $\MOspec\coloneqq\{\MO(k)|k\in\Nat\}$ be the universal Thom spectrum.
  \end{itemize}
\end{Not*}

\begin{Rem*}
Recall that the Thom isomorphism extends to an isomorphism
\begin{gather*}
  \thomiso\colon \H^*(\BO) \to \H^*(\MOspec)
\end{gather*}
(see \forexample \cite{milnor}).
Furthermore, note that $\H^*(\MOspec)$ is known to be a free module
over the Steenrod algebra $\A$ of the form
\begin{gather*}
  \H^*(\MOspec)
  = (\sigma_i| i\neq 2^s-1)_{\A}
  \cong \A\otimes \Zmod2[\sigma_i|i\neq 2^s-1]
  \cong \bigoplus_{\mathclap{\deg(I)\neq 2^s-1}}
  \A_{(*+\deg(I))}\sigma^I
\end{gather*}
where the $\sigma_i$ are algebraically independent elements,
$\deg(\sigma_i)=i$, and $\deg(I)\coloneqq\deg(\sigma^I)$.
Furthermore, the last isomorphism is one of $\A$-modules, and the sum
is over arbitrary sequences $I$ of positive integers
(see \cite[p.~82]{immersionconj}). % TODO
\end{Rem*}

Brown and Peterson investigated the image of $\I n$ under the Thom
isomorphism.
The meaning of this ideal is the following:
Consider an $n$-manifold $M$, and the map
\begin{gather*}
  \T{\N M}\colon\Tspec{\N M}\to\MOspec
\end{gather*}
induced by the pullback of $\N M$ along the classifying map.
Then $\ker\pb{(\T{\N M})} = \thomiso(\I n)$.
Their result was the following.
\begin{Thm*}[{\cite[Theorem~2.3]{immersionconj}}]
  Define
  \begin{gather*}
    \J r
    \coloneqq \bigcap_{\text{$M^n$ mfd.}}
    \left\{ a\in\A \middle| a(\u{\N M}) = 0 \right\}
    \subset \A
    \;.
  \end{gather*}
  Then
  \begin{align*}
    \J r
    &= \A \cdot \left\{\antipode(\Sq i) \middle| 2i>r\right\}
    &\text{and}&
    &\H^*(\MOspec)/\thomiso(\I n)
    &\cong \bigoplus_{\mathclap{n\geq\deg(I)\neq 2^s-1}}
      \left( \A / \J{n-\deg(I)} \right)_{*+\deg(I)}
      \;.
  \end{align*}
  Or in other words
  \begin{gather*}
    \thomiso(\I n)
    = \H^{*>n}(\MOspec)
    + \A\cdot \left\{
      \antipode(\Sq i)(\sigma^I) \middle| 2i>n-\deg(\sigma^I)
    \right\}
  \end{gather*}
\end{Thm*}

For now note that Massey's theorem~\ref{thm:massey} is a corollary of
the above statement:
\begin{Rem*}
  Note that the projection of
  \begin{gather*}
    \left\{
      \antipode(\Sq I) \middle|
      I=(i_1,\dotsc,i_l) \text{admissible, } 2i_1\leq r
    \right\}
  \end{gather*}
  is a $\Zmod2$-vector space basis of $\A/\J r$.
  Taking a closer look at these elements, one notes that the
  largest dimension of elements occurring is $r-\alpha(r)$.
  Therefore, for $i>r-\alpha(r)$, $\Sq i$ is sent to zero under the
  projection of $\A/\J r$, respectively lies in $\J r$.
  Thus, $\Sq i\in\thomiso(\I n)$ for
  $i>n-\alpha(n)=\max_k\{(n-k)-\alpha(n-k)\}$.
  This yields the statement of Massey's theorem, since for an
  $n$-manifold $M$
  \begin{gather*}
    \thomiso(\w{i}{\N M})
    = \Sq i(\u{\N M})
    \in \Sq i(\Im\pb{(\T{\N M})})
    = \pb{(\T{\N M})} (\Sq i\cdot \H^*(\MOspec))
    = 0
    \;.
  \end{gather*}
\end{Rem*}

\section*{The Construction of $\MOmodI n$}
The summands in $\H^*(\MOspec)/\thomiso(\I n)$ are isomorphic to the
cohomology rings of the Brown-Gitler spectra.
Thus, the form of $\H^*(\MOspec)/\thomiso(\I n)$ gives a good hint on how
to proceed in the search of some universal spectrum for Thom spaces of
normal bundles.

\begin{Not*}
  % Let $\Braid k$ be the $k$th braid group and
  % $f\colon\Braid k\to\Orth(k)$ the map representing elements in
  % $\Braid k$ as permutation matrices (see
  % \cite[§2]{immersionconj} for a detailed construction).
  In the following denote by $\BrownGitler k$ the $k$th Brown-Gitler
  spectrum, thus especially $\H^*(\BrownGitler k) \cong \A/\J k$
  (see \cite{browngitler} or \cite[p.~101]{immersionconj} for
  definition and a construction).
  Further, let $\EMspec$ be the Eilenberg-MacLane spectrum for
  $\Zmod2$.
\end{Not*}

\begin{Rem*}
  Recall that Thom calculated
  \begin{gather*}
    \MOspec
    \cong \bigvee_{\mathclap{\deg(I)\neq 2^s-1}}
    \susp^{\deg(I)} \EMspec
  \end{gather*}
  (see \cite[p.~81f]{immersionconj} for a good summary of the arguments).
\end{Rem*}

\begin{Def*}
  For $k\in\Nat$ let $y_k$ be the generator of
  $\H^*(\BrownGitler k)\cong \A/\J k$ as $\A$-module,
  \idest $\H^*(\BrownGitler)=\A\cdot y_k$.
  Define
  \begin{gather*}
    \bm{\rho}_n\colon
    \MOmodI n
    \coloneqq \bigvee_{\mathclap{n\geq\deg(I)\neq 2^s-1}}
    \susp^{\deg(I)} \BrownGitler{n-\deg{I}}
    \to \MOspec
  \end{gather*}
  where the map is the wedge product given by the cohomology elements
  \begin{gather*}
    y_k\colon\BrownGitler k \to \EMspec
    \;.
  \end{gather*}
\end{Def*}

\begin{Thm*}
  $\bm{\rho}_n\colon\MOmodI n\to\MOspec$ fulfils 
  \begin{enumerate}
  \item\label{item:momodinprop:1}
    $\pb{\bm{\rho}}_n=\proj\colon
    \H^*(\MOspec)
    \to \H^*(\MOspec)/\thomiso(\I n)\cong\H^*(\MOmodI n)$
  \item\label{item:momodinprop:2}
    $\MOmodI n$ together with $\bm{\rho}$ is a universal Thom space
    for Thom spaces of normal bundles of $n$-manifolds.
    \Idest for every $n$-manifold there exists a lift $\bm{f}_M$ of
    $\thommap{\N M}$ 
    such that the following diagram commutes up to homotopy
    \begin{center}
      \begin{tikzcd}
        \Tspec{\N M} \ar[r, "\bm{f}_M"]
        \ar[dr, bend right=20, "\thommap{\N M}"{below}]
        &\MOmodI n
        \ar[d, "\bm{\rho}"]\\
        &\MOspec
      \end{tikzcd}
    \end{center}
  \item\label{item:momodinprop:3}
    $\bm{\rho}$ factors up to homotopy over
    $\MOspec(n-\alpha(n))\to\MOspec$.
  \end{enumerate}
  \begin{proof}[Outline of the proof]
    For a proof see \cite[Lemma~2.28, Theorem~2.29]{immersionconj}.
    Statement \ref{item:momodinprop:1} requires mainly the
    considerations from above, thus follows rather directly from the
    definition.
    For \ref{item:momodinprop:2} one constructs a lift piece-wise by
    finding lifts on the wedge-product factors, \idest for any
    sequence $I$ with $n\geq\deg(I)\neq2^s-1$:
    \begin{center}
      % \begin{tikzcd}
      %   &&\susp^{\deg(I)}\BrownGitler{n-\deg(I)}
      %   %\ar[d, "\susp^{\deg(I)}y_k"]
      %   \\
      %   %\ar[urr, bend left=20, dashed]
      %   \Tspec{\N M} \ar[r, "f_M"]
      %   %\ar[r, "\thommap{\N M}"{below}]
      %   &\MOspec
      %   %\ar[r, "\proj"]
      %   &\susp^{\deg(I)}\EMspec}
      % \end{tikzcd}
    \end{center}
    Lastly, \ref{item:momodinprop:3} uses that $\BrownGitler k$ has
    the homotopy type of a CW complex of dimension $k-\alpha(k)$,
    which makes it possible to apply an obstruction argument to find
    the required lift.
  \end{proof}
\end{Thm*}

This gives the desired intermediate result.


\section*{A Construction of $\BOmodI n$}
Now that a nice universal spectrum for Thom spaces of normal bundles
of $n$-manifolds is found, one has to somehow \emph{de-Thom-ify} this
result to a universal space of normal bundles with the desired
properties.

Let us first specify de-Thom-ification.
\begin{Def}
  For a bundle $\xi\colon X\to\BO$, a map of spectra
  $\bm{\tau}\colon\spectrum{Z}\to\Tspec{\xi}$ is said to
  \emph{de-Thom-ify} through dimension $k$ if there is
  \begin{itemize}
  \item a map of spaces $g\colon Y\to X$, and
  \item a $k$-connected map of spectra
    $\bm{\kappa}\colon\Tspec{(\pb g\xi)}\to\spectrum{Z}$,
  \end{itemize}
  such that the following diagram homotopy commutes
  \begin{center}
    \begin{tikzcd}
      \spectrum{Z} \ar[r, "\bm{\tau}"]
      \ar[d, "\bm{\kappa}"]
      &\Tspec{\xi}\\
      \ar[ur, bend right=20, "\thommap{g}"{below}]
      \Tspec{(\pb g\xi)}
    \end{tikzcd}
  \end{center}
  .
\end{Def}


It is a key de-Thom-ification obstruction
that a map on Thom spectra induced by a map on vector bundles does
admit linearity with respect to an additional module structure
(see \cite[p.~107ff]{immersionconj}).
\begin{Lem*}
  \begin{enumerate}
  \item On the tensor product $\A\otimes\H^*(\BO)$ define the multiplication
    \begin{gather*}
      (a\otimes x) \cdot (b\otimes y)
      \coloneqq \sum_{i\in A}
      \left(a\circ b_i'\right) \otimes \left(\antipode(b_i'')(x)\cdot y\right)
    \end{gather*}
    where $b_i', b_i''$ is such that $\sum_{i\in A}b_i'\otimes b_i''$ is
    the image of $b$ under the diagonal map of $\A$.
    This turns $\A\otimes\H^*(\BO)$ into a ring.
  \item For any classifying map $\xi\colon X\to\BO$, the
    corresponding $\H^*(\BO)$-module structure on $\H^*(X)$
    induces the $(\A\otimes\H^*(\BO))$-structure on $\H^*(\Tspec{\xi})$
    \begin{gather*}
      (a\otimes x)\cdot \thomiso(y) = a\left( \thomiso(\pb\xi x\cup y \right)
    \end{gather*}
    via the Thom isomorphism.
  \item For any map of vector bundles, the induced map on Thom spaces
    is $(\A\otimes\H^*(\BO))$-linear on cohomology.
  \end{enumerate}
\end{Lem*}

In the spirit of the above observation, one gets even more additional
structure around Thom maps induced by pullbacks of vector bundles.
An essential result is the following de-Thom-ification criterion:
\begin{Thm*}
  Let $k\in\Nat$ and $\xi\colon X\to\BO$ be $k$-connected.
  Assume there is a map of spectra
  $\bm{\tau}\colon\spectrum{Z}\to\Tspec{\xi}$.
  Then $\bm{\tau}$ de-Thom-ifies through dimension $2k$ if and only if
  in dimensions lower or equals $2k$ the cohomology ring
  $\H^*(\spectrum{C}_{\bm{\tau}})$ of the mapping cone of $\bm{\tau}$ is
  a free module over $\A\otimes\H^*(\BO)$.  
  \begin{proof}[Outline of the proof]
    See \cite[Theorem~3.5]{immersionconj}.
    For the necessity one should observe that a pullback $g\colon Y\to X$ of
    vector bundles can be modified to a certain 2-stage Postnikov tower
    \begin{gather*}
      Y'\longto X\overset{\xi}\longto\BO
    \end{gather*}
    without changing the homotopy type of $\Tspec{(\pb g\xi)}$ up to
    dimension $2k$.
    Then one can use a theorem by Brown and Peterson, which yields the
    needed result for the cohomology of the modified Thom mapping cone
    (see \cite[Theorem~3.3]{immersionconj}).
    For the other direction, a construction that rebuilds a system as
    the above one using the Thom isomorphism can be applied.
  \end{proof}
\end{Thm*}


Brown and Peterson found a nice Adams resolution of spectra
\begin{gather*}
  \MOmodI n \to\dotsc \to \spectrum{Z}_1 \to \spectrum{Z}_0 = \MOspec
\end{gather*}
which in each stage de-Thom-ifies through dimension $n$ according to
the above criterion.
This finally gives:
\begin{Def}
  Let
  \begin{gather*}
    \dotsc \to Y_1 \to Y_0 = \BO
  \end{gather*}
  be the result of the de-Thom-ification of the above resolution.
  Define $\BOmodI n$ as the $n$-skeleton of the inverse limit of the
  $Y_i$, and the map $\rho\colon\BOmodI n\to\BO$ by
  \begin{gather*}
    \rho\colon\BOmodI n\to\dotsc\to Y_1\to Y_0=\BO
    \;.
  \end{gather*}
\end{Def}

\begin{Thm*}
  The special choice of the chosen resolution equips
  $\rho\colon\BOmodI n\to\BO$ with the following properties:
  \begin{enumerate}
  \item
    $\thommap{\rho} = \bm{\rho}\colon \Tspec{\rho}\simeq\MOmodI n\to\MOspec$.
  \item
    $\pb\rho=\proj\colon
    \H^*(\BOmodI n) \to \H^*(\BO)/\I n\cong\H^*(\BOmodI n)$
    (this comes from the choice as de-Thom-ification of
    $\bm{\rho}\colon\MOmodI n\to\MOspec$ and application of the Thom
    isomorphism).
  \item
    $\rho$ acts as universal bundle for stable normal bundles of
    $n$-manifolds.
    \Idest for every $n$-manifold there exists a lift $f_M$ of
    $\N M$ 
    such that the following diagram commutes up to homotopy
    \begin{center}
      \begin{tikzcd}
        M \ar[r, "f_M"]
        \ar[dr, bend right=20, "\N M"{below}]
        &\BOmodI n
        \ar[d, "\rho"]\\
        &\BO
      \end{tikzcd}
    \end{center}
  \end{enumerate}
\end{Thm*}

% TODO: say that lift is missing

\section*{A Construction of the Lift $\BOmodI n\to\BO(n-\alpha(n))$}
% TODO

%\optcite[Theorem~3.34]{The Topology of Fiber Bundles Lecture Notes, Cohen}

%%% Local Variables:
%%% mode: latex
%%% TeX-master: "thesis"
%%% ispell-local-dictionary: "en_GB"
%%% End:
