%%%%%%%%%%%%%%%%%%%%%%%%%%%%%%%%%% 
% Master Thesis in Mathematics
% "Immersions and Stiefel-Whitney classes of Manifolds"
% -- Chapter 4: Outlook --
% 
% Author: Gesina Schwalbe
% Supervisor: Georgios Raptis
% University of Regensburg 2018
%%%%%%%%%%%%%%%%%%%%%%%%%%%%%%%%%% 

\chapter
{An Outline of the Proof of the Immersion Conjecture}
\label{chap:outlook}
% Requirements:
% (Optional) Give an outline of the work of Brown-Peterson and Cohen for
% the proof of the Immersion Conjecture.

This chapter gives an overview of the main steps towards a proof 
of the immersion conjecture.
In the course of this it becomes clear how the key theorems discussed
in this thesis---Theorem~\ref{thm:massey} by Massey and
Theorem~\ref{thm:brown} by R.~L.~Brown---contributed to the
development in this area.
The proof discussed will be the one based on the work of
E.~H.~Brown~Jr., F.~P.~Peterson \cite{brownpeterson}, which was
finalised by R.~L.~Cohen in his dissertation \cite{cohen}. The
outlines here are geared to Cohen's highly recommendable lecture notes
\cite{immersionconj} on this topic.

Proofs of the following statements are either sketched or omitted. For
details the reader is advised to consult the latter reference if no
other is given. Even though the proof of the immersion conjecture
itself requires profound knowledge of the theory of spectra, basic
knowledge should suffice to follow the outlines below. See
\cite[p.~77ff]{immersionconj} for a very brief introduction, or
\cite{switzer} for a more complete picture.


\begin{Not*}
  Normal Bundles will be considered to arise from immersions of
  manifolds into Euclidean space, and
  vector bundles will be identified with their classifying maps.
  For a vector bundle $\xi\colon X\to\BO$ denote its Thom spectrum by
  $\Tspec{\xi}$,
  and for a map of spaces $f\colon Y\to X$ denote the induced map of
  Thom spaces resp. spectra by
  $\thommap{f}\colon\Tspec{(\pb f\xi)}\to\Tspec{\xi}$.
  Denote by $\EMspec$ the Eilenberg-MacLane spectrum for $\Zmod2$.
  For $k\in\Nat$ let
  \begin{itemize}
  \item $\univbdl{k}$ be the universal bundle of $\Orth(k)$, 
  \item $\MO(k)\cong\Thomspace{\univbdl{}}$ the corresponding Thom space,
  \item $\MOspec(k)$ the corresponding Thom spectrum,
  \item $\MOspec\coloneqq\{\MO(k)|k\in\Nat\}$ be the universal Thom
    spectrum, and
  \item $\Spherespec{k}$ be the $k$th suspension of the sphere spectrum.
  \end{itemize}
  For a sequence $I=(i_1,\dotsc,i_l)$ of positive integers, denote by
  $\Deg{I}\coloneqq\deg(\sigma^I)$ the degree of the monomial
  $\sigma^{i_1}\dotsm\sigma^{i_l}$ in variables $\sigma_i$ with
  $\deg(\sigma_i)=i$.
  Lastly, fix some $n\in\Nat$ throughout this chapter.
\end{Not*}


Recall from \autoref{chap:reformulation} that the statement in
question is:
\begin{Thm*}
  Every closed smooth $n$-manifold immerses into
  $\R^{2n-\alpha(n)}$.
\end{Thm*}
Furthermore, it was shown that this is equivalent to:
\begin{Thm*}
  For any closed smooth $n$-manifold,
  the classifying map $\N M\colon M\to\BO$ factors over the
  inclusion $\BO(n-\alpha(n))\to\BO$.
\end{Thm*}
This theorem essentially says that the universal
$\Orth(n-\alpha(n))$-vector bundle is a better choice for
classifying normal bundles of $n$-manifolds than
$\BO$.
% This is not astonishing, as $\BO$ is a 
% universal space of \emph{all} vector bundles up to stable
% equivalence over \emph{all} (sufficiently nice) spaces.
% Due to Whitney's immersion theorem, a first improvement of $\BO$ is
% the universal bundle over $\BO(n-1)$.
The idea for the proof of the immersion conjecture is to
find a vector bundle $\BOmodI n\to\BO$ which
\begin{enumerate}[1.]
\item
  acts as universal bundle for normal bundles of $n$-manifolds, in
  the sense that all normal bundle maps factor over it, and which
\item
  is even better than $\BO(n-\alpha(n))$ in classifying normal
  bundles, \idest it lifts to $\BO(n-\alpha(n))$.
\end{enumerate}
In other words, we are looking for a space later called $\BOmodI n$
and a map
\begin{gather*}
  \rho_n\colon \BOmodI n\longto\BO(n-\alpha(n))\longto\BO
\end{gather*}
over which all classifying maps of stable normal bundles factor.
More precisely, for a fixed $n\in\Nat$ the proof
encompasses the following main steps:
\begin{steps}
\item
  Get an idea of the statement on cohomology, and determine up to a
  Thom isomorphism the ideal $\I n\subset\H^*(\BO)$ of characteristic
  classes that vanish on all normal bundles of $n$-manifolds.
  This turns out to be a generalisation of Massey's theorem on
  characteristic classes of normal bundles (Theorem~\ref{thm:massey}).
\item
  Find a spectrum $\MOmodI n$ with a map
  $\bm{\rho}_n\colon \MOmodI n\to\MOspec$ which acts as a universal
  spectrum for Thom spectra of normal bundles of $n$-manifolds,
  \idest all maps $\Tspec{\N{M^n}}\to \MOspec$ of a Thom spectrum of a
  normal bundle factor over the above map.
  Furthermore, $\bm{\rho}_n$ should factor
  over $\MOspec(n-\alpha(n))\to\MOspec$.
\item
  Find a sufficient criterion when a map of a spectrum into the Thom
  spectrum of a bundle \emph{de-Thom-ifies}, \idest sort of
  comes from a map of Thom spectra which is induced by a map
  of vector bundles.
\item
  Construct a de-Thom-ification bundle $\BOmodI n\to\BO$ of $\MOmodI
  n\to\MOspec$ by de-Thom-ifying a suitable resolution,
  and show that this acts as a universal space for normal
  bundles of manifolds.
\item
  Show that there is a space $X^n$ and a map
  $X^n\to\BOmodI n\to\BO$, which factors over $\BO(n-\alpha(n))$, and
  admits some nice splitting and  commutativity properties on the Thom
  spectra corresponding to the classifying maps.
\item
  Show that the latter implies that the map $\BOmodI n\to\BO$ factors
  over the universal bundle $\BO(n-\alpha(n))\to\BO$.
\end{steps}
In the end, one has a space $\BOmodI n$ which admits for
any $n$-manifold a commutative diagram
\begin{center}
  \begin{tikzcd}[column sep=small]
    M \ar[r]
    \ar[drr, "\N M", bend right=15]
    &\BOmodI n
    \ar[r]
    &\BO(n-\alpha(n))
    \ar[d, "\incl"]\\
    &&\BO
  \end{tikzcd}
\end{center}
as was to be shown for the immersion conjecture.

The last two steps turn out to be quite tricky and are the topic of
Cohen's dissertation \cite{cohen}. This is also the point where
R.~L.~Brown's theorem from \autoref{chap:brown} comes into play.
It is needed for some choices in the construction of $X^n$.
The other results are due to E.~H.~Brown~Jr. and F.~P.~Peterson.

In the following a couple of aspects of the above steps are filled
with more details.


\section*{A Generalisation of Massey's Theorem}
In search of properties of such a universal bundle
$\rho_n\colon X_n\to\BO$ for stable normal bundles of $n$-manifolds,
one idea is that the \enquote{best} such one should fulfil
\begin{gather*}
  \ker(\pb\rho_n)
  = \bigcap_{\text{$M$ mfd.}}
  \ker\left(\pb{\N M}\colon\H^*(\BO)\to \H^*(M)\right)
  \eqqcolon \I n
  \;,
\end{gather*}
\idest all characteristic classes from $\H^*(\BO)$ that vanish for
normal bundles, already vanish for the bundle $\rho_n$.
Note that by Massey's theorem the ideal
$\left(\ws{i}\,\middle|\,i>n-\alpha(n)\right)$ lies within $\I n$.

\begin{Rem*}
Recall that the Thom isomorphism extends to an isomorphism
\begin{gather*}
  \thomiso\colon \H^*(\BO) \to \H^*(\MOspec)
\end{gather*}
(see \forexample \cite{milnor}).
Furthermore, note that $\H^*(\MOspec)$ is known to be a free module
over the Steenrod algebra $\A$ of the form
\begin{gather*}
  \H^*(\MOspec)
  = (\sigma_i| i\neq 2^s-1)_{\A}
  \cong \A\otimes \Zmod2[\sigma_i|i\neq 2^s-1]
  \cong \bigoplus_{\mathclap{\Deg{I}\neq 2^s-1}}
  \A_{(*+\Deg{I})}\sigma^I
\end{gather*}
where the $\sigma_i$ are algebraically independent elements,
$\deg(\sigma_i)=i$.
Furthermore, the last isomorphism is one of $\A$-modules, and the sum
is over arbitrary sequences $I$ of positive integers
(see \forexample \cite[p.~82]{immersionconj} or
\cite[Chap.~20]{switzer}).
\end{Rem*}

Brown and Peterson investigated the image of $\I n$ under the Thom
isomorphism, which is characterised by the property that for any
$n$-manifold $M$, $\thomiso(\I n)=\ker(\pb{\thommap{\N M}})$.
Their result was the following.
\begin{Thm*}[{\cite[Theorem~2.3]{immersionconj}}]
  One has the following equality of left-ideals of $\A$:
  \begin{gather*}
    \J r
    \coloneqq \bigcap_{\text{$M^n$ mfd.}}
    \left\{ a\in\A \middle| a(\u{\N M}) = 0 \right\}
    = \A \cdot \left\{\antipode(\Sq i) \middle| 2i>r\right\}
    \subset \A
    \;.
  \end{gather*}
  Furthermore, with notation as above,
  \begin{align*}
    \H^*(\MOspec)/\thomiso(\I n)
    &\cong \bigoplus_{\mathrlap{n\geq\Deg{I}\neq 2^s-1}}
      \left( \A / \J{n-\Deg{I}} \right)_{*+\Deg{I}}
      \;,\qquad\text{thus}\\
    \thomiso(\I n)
    &= \H^{*>n}(\MOspec)
    + \A\cdot \left\{
      \antipode(\Sq i)(\sigma^I) \middle| 2i>n-\deg(\sigma^I)
    \right\}
      \;.
  \end{align*}
\end{Thm*}

Note that Massey's theorem~\ref{thm:massey} can be obtained
from the above statement as follows:
\begin{Rem*}
  Note that the projection of
  \begin{gather*}
    \left\{
      \antipode(\Sq I) \middle|
      I=(i_1,\dotsc,i_l) \text{ admissible, } 2i_1\leq r
    \right\}
  \end{gather*}
  is a $\Zmod2$-vector space basis of $\A/\J r$.
  Taking a closer look at these elements, one notes that the
  largest dimension occurring is $r-\alpha(r)$.
  Therefore, for $i>r-\alpha(r)$, $\Sq i$ is sent to zero under the
  projection of $\A/\J r$, respectively lies in $\J r$.
  Thus, $\Sq i$ lies in $\thomiso(\I n)$ for
  $i>n-\alpha(n)=\max_k\{(n-k)-\alpha(n-k)\}$.
  This yields the statement of Massey's theorem, since for an
  $n$-manifold $M$ one gets
  \begin{gather*}
    \thomiso(\w{i}{\N M})
    \cequalsby{\ref{thm:altdefswclasses}} \Sq i(\u{\N M})
    \in \Sq i(\Im\pb{(\T{\N M})})
    = \pb{(\T{\N M})} (\Sq i\cdot \H^*(\MOspec))
    = 0
    \;.
  \end{gather*}
  See also \cite[p.~93]{immersionconj}.
\end{Rem*}

\section*{The Construction of $\MOmodI n$}
The summands in $\H^*(\MOspec)/\thomiso(\I n)$ are isomorphic to the
cohomology rings of the Brown-Gitler spectra.
Thus, the form of $\H^*(\MOspec)/\thomiso(\I n)$ gives a good hint on how
to proceed in the search of some universal spectrum for Thom spaces of
normal bundles.

\begin{Rem*}
  Recall that Thom calculated
  \begin{gather*}
    \MOspec
    \cong \bigvee_{\mathclap{\Deg{I}\neq 2^s-1}}
    \Spherespec{\Deg{I}} \smashprod \EMspec
  \end{gather*}
  (see \cite[p.~81f]{immersionconj} for a good summary of the arguments).
\end{Rem*}

\begin{Def*}
  In the following denote by $\BrownGitler k$ the $k$th
  Brown-Gitler spectrum, thus especially
  $\H^*(\BrownGitler k)\cong\A/\J k$
  (see \cite{browngitler} or \cite[p.~101]{immersionconj} for
  definition and a construction).
  For $k\in\Nat$ let $y_k$ be the generator of
  $\H^*(\BrownGitler k)\cong \A/\J k$ as $\A$-module,
  \idest $\H^*(\BrownGitler k)=\A\cdot y_k$.
  Define
  \begin{gather*}
    \bm{\rho}_n\colon
    \MOmodI n
    \coloneqq \bigvee_{\mathrlap{n\geq\Deg{I}\neq 2^s-1}}
    \Spherespec{\Deg{I}} \smashprod \BrownGitler{n-\Deg{I}}
    \longto
    \bigvee_{\mathrlap{\Deg{I}\neq 2^s-1}}
    \Spherespec{\Deg{I}} \smashprod \EMspec
    \cong \MOspec
  \end{gather*}
  where the map is the wedge product given by the cohomology elements
  \begin{gather*}
    y_k\colon\BrownGitler k \to \EMspec
    \;.
  \end{gather*}
\end{Def*}

\begin{Thm*}
  $\bm{\rho}_n\colon\MOmodI n\to\MOspec$ fulfils 
  \begin{enumerate}
  \item\label{item:momodinprop:1}
    $\pb{\bm{\rho}}_n=\proj\colon
    \H^*(\MOspec)
    \to \H^*(\MOspec)/\thomiso(\I n)\cong\H^*(\MOmodI n)$
  \item\label{item:momodinprop:2}
    $\MOmodI n$ together with $\bm{\rho}_n$ is a universal Thom spectrum
    for Thom spectra of normal bundles of $n$-manifolds.
    \Idest for every $n$-manifold there exists a lift $\bm{f}_M$ of
    $\thommap{\N M}$ 
    such that the following diagram commutes up to homotopy
    \begin{center}
      \begin{tikzcd}
        \Tspec{\N M} \ar[r, "\bm{f}_M"]
        \ar[dr, bend right=20, "\thommap{\N M}"{below}]
        &\MOmodI n
        \ar[d, "\bm{\rho}_n"]\\
        &\MOspec
      \end{tikzcd}
    \end{center}
  \item\label{item:momodinprop:3}
    $\bm{\rho}_n$ factors up to homotopy over
    $\MOspec(n-\alpha(n))\to\MOspec$.
  \end{enumerate}
  \begin{proof}[Outline of the proof]
    For a proof see \cite[Lemma~2.28, Theorem~2.29]{immersionconj}.
    Statement \ref{item:momodinprop:1} requires mainly the
    considerations from above, thus follows rather directly from the
    definition.
    For \ref{item:momodinprop:2}, one piece-wise constructs a lift on
    the wedge-product factors, \idest for any
    sequence $I$ with $n\geq d=\Deg{I}\neq2^s-1$, find a map
    $\proj\circ\bm{f}_M$ making the following diagram commute:
    \begin{center}
      \begin{tikzcd}
        &&\susp^{d}\BrownGitler{n-d}
        \ar[d, "\susp^{d}y_k"]
        \\
        \ar[urr, bend left=20, dashed, "\proj\circ f_M"]
        \Tspec{\N M}
        \ar[r, "\thommap{\N M}"{below}]
        &\MOspec
        \ar[r, "\proj"]
        &\susp^{d}\EMspec
      \end{tikzcd}
    \end{center}
    Lastly, \ref{item:momodinprop:3} uses that $\BrownGitler k$ has
    modulo 2 the homotopy type of a CW complex of dimension $k-\alpha(k)$,
    which makes it possible to apply an obstruction argument to find
    the required lift.
  \end{proof}
\end{Thm*}

This gives the desired intermediate result.


\section*{The Construction of $\BOmodI n$}
Now that a nice universal spectrum for Thom spaces of normal bundles
of $n$-manifolds is found, one has to somehow \emph{de-Thom-ify} this
result to a universal space of normal bundles with the desired
properties.

Let us first specify the meaning of de-Thom-ification.
\begin{Def*}
  For a classifying map $\xi\colon X\to\BO$, a map of spectra
  $\bm{\tau}\colon\spectrum{Z}\to\Tspec{\xi}$ is said to
  \emph{de-Thom-ify} through dimension $k$ if there is
  \begin{itemize}
  \item a map of spaces $g\colon Y\to X$, and
  \item a $k$-connected map of spectra
    $\bm{\kappa}\colon\Tspec{(\pb g\xi)}\to\spectrum{Z}$,
  \end{itemize}
  such that the following diagram homotopy commutes
  \begin{center}
    \begin{tikzcd}
      \spectrum{Z} \ar[r, "\bm{\tau}"]
      \ar[d, "\bm{\kappa}"]
      &\Tspec{\xi}\\
      \ar[ur, bend right=30, "~\thommap{g}"{right}]
      \Tspec{(\pb g\xi)}
    \end{tikzcd}
    .
  \end{center}
\end{Def*}


On the way a sufficient criterion for de-Thom-ification it
is necessary to observe what additional structure a
map of Thom spectra has if it is induced by a map on vector bundles.
Indeed, such maps are linear with respect to an additional module structure
(see \cite[p.~107ff]{immersionconj}).
\begin{Lem*}
  \begin{enumerate}
  \item On the tensor product $\A\otimes\H^*(\BO)$ define the multiplication
    \begin{gather*}
      (a\otimes x) \cdot (b\otimes y)
      \coloneqq \sum_{i\in A}
      \left(a\circ b_i'\right) \otimes \left(\antipode(b_i'')(x)\cdot y\right)
    \end{gather*}
    where $b_i', b_i''$ is such that $\sum_{i\in A}b_i'\otimes b_i''$ is
    the image of $b$ under the diagonal map of $\A$
    (see Definition~\ref{def:steenrodalgebra}).
    This turns $\A\otimes\H^*(\BO)$ into a ring.
  \item For any classifying map $\xi\colon X\to\BO$, the
    corresponding $\H^*(\BO)$-module structure on $\H^*(X)$
    induces the $(\A\otimes\H^*(\BO))$-structure on $\H^*(\Tspec{\xi})$
    \begin{gather*}
      (a\otimes x)\cdot \thomiso(y) = a\left( \thomiso(\pb\xi x\cup y \right)
    \end{gather*}
    via the Thom isomorphism.
  \item For any map of vector bundles, the induced map on Thom spaces
    is $(\A\otimes\H^*(\BO))$-linear on cohomology.
  \end{enumerate}
\end{Lem*}

In the spirit of the above observation, one gets even more additional
structure around Thom maps induced by pullbacks of vector bundles.
An essential result is the following de-Thom-ification criterion:
\begin{Thm*}
  Let $k\in\Nat$ and $\xi\colon X\to\BO$ be $k$-connected.
  Assume there is a map of spectra
  $\bm{\tau}\colon\spectrum{Z}\to\Tspec{\xi}$.
  Then $\bm{\tau}$ de-Thom-ifies through dimension $2k$ if and only if
  in dimensions lower or equals $2k$ the cohomology ring
  $\H^*(\spectrum{C}_{\bm{\tau}})$ of the mapping cone of $\bm{\tau}$ is
  a free module over $\A\otimes\H^*(\BO)$.  
  \begin{proof}[Outline of the proof]
    See \cite[Theorem~3.5]{immersionconj}.
    For the necessity one should observe that a pullback $g\colon Y\to X$ of
    vector bundles can be modified to a certain 2-stage Postnikov tower
    \begin{gather*}
      Y'\longto X\overset{\xi}\longto\BO
    \end{gather*}
    without changing the homotopy type of $\Tspec{(\pb g\xi)}$ up to
    dimension $2k$.
    Then one can use a theorem by Brown and Peterson, which yields the
    needed result for the cohomology of the modified Thom mapping cone
    (see \cite[Theorem~3.3]{immersionconj}).
    For the other direction, a construction that rebuilds a system as
    the above one using the Thom isomorphism can be applied.
  \end{proof}
\end{Thm*}


Brown and Peterson found a nice Adams resolution of spectra
\begin{gather*}
  \MOmodI n \to\dotsc \to \spectrum{Z}_1 \to \spectrum{Z}_0 = \MOspec
\end{gather*}
which in each stage de-Thom-ifies through dimension $n$ according to
the above criterion.
This finally gives:
\begin{Def}
  Let
  \begin{gather*}
    \dotsc \to Y_1 \to Y_0 = \BO
 \end{gather*}
  be the result of the de-Thom-ification of the above resolution.
  Define $\BOmodI n$ as the $n$-skeleton of the inverse limit of the
  $Y_i$, and the map $\rho_n\colon\BOmodI n\to\BO$ by
  \begin{gather*}
    \rho_n\colon\BOmodI n\to\dotsc\to Y_1\to Y_0=\BO
    \;.
  \end{gather*}
\end{Def}

\begin{Thm*}
  The special choice of the resolution equips
  $\rho_n\colon\BOmodI n\to\BO$ with the following properties:
  \begin{enumerate}
  \item\label{item:bomodinprop:1}
    $\thommap{\rho_n} = \bm{\rho}_n\colon \Tspec{\rho_n}\simeq\MOmodI n\to\MOspec$.
  \item\label{item:bomodinprop:2}
    $\pb\rho_n=\proj\colon
    \H^*(\BOmodI n) \to \H^*(\BO)/\I n\cong\H^*(\BOmodI n)$.
  \item\label{item:bomodinprop:3}
    $\rho_n$ acts as universal bundle for stable normal bundles of
    $n$-manifolds.
    \Idest for every $n$-manifold there exists a lift $f_M$ of
    $\N M$ 
    such that the following diagram commutes up to homotopy
    \begin{center}
      \begin{tikzcd}
        M \ar[r, "f_M"]
        \ar[dr, bend right=30, "\N M"{left}]
        &\BOmodI n
        \ar[d, "\rho_n"]\\
        &\BO
      \end{tikzcd}
    \end{center}
  \end{enumerate}
  \begin{proof}[Outline of the proof]
    Statements \ref{item:bomodinprop:1} and \ref{item:bomodinprop:2}
    follow from the choice of $\BOmodI n$ as de-Thom-ification of
    $\bm{\rho}_n\colon\MOmodI n\to\MOspec$, and application of the Thom
    isomorphism.
    For \ref{item:bomodinprop:3}, consider the classifying map $\N M$
    of a stable normal bundle of an $n$-manifold. This can inductively
    be lifted to the spaces $Y_i$, again
    using obstruction theory,
    which in this case only needs to be checked on the Thom spectrum
    level. However, for the latter a nice property of the chosen
    resolution is applicable.
  \end{proof}
\end{Thm*}

This is already very close to the desired final result.
However, note that Brown and Peterson could not directly conclude the
needed factorisation over $\BO(n-\alpha(n))$ from the definition
(compare \cite{brownpeterson}).
This still required quite a lot of work done by Cohen.


\section*{The Construction of the Lift $\BOmodI n\to\BO(n-\alpha(n))$}
Cohen's approach to finding the factorisation of
$\rho_n\colon\BOmodI n\to\BO$ over $\BO(n-\alpha(n))$ is split into two
main parts:
\begin{itemize}
\item Show an indirect version of the statement using a carefully
  constructed helper space $X_n$ (\cite[Lemma~B]{cohen}), and
\item prove that this already implies the factorisation.
\end{itemize}

\begin{Def*}
  Let $P_n\coloneqq \BOmodI n \times_{\BO}\BO(n-\alpha(n))$
  be the usual homotopy pullback of
  \begin{center}
    \begin{tikzcd}
      \BOmodI n
      \ar[r, "{\rho_n}"]
      &\BO
      &\ar[l, "{\incl}"{above}]
      \BO(n-\alpha(n))
    \end{tikzcd}
    ,
  \end{center}
  \idest defined by
  \begin{gather*}
    P_n
    \coloneqq \left\{
      (x, y, \gamma)
      \in \BOmodI n\times \BO(n-\alpha(n))\times \BO^I
      \;\middle|\;
      \gamma(0) = \rho_n(x), \gamma(1) = \incl(y)
    \right\}
    \;.
  \end{gather*}
\end{Def*}
Cohen's Lemma~B is as follows.
\begin{Thm*}
  There exists a space and a map $p_n\colon X_n\to P_n$
  respectively maps $f_n$, $g_n$ fitting into the resulting
  homotopy commutative diagram
  \begin{center}
    \begin{tikzcd}[column sep=small]
      X_n \ar[dr, "p_n"]
      \ar[drr, bend left=20,  "f_n"]
      \ar[ddr, bend right=20, "g_n"{left}]\\
      &P_n \ar[r]\ar[d]
      &\BO(n-\alpha(n)) \ar[d, "\incl"{right}]\\
      &\BOmodI n \ar[r, "\rho_n"]
      &\BO
    \end{tikzcd}
  \end{center}
  satisfying the following properties.
  Let $\xi_n\coloneqq \rho_n\circ g_n=\incl\circ f_n$ be the vector bundle on
  $X$ defined by the map into $\BO$.
  \begin{enumerate}
  \item\label{item:bomodinfactorisation:1}
    There is a map $\bm{s}_n\colon \MOmodI n\to\Tspec{\xi_n}$ of
    spectra, such that
    \begin{center}
      \begin{tikzcd}
        \MOmodI n
        \ar[r, "\bm{s}_n"]
        &\Tspec{\xi_n}
        \ar[r, "\thommap{g_n}"]
        &\MOmodI n
      \end{tikzcd}
    \end{center}
    is homotopic to the identity, \idest defines a split.
  \item\label{item:bomodinfactorisation:2}
    Furthermore, $\bm{s}_n\circ \thommap{g_n}$ is close to the
    identity in the sense that the following diagram of spectra
    commutes up to homotopy
    \begin{center}
      \begin{tikzcd}
        \Tspec{\xi_n}
        \ar[r, "\thommap{p_n}"]
        \ar[d, "\thommap{g_n}"{left}]
        & \Tspec{\xi_n} \\
        \MOmodI n
        \ar[d, "\bm{s}_n"{left}] \\
        \Tspec{\xi_n}
        \ar[uur, bend right=15, "\thommap{p_n}"{right, near end}]
      \end{tikzcd}
    \end{center}
  \end{enumerate}
\end{Thm*}
Cohen showed that Lemma~B implies the existence of a lift of $\rho_n$
to $\BO(n-\alpha(n))$ along $\incl$. This was done by inductively
constructing lifts along a slightly modified Postnikov tower of
the map $\incl\colon\BO(n-\alpha(n))\to\BO$.
The induction step uses that the obstruction to such a lift is
a map from a Thom spectrum to a spectrum which again admits a
free $(\A\otimes\H^*(\BO))$-module structure in lower dimensions.
The latter is then used to get rid of the obstruction
(compare \cite[p.~118f]{immersionconj}).

\begin{proof}[Outline of the proof of Lemma~B]
  For more details see \cite[Chap.~III, §2]{immersionconj}.
  We sketch the construction of $X_n$ below.
  The idea is to dissect the map $\MOmodI n\to\MO$ into its
  wedge summands in order to see that this actually factors over a
  wedge of Thom maps of vector bundles.
  $\xi_n\colon X_n\to\BO$ can be constructed from these vector bundles
  as disjoint sum of products in the correct dimension.
  Thus, first observe that
  \begin{gather*}
    \MOmodI n
    \coloneqq
    % \bigvee_{\mathrlap{n\geq\Deg{I}\neq 2^s-1}}
    % \susp^{\Deg{I}} \BrownGitler{n-\deg{I}}
    \bigvee_{\mathrlap{n\geq\Deg{I}\neq 2^s-1}}
    \Spherespec{\Deg{I}} \smashprod \BrownGitler{n-\deg{I}}
    \longto
    \bigvee_{\mathrlap{\Deg{I}\neq 2^s-1}}
    \Spherespec{\Deg{I}} \smashprod  \EMspec
    \cong
    \MOspec
  \end{gather*}
  on each wedge factor looks like
  \begin{center}
    \begin{tikzcd}[column sep=large]
      \Spherespec{\Deg{I}} \smashprod \BrownGitler{n-\deg{I}}
      \ar[r, "M_{\Deg{I}}\smashprod y_k"]
      &\MOspec \smashprod \MOspec
      \ar[r]
      &\MOspec
    \end{tikzcd}
  \end{center}
  where the second map is the multiplication of the ring spectrum
  $\MO$.
  Now, both $M_{\Deg{I}}$ as well as $y_k$ originate
  from classifying maps of vector bundles, which furthermore factor
  over $\BO(n-\alpha(n))$:
  \begin{description}
  \item[$y_k$:]
    Using a specific construction shows that
    the $k$th Brown-Gitler spectrum is in fact the Thom spectrum of a
    vector bundle $\tilde y_k\colon\BraidSpace k\to\BO$ over the
    Eilenberg-MacLane space $\BraidSpace k\coloneqq K(\Braid{k},1)$,
    where $\Braid k$ is the $k$th braid group
    (see \cite[Chap.~II, §2]{immersionconj}).
    The homotopy type of $\BraidSpace{k}$ shows that $\tilde y_k$
    factors over $\BO(k-\alpha(k))$ by obstruction theory.
  \item[$M_{\Deg{I}}$:]
    The stable homotopy group $\pi_k^s(\MOspec)$ is isomorphic
    to the $k$th cobordism group $\c_k$
    via the Thom-Pontryagin isomorphism (see \cite[Chap.~II]{stong}).
    The latter also says that a map $\Spherespec{k}\to\MOspec$
    represented by a manifold $M^k$ representing under the above
    isomorphism factors up to homotopy as
    \begin{center}
      \begin{tikzcd}
        \Spherespec{k} \ar[r, "\spectrum{c}"]
        &\Tspec{\N M} \ar[r, "\thommap{\N M}"]
        &\MOspec
      \end{tikzcd}
    \end{center}
    where $\bm{c}$ is the stable version of a collapse map.
    Furthermore, due to Theorem~\ref{thm:brown}, one can always choose
    a representative $M^k$ for which the stable normal bundle
    $\N M\colon M\to\BO$ factors over $\BO(n-\alpha(n))$.
  \end{description}
  The map $\xi_n\colon X_n\to\BO$ will look like
  \begin{gather*}
    \coprod_{\mathclap{n\geq\Deg{I}\neq 2^s-1}}
    M_{\Deg{I}} \times \BraidSpace{n-\Deg{I}}
    \xrightarrow{\coprod \N{M_{\Deg{I}}} \times \tilde y_{n-\Deg{I}}}
    \BO
  \end{gather*}
  where $M_{\Deg{I}}$ is a manifold representing the map
  $M_{\Deg{I}}$, chosen such that it fulfils the immersion
  property.
  A choice like this will factor over
  $\incl\colon\BO(n-\alpha(n))\to\BO$ as explained above.
  And a factorisation over $\rho_n\colon\BOmodI n\to\BO$ is directly
  given if one observes that
  \begin{itemize}
  \item $\rho_i\times\rho_j\colon\BOmodI i\times\BOmodI j\to\BO$ factors
    over $\rho_{i+j}$,
  \item $\tilde y_k$ factors over $\rho_k$, and that
  \item all stable normal bundles of $k$-manifolds factor over
    $\rho_k$ by its universality property.
  \end{itemize}
  So, bundles of the form described above are candidates
  for $\xi_n$.
  They can also be shown to fulfil the splitting
  property \ref{item:bomodinfactorisation:1} using the well-known
  formula
  \begin{gather*}
    \Tspec{(\N{M_i} \times \tilde y_j)}
    = \Tspec{\N{M_i}} \smashprod \Tspec{\tilde y_j}
    \;.
  \end{gather*}
  However, if one wants to obtain a vector bundle satisfying
  condition~\ref{item:bomodinfactorisation:2} using the above
  construction, one has to be careful when choosing the
  set of representatives $\{M_{\Deg{I}}\,|\, n\geq\Deg{I}\neq 2^s-1\}$.
  Cohen's approach here is an induction on $n$:
  For $\Deg{I}=n$,
  \begin{itemize}
  \item the decomposable classes are represented as product
    manifold of the lower dimensional representatives, and
  \item the (unique) indecomposable class $\Spherespec{n}\to\MO$ of degree
    $n$ is shown to originate from a split
    \begin{gather*}
      \bm{j}\colon
      \Tspec{(\rho_n|_{\subcxBOmodI n})} \vee \Spherespec{n}
      \overset{\cong}\longrightarrow \MOmodI n
    \end{gather*}
    where $\subcxBOmodI n$ is a nice subcomplex of $\BOmodI n$.
  \end{itemize}
  The neat properties of $\subcxBOmodI n$ and the induction assumption
  then imply \ref{item:bomodinfactorisation:2} for this choice of
  $\xi_n$.
\end{proof}

With this Cohen finalised the proof of the immersion conjecture.


%\optcite[Theorem~3.34]{The Topology of Fiber Bundles Lecture Notes, Cohen}

%%% Local Variables:
%%% mode: latex
%%% TeX-master: "thesis"
%%% ispell-local-dictionary: "en_GB"
%%% End:
