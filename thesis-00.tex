%%%%%%%%%%%%%%%%%%%%%%%%%%%%%%%%%% 
% Master Thesis in Mathematics
% "Immersions and Stiefel-Whitney classes of Manifolds"
% -- Chapter 0: Introduction --
% 
% Author: Gesina Schwalbe
% Supervisor: Georgios Raptis
% University of Regensburg 2018
%%%%%%%%%%%%%%%%%%%%%%%%%%%%%%%%%% 

\chapter*{Introduction} % TODO
% Immersions are roughly speaking mappings between
% manifolds that at least locally look like embeddings, globally however
% allow self-intersection or -convergence of the image.
After Whitney's immersion theorem~\cite{whitneyimmersiontheorem}
showed that every $n$-manifold can be immersed into real space of
codimension $2n$, naturally the following question arose:
\begin{quote}
  What is the minimum codimension $k$ such that for every
  $n$-dimensional compact manifold there exists an
  immersion $M\immto\R^{n+k}$ into real space?
\end{quote}
The immersion conjecture asserts that $k=n-\alpha(n)$ where
$\alpha(n)$ is the number of ones in the binary notation of $n$.
The goal of this thesis is to reformulate the question to a homotopy
theoretic context, and to show several ways how characteristic classes
can be involved in the investigation of this problem.
More precisely, there are given details on the following two aspects:
On the one hand, a theorem of Massey~\cite{massey} voids an important
obstruction through Stiefel-Whitney characteristic classes for
codimensions $k\geq n-\alpha(n)$, thus motivating the value of $k$ in
the conjecture.
And secondly, a criterion given by characteristic classes for
indecomposability in the cobordism ring due to Thom~\cite{thom}
is the main ingredient for a theorem of R.~L.~Brown~\cite{brown},
which states that the immersion conjecture is true up to cobordism.


Timeline:
\begin{description}
\item[1944] Whitney's immersion theorem \cite{whitneyimmersiontheorem}
\item[1950] Wu's theorem \cite{wu}
\item[1954] Structure of the cobordism ring by Thom \cite{thomfrench}
\item[1959] Hirsch-Smale theorem by M.~W.~Hirsch \cite{hirschimmersions} (PhD thesis)
\item[1960] Massey's theorem \cite{massey}
\item[1971] R.~L.~Brown's theorem \cite{brown}
\item[1985] proof of immersion conjecture by R.~L.~Cohen \cite{immersionconj}
\end{description}

\begin{itemize}
\item immersion problem and relation to characteristic classes
\item state \& reformulate immersion conjecture
\item two steps necessary for proof (not given; due to Cohen)
\item (Massey) show that obstructions by characteristic classes vanish for
  $n-\alpha(n)$ and that this is possible best result
\item (Brown) show that is true up to cobordism and how certain
  characteristic classes play a major role (to detect indecomposable
  elements of the cobordism ring)
\item Reader assumed to be familiar with
  \begin{itemize}
  \item Steenrod squares
  \item characteristic classes, especially Stiefel-Whitney classes
  \item the cobordism ring
  \end{itemize}
\end{itemize}

\section*{Notation}
\begin{itemize}
\item \enquote{Space} always means topological space, and all maps
  between spaces are continuous if not stated otherwise.
  % \item $\Zmod2 \coloneqq \Z/2\Z$
\item For $X$, $Y$ spaces $[X,Y]$ denotes the set of homotopy classes
  of base point preserving maps $X\to Y$.
\item All manifolds are topological manifolds.
\item All homology groups have $\Zmod2$ coefficients if not stated
  otherwise. Singular homology in degree $r$ of a pair $(X,A)$ with
  coefficients in a ring $R$ is denoted by $H_r(X,A;R)$, with the
  short form $H_r(X;R)\coloneqq H_r(X,\emptyset;R)$. Relative homology
  of a pointed space $(X,\ast)$ is denoted by $\relH_r(X;R)\coloneqq
  H_r(X,\ast;R)$. Analogously for singular cohomology.
\item All vector bundles are real if not stated otherwise.
\item The following notation for a fibre bundle $\xi\colon E\to B$ is used:
  \begin{itemize}
  \item $\E\xi\coloneqq E$ denotes the total space.
  \item $\B\xi\coloneqq B$ denotes the base space.
  \item $p_\xi\coloneqq p$ denotes the underlying surjection.
  \item $\E_b$ denotes the fibre over a point $b\in \B$.
  \item $0_\xi$ denotes the zero section if it exists.
  \item $\minuszerosec{E}\coloneqq E\setminus\Im\zerosec{\xi}$
    is the total space without the zero section.
  \end{itemize}
\item The following notation for cup-~and cap-product will be used,
  where $(X,A)$, $(X,B)$ are pairs of spaces, $R$ a unital commutative
  ring, and $i,j\in\Nat$:
  \begin{itemize}
  \item The (relative) cup-product is denoted as
    \begin{align*}
      -\cup-\colon
      H^i(X,A;R)\times H^j(X,B;R) &\longto H^{i+j}(X,A\cup B;R)\\
      (x, y) &\longmapsto x\cup y
    \end{align*}
  \item The (relative) cap-product is denoted as
    % TODO: cap-product vs. Kronecker pairing?
    \begin{align*}
      -\cap-\colon
      H^i(X,A;R)\times H_{i+j}(X,A\cup B;R) &\longto H_j(X,B;R)\\
      (x, \alpha) &\longmapsto x\cap \alpha \eqqcolon \capped x \alpha
    \end{align*}
  \end{itemize}
\end{itemize}



% Nomenclature:
\nomenclature{$H^i(-; R)$}{Singular cohomology of degree $i$ with coefficients
  in $R$}
\nomenclature{$K(i,G)$}{$ith$ Eilenberg-MacLane space over group $G$}
\nomenclature{$\GL n$}{Group of $n$-dimensional matrices over $\R$}
\nomenclature{$\Orth(n)$}{Group of $n$-dimensional orthogonal matrices over $\R$}
\nomenclature{$\Orth$}{Group of orthogonal matrices over $\R$}
\nomenclature{$\Sigma$}{Suspension of a space $X$,
  respectively suspension isomorphism on co-/homology
}
\nomenclature{$\zerosec\xi$}{Zero section of a vector bundle $\xi$}
\nomenclature{$\cap$}{cap-product $H^i(X;R)\times H_j(X;R) \to H_{j-i}(X;R)$}

%%% Local Variables:
%%% mode: latex
%%% TeX-master: "thesis"
%%% End:
  