%%%%%%%%%%%%%%%%%%%%%%%%%%%%%%%%%% 
% Master Thesis in Mathematics
% "Immersions and Stiefel-Whitney classes of Manifolds"
% -- Chapter 0: Introduction --
% 
% Author: Gesina Schwalbe
% Supervisor: Georgios Raptis
% University of Regensburg 2018
%%%%%%%%%%%%%%%%%%%%%%%%%%%%%%%%%% 

\chapter*{Introduction}
% a survey of literature with comments related to the current research questions
% popular presentation of project and its results
% involved fields/theories, preliminary knowledge, notation
%Timeline:
%\begin{description}
%\item[1944] Whitney's immersion theorem \cite{whitneyimmersiontheorem}
%\item[1950] Wu's theorem \cite{wu}
%\item[1954] Structure of the cobordism ring by Thom \cite{thomfrench}
%\item[1959] Hirsch-Smale theorem by M.~W.~Hirsch \cite{hirschimmersions} (PhD thesis)
%\item[1960] Massey's theorem \cite{massey}
%\item[1971] R.~L.~Brown's theorem \cite{brown}
%\item[1985] proof of immersion conjecture by R.~L.~Cohen \cite{immersionconj}
%\end{description}


Immersions are roughly speaking mappings between smooth
manifolds that locally look like embeddings, globally however
allow self-intersection, or \enquote{worse} in case the domain is not
compact.
First cornerstones of modern immersion theory are Whitney's papers on this
topic, remarkably his embedding and immersion
theorems~\cite{whitneyimmersiontheorem} published in 1944, making
immersions of certain codimension into real space always available for the
investigation of manifolds
\cite{immersiontheoryhistory,hirsch}.

Attempts to reduce the above codimension of $n-1$ for $n$ manifolds
were heated up, when a theorem of
M.~W.~Hirsch~\cite{hirschimmersions} building on important work of
Smale allowed to answer the question about the existence of an
immersion of certain codimension by the 
existence of a vector bundle monomorphism into Euclidean space of
certain dimension.
Using the theory of characteristic classes the latter can again be
reformulated to the existence of a vector bundle of this certain
dimension which is dual to the tangent bundle, \idest looks
like the pullback of a normal bundle.
Amongst other valuable tooling, this brought in obstructions by
Stiefel-Whitney classes.

In 1960, Massey~\cite{massey} showed with the help of Wu's
theorem~\cite{wu} from 1950 that these obstructions for compact
$n$-manifolds vanish for exactly all codimensions greater than
$n-\alpha(n)$, where $\alpha(n)$ is the number of ones in the binary
notation of $n$.
This motivated the conjecture that the answer to the question
\begin{quote}
    What is the minimum codimension $k$, such that for every
    $n$-dimensional compact manifold there exists an
    immersion $M\immto\R^{n+k}$ into Euclidean space?
\end{quote}
is $k=n-\alpha(n)$, later known as the immersion conjecture
\cite{cohen}.

The guess was partly reassured in 1971 by a result of R.~L.~Brown's
investigations of the cobordism ring~\cite{brown}, proving the
conjecture up to cobordism. It heavily relies on Thom's results from
1954 on the structure of the cobordism ring \cite{thomfrench},
and its idea is to provide generators of the
cobordism ring that fulfill the immersion property, partly inspired by
constructions introduced in Dold's work on a complete generating set
of the cobordism ring published in 1956 \cite{dold}.
However, the key point prior to the construction of generators is to
find a criterion for indecomposability of cobordism classes, which is
given by an indicator characteristic class.

After a long series of steps, notably by E.~H.~Brown~Jr. and
F.~P.~Peterson beginning in 1963 (see \cite{cohen}), the proof of the
conjecture was eventually finalized in 1985 by R.~L.~Cohen, by showing
that all stable normal bundles of $n$-manifolds factor over
$\B\Orth(n-\alpha(n))$.
Its main idea is to refine Steenrod's classification theorem for
vector bundles by finding for each $n$ a classifying space specifically
for stable normal bundles of $n$-manifolds, over which all classifying
maps factor. It is then shown that this factorization admits a lift to
$\B\Orth(n-\alpha(n))$.
Important intermediate steps are to prove both, the existence and the
lift, up to cobordism, and then \emph{de-Thom-ify}, as they called it,
the results.
In \cite{brownpeterson} E.~H.~Brown~Jr. and F.~P.~Peterson conducted
all these steps except for de-Thom-ifying the lift,
which was done by R.~L.~Cohen in \cite{cohen}.

The focus of this thesis is on the results by Massey and
R.~L.~Brown, especially the role of characteristic classes in tackling
these problems.
The needed reformulation using the Hirsch-Smale theorem, as well as
the obstruction by characteristic classes, is given in
\autoref{chap:reformulation}.
A detailed proof of Massey's theorem follows in \autoref{chap:massey},
and a proof of Brown's theorem finalizes the efforts in
\autoref{chap:brown}.
A final, more detailed outlook on the proof of the immersion
conjecture can be found in \autoref{chap:outlook}.

The reader is assumed to be familiar with the concepts of Steenrod
squares, characteristic classes, especially Stiefel-Whitney classes,
and the unoriented cobordism ring. Those theories will merely be
recapitulated replacing a couple of proofs by references.

\section*{Notation}
If not stated otherwise, the following notation is used.
\begin{itemize}
\item \enquote{Space} always means topological space, and all maps
  between spaces are continuous.
\item For $X$, $Y$ spaces, $[X,Y]$ denotes the set of homotopy classes
  of base point preserving maps $X\to Y$.
\item All manifolds are smooth and closed.
\item Cobordism always means unoriented cobordism.
\item All vector bundles are real.
\item The bundle $\trivbdl^r$ over a space $X$ is the rank-$r$ trivial
  bundle $X\times\R^r\to X$.
\item $\oplus$ denotes the usual Whitney sum of vector bundles.
\item For a fiber bundle $\xi\colon E\to B$ denote by:
  $\E\xi\coloneqq E$ the total space,
  $\B\xi\coloneqq B$ the base space,
  $p_\xi\coloneqq p$ the underlying surjection,
  $\E_b\coloneqq p^{-1}(b)$ the fiber over a point $b\in \B$,
  $0_\xi$ the zero section if it is exists, and by
  $\minuszerosec{E}\coloneqq E\setminus\Im\zerosec{\xi}$
  the total space without the zero section.
\item $\Zmod2 \coloneqq \Z/2\Z$.
\item All homology groups have $\Zmod2$ coefficients. Singular
  cohomology in degree $r$ of a pair $(X,A)$ with 
  coefficients in a ring $R$ is denoted by $\H^r(X,A;R)$, with the
  short form $\H^r(X;R)\coloneqq\H^r(X,\emptyset;R)$. Relative cohomology
  of a pointed space $(X,\pt)$ is denoted by
  $\relH^r(X;R)\cong\H^r(X,\pt;R)$.
  Analogously for singular homology.
\item For pairs of spaces $(X,A)$, $(X,B)$, $R$ a unital commutative
  ring, and $i,j\in\Nat$ denote
  \begin{itemize}
  \item the (relative) cup-product by
    \begin{align*}
      -\cup-\colon
      H^i(X,A;R)\times H^j(X,B;R) &\longto H^{i+j}(X,A\cup B;R)\\
      (x, y) &\longmapsto x\cup y\;,
    \end{align*}
  \item and the (relative) cap-product by
    \begin{align*}
      -\cap-\colon
      H^i(X,A;R)\times H_{i+j}(X,A\cup B;R) &\longto H_j(X,B;R)\\
      (x, \alpha) &\longmapsto x\cap \alpha \eqqcolon \capped x \alpha
                    \;.
    \end{align*}
  \item For a manifold $M^n$, $\fundcl M\in\H_n(M)$ is the unique
    fundamental class and generator.
  \end{itemize}
\end{itemize}



% Nomenclature:
%\nomenclature{$H^i(-; R)$}{Singular cohomology of degree $i$ with coefficients
%  in $R$}
%\nomenclature{$K(i,G)$}{$ith$ Eilenberg-MacLane space over group $G$}
%\nomenclature{$\GL n$}{Group of $n$-dimensional matrices over $\R$}
%\nomenclature{$\Orth(n)$}{Group of $n$-dimensional orthogonal matrices over $\R$}
%\nomenclature{$\Orth$}{Group of orthogonal matrices over $\R$}
%\nomenclature{$\Sigma$}{Suspension of a space $X$,
%  respectively suspension isomorphism on co-/homology
%}
%\nomenclature{$\zerosec\xi$}{Zero section of a vector bundle $\xi$}
%\nomenclature{$\cap$}{cap-product $H^i(X;R)\times H_j(X;R) \to H_{j-i}(X;R)$}

%%% Local Variables:
%%% mode: latex
%%% TeX-master: "thesis"
%%% ispell-local-dictionary: "en_US"
%%% End:
