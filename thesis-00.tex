%%%%%%%%%%%%%%%%%%%%%%%%%%%%%%%%%% 
% Master Thesis in Mathematics
% "Immersions and Stiefel-Whitney classes of Manifolds"
% -- Chapter 0: Introduction --
% 
% Author: Gesina Schwalbe
% Supervisor: Georgios Raptis
% University of Regensburg 2018
%%%%%%%%%%%%%%%%%%%%%%%%%%%%%%%%%% 

\chapter*{Introduction} % TOD
% a survey of literature with comments related to the current research questions
% popular presentation of project and its results
% involved fields/theories, preliminary knowledge, notation
%Timeline:
%\begin{description}
%\item[1944] Whitney's immersion theorem \cite{whitneyimmersiontheorem}
%\item[1950] Wu's theorem \cite{wu}
%\item[1954] Structure of the cobordism ring by Thom \cite{thomfrench}
%\item[1959] Hirsch-Smale theorem by M.~W.~Hirsch \cite{hirschimmersions} (PhD thesis)
%\item[1960] Massey's theorem \cite{massey}
%\item[1971] R.~L.~Brown's theorem \cite{brown}
%\item[1985] proof of immersion conjecture by R.~L.~Cohen \cite{immersionconj}
%\end{description}


Immersions are roughly speaking mappings between
manifolds that at least locally look like embeddings, globally however
allow self-intersection or, if the domain is not compact,
self-convergence of the image.
They first became of interest with Whitney's embedding and immersion
theorems~\cite{whitneyimmersiontheorem} in 1944, making immersions
of certain codimension into real space always available for the
investigation of manifolds.

Attempts to reduce the above codimension of $2n-1$ for $n$ manifolds
were heated up by new available tools, when a theorem of
M.~W.~Hirsch~\cite{hirschimmersions} allowed to answer the question
about the existence of an immersion of certain codimension by the
existence of a normal-bundle-like vector bundle of certain dimension.
Amongst others, this brought in a main obstruction by dual
Stiefel-Whitney classes.

In 1960 then, Massey~\cite{massey} showed with the help of Wu's
theorem~\cite{wu} from 1950 that this obstruction for compact
$n$-manifolds is void for exactly all codimensions greater than
$n-\alpha(n)$, where $\alpha(n)$ is the number of ones in the binary
notation of $n$.
This motivated the conjecture that the answer to the question
\begin{quote}
    What is the minimum codimension $k$ such that for every
    $n$-dimensional compact manifold there exists an
    immersion $M\immto\R^{n+k}$ into real space?
\end{quote}
is $k=n-\alpha(n)$, thereafter commonly called the immersion
conjecture.

The guess was partly reassured in 1971 by a result of R.~L.~Brown's
investigations of the cobordism ring~\cite{brown}, proving the
conjecture up to cobordism. It heavily relies on Thom's results in
1954 on the structure of the cobordism ring~\cite{thomfrench} from and
previous work by Dold, and its idea is to provide generators of the
cobordism ring that fulfil the immersion property. The representatives
for generators given by Brown can be even more simplified as
demonstrated in \cite{immersionconj}.
Nevertheless, the main work lies in finding a criterion for
indecomposability of cobordism classes, which is given by an indicator
characteristic class, once more utilising the new tools.

After a long series of steps, notably amongst others by
E.~H.~Brown~Jr. and F.~P.~Peterson and the above, the conjecture was
finally proven in 1985 by R.~L.~Cohen, showing that all stable normal
bundles of $n$-manifolds factor over $\B\Orth(n-\alpha(n))$.
Its main idea is to refine Steenrod's classification theorem for
vector bundles and find for each $n$ a classifying space specifically
for stable normal bundles of $n$-manifolds, over which all classifying
maps factor. It is then shown that this factorisation admits a lift to
 $\B\Orth(n-\alpha(n))$.
Important intermediate steps are to prove this up to cobordism and then
\emph{de-Thom-ifying}, as they called it, the result.

The attention of this thesis lies on the results by Massey and
R.~L.~Brown, especially the role of characteristic classes in tackling
these problems.
The needed reformulation using the Hirsch-Smale theorem, as well as
the obstruction by characteristic classes, is given in
\autoref{chap:reformulation}.
A detailed proof of Massey's theorem follows in \autoref{chap:massey},
and a proof of Brown's theorem finalises the efforts in
\autoref{chap:brown}.

%\begin{itemize}
%\item immersion problem and relation to characteristic classes
%\item state \& reformulate immersion conjecture
%\item two steps necessary for proof (not given; due to Cohen)
%\item (Massey) show that obstructions by characteristic classes vanish for
%  $n-\alpha(n)$ and that this is possible best result
%\item (Brown) show that is true up to cobordism and how certain
%  characteristic classes play a major role (to detect indecomposable
%  elements of the cobordism ring)
%\end{itemize}
The reader is assumed to be familiar with the concepts of Steenrod
squares, characteristic classes, especially Stiefel-Whitney classes,
and the cobordism ring. Those theories will merely be recapitulated
replacing a couple of proofs by references.

\section*{Notation}
\begin{itemize}
\item \enquote{Space} always means topological space, and all maps
  between spaces are continuous if not stated otherwise.
\item For $X$, $Y$ spaces $[X,Y]$ denotes the set of homotopy classes
  of base point preserving maps $X\to Y$.
\item All manifolds are smooth, closed manifolds if not stated otherwise.
\item All vector bundles are real if not stated otherwise.
\item $\Zmod2 \coloneqq \Z/2\Z$.
\item All homology groups have $\Zmod2$ coefficients if not stated
  otherwise. Singular cohomology in degree $r$ of a pair $(X,A)$ with
  coefficients in a ring $R$ is denoted by $\H^r(X,A;R)$, with the
  short form $\H^r(X;R)\coloneqq\H^r(X,\emptyset;R)$. Relative cohomology
  of a pointed space $(X,\ast)$ is denoted by $\relH^r(X;R)$.
  Analogously for singular homology.
\item The following notation for a fibre bundle $\xi\colon E\to B$ is used:
  $\E\xi\coloneqq E$ denotes the total space,
  $\B\xi\coloneqq B$ the base space,
  $p_\xi\coloneqq p$ the underlying surjection,
  $\E_b\coloneqq p^{-1}(b)$ the fibre over a point $b\in \B$,
  $0_\xi$ denotes the zero section if it exists, and
  $\minuszerosec{E}\coloneqq E\setminus\Im\zerosec{\xi}$
  is the total space without the zero section.
\item The following notation for cup-~and cap-product will be used,
  where $(X,A)$, $(X,B)$ are pairs of spaces, $R$ a unital commutative
  ring, and $i,j\in\Nat$:
  \begin{itemize}
  \item The (relative) cup-product is denoted as
    \begin{align*}
      -\cup-\colon
      H^i(X,A;R)\times H^j(X,B;R) &\longto H^{i+j}(X,A\cup B;R)\\
      (x, y) &\longmapsto x\cup y
    \end{align*}
  \item The (relative) cap-product is denoted as
    \begin{align*}
      -\cap-\colon
      H^i(X,A;R)\times H_{i+j}(X,A\cup B;R) &\longto H_j(X,B;R)\\
      (x, \alpha) &\longmapsto x\cap \alpha \eqqcolon \capped x \alpha
    \end{align*}
  \end{itemize}
\end{itemize}



% Nomenclature:
%\nomenclature{$H^i(-; R)$}{Singular cohomology of degree $i$ with coefficients
%  in $R$}
%\nomenclature{$K(i,G)$}{$ith$ Eilenberg-MacLane space over group $G$}
%\nomenclature{$\GL n$}{Group of $n$-dimensional matrices over $\R$}
%\nomenclature{$\Orth(n)$}{Group of $n$-dimensional orthogonal matrices over $\R$}
%\nomenclature{$\Orth$}{Group of orthogonal matrices over $\R$}
%\nomenclature{$\Sigma$}{Suspension of a space $X$,
%  respectively suspension isomorphism on co-/homology
%}
%\nomenclature{$\zerosec\xi$}{Zero section of a vector bundle $\xi$}
%\nomenclature{$\cap$}{cap-product $H^i(X;R)\times H_j(X;R) \to H_{j-i}(X;R)$}

%%% Local Variables:
%%% mode: latex
%%% TeX-master: "thesis"
%%% End:
